
\documentclass{article}
\usepackage{amsmath, amssymb}
\usepackage{geometry}
\geometry{a4paper, margin=0.75in}
\usepackage{enumitem}
\usepackage{hyperref}
\usepackage{fancyhdr}
\usepackage{tikz}
\usepackage{graphicx}
\usepackage{asymptote}
\begin{asydef}
  // Global Asymptote settings
  settings.outformat = "pdf";
  settings.render = 0;
  settings.prc = false;
  import olympiad;
  import cse5;
  size(8cm);
\end{asydef}
\pagestyle{fancy}
\fancyhead[L]{\textbf{AIME Problems}}
\fancyhead[R]{\textbf{2017}}
\fancyfoot[C]{\thepage}
\renewcommand{\headrulewidth}{0.4pt}
\renewcommand{\footrulewidth}{0.4pt}

\title{AIME Problems \\ 2017}
\date{}
\begin{document}\maketitle\thispagestyle{fancy}\newpage\section*{Problems}\begin{enumerate}[label=\arabic*., itemsep=0.5em]\item Find the number of subsets of $\{1, 2, 3, 4, 5, 6, 7, 8\}$ that are subsets of neither $\{1, 2, 3, 4, 5\}$ nor $\{4, 5, 6, 7, 8\}$.\par \vspace{0.5em}\item Teams $T_1$, $T_2$, $T_3$, and $T_4$ are in the playoffs. In the semifinal matches, $T_1$ plays $T_4$, and $T_2$ plays $T_3$. The winners of those two matches will play each other in the final match to determine the champion. When $T_i$ plays $T_j$, the probability that $T_i$ wins is $\frac{i}{i+j}$, and the outcomes of all the matches are independent. The probability that $T_4$ will be the champion is $\frac{p}{q}$, where $p$ and $q$ are relatively prime positive integers. Find $p+q$.\par \vspace{0.5em}\item A triangle has vertices $A(0,0)$, $B(12,0)$, and $C(8,10)$. The probability that a randomly chosen point inside the triangle is closer to vertex $B$ than to either vertex $A$ or vertex $C$ can be written as $\frac{p}{q}$, where $p$ and $q$ are relatively prime positive integers. Find $p+q$.\par \vspace{0.5em}\item Find the number of positive integers less than or equal to $2017$ whose base-three representation contains no digit equal to $0$.\par \vspace{0.5em}\item A set contains four numbers. The six pairwise sums of distinct elements of the set, in no particular order, are $189$, $320$, $287$, $234$, $x$, and $y$. Find the greatest possible value of $x+y$.\par \vspace{0.5em}\item Find the sum of all positive integers $n$ such that $\sqrt{n^2+85n+2017}$ is an integer.\par \vspace{0.5em}\item Find the number of integer values of $k$ in the closed interval $[-500,500]$ for which the equation $\log(kx)=2\log(x+2)$ has exactly one real solution.\par \vspace{0.5em}\item Find the number of positive integers $n$ less than $2017$ such that 
\begin{equation*}
1+n+\frac{n^2}{2!}+\frac{n^3}{3!}+\frac{n^4}{4!}+\frac{n^5}{5!}+\frac{n^6}{6!}
\end{equation*}
 is an integer.\par \vspace{0.5em}\item A special deck of cards contains $49$ cards, each labeled with a number from $1$ to $7$ and colored with one of seven colors. Each number-color combination appears on exactly one card. Sharon will select a set of eight cards from the deck at random. Given that she gets at least one card of each color and at least one card with each number, the probability that Sharon can discard one of her cards and $\textit{still}$ have at least one card of each color and at least one card with each number is $\frac{p}{q}$, where $p$ and $q$ are relatively prime positive integers. Find $p+q$.\par \vspace{0.5em}\item Rectangle $ABCD$ has side lengths $AB=84$ and $AD=42$. Point $M$ is the midpoint of $\overline{AD}$, point $N$ is the trisection point of $\overline{AB}$ closer to $A$, and point $O$ is the intersection of $\overline{CM}$ and $\overline{DN}$. Point $P$ lies on the quadrilateral $BCON$, and $\overline{BP}$ bisects the area of $BCON$. Find the area of $\triangle CDP$.\par \vspace{0.5em}\item Five towns are connected by a system of roads. There is exactly one road connecting each pair of towns. Find the number of ways there are to make all the roads one-way in such a way that it is still possible to get from any town to any other town using the roads (possibly passing through other towns on the way).\par \vspace{0.5em}\item Circle $C_0$ has radius $1$, and the point $A_0$ is a point on the circle. Circle $C_1$ has radius $r<1$ and is internally tangent to $C_0$ at point $A_0$. Point $A_1$ lies on circle $C_1$ so that $A_1$ is located $90^{\circ}$ counterclockwise from $A_0$ on $C_1$. Circle $C_2$ has radius $r^2$ and is internally tangent to $C_1$ at point $A_1$. In this way a sequence of circles $C_1,C_2,C_3,\ldots$ and a sequence of points on the circles $A_1,A_2,A_3,\ldots$ are constructed, where circle $C_n$ has radius $r^n$ and is internally tangent to circle $C_{n-1}$ at point $A_{n-1}$, and point $A_n$ lies on $C_n$ $90^{\circ}$ counterclockwise from point $A_{n-1}$, as shown in the figure below. There is one point $B$ inside all of these circles. When $r = \frac{11}{60}$, the distance from the center $C_0$ to $B$ is $\frac{m}{n}$, where $m$ and $n$ are relatively prime positive integers. Find $m+n$.


\begin{center}
\begin{asy}
import olympiad;
import cse5;
draw(Circle((0,0),125));
draw(Circle((25,0),100));
draw(Circle((25,20),80));
draw(Circle((9,20),64));
dot((125,0));
label("$A_0$",(125,0),E);
dot((25,100));
label("$A_1$",(25,100),SE);
dot((-55,20));
label("$A_2$",(-55,20),E);
\end{asy}
\end{center}
\par \vspace{0.5em}\item For each integer $n\geq3$, let $f(n)$ be the number of $3$-element subsets of the vertices of a regular $n$-gon that are the vertices of an isosceles triangle (including equilateral triangles). Find the sum of all values of $n$ such that $f(n+1)=f(n)+78$.\par \vspace{0.5em}\item A $10\times10\times10$ grid of points consists of all points in space of the form $(i,j,k)$, where $i$, $j$, and $k$ are integers between $1$ and $10$, inclusive. Find the number of different lines that contain exactly $8$ of these points.\par \vspace{0.5em}\item Tetrahedron $ABCD$ has $AD=BC=28$, $AC=BD=44$, and $AB=CD=52$. For any point $X$ in space, define $f(X)=AX+BX+CX+DX$. The least possible value of $f(X)$ can be expressed as $m\sqrt{n}$, where $m$ and $n$ are positive integers, and $n$ is not divisible by the square of any prime. Find $m+n$.



{{AIME box|year=2017|n=II|before=|after=}}
{{MAA Notice}}\par \vspace{0.5em}\end{enumerate}
\end{document}
