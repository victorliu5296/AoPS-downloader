
\documentclass{article}
\usepackage{amsmath, amssymb}
\usepackage{geometry}
\geometry{a4paper, margin=0.75in}
\usepackage{enumitem}
\usepackage{hyperref}
\usepackage{fancyhdr}
\usepackage{tikz}
\usepackage{graphicx}
\usepackage{asymptote}
\begin{asydef}
  // Global Asymptote settings
  settings.outformat = "pdf";
  settings.render = 0;
  settings.prc = false;
  import olympiad;
  import cse5;
  size(8cm);
\end{asydef}
\pagestyle{fancy}
\fancyhead[L]{\textbf{AIME Problems}}
\fancyhead[R]{\textbf{2019}}
\fancyfoot[C]{\thepage}
\renewcommand{\headrulewidth}{0.4pt}
\renewcommand{\footrulewidth}{0.4pt}

\title{AIME Problems \\ 2019}
\date{}
\begin{document}\maketitle\thispagestyle{fancy}\newpage\section*{Problems}\begin{enumerate}[label=\arabic*., itemsep=0.5em]\item Consider the integer 
\begin{equation*}
N = 9 + 99 + 999 + 9999 + \cdots + \underbrace{99\ldots 99}_\text{321 digits}.
\end{equation*}
Find the sum of the digits of $N$.\par \vspace{0.5em}\item Jenn randomly chooses a number $J$ from $1, 2, 3,\ldots, 19, 20$. Bela then randomly chooses a number $B$ from $1, 2, 3,\ldots, 19, 20$ distinct from $J$. The value of $B - J$ is at least $2$ with a probability that can be expressed in the form $\tfrac{m}{n}$, where $m$ and $n$ are relatively prime positive integers. Find $m+n$.\par \vspace{0.5em}\item In $\triangle PQR$, $PR=15$, $QR=20$, and $PQ=25$. Points $A$ and $B$ lie on $\overline{PQ}$, points $C$ and $D$ lie on $\overline{QR}$, and points $E$ and $F$ lie on $\overline{PR}$, with $PA=QB=QC=RD=RE=PF=5$. Find the area of hexagon $ABCDEF$.\par \vspace{0.5em}\item A soccer team has $22$ available players. A fixed set of $11$ players starts the game, while the other $11$ are available as substitutes. During the game, the coach may make as many as $3$ substitutions, where any one of the $11$ players in the game is replaced by one of the substitutes. No player removed from the game may reenter the game, although a substitute entering the game may be replaced later. No two substitutions can happen at the same time. The players involved and the order of the substitutions matter. Let $n$ be the number of ways the coach can make substitutions during the game (including the possibility of making no substitutions). Find the remainder when $n$ is divided by $1000$.\par \vspace{0.5em}\item A moving particle starts at the point $(4,4)$ and moves until it hits one of the coordinate axes for the first time. When the particle is at the point $(a,b)$, it moves at random to one of the points $(a-1,b)$, $(a,b-1)$, or $(a-1,b-1)$, each with probability $\tfrac{1}{3}$, independently of its previous moves. The probability that it will hit the coordinate axes at $(0,0)$ is $\tfrac{m}{3^n}$, where $m$ and $n$ are positive integers, and $m$ is not divisible by $3$. Find $m + n$.\par \vspace{0.5em}\item In convex quadrilateral $KLMN$, side $\overline{MN}$ is perpendicular to diagonal $\overline{KM}$, side $\overline{KL}$ is perpendicular to diagonal $\overline{LN}$, $MN = 65$, and $KL = 28$. The line through $L$ perpendicular to side $\overline{KN}$ intersects diagonal $\overline{KM}$ at $O$ with $KO = 8$. Find $MO$.\par \vspace{0.5em}\item There are positive integers $x$ and $y$ that satisfy the system of equations

\begin{equation*}
\log_{10} x + 2 \log_{10} (\gcd(x,y)) = 60
\end{equation*}
 
\begin{equation*}
\log_{10} y + 2 \log_{10} (\text{lcm}(x,y)) = 570.
\end{equation*}

Let $m$ be the number of (not necessarily distinct) prime factors in the prime factorization of $x$, and let $n$ be the number of (not necessarily distinct) prime factors in the prime factorization of $y$. Find $3m+2n$.\par \vspace{0.5em}\item Let $x$ be a real number such that $\sin^{10}x+\cos^{10} x = \tfrac{11}{36}$. Then $\sin^{12}x+\cos^{12} x = \tfrac{m}{n}$ where $m$ and $n$ are relatively prime positive integers. Find $m+n$.\par \vspace{0.5em}\item Let $\tau(n)$ denote the number of positive integer divisors of $n$. Find the sum of the six least positive integers $n$ that are solutions to $\tau (n) + \tau (n+1) = 7$.\par \vspace{0.5em}\item For distinct complex numbers $z_1,z_2,\dots,z_{673}$, the polynomial 

\begin{equation*}
(x-z_1)^3(x-z_2)^3 \cdots (x-z_{673})^3
\end{equation*}

can be expressed as $x^{2019} + 20x^{2018} + 19x^{2017}+g(x)$, where $g(x)$ is a polynomial with complex coefficients and with degree at most $2016$. The value of 

\begin{equation*}
\left| \sum_{1 \le j <k \le 673} z_jz_k \right|
\end{equation*}

can be expressed in the form $\tfrac{m}{n}$, where $m$ and $n$ are relatively prime positive integers. Find $m+n$.\par \vspace{0.5em}\item In $\triangle ABC$, the sides have integer lengths and $AB=AC$. Circle $\omega$ has its center at the incenter of $\triangle ABC$. An ''excircle'' of $\triangle ABC$ is a circle in the exterior of $\triangle ABC$ that is tangent to one side of the triangle and tangent to the extensions of the other two sides. Suppose that the excircle tangent to $\overline{BC}$ is internally tangent to $\omega$, and the other two excircles are both externally tangent to $\omega$. Find the minimum possible value of the perimeter of $\triangle ABC$.\par \vspace{0.5em}\item Given $f(z) = z^2-19z$, there are complex numbers $z$ with the property that $z$, $f(z)$, and $f(f(z))$ are the vertices of a right triangle in the complex plane with a right angle at $f(z)$. There are positive integers $m$ and $n$ such that one such value of $z$ is $m+\sqrt{n}+11i$. Find $m+n$.\par \vspace{0.5em}\item Triangle $ABC$ has side lengths $AB=4$, $BC=5$, and $CA=6$. Points $D$ and $E$ are on ray $AB$ with $AB<AD<AE$. The point $F \neq C$ is a point of intersection of the circumcircles of $\triangle ACD$ and $\triangle EBC$ satisfying $DF=2$ and $EF=7$. Then $BE$ can be expressed as $\tfrac{a+b\sqrt{c}}{d}$, where $a$, $b$, $c$, and $d$ are positive integers such that $a$ and $d$ are relatively prime, and $c$ is not divisible by the square of any prime. Find $a+b+c+d$.\par \vspace{0.5em}\item Find the least odd prime factor of $2019^8 + 1$.\par \vspace{0.5em}\item Let $\overline{AB}$ be a chord of a circle $\omega$, and let $P$ be a point on the chord $\overline{AB}$. Circle $\omega_1$ passes through $A$ and $P$ and is internally tangent to $\omega$. Circle $\omega_2$ passes through $B$ and $P$ and is internally tangent to $\omega$. Circles $\omega_1$ and $\omega_2$ intersect at points $P$ and $Q$. Line $PQ$ intersects $\omega$ at $X$ and $Y$. Assume that $AP=5$, $PB=3$, $XY=11$, and $PQ^2 = \tfrac{m}{n}$, where $m$ and $n$ are relatively prime positive integers. Find $m+n$.



{{AIME box|year=2019|n=I|before=|after=}}
{{MAA Notice}}\par \vspace{0.5em}\end{enumerate}
\end{document}
