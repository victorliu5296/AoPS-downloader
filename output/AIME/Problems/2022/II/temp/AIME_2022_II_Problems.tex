
\documentclass{article}
\usepackage{amsmath, amssymb}
\usepackage{geometry}
\geometry{a4paper, margin=0.75in}
\usepackage{enumitem}
\usepackage{hyperref}
\usepackage{fancyhdr}
\usepackage{tikz}
\usepackage{graphicx}
\usepackage{asymptote}
\begin{asydef}
  // Global Asymptote settings
  settings.outformat = "pdf";
  settings.render = 0;
  settings.prc = false;
  import olympiad;
  import cse5;
  size(8cm);
\end{asydef}
\pagestyle{fancy}
\fancyhead[L]{\textbf{AIME Problems}}
\fancyhead[R]{\textbf{2022}}
\fancyfoot[C]{\thepage}
\renewcommand{\headrulewidth}{0.4pt}
\renewcommand{\footrulewidth}{0.4pt}

\title{AIME Problems \\ 2022}
\date{}
\begin{document}\maketitle\thispagestyle{fancy}\newpage\section*{Problems}\begin{enumerate}[label=\arabic*., itemsep=0.5em]\item Adults made up $\frac5{12}$ of the crowd of people at a concert. After a bus carrying $50$ more people arrived, adults made up $\frac{11}{25}$ of the people at the concert. Find the minimum number of adults who could have been at the concert after the bus arrived.\par \vspace{0.5em}\item Azar, Carl, Jon, and Sergey are the four players left in a singles tennis tournament. They are randomly assigned opponents in the semifinal matches, and the winners of those matches play each other in the final match to determine the winner of the tournament. When Azar plays Carl, Azar will win the match with probability $\frac23$. When either Azar or Carl plays either Jon or Sergey, Azar or Carl will win the match with probability $\frac34$. Assume that outcomes of different matches are independent. The probability that Carl will win the tournament is $\frac{p}{q}$, where $p$ and $q$ are relatively prime positive integers. Find $p+q$.\par \vspace{0.5em}\item A right square pyramid with volume $54$ has a base with side length $6.$ The five vertices of the pyramid all lie on a sphere with radius $\frac mn$, where $m$ and $n$ are relatively prime positive integers. Find $m+n$.\par \vspace{0.5em}\item There is a positive real number $x$ not equal to either $\tfrac{1}{20}$ or $\tfrac{1}{2}$ such that
\begin{equation*}
\log_{20x} (22x)=\log_{2x} (202x).
\end{equation*}
The value $\log_{20x} (22x)$ can be written as $\log_{10} (\tfrac{m}{n})$, where $m$ and $n$ are relatively prime positive integers. Find $m+n$.\par \vspace{0.5em}\item Twenty distinct points are marked on a circle and labeled $1$ through $20$ in clockwise order. A line segment is drawn between every pair of points whose labels differ by a prime number. Find the number of triangles formed whose vertices are among the original $20$ points.\par \vspace{0.5em}\item Let $x_1\leq x_2\leq \cdots\leq x_{100}$ be real numbers such that $|x_1| + |x_2| + \cdots + |x_{100}| = 1$ and $x_1 + x_2 + \cdots + x_{100} = 0$. Among all such $100$-tuples of numbers, the greatest value that $x_{76} - x_{16}$ can achieve is $\tfrac mn$, where $m$ and $n$ are relatively prime positive integers. Find $m+n$.\par \vspace{0.5em}\item A circle with radius $6$ is externally tangent to a circle with radius $24$. Find the area of the triangular region bounded by the three common tangent lines of these two circles.\par \vspace{0.5em}\item Find the number of positive integers $n \le 600$ whose value can be uniquely determined when the values of $\left\lfloor \frac n4\right\rfloor$, $\left\lfloor\frac n5\right\rfloor$, and $\left\lfloor\frac n6\right\rfloor$ are given, where $\lfloor x \rfloor$ denotes the greatest integer less than or equal to the real number $x$.\par \vspace{0.5em}\item Let $\ell_A$ and $\ell_B$ be two distinct parallel lines. For positive integers $m$ and $n$, distinct points $A_1, A_2, \allowbreak A_3, \allowbreak \ldots, \allowbreak A_m$ lie on $\ell_A$, and distinct points $B_1, B_2, B_3, \ldots, B_n$ lie on $\ell_B$. Additionally, when segments $\overline{A_iB_j}$ are drawn for all $i=1,2,3,\ldots, m$ and $j=1,\allowbreak 2,\allowbreak 3, \ldots, \allowbreak n$, no point strictly between $\ell_A$ and $\ell_B$ lies on more than two of the segments. Find the number of bounded regions into which this figure divides the plane when $m=7$ and $n=5$. The figure shows that there are 8 regions when $m=3$ and $n=2$.

\begin{center}
\begin{asy}
import olympiad;
import cse5;
import geometry;
size(10cm);
draw((-2,0)--(13,0));
draw((0,4)--(10,4));
label("$\ell_A$",(-2,0),W);
label("$\ell_B$",(0,4),W);
point A1=(0,0),A2=(5,0),A3=(11,0),B1=(2,4),B2=(8,4),I1=extension(B1,A2,A1,B2),I2=extension(B1,A3,A1,B2),I3=extension(B1,A3,A2,B2);
draw(B1--A1--B2);
draw(B1--A2--B2);
draw(B1--A3--B2);
label("$A_1$",A1,S);
label("$A_2$",A2,S);
label("$A_3$",A3,S);
label("$B_1$",B1,N);
label("$B_2$",B2,N);
label("1",centroid(A1,B1,I1));
label("2",centroid(B1,I1,I3));
label("3",centroid(B1,B2,I3));
label("4",centroid(A1,A2,I1));
label("5",(A2+I1+I2+I3)/4);
label("6",centroid(B2,I2,I3));
label("7",centroid(A2,A3,I2));
label("8",centroid(A3,B2,I2));
dot(A1);
dot(A2);
dot(A3);
dot(B1);
dot(B2);
\end{asy}
\end{center}
\par \vspace{0.5em}\item Find the remainder when
\begin{equation*}
\binom{\binom{3}{2}}{2} + \binom{\binom{4}{2}}{2} + \dots +  \binom{\binom{40}{2}}{2}
\end{equation*}
is divided by $1000$.\par \vspace{0.5em}\item Let $ABCD$ be a convex quadrilateral with $AB=2, AD=7,$ and $CD=3$ such that the bisectors of acute angles $\angle{DAB}$ and $\angle{ADC}$ intersect at the midpoint of $\overline{BC}.$ Find the square of the area of $ABCD.$\par \vspace{0.5em}\item Let $a, b, x,$ and $y$ be real numbers with $a>4$ and $b>1$ such that
\begin{equation*}
\frac{x^2}{a^2}+\frac{y^2}{a^2-16}=\frac{(x-20)^2}{b^2-1}+\frac{(y-11)^2}{b^2}=1.
\end{equation*}
Find the least possible value of $a+b.$\par \vspace{0.5em}\item There is a polynomial $P(x)$ with integer coefficients such that
\begin{equation*}
P(x)=\frac{(x^{2310}-1)^6}{(x^{105}-1)(x^{70}-1)(x^{42}-1)(x^{30}-1)}
\end{equation*}
holds for every $0<x<1.$ Find the coefficient of $x^{2022}$ in $P(x)$.\par \vspace{0.5em}\item For positive integers $a$, $b$, and $c$ with $a < b < c$, consider collections of postage stamps in denominations $a$, $b$, and $c$ cents that contain at least one stamp of each denomination. If there exists such a collection that contains sub-collections worth every whole number of cents up to $1000$ cents, let $f(a, b, c)$ be the minimum number of stamps in such a collection. Find the sum of the three least values of $c$ such that $f(a, b, c) = 97$ for some choice of $a$ and $b$.\par \vspace{0.5em}\item Two externally tangent circles $\omega_1$ and $\omega_2$ have centers $O_1$ and $O_2$, respectively. A third circle $\Omega$ passing through $O_1$ and $O_2$ intersects $\omega_1$ at $B$ and $C$ and $\omega_2$ at $A$ and $D$, as shown. Suppose that $AB = 2$, $O_1O_2 = 15$, $CD = 16$, and $ABO_1CDO_2$ is a convex hexagon. Find the area of this hexagon.

\begin{center}
\begin{asy}
import olympiad;
import cse5;
import geometry;
size(10cm);
point O1=(0,0),O2=(15,0),B=9*dir(30);
circle w1=circle(O1,9),w2=circle(O2,6),o=circle(O1,O2,B);
point A=intersectionpoints(o,w2)[1],D=intersectionpoints(o,w2)[0],C=intersectionpoints(o,w1)[0];
filldraw(A--B--O1--C--D--O2--cycle,0.2*red+white,black);
draw(w1);
draw(w2);
draw(O1--O2,dashed);
draw(o);
dot(O1);
dot(O2);
dot(A);
dot(D);
dot(C);
dot(B);
label("$\omega_1$",8*dir(110),SW);
label("$\omega_2$",5*dir(70)+(15,0),SE);
label("$O_1$",O1,W);
label("$O_2$",O2,E);
label("$B$",B,N+1/2*E);
label("$A$",A,N+1/2*W);
label("$C$",C,S+1/4*W);
label("$D$",D,S+1/4*E);
label("$15$",midpoint(O1--O2),N);
label("$16$",midpoint(C--D),N);
label("$2$",midpoint(A--B),S);
label("$\Omega$",o.C+(o.r-1)*dir(270));
\end{asy}
\end{center}
\par \vspace{0.5em}\end{enumerate}
\end{document}
