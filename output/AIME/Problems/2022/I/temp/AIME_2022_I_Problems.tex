
\documentclass{article}
\usepackage{amsmath, amssymb}
\usepackage{geometry}
\geometry{a4paper, margin=0.75in}
\usepackage{enumitem}
\usepackage{hyperref}
\usepackage{fancyhdr}
\usepackage{tikz}
\usepackage{graphicx}
\usepackage{asymptote}
\begin{asydef}
  // Global Asymptote settings
  settings.outformat = "pdf";
  settings.render = 0;
  settings.prc = false;
  import olympiad;
  import cse5;
  size(8cm);
\end{asydef}
\pagestyle{fancy}
\fancyhead[L]{\textbf{AIME Problems}}
\fancyhead[R]{\textbf{2022}}
\fancyfoot[C]{\thepage}
\renewcommand{\headrulewidth}{0.4pt}
\renewcommand{\footrulewidth}{0.4pt}

\title{AIME Problems \\ 2022}
\date{}
\begin{document}\maketitle\thispagestyle{fancy}\newpage\section*{Problems}\begin{enumerate}[label=\arabic*., itemsep=0.5em]\item Quadratic polynomials $P(x)$ and $Q(x)$ have leading coefficients $2$ and $-2,$ respectively. The graphs of both polynomials pass through the two points $(16,54)$ and $(20,53).$ Find $P(0) + Q(0).$\par \vspace{0.5em}\item Find the three-digit positive integer $\underline{a}\,\underline{b}\,\underline{c}$ whose representation in base nine is $\underline{b}\,\underline{c}\,\underline{a}_{\,\text{nine}},$ where $a,$ $b,$ and $c$ are (not necessarily distinct) digits.\par \vspace{0.5em}\item In isosceles trapezoid $ABCD,$ parallel bases $\overline{AB}$ and $\overline{CD}$ have lengths $500$ and $650,$ respectively, and $AD=BC=333.$ The angle bisectors of $\angle A$ and $\angle D$ meet at $P,$ and the angle bisectors of $\angle B$ and $\angle C$ meet at $Q.$ Find $PQ.$\par \vspace{0.5em}\item Let $w = \dfrac{\sqrt{3} + i}{2}$ and $z = \dfrac{-1 + i\sqrt{3}}{2},$ where $i = \sqrt{-1}.$ Find the number of ordered pairs $(r,s)$ of positive integers not exceeding $100$ that satisfy the equation $i \cdot w^r = z^s.$\par \vspace{0.5em}\item A straight river that is $264$ meters wide flows from west to east at a rate of $14$ meters per minute. Melanie and Sherry sit on the south bank of the river with Melanie a distance of $D$ meters downstream from Sherry. Relative to the water, Melanie swims at $80$ meters per minute, and Sherry swims at $60$ meters per minute. At the same time, Melanie and Sherry begin swimming in straight lines to a point on the north bank of the river that is equidistant from their starting positions. The two women arrive at this point simultaneously. Find $D.$\par \vspace{0.5em}\item Find the number of ordered pairs of integers $(a,b)$ such that the sequence 
\begin{equation*}
3,4,5,a,b,30,40,50
\end{equation*}
 is strictly increasing and no set of four (not necessarily consecutive) terms forms an arithmetic progression.\par \vspace{0.5em}\item Let $a,b,c,d,e,f,g,h,i$ be distinct integers from $1$ to $9.$ The minimum possible positive value of 
\begin{equation*}
\dfrac{a \cdot b \cdot c - d \cdot e \cdot f}{g \cdot h \cdot i}
\end{equation*}
 can be written as $\frac{m}{n},$ where $m$ and $n$ are relatively prime positive integers. Find $m+n.$\par \vspace{0.5em}\item Equilateral triangle $\triangle ABC$ is inscribed in circle $\omega$ with radius $18.$ Circle $\omega_A$ is tangent to sides $\overline{AB}$ and $\overline{AC}$ and is internally tangent to $\omega.$ Circles $\omega_B$ and $\omega_C$ are defined analogously. Circles $\omega_A,$ $\omega_B,$ and $\omega_C$ meet in six points---two points for each pair of circles. The three intersection points closest to the vertices of $\triangle ABC$ are the vertices of a large equilateral triangle in the interior of $\triangle ABC,$ and the other three intersection points are the vertices of a smaller equilateral triangle in the interior of $\triangle ABC.$ The side length of the smaller equilateral triangle can be written as $\sqrt{a} - \sqrt{b},$ where $a$ and $b$ are positive integers. Find $a+b.$\par \vspace{0.5em}\item Ellina has twelve blocks, two each of red ($\textbf{R}$), blue ($\textbf{B}$), yellow ($\textbf{Y}$), green ($\textbf{G}$), orange ($\textbf{O}$), and purple ($\textbf{P}$). Call an arrangement of blocks $\textit{even}$ if there is an even number of blocks between each pair of blocks of the same color. For example, the arrangement

\begin{equation*}
\textbf{R B B Y G G Y R O P P O}
\end{equation*}

is even. Ellina arranges her blocks in a row in random order. The probability that her arrangement is even is $\frac{m}{n},$ where $m$ and $n$ are relatively prime positive integers. Find $m+n.$\par \vspace{0.5em}\item Three spheres with radii $11,$ $13,$ and $19$ are mutually externally tangent. A plane intersects the spheres in three congruent circles centered at $A,$ $B,$ and $C,$ respectively, and the centers of the spheres all lie on the same side of this plane. Suppose that $AB^2 = 560.$ Find $AC^2.$\par \vspace{0.5em}\item Let $ABCD$ be a parallelogram with $\angle BAD < 90^\circ.$ A circle tangent to sides $\overline{DA},$ $\overline{AB},$ and $\overline{BC}$ intersects diagonal $\overline{AC}$ at points $P$ and $Q$ with $AP < AQ,$ as shown. Suppose that $AP=3,$ $PQ=9,$ and $QC=16.$ Then the area of $ABCD$ can be expressed in the form $m\sqrt{n},$ where $m$ and $n$ are positive integers, and $n$ is not divisible by the square of any prime. Find $m+n.$


\begin{center}
\begin{asy}
import olympiad;
import cse5;
defaultpen(linewidth(0.6)+fontsize(11));
size(8cm);
pair A,B,C,D,P,Q;
A=(0,0);
label("$A$", A, SW);
B=(6,15);
label("$B$", B, NW);
C=(30,15);
label("$C$", C, NE);
D=(24,0);
label("$D$", D, SE);
P=(5.2,2.6);
label("$P$", (5.8,2.6), N);
Q=(18.3,9.1);
label("$Q$", (18.1,9.7), W);
draw(A--B--C--D--cycle);
draw(C--A);
draw(Circle((10.95,7.45), 7.45));
dot(A^^B^^C^^D^^P^^Q);
\end{asy}
\end{center}
\par \vspace{0.5em}\item For any finite set $X,$ let $|X|$ denote the number of elements in $X.$ Define 
\begin{equation*}
S_n = \sum |A \cap B|,
\end{equation*}
 where the sum is taken over all ordered pairs $(A,B)$ such that $A$ and $B$ are subsets of $\{1,2,3,\ldots,n\}$ with $|A|=|B|.$ For example, $S_2 = 4$ because the sum is taken over the pairs of subsets 
\begin{equation*}
(A,B) \in \left\{(\emptyset,\emptyset),(\{1\},\{1\}),(\{1\},\{2\}),(\{2\},\{1\}),(\{2\},\{2\}),(\{1,2\},\{1,2\})\right\},
\end{equation*}
 giving $S_2 = 0+1+0+0+1+2=4.$ Let $\frac{S_{2022}}{S_{2021}} = \frac{p}{q},$ where $p$ and $q$ are relatively prime positive integers. Find the remainder when $p+q$ is divided by $1000.$\par \vspace{0.5em}\item Let $S$ be the set of all rational numbers that can be expressed as a repeating decimal in the form $0.\overline{abcd},$ where at least one of the digits $a,$ $b,$ $c,$ or $d$ is nonzero. Let $N$ be the number of distinct numerators obtained when numbers in $S$ are written as fractions in lowest terms. For example, both $4$ and $410$ are counted among the distinct numerators for numbers in $S$ because $0.\overline{3636} = \frac{4}{11}$ and $0.\overline{1230} = \frac{410}{3333}.$ Find the remainder when $N$ is divided by $1000.$\par \vspace{0.5em}\item Given $\triangle ABC$ and a point $P$ on one of its sides, call line $\ell$ the $\textit{splitting line}$ of $\triangle ABC$ through $P$ if $\ell$ passes through $P$ and divides $\triangle ABC$ into two polygons of equal perimeter. Let $\triangle ABC$ be a triangle where $BC = 219$ and $AB$ and $AC$ are positive integers. Let $M$ and $N$ be the midpoints of $\overline{AB}$ and $\overline{AC},$ respectively, and suppose that the splitting lines of $\triangle ABC$ through $M$ and $N$ intersect at $30^\circ.$ Find the perimeter of $\triangle ABC.$\par \vspace{0.5em}\item Let $x,$ $y,$ and $z$ be positive real numbers satisfying the system of equations:

\begin{equation*}
\sqrt{2x-xy} + \sqrt{2y-xy} &= 1 \\
\sqrt{2y-yz} + \sqrt{2z-yz} &= \sqrt2 \\
\sqrt{2z-zx} + \sqrt{2x-zx} &= \sqrt3.
\end{equation*}
 
Then $\left[ (1-x)(1-y)(1-z) \right]^2$ can be written as $\frac{m}{n},$ where $m$ and $n$ are relatively prime positive integers. Find $m+n.$\par \vspace{0.5em}\end{enumerate}
\end{document}
