
\documentclass{article}
\usepackage{amsmath, amssymb}
\usepackage{geometry}
\geometry{a4paper, margin=0.75in}
\usepackage{enumitem}
\usepackage{hyperref}
\usepackage{fancyhdr}
\usepackage{tikz}
\usepackage{graphicx}
\usepackage{asymptote}
\begin{asydef}
  // Global Asymptote settings
  settings.outformat = "pdf";
  settings.render = 0;
  settings.prc = false;
  import olympiad;
  import cse5;
  size(8cm);
\end{asydef}
\pagestyle{fancy}
\fancyhead[L]{\textbf{AIME Problems}}
\fancyhead[R]{\textbf{2011}}
\fancyfoot[C]{\thepage}
\renewcommand{\headrulewidth}{0.4pt}
\renewcommand{\footrulewidth}{0.4pt}

\title{AIME Problems \\ 2011}
\date{}
\begin{document}\maketitle\thispagestyle{fancy}\newpage\section*{Problems}\begin{enumerate}[label=\arabic*., itemsep=0.5em]\item Gary purchased a large beverage, but only drank $m/n$ of it, where $m$ and $n$ are relatively prime positive integers. If he had purchased half as much and drunk twice as much, he would have wasted only $2/9$ as much beverage. Find $m+n$.\par \vspace{0.5em}\item On square $ABCD$, point $E$ lies on side $AD$ and point $F$ lies on side $BC$, so that $BE=EF=FD=30$. Find the area of the square $ABCD$.\par \vspace{0.5em}\item The degree measures of the angles in a convex 18-sided polygon form an increasing arithmetic sequence with integer values. Find the degree measure of the smallest angle.\par \vspace{0.5em}\item In triangle $ABC$, $AB=20$ and $AC=11$. The angle bisector of angle $A$ intersects $BC$ at point $D$, and point $M$ is the midpoint of $AD$. Let $P$ be the point of intersection of $AC$ and the line $BM$. The ratio of $CP$ to $PA$ can be expressed in the form $\frac{m}{n}$, where $m$ and $n$ are relatively prime positive integers. Find $m+n$.\par \vspace{0.5em}\item The sum of the first $2011$ terms of a geometric sequence is $200$. The sum of the first $4022$ terms is $380$. Find the sum of the first $6033$ terms.\par \vspace{0.5em}\item Define an ordered quadruple of integers $(a, b, c, d)$ as ''interesting'' if $1 \le a<b<c<d \le 10$, and $ a+d>b+c $. How many interesting ordered quadruples are there?\par \vspace{0.5em}\item Ed has five identical green marbles, and a large supply of identical red marbles. He arranges the green marbles and some of the red ones in a row and finds that the number of marbles whose right hand neighbor is the same color as themselves is equal to the number of marbles whose right hand neighbor is the other color. An example of such an arrangement is GGRRRGGRG. Let $m$ be the maximum number of red marbles for which such an arrangement is possible, and let $N$ be the number of ways he can arrange the $m+5$ marbles to satisfy the requirement. Find the remainder when $N$ is divided by $1000$.\par \vspace{0.5em}\item Let $z_1,z_2,z_3,\dots,z_{12}$ be the 12 zeroes of the polynomial $z^{12}-2^{36}$. For each $j$, let $w_j$ be one of $z_j$ or $i z_j$. Then the maximum possible value of the real part of $\sum_{j=1}^{12} w_j$ can be written as $m+\sqrt{n}$ where $m$ and $n$ are positive integers. Find $m+n$.\par \vspace{0.5em}\item Let $x_1$, $x_2$, $\dots$, $x_6$ be nonnegative real numbers such that $x_1 + x_2 + x_3 + x_4 + x_5 + x_6 = 1$, and $x_1x_3x_5 + x_2x_4x_6 \ge {\frac{1}{540}}$. Let $p$ and $q$ be relatively prime positive integers such that $\frac{p}{q}$ is the maximum possible value of $x_1x_2x_3 + x_2x_3x_4 + x_3x_4x_5 + x_4x_5x_6 + x_5x_6x_1 + x_6x_1x_2$. Find $p + q$.\par \vspace{0.5em}\item A circle with center $O$ has radius 25. Chord $\overline{AB}$ of length 30 and chord $\overline{CD}$ of length 14 intersect at point $P$. The distance between the midpoints of the two chords is 12. The quantity $OP^2$ can be represented as $\frac{m}{n}$, where $m$ and $n$ are relatively prime positive integers. Find the remainder when $m + n$ is divided by 1000.\par \vspace{0.5em}\item Let $M_n$ be the $n \times n$ matrix with entries as follows: for $1 \le i \le n$, $m_{i,i} = 10$; for $1 \le i \le n - 1$, $m_{i+1,i} = m_{i,i+1} = 3$; all other entries in $M_n$ are zero. Let $D_n$ be the determinant of matrix $M_n$. Then $\sum_{n=1}^{\infty} \frac{1}{8D_n+1}$ can be represented as $\frac{p}{q}$, where $p$ and $q$ are relatively prime positive integers. Find $p + q$. 

Note: The determinant of the $1 \times 1$ matrix $[a]$ is $a$, and the determinant of the $2 \times 2$ matrix $\left[ {\begin{array}{cc}
 a & b  \\
 c & d  \\
 \end{array} } \right] = ad - bc$; for $n \ge 2$, the determinant of an $n \times n$ matrix with first row or first column $a_1$ $a_2$ $a_3$ $\dots$ $a_n$ is equal to $a_1C_1 - a_2C_2 + a_3C_3 - \dots + (-1)^{n+1}a_nC_n$, where $C_i$ is the determinant of the $(n - 1) \times (n - 1)$ matrix formed by eliminating the row and column containing $a_i$.\par \vspace{0.5em}\item Nine delegates, three each from three different countries, randomly select chairs at a round table that seats nine people. Let the probability that each delegate sits next to at least one delegate from another country be $\frac{m}{n}$, where $m$ and $n$ are relatively prime positive integers. Find $m + n$.\par \vspace{0.5em}\item Point $P$ lies on the diagonal $AC$ of square $ABCD$ with $AP > CP$. Let $O_1$ and $O_2$ be the circumcenters of triangles $ABP$ and $CDP$, respectively. Given that $AB = 12$ and $\angle O_1PO_2 = 120 ^{\circ}$, then $AP = \sqrt{a} + \sqrt{b}$, where $a$ and $b$ are positive integers. Find $a + b$.\par \vspace{0.5em}\item There are $N$ permutations $(a_1, a_2, \dots, a_{30})$ of $1, 2, \dots, 30$ such that for $m \in \{2,3,5\}$, $m$ divides $a_{n+m} - a_n$ for all integers $n$ with $1 \le n < n+m \le 30$. Find the remainder when $N$ is divided by 1000.\par \vspace{0.5em}\item Let $P(x) = x^2 - 3x - 9$. A real number $x$ is chosen at random from the interval $5 \le x \le 15$. The probability that $\left\lfloor\sqrt{P(x)}\right\rfloor = \sqrt{P(\lfloor x \rfloor)}$ is equal to $\frac{\sqrt{a} + \sqrt{b} + \sqrt{c} - d}{e}$ , where $a$, $b$, $c$, $d$, and $e$ are positive integers. Find $a + b + c + d + e$.\par \vspace{0.5em}\end{enumerate}
\end{document}
