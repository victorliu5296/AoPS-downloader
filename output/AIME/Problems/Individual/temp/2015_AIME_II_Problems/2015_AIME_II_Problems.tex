
\documentclass{article}
\usepackage{amsmath, amssymb}
\usepackage{geometry}
\geometry{a4paper, margin=0.75in}
\usepackage{enumitem}
\usepackage[hypertexnames=true, linktoc=all]{hyperref}
\usepackage{fancyhdr}
\usepackage{tikz}
\usepackage{graphicx}
\usepackage{asymptote}
\usepackage{arcs}
\usepackage{xwatermark}
\begin{asydef}
  // Global Asymptote settings
  settings.outformat = "pdf";
  settings.render = 0;
  settings.prc = false;
  import olympiad;
  import cse5;
  size(8cm);
\end{asydef}
\pagestyle{fancy}
\fancyhead[L]{\textbf{AIME Problems}}
\fancyhead[R]{\textbf{2015}}
\fancyfoot[C]{\thepage}
\renewcommand{\headrulewidth}{0.4pt}
\renewcommand{\footrulewidth}{0.4pt}

\title{AIME Problems \\ 2015}
\date{}
\begin{document}\maketitle\thispagestyle{fancy}\newpage\section*{2015 AIME II}
\begin{enumerate}[label=\arabic*., itemsep=0.5em]
\item Let \(N\) be the least positive integer that is both \(22\) percent less than one integer and \(16\) percent greater than another integer. Find the remainder when \(N\) is divided by \(1000\).\par \vspace{0.5em}\item In a new school, \(40\) percent of the students are freshmen, \(30\) percent are sophomores, \(20\) percent are juniors, and \(10\) percent are seniors. All freshmen are required to take Latin, and \(80\) percent of sophomores, \(50\) percent of the juniors, and \(20\) percent of the seniors elect to take Latin. The probability that a randomly chosen Latin student is a sophomore is \(\frac{m}{n}\), where \(m\) and \(n\) are relatively prime positive integers. Find \(m+n\).\par \vspace{0.5em}\item Let \(m\) be the least positive integer divisible by \(17\) whose digits sum to \(17\). Find \(m\).\par \vspace{0.5em}\item In an isosceles trapezoid, the parallel bases have lengths \(\log 3\) and \(\log 192\), and the altitude to these bases has length \(\log 16\). The perimeter of the trapezoid can be written in the form \(\log 2^p 3^q\), where \(p\) and \(q\) are positive integers. Find \(p + q\).\par \vspace{0.5em}\item Two unit squares are selected at random without replacement from an \(n \times n\) grid of unit squares. Find the least positive integer \(n\) such that the probability that the two selected unit squares are horizontally or vertically adjacent is less than \(\frac{1}{2015}\).\par \vspace{0.5em}\item Steve says to Jon, "I am thinking of a polynomial whose roots are all positive integers. The polynomial has the form \(P(x) = 2x^3-2ax^2+(a^2-81)x-c\) for some positive integers \(a\) and \(c\). Can you tell me the values of \(a\) and \(c\)?"

After some calculations, Jon says, "There is more than one such polynomial."

Steve says, "You're right.  Here is the value of \(a\)." He writes down a positive integer and asks, "Can you tell me the value of \(c\)?"

Jon says, "There are still two possible values of \(c\)."

Find the sum of the two possible values of \(c\).\par \vspace{0.5em}\item Triangle \(ABC\) has side lengths \(AB = 12\), \(BC = 25\), and \(CA = 17\). Rectangle \(PQRS\) has vertex \(P\) on \(\overline{AB}\), vertex \(Q\) on \(\overline{AC}\), and vertices \(R\) and \(S\) on \(\overline{BC}\). In terms of the side length \(PQ = w\), the area of \(PQRS\) can be expressed as the quadratic polynomial


\begin{equation*}
\text{Area}(PQRS) = \alpha w - \beta \cdot w^2.
\end{equation*}


Then the coefficient \(\beta = \frac{m}{n}\), where \(m\) and \(n\) are relatively prime positive integers. Find \(m+n\).\par \vspace{0.5em}\item Let \(a\) and \(b\) be positive integers satisfying \(\frac{ab+1}{a+b} < \frac{3}{2}\). The maximum possible value of \(\frac{a^3b^3+1}{a^3+b^3}\) is \(\frac{p}{q}\), where \(p\) and \(q\) are relatively prime positive integers. Find \(p+q\).\par \vspace{0.5em}\item A cylindrical barrel with radius \(4\) feet and height \(10\) feet is full of water. A solid cube with side length \(8\) feet is set into the barrel so that the diagonal of the cube is vertical. The volume of water thus displaced is \(v\) cubic feet. Find \(v^2\).


\begin{center}
\begin{asy}
import olympiad;
import cse5;
import three; import solids;
size(5cm);
currentprojection=orthographic(1,-1/6,1/6);

draw(surface(revolution((0,0,0),(-2,-2*sqrt(3),0)--(-2,-2*sqrt(3),-10),Z,0,360)),white,nolight);

triple A =(8*sqrt(6)/3,0,8*sqrt(3)/3), B = (-4*sqrt(6)/3,4*sqrt(2),8*sqrt(3)/3), C = (-4*sqrt(6)/3,-4*sqrt(2),8*sqrt(3)/3), X = (0,0,-2*sqrt(2));

draw(X--X+A--X+A+B--X+A+B+C);
draw(X--X+B--X+A+B);
draw(X--X+C--X+A+C--X+A+B+C);
draw(X+A--X+A+C);
draw(X+C--X+C+B--X+A+B+C,linetype("2 4"));
draw(X+B--X+C+B,linetype("2 4"));

draw(surface(revolution((0,0,0),(-2,-2*sqrt(3),0)--(-2,-2*sqrt(3),-10),Z,0,240)),white,nolight);
draw((-2,-2*sqrt(3),0)..(4,0,0)..(-2,2*sqrt(3),0));
draw((-4*cos(atan(5)),-4*sin(atan(5)),0)--(-4*cos(atan(5)),-4*sin(atan(5)),-10)..(4,0,-10)..(4*cos(atan(5)),4*sin(atan(5)),-10)--(4*cos(atan(5)),4*sin(atan(5)),0));
draw((-2,-2*sqrt(3),0)..(-4,0,0)..(-2,2*sqrt(3),0),linetype("2 4"));
\end{asy}
\end{center}
\par \vspace{0.5em}\item Call a permutation \(a_1, a_2, \ldots, a_n\) of the integers \(1, 2, \ldots, n\) ''quasi-increasing'' if \(a_k \leq a_{k+1} + 2\) for each \(1 \leq k \leq n-1\). For example, \(53421\) and \(14253\) are quasi-increasing permutations of the integers \(1, 2, 3, 4, 5\), but \(45123\) is not. Find the number of quasi-increasing permutations of the integers \(1, 2, \ldots, 7\).\par \vspace{0.5em}\item The circumcircle of acute \(\triangle ABC\) has center \(O\). The line passing through point \(O\) perpendicular to \(\overline{OB}\) intersects lines \(AB\) and \(BC\) at \(P\) and \(Q\), respectively. Also \(AB=5\), \(BC=4\), \(BQ=4.5\), and \(BP=\frac{m}{n}\), where \(m\) and \(n\) are relatively prime positive integers. Find \(m+n\).\par \vspace{0.5em}\item There are \(2^{10} = 1024\) possible \(10\)-letter strings in which each letter is either an A or a B. Find the number of such strings that do not have more than \(3\) adjacent letters that are identical.\par \vspace{0.5em}\item Define the sequence \(a_1, a_2, a_3, \ldots\) by \(a_n = \sum\limits_{k=1}^n \sin{k}\), where \(k\) represents radian measure. Find the index of the 100th term for which \(a_n < 0\).\par \vspace{0.5em}\item Let \(x\) and \(y\) be real numbers satisfying \(x^4y^5+y^4x^5=810\) and \(x^3y^6+y^3x^6=945\). Evaluate \(2x^3+(xy)^3+2y^3\).\par \vspace{0.5em}\item Circles \(\mathcal{P}\) and \(\mathcal{Q}\) have radii \(1\) and \(4\), respectively, and are externally tangent at point \(A\). Point \(B\) is on \(\mathcal{P}\) and point \(C\) is on \(\mathcal{Q}\) such that \(BC\) is a common external tangent of the two circles. A line \(\ell\) through \(A\) intersects \(\mathcal{P}\) again at \(D\) and intersects \(\mathcal{Q}\) again at \(E\). Points \(B\) and \(C\) lie on the same side of \(\ell\), and the areas of \(\triangle DBA\) and \(\triangle ACE\) are equal. This common area is \(\frac{m}{n}\), where \(m\) and \(n\) are relatively prime positive integers. Find \(m+n\).


\begin{center}
\begin{asy}
import olympiad;
import cse5;
import cse5;
pathpen=black; pointpen=black;
size(6cm);

pair E = IP(L((-.2476,1.9689),(0.8,1.6),-3,5.5),CR((4,4),4)), D = (-.2476,1.9689);

filldraw(D--(0.8,1.6)--(0,0)--cycle,gray(0.7));
filldraw(E--(0.8,1.6)--(4,0)--cycle,gray(0.7));
D(CR((0,1),1)); D(CR((4,4),4,150,390));
D(L(MP("D",D(D),N),MP("A",D((0.8,1.6)),NE),1,5.5));
D((-1.2,0)--MP("B",D((0,0)),S)--MP("C",D((4,0)),S)--(8,0));
D(MP("E",E,N));
\end{asy}
\end{center}




\{\{AIME box|year=2015|n=II|before=|after=\}\}
\{\{MAA Notice\}\}\par \vspace{0.5em}
\end{enumerate}

\end{document}
