
\documentclass{article}
\usepackage{amsmath, amssymb}
\usepackage{geometry}
\geometry{a4paper, margin=0.75in}
\usepackage{enumitem}
\usepackage[hypertexnames=true, linktoc=all]{hyperref}
\usepackage{fancyhdr}
\usepackage{tikz}
\usepackage{graphicx}
\usepackage{asymptote}
\usepackage{arcs}
\usepackage{xwatermark}
\begin{asydef}
  // Global Asymptote settings
  settings.outformat = "pdf";
  settings.render = 0;
  settings.prc = false;
  import olympiad;
  import cse5;
  size(8cm);
\end{asydef}
\pagestyle{fancy}
\fancyhead[L]{\textbf{AIME Problems}}
\fancyhead[R]{\textbf{2016}}
\fancyfoot[C]{\thepage}
\renewcommand{\headrulewidth}{0.4pt}
\renewcommand{\footrulewidth}{0.4pt}

\title{AIME Problems \\ 2016}
\date{}
\begin{document}\maketitle\thispagestyle{fancy}\newpage\section*{2016 AIME II}
\begin{enumerate}[label=\arabic*., itemsep=0.5em]
\item Initially Alex, Betty, and Charlie had a total of \(444\) peanuts. Charlie had the most peanuts, and Alex had the least. The three numbers of peanuts that each person had formed a geometric progression. Alex eats \(5\) of his peanuts, Betty eats \(9\) of her peanuts, and Charlie eats \(25\) of his peanuts. Now the three numbers of peanuts each person has forms an arithmetic progression. Find the number of peanuts Alex had initially.\par \vspace{0.5em}\item There is a \(40\%\) chance of rain on Saturday and a \(30\%\) chance of rain on Sunday. However, it is twice as likely to rain on Sunday if it rains on Saturday than if it does not rain on Saturday. The probability that it rains at least one day this weekend is \(\frac{a}{b}\), where \(a\) and \(b\) are relatively prime positive integers. Find \(a+b\).\par \vspace{0.5em}\item Let \(x,y,\) and \(z\) be real numbers satisfying the system

\begin{align*}
\log_2(xyz-3+\log_5 x)&=5,\\
\log_3(xyz-3+\log_5 y)&=4,\\
\log_4(xyz-3+\log_5 z)&=4.
\end{align*}

Find the value of \(|\log_5 x|+|\log_5 y|+|\log_5 z|\).\par \vspace{0.5em}\item An \(a \times b \times c\) rectangular box is built from \(a \cdot b \cdot c\) unit cubes. Each unit cube is colored red, green, or yellow. Each of the \(a\) layers of size \(1 \times b \times c\) parallel to the \((b \times c)\) faces of the box contains exactly \(9\) red cubes, exactly \(12\) green cubes, and some yellow cubes. Each of the \(b\) layers of size \(a \times 1 \times c\) parallel to the \((a \times c)\) faces of the box contains exactly \(20\) green cubes, exactly \(25\) yellow cubes, and some red cubes. Find the smallest possible volume of the box.\par \vspace{0.5em}\item Triangle \(ABC_0\) has a right angle at \(C_0\). Its side lengths are pairwise relatively prime positive integers, and its perimeter is \(p\). Let \(C_1\) be the foot of the altitude to \(\overline{AB}\), and for \(n \geq 2\), let \(C_n\) be the foot of the altitude to \(\overline{C_{n-2}B}\) in \(\triangle C_{n-2}C_{n-1}B\). The sum \(\sum_{n=2}^\infty C_{n-2}C_{n-1} = 6p\). Find \(p\).\par \vspace{0.5em}\item For polynomial \(P(x)=1-\dfrac{1}{3}x+\dfrac{1}{6}x^{2}\), define
\(Q(x)=P(x)P(x^{3})P(x^{5})P(x^{7})P(x^{9})=\sum_{i=0}^{50} a_ix^{i}\).
Then \(\sum_{i=0}^{50} |a_i|=\dfrac{m}{n}\), where \(m\) and \(n\) are relatively prime positive integers. Find \(m+n\).\par \vspace{0.5em}\item Squares \(ABCD\) and \(EFGH\) have a common center and \(\overline{AB} || \overline{EF}\). The area of \(ABCD\) is 2016, and the area of \(EFGH\) is a smaller positive integer. Square \(IJKL\) is constructed so that each of its vertices lies on a side of \(ABCD\) and each vertex of \(EFGH\) lies on a side of \(IJKL\). Find the difference between the largest and smallest positive integer values for the area of \(IJKL\).\par \vspace{0.5em}\item Find the number of sets \(\{a,b,c\}\) of three distinct positive integers with the property that the product of \(a,b,\) and \(c\) is equal to the product of \(11,21,31,41,51,\) and \(61\).\par \vspace{0.5em}\item The sequences of positive integers \(1,a_2, a_3,...\) and \(1,b_2, b_3,...\) are an increasing arithmetic sequence and an increasing geometric sequence, respectively. Let \(c_n=a_n+b_n\). There is an integer \(k\) such that \(c_{k-1}=100\) and \(c_{k+1}=1000\). Find \(c_k\).\par \vspace{0.5em}\item Triangle \(ABC\) is inscribed in circle \(\omega\). Points \(P\) and \(Q\) are on side \(\overline{AB}\) with \(AP<AQ\). Rays \(CP\) and \(CQ\) meet \(\omega\) again at \(S\) and \(T\) (other than \(C\)), respectively. If \(AP=4,PQ=3,QB=6,BT=5,\) and \(AS=7\), then \(ST=\frac{m}{n}\), where \(m\) and \(n\) are relatively prime positive integers. Find \(m+n\).\par \vspace{0.5em}\item For positive integers \(N\) and \(k\), define \(N\) to be \(k\)-nice if there exists a positive integer \(a\) such that \(a^{k}\) has exactly \(N\) positive divisors. Find the number of positive integers less than \(1000\) that are neither \(7\)-nice nor \(8\)-nice.\par \vspace{0.5em}\item The figure below shows a ring made of six small sections which you are to paint on a wall. You have four paint colors available and you will paint each of the six sections a solid color. Find the number of ways you can choose to paint the sections if no two adjacent sections can be painted with the same color.


\begin{center}
\begin{asy}
import olympiad;
import cse5;
draw(Circle((0,0), 4));
draw(Circle((0,0), 3));
draw((0,4)--(0,3));
draw((0,-4)--(0,-3));
draw((-2.598, 1.5)--(-3.4641, 2));
draw((-2.598, -1.5)--(-3.4641, -2));
draw((2.598, -1.5)--(3.4641, -2));
draw((2.598, 1.5)--(3.4641, 2));
\end{asy}
\end{center}
\par \vspace{0.5em}\item Beatrix is going to place six rooks on a \(6 \times 6\) chessboard where both the rows and columns are labeled \(1\) to \(6\); the rooks are placed so that no two rooks are in the same row or the same column. The ''value'' of a square is the sum of its row number and column number. The ''score'' of an arrangement of rooks is the least value of any occupied square. The average score over all valid configurations is \(\frac{p}{q}\), where \(p\) and \(q\) are relatively prime positive integers. Find \(p+q\).\par \vspace{0.5em}\item Equilateral \(\triangle ABC\) has side length \(600\). Points \(P\) and \(Q\) lie outside the plane of \(\triangle ABC\) and are on opposite sides of the plane. Furthermore, \(PA=PB=PC\), and \(QA=QB=QC\), and the planes of \(\triangle PAB\) and \(\triangle QAB\) form a \(120^{\circ}\) dihedral angle (the angle between the two planes). There is a point \(O\) whose distance from each of \(A,B,C,P,\) and \(Q\) is \(d\). Find \(d\).\par \vspace{0.5em}\item For \(1 \leq i \leq 215\) let \(a_i = \dfrac{1}{2^{i}}\) and \(a_{216} = \dfrac{1}{2^{215}}\). Let \(x_1, x_2, ..., x_{216}\) be positive real numbers such that \(\sum_{i=1}^{216} x_i=1\) and \(\sum_{1 \leq i < j \leq 216} x_ix_j = \dfrac{107}{215} + \sum_{i=1}^{216} \dfrac{a_i x_i^{2}}{2(1-a_i)}\). The maximum possible value of \(x_2=\dfrac{m}{n}\), where \(m\) and \(n\) are relatively prime positive integers. Find \(m+n\).




\{\{AIME box|year=2016|n=II|before=|after=\}\}
\{\{MAA Notice\}\}\par \vspace{0.5em}
\end{enumerate}

\end{document}
