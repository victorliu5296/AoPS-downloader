
\documentclass{article}
\usepackage{amsmath, amssymb}
\usepackage{geometry}
\geometry{a4paper, margin=0.75in}
\usepackage{enumitem}
\usepackage[hypertexnames=true, linktoc=all]{hyperref}
\usepackage{fancyhdr}
\usepackage{tikz}
\usepackage{graphicx}
\usepackage{asymptote}
\usepackage{arcs}
\usepackage{xwatermark}
\begin{asydef}
  // Global Asymptote settings
  settings.outformat = "pdf";
  settings.render = 0;
  settings.prc = false;
  import olympiad;
  import cse5;
  size(8cm);
\end{asydef}
\pagestyle{fancy}
\fancyhead[L]{\textbf{AIME Problems}}
\fancyhead[R]{\textbf{2010}}
\fancyfoot[C]{\thepage}
\renewcommand{\headrulewidth}{0.4pt}
\renewcommand{\footrulewidth}{0.4pt}

\title{AIME Problems \\ 2010}
\date{}
\begin{document}\maketitle\thispagestyle{fancy}\newpage\section*{2010 AIME I}
\begin{enumerate}[label=\arabic*., itemsep=0.5em]
\item Maya lists all the positive divisors of \(2010^2\). She then randomly selects two distinct divisors from this list. Let \(p\) be the probability that exactly one of the selected divisors is a perfect square. The probability \(p\) can be expressed in the form \(\frac {m}{n}\), where \(m\) and \(n\) are relatively prime positive integers. Find \(m + n\).\par \vspace{0.5em}\item Find the remainder when \(9 \times 99 \times 999 \times \cdots \times \underbrace{99\cdots9}_{\text{999 9's}}\) is divided by \(1000\).\par \vspace{0.5em}\item Suppose that \(y = \frac34x\) and \(x^y = y^x\). The quantity \(x + y\) can be expressed as a rational number \(\frac {r}{s}\), where \(r\) and \(s\) are relatively prime positive integers. Find \(r + s\).\par \vspace{0.5em}\item Jackie and Phil have two fair coins and a third coin that comes up heads with probability \(\frac47\). Jackie flips the three coins, and then Phil flips the three coins. Let \(\frac {m}{n}\) be the probability that Jackie gets the same number of heads as Phil, where \(m\) and \(n\) are relatively prime positive integers. Find \(m + n\).\par \vspace{0.5em}\item Positive integers \(a\), \(b\), \(c\), and \(d\) satisfy \(a > b > c > d\), \(a + b + c + d = 2010\), and \(a^2 - b^2 + c^2 - d^2 = 2010\). Find the number of possible values of \(a\).\par \vspace{0.5em}\item Let \(P(x)\) be a quadratic polynomial with real coefficients satisfying \(x^2 - 2x + 2 \le P(x) \le 2x^2 - 4x + 3\) for all real  numbers \(x\), and suppose \(P(11) = 181\). Find \(P(16)\).\par \vspace{0.5em}\item Define an ordered triple \((A, B, C)\) of sets to be \(\textit{minimally intersecting}\) if \(|A \cap B| = |B \cap C| = |C \cap A| = 1\) and \(A \cap B \cap C = \emptyset\). For example, \((\{1,2\},\{2,3\},\{1,3,4\})\) is a minimally intersecting triple. Let \(N\) be the number of minimally intersecting ordered triples of sets for which each set is a subset of \(\{1,2,3,4,5,6,7\}\). Find the remainder when \(N\) is divided by \(1000\).

'''Note''': \(|S|\) represents the number of elements in the set \(S\).\par \vspace{0.5em}\item For a real number \(a\), let \(\lfloor a \rfloor\) denote the greatest integer less than or equal to \(a\). Let \(\mathcal{R}\) denote the region in the coordinate plane consisting of points \((x,y)\) such that \(\lfloor x \rfloor ^2 + \lfloor y \rfloor ^2 = 25\). The region \(\mathcal{R}\) is completely contained in a disk of radius \(r\) (a disk is the union of a circle and its interior). The minimum value of \(r\) can be written as \(\frac {\sqrt {m}}{n}\), where \(m\) and \(n\) are integers and \(m\) is not divisible by the square of any prime. Find \(m + n\).\par \vspace{0.5em}\item Let \((a,b,c)\) be a real solution of the system of equations \(x^3 - xyz = 2\), \(y^3 - xyz = 6\), \(z^3 - xyz = 20\). The greatest possible value of \(a^3 + b^3 + c^3\) can be written in the form \(\frac {m}{n}\), where \(m\) and \(n\) are relatively prime positive integers. Find \(m + n\).\par \vspace{0.5em}\item Let \(N\) be the number of ways to write \(2010\) in the form \(2010 = a_3 \cdot 10^3 + a_2 \cdot 10^2 + a_1 \cdot 10 + a_0\), where the \(a_i\)'s are integers, and \(0 \le a_i \le 99\). An example of such a representation is \(1\cdot 10^3 + 3\cdot 10^2 + 67\cdot 10^1 + 40\cdot 10^0\). Find \(N\).\par \vspace{0.5em}\item Let \(\mathcal{R}\) be the region consisting of the set of points in the coordinate plane that satisfy both \(|8 - x| + y \le 10\) and \(3y - x \ge 15\). When \(\mathcal{R}\) is revolved around the line whose equation is \(3y - x = 15\), the volume of the resulting solid is \(\frac {m\pi}{n\sqrt {p}}\), where \(m\), \(n\), and \(p\) are positive integers, \(m\) and \(n\) are relatively prime, and \(p\) is not divisible by the square of any prime. Find \(m + n + p\).\par \vspace{0.5em}\item Let \(m \ge 3\) be an integer and let \(S = \{3,4,5,\ldots,m\}\). Find the smallest value of \(m\) such that for every partition of \(S\) into two subsets, at least one of the subsets contains integers \(a\), \(b\), and \(c\) (not necessarily distinct) such that \(ab = c\).

'''Note''': a partition of \(S\) is a pair of sets \(A\), \(B\) such that \(A \cap B = \emptyset\), \(A \cup B = S\).\par \vspace{0.5em}\item Rectangle \(ABCD\) and a semicircle with diameter \(AB\) are coplanar and have nonoverlapping interiors. Let \(\mathcal{R}\) denote the region enclosed by the semicircle and the rectangle. Line \(\ell\) meets the semicircle, segment \(AB\), and segment \(CD\) at distinct points \(N\), \(U\), and \(T\), respectively. Line \(\ell\) divides region \(\mathcal{R}\) into two regions with areas in the ratio \(1: 2\). Suppose that \(AU = 84\), \(AN = 126\), and \(UB = 168\). Then \(DA\) can be represented as \(m\sqrt {n}\), where \(m\) and \(n\) are positive integers and \(n\) is not divisible by the square of any prime. Find \(m + n\).\par \vspace{0.5em}\item For each positive integer \(n,\) let \(f(n) = \sum_{k = 1}^{100} \lfloor \log_{10} (kn) \rfloor\). Find the largest value of \(n\) for which \(f(n) \le 300\).

'''Note:''' \(\lfloor x \rfloor\) is the greatest integer less than or equal to \(x\).\par \vspace{0.5em}\item In \(\triangle{ABC}\) with \(AB = 12\), \(BC = 13\), and \(AC = 15\), let \(M\) be a point on \(\overline{AC}\) such that the incircles of \(\triangle{ABM}\) and \(\triangle{BCM}\) have equal radii. Then \(\frac{AM}{CM} = \frac{p}{q}\), where \(p\) and \(q\) are relatively prime positive integers. Find \(p + q\).\par \vspace{0.5em}
\end{enumerate}

\end{document}
