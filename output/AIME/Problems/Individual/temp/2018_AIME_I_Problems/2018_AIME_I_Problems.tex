
\documentclass{article}
\usepackage{amsmath, amssymb}
\usepackage{geometry}
\geometry{a4paper, margin=0.75in}
\usepackage{enumitem}
\usepackage[hypertexnames=true, linktoc=all]{hyperref}
\usepackage{fancyhdr}
\usepackage{tikz}
\usepackage{graphicx}
\usepackage{asymptote}
\usepackage{arcs}
\usepackage{xwatermark}
\begin{asydef}
  // Global Asymptote settings
  settings.outformat = "pdf";
  settings.render = 0;
  settings.prc = false;
  import olympiad;
  import cse5;
  size(8cm);
\end{asydef}
\pagestyle{fancy}
\fancyhead[L]{\textbf{AIME Problems}}
\fancyhead[R]{\textbf{2018}}
\fancyfoot[C]{\thepage}
\renewcommand{\headrulewidth}{0.4pt}
\renewcommand{\footrulewidth}{0.4pt}

\title{AIME Problems \\ 2018}
\date{}
\begin{document}\maketitle\thispagestyle{fancy}\newpage\section*{2018 AIME I}
\begin{enumerate}[label=\arabic*., itemsep=0.5em]
\item Let \(S\) be the number of ordered pairs of integers \((a,b)\) with \(1 \leq a \leq 100\) and \(b \geq 0\) such that the polynomial \(x^2+ax+b\) can be factored into the product of two (not necessarily distinct) linear factors with integer coefficients. Find the remainder when \(S\) is divided by \(1000\).\par \vspace{0.5em}\item The number \(n\) can be written in base \(14\) as \(\underline{a}\text{ }\underline{b}\text{ }\underline{c}\), can be written in base \(15\) as \(\underline{a}\text{ }\underline{c}\text{ }\underline{b}\), and can be written in base \(6\) as \(\underline{a}\text{ }\underline{c}\text{ }\underline{a}\text{ }\underline{c}\text{ }\), where \(a > 0\). Find the base-\(10\) representation of \(n\).\par \vspace{0.5em}\item Kathy has \(5\) red cards and \(5\) green cards. She shuffles the \(10\) cards and lays out \(5\) of the cards in a row in a random order. She will be happy if and only if all the red cards laid out are adjacent and all the green cards laid out are adjacent. For example, card orders RRGGG, GGGGR, or RRRRR will make Kathy happy, but RRRGR will not. The probability that Kathy will be happy is \( \frac{m}{n}\), where \(m\) and \(n\) are relatively prime positive integers. Find \(m + n\).\par \vspace{0.5em}\item In \(\triangle ABC, AB = AC = 10\) and \(BC = 12\). Point \(D\) lies strictly between \(A\) and \(B\) on \(\overline{AB}\) and point \(E\) lies strictly between \(A\) and \(C\) on \(\overline{AC}\) so that \(AD = DE = EC\). Then \(AD\) can be expressed in the form \(\dfrac{p}{q}\), where \(p\) and \(q\) are relatively prime positive integers. Find \(p+q\).\par \vspace{0.5em}\item For each ordered pair of real numbers \((x,y)\) satisfying

\begin{equation*}
\log_2(2x+y) = \log_4(x^2+xy+7y^2)
\end{equation*}
there is a real number \(K\) such that

\begin{equation*}
\log_3(3x+y) = \log_9(3x^2+4xy+Ky^2).
\end{equation*}
Find the product of all possible values of \(K\).\par \vspace{0.5em}\item Let \(N\) be the number of complex numbers \(z\) with the properties that \(|z|=1\) and \(z^{6!}-z^{5!}\) is a real number. Find the remainder when \(N\) is divided by \(1000\).\par \vspace{0.5em}\item A right hexagonal prism has height \(2\). The bases are regular hexagons with side length \(1\). Any \(3\) of the \(12\) vertices determine a triangle. Find the number of these triangles that are isosceles (including equilateral triangles).\par \vspace{0.5em}\item Let \(ABCDEF\) be an equiangular hexagon such that \(AB=6, BC=8, CD=10\), and \(DE=12\). Denote \(d\) the diameter of the largest circle that fits inside the hexagon. Find \(d^2\).\par \vspace{0.5em}\item Find the number of four-element subsets of \(\{1,2,3,4,\dots, 20\}\) with the property that two distinct elements of a subset have a sum of \(16\), and two distinct elements of a subset have a sum of \(24\). For example, \(\{3,5,13,19\}\) and \(\{6,10,20,18\}\) are two such subsets.\par \vspace{0.5em}\item The wheel shown below consists of two circles and five spokes, with a label at each point where a spoke meets a circle. A bug walks along the wheel, starting at point \(A\). At every step of the process, the bug walks from one labeled point to an adjacent labeled point. Along the inner circle the bug only walks in a counterclockwise direction, and along the outer circle the bug only walks in a clockwise direction. For example, the bug could travel along the path \(AJABCHCHIJA\), which has \(10\) steps. Let \(n\) be the number of paths with \(15\) steps that begin and end at point \(A.\) Find the remainder when \(n\) is divided by \(1000\).


\begin{center}
\begin{asy}
import olympiad;
import cse5;
size(6cm);

draw(unitcircle);
draw(scale(2) * unitcircle);
for(int d = 90; d < 360 + 90; d += 72){
draw(2 * dir(d) -- dir(d));
}

dot(1 * dir( 90), linewidth(5));
dot(1 * dir(162), linewidth(5));
dot(1 * dir(234), linewidth(5));
dot(1 * dir(306), linewidth(5));
dot(1 * dir(378), linewidth(5));
dot(2 * dir(378), linewidth(5));
dot(2 * dir(306), linewidth(5));
dot(2 * dir(234), linewidth(5));
dot(2 * dir(162), linewidth(5));
dot(2 * dir( 90), linewidth(5));

label("$A$", 1 * dir( 90), -dir( 90));
label("$B$", 1 * dir(162), -dir(162));
label("$C$", 1 * dir(234), -dir(234));
label("$D$", 1 * dir(306), -dir(306));
label("$E$", 1 * dir(378), -dir(378));
label("$F$", 2 * dir(378), dir(378));
label("$G$", 2 * dir(306), dir(306));
label("$H$", 2 * dir(234), dir(234));
label("$I$", 2 * dir(162), dir(162));
label("$J$", 2 * dir( 90), dir( 90));
\end{asy}
\end{center}
\par \vspace{0.5em}\item Find the least positive integer \(n\) such that when \(3^n\) is written in base \(143\), its two right-most digits in base \(143\) are \(01\).\par \vspace{0.5em}\item For every subset \(T\) of \(U = \{ 1,2,3,\ldots,18 \}\), let \(s(T)\) be the sum of the elements of \(T\), with \(s(\emptyset)\) defined to be \(0\). If \(T\) is chosen at random among all subsets of \(U\), the probability that \(s(T)\) is divisible by \(3\) is \(\frac{m}{n}\), where \(m\) and \(n\) are relatively prime positive integers. Find \(m\).\par \vspace{0.5em}\item Let \(\triangle ABC\) have side lengths \(AB=30\), \(BC=32\), and \(AC=34\). Point \(X\) lies in the interior of \(\overline{BC}\), and points \(I_1\) and \(I_2\) are the incenters of \(\triangle ABX\) and \(\triangle ACX\), respectively. Find the minimum possible area of \(\triangle AI_1I_2\) as \(X\) varies along \(\overline{BC}\).\par \vspace{0.5em}\item Let \(SP_1P_2P_3EP_4P_5\) be a heptagon. A frog starts jumping at vertex \(S\). From any vertex of the heptagon except \(E\), the frog may jump to either of the two adjacent vertices. When it reaches vertex \(E\), the frog stops and stays there. Find the number of distinct sequences of jumps of no more than \(12\) jumps that end at \(E\).\par \vspace{0.5em}\item David found four sticks of different lengths that can be used to form three non-congruent convex cyclic quadrilaterals, \(A,\text{ }B,\text{ }C\), which can each be inscribed in a circle with radius \(1\). Let \(\varphi_A\) denote the measure of the acute angle made by the diagonals of quadrilateral \(A\), and define \(\varphi_B\) and \(\varphi_C\) similarly. Suppose that \(\sin\varphi_A=\frac{2}{3}\), \(\sin\varphi_B=\frac{3}{5}\), and \(\sin\varphi_C=\frac{6}{7}\). All three quadrilaterals have the same area \(K\), which can be written in the form \(\dfrac{m}{n}\), where \(m\) and \(n\) are relatively prime positive integers. Find \(m+n\).



\{\{AIME box|year=2018|n=I|before=|after=\}\}
\{\{MAA Notice\}\}\par \vspace{0.5em}
\end{enumerate}

\end{document}
