
\documentclass{article}
\usepackage{amsmath, amssymb}
\usepackage{geometry}
\geometry{a4paper, margin=0.75in}
\usepackage{enumitem}
\usepackage[hypertexnames=true, linktoc=all]{hyperref}
\usepackage{fancyhdr}
\usepackage{tikz}
\usepackage{graphicx}
\usepackage{asymptote}
\usepackage{arcs}
\usepackage{xwatermark}
\begin{asydef}
  // Global Asymptote settings
  settings.outformat = "pdf";
  settings.render = 0;
  settings.prc = false;
  import olympiad;
  import cse5;
  size(8cm);
\end{asydef}
\pagestyle{fancy}
\fancyhead[L]{\textbf{AIME Problems}}
\fancyhead[R]{\textbf{2011}}
\fancyfoot[C]{\thepage}
\renewcommand{\headrulewidth}{0.4pt}
\renewcommand{\footrulewidth}{0.4pt}

\title{AIME Problems \\ 2011}
\date{}
\begin{document}\maketitle\thispagestyle{fancy}\newpage\section*{2011 AIME I}\begin{enumerate}[label=\arabic*., itemsep=0.5em]\item Jar \(A\) contains four liters of a solution that is \(45\%\) acid. Jar \(B\) contains five liters of a solution that is \(48\%\) acid. Jar \(C\) contains one liter of a solution that is \(k\%\) acid. From jar \(C\), \(\frac{m}{n}\) liters of the solution is added to jar \(A\), and the remainder of the solution in jar \(C\) is added to jar B. At the end both jar \(A\) and jar \(B\) contain solutions that are \(50\%\) acid. Given that \(m\) and \(n\) are relatively prime positive integers, find \(k + m + n\).\par \vspace{0.5em}\item In rectangle \(ABCD\), \(AB = 12\) and \(BC = 10\).  Points \(E\) and \(F\) lie inside rectangle \(ABCD\) so that \(BE = 9\), \(DF = 8\), \(\overline{BE} \parallel \overline{DF}\), \(\overline{EF} \parallel \overline{AB}\), and line \(BE\) intersects segment \(\overline{AD}\).  The length \(EF\) can be expressed in the form \(m \sqrt{n} - p\), where \(m\), \(n\), and \(p\) are positive integers and \(n\) is not divisible by the square of any prime.  Find \(m + n + p\).\par \vspace{0.5em}\item Let \(L\) be the line with slope \(\frac{5}{12}\) that contains the point \(A = (24,-1)\), and let \(M\) be the line perpendicular to line \(L\) that contains the point \(B = (5,6)\).  The original coordinate axes are erased, and line \(L\) is made the \(x\)-axis and line \(M\) the \(y\)-axis.  In the new coordinate system, point \(A\) is on the positive \(x\)-axis, and point \(B\) is on the positive \(y\)-axis.  The point \(P\) with coordinates \((-14,27)\) in the original system has coordinates \((\alpha,\beta)\) in the new coordinate system.  Find \(\alpha + \beta\).\par \vspace{0.5em}\item In triangle \(ABC\), \(AB = 125\), \(AC = 117\), and \(BC = 120\).  The angle bisector of angle \(A\) intersects \(\overline{BC}\) at point \(L\), and the angle bisector of angle \(B\) intersects \(\overline{AC}\) at point \(K\).  Let \(M\) and \(N\) be the feet of the perpendiculars from \(C\) to \(\overline{BK}\) and \(\overline{AL}\), respectively.  Find \(MN\).\par \vspace{0.5em}\item The vertices of a regular nonagon (9-sided polygon) are to be labeled with the digits 1 through 9 in such a way that the sum of the numbers on every three consecutive vertices is a multiple of 3.  Two acceptable arrangements are considered to be indistinguishable if one can be obtained from the other by rotating the nonagon in the plane.  Find the number of distinguishable acceptable arrangements.\par \vspace{0.5em}\item Suppose that a parabola has vertex \(\left(\frac{1}{4},-\frac{9}{8}\right)\) and equation \(y = ax^2 + bx + c\), where \(a > 0\) and \(a + b + c\) is an integer.  The minimum possible value of \(a\) can be written in the form \(\frac{p}{q}\), where \(p\) and \(q\) are relatively prime positive integers.  Find \(p + q\).\par \vspace{0.5em}\item Find the number of positive integers \(m\) for which there exist nonnegative integers \(x_0\), \(x_1\) , \(\dots\) , \(x_{2011}\) such that

\begin{equation*}
m^{x_0} = \sum_{k = 1}^{2011} m^{x_k}.
\end{equation*}
\par \vspace{0.5em}\item In triangle \(ABC\), \(BC = 23\), \(CA = 27\), and \(AB = 30\).  Points \(V\) and \(W\) are on \(\overline{AC}\) with \(V\) on \(\overline{AW}\), points \(X\) and \(Y\) are on \(\overline{BC}\) with \(X\) on \(\overline{CY}\), and points \(Z\) and \(U\) are on \(\overline{AB}\) with \(Z\) on \(\overline{BU}\).  In addition, the points are positioned so that \(\overline{UV} \parallel \overline{BC}\), \(\overline{WX} \parallel \overline{AB}\), and \(\overline{YZ} \parallel \overline{CA}\).  Right angle folds are then made along \(\overline{UV}\), \(\overline{WX}\), and \(\overline{YZ}\).  The resulting figure is placed on a level floor to make a table with triangular legs.  Let \(h\) be the maximum possible height of a table constructed from triangle \(ABC\) whose top is parallel to the floor.  Then \(h\) can be written in the form \(\frac{k \sqrt{m}}{n}\), where \(k\) and \(n\) are relatively prime positive integers and \(m\) is a positive integer that is not divisible by the square of any prime.  Find \(k + m + n\).

\begin{center}

\begin{center}
\begin{asy}
import olympiad;
import cse5;
unitsize(1 cm);
pair translate;
pair[] A, B, C, U, V, W, X, Y, Z;
A[0] = (1.5,2.8);
B[0] = (3.2,0);
C[0] = (0,0);
U[0] = (0.69*A[0] + 0.31*B[0]);
V[0] = (0.69*A[0] + 0.31*C[0]);
W[0] = (0.69*C[0] + 0.31*A[0]);
X[0] = (0.69*C[0] + 0.31*B[0]);
Y[0] = (0.69*B[0] + 0.31*C[0]);
Z[0] = (0.69*B[0] + 0.31*A[0]);
translate = (7,0);
A[1] = (1.3,1.1) + translate;
B[1] = (2.4,-0.7) + translate;
C[1] = (0.6,-0.7) + translate;
U[1] = U[0] + translate;
V[1] = V[0] + translate;
W[1] = W[0] + translate;
X[1] = X[0] + translate;
Y[1] = Y[0] + translate;
Z[1] = Z[0] + translate;
draw (A[0]--B[0]--C[0]--cycle);
draw (U[0]--V[0],dashed);
draw (W[0]--X[0],dashed);
draw (Y[0]--Z[0],dashed);
draw (U[1]--V[1]--W[1]--X[1]--Y[1]--Z[1]--cycle);
draw (U[1]--A[1]--V[1],dashed);
draw (W[1]--C[1]--X[1]);
draw (Y[1]--B[1]--Z[1]);
dot("$A$",A[0],N);
dot("$B$",B[0],SE);
dot("$C$",C[0],SW);
dot("$U$",U[0],NE);
dot("$V$",V[0],NW);
dot("$W$",W[0],NW);
dot("$X$",X[0],S);
dot("$Y$",Y[0],S);
dot("$Z$",Z[0],NE);
dot(A[1]);
dot(B[1]);
dot(C[1]);
dot("$U$",U[1],NE);
dot("$V$",V[1],NW);
dot("$W$",W[1],NW);
dot("$X$",X[1],dir(-70));
dot("$Y$",Y[1],dir(250));
dot("$Z$",Z[1],NE);
\end{asy}
\end{center}

\end{center}\par \vspace{0.5em}\item Suppose \(x\) is in the interval \([0,\pi/2]\) and \(\log_{24 \sin x} (24 \cos x) = \frac{3}{2}\).  Find \(24 \cot^2 x\).\par \vspace{0.5em}\item The probability that a set of three distinct vertices chosen at random from among the vertices of a regular \(n\)-gon determine an obtuse triangle is \(\frac{93}{125}\).  Find the sum of all possible values of \(n\).\par \vspace{0.5em}\item Let \(R\) be the set of all possible remainders when a number of the form \(2^n\), \(n\) a nonnegative integer, is divided by 1000. Let \(S\) be the sum of the elements in \(R\). Find the remainder when \(S\) is divided by 1000.\par \vspace{0.5em}\item Six men and some number of women stand in a line in random order.  Let \(p\) be the probability that a group of at least four men stand together in the line, given that every man stands next to at least one other man.  Find the least number of women in the line such that \(p\) does not exceed 1 percent.\par \vspace{0.5em}\item A cube with side length 10 is suspended above a plane.  The vertex closest to the plane is labeled \(A\). The three vertices adjacent to vertex \(A\) are at heights 10, 11, and 12 above the plane.  The distance from vertex \(A\) to the plane can be expressed as \(\frac{r - \sqrt{s}}{t}\), where \(r\), \(s\), and \(t\) are positive integers. Find \(r + s + t\).\par \vspace{0.5em}\item Let \(A_1 A_2 A_3 A_4 A_5 A_6 A_7 A_8\) be a regular octagon.  Let \(M_1\), \(M_3\), \(M_5\), and \(M_7\) be the midpoints of sides \(\overline{A_1 A_2}\), \(\overline{A_3 A_4}\), \(\overline{A_5 A_6}\), and \(\overline{A_7 A_8}\), respectively.  For \(i = 1, 3, 5, 7\), ray \(R_i\) is constructed from \(M_i\) towards the interior of the octagon such that \(R_1 \perp R_3\), \(R_3 \perp R_5\), \(R_5 \perp R_7\), and \(R_7 \perp R_1\).  Pairs of rays \(R_1\) and \(R_3\), \(R_3\) and \(R_5\), \(R_5\) and \(R_7\), and \(R_7\) and \(R_1\) meet at \(B_1\), \(B_3\), \(B_5\), \(B_7\) respectively.  If \(B_1 B_3 = A_1 A_2\), then \(\cos 2 \angle A_3 M_3 B_1\) can be written in the form \(m - \sqrt{n}\), where \(m\) and \(n\) are positive integers.  Find \(m + n\).\par \vspace{0.5em}\item For some integer \(m\), the polynomial \(x^3 - 2011x + m\) has the three integer roots \(a\), \(b\), and \(c\).  Find \(|a| + |b| + |c|\).\par \vspace{0.5em}\end{enumerate}
\end{document}
