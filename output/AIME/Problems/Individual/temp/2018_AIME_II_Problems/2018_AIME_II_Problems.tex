
\documentclass{article}
\usepackage{amsmath, amssymb}
\usepackage{geometry}
\geometry{a4paper, margin=0.75in}
\usepackage{enumitem}
\usepackage[hypertexnames=true, linktoc=all]{hyperref}
\usepackage{fancyhdr}
\usepackage{tikz}
\usepackage{graphicx}
\usepackage{asymptote}
\usepackage{arcs}
\usepackage{xwatermark}
\begin{asydef}
  // Global Asymptote settings
  settings.outformat = "pdf";
  settings.render = 0;
  settings.prc = false;
  import olympiad;
  import cse5;
  size(8cm);
\end{asydef}
\pagestyle{fancy}
\fancyhead[L]{\textbf{AIME Problems}}
\fancyhead[R]{\textbf{2018}}
\fancyfoot[C]{\thepage}
\renewcommand{\headrulewidth}{0.4pt}
\renewcommand{\footrulewidth}{0.4pt}

\title{AIME Problems \\ 2018}
\date{}
\begin{document}\maketitle\thispagestyle{fancy}\newpage\section*{2018 AIME II}
\begin{enumerate}[label=\arabic*., itemsep=0.5em]
\item Points \(A\), \(B\), and \(C\) lie in that order along a straight path where the distance from \(A\) to \(C\) is \(1800\) meters. Ina runs twice as fast as Eve, and Paul runs twice as fast as Ina. The three runners start running at the same time with Ina starting at \(A\) and running toward \(C\), Paul starting at \(B\) and running toward \(C\), and Eve starting at \(C\) and running toward \(A\). When Paul meets Eve, he turns around and runs toward \(A\). Paul and Ina both arrive at \(B\) at the same time. Find the number of meters from \(A\) to \(B\).\par \vspace{0.5em}\item Let \(a_{0} = 2\), \(a_{1} = 5\), and \(a_{2} = 8\), and for \(n > 2\) define \(a_{n}\) recursively to be the remainder when \(4(a_{n-1} + a_{n-2} + a_{n-3})\) is divided by \(11\). Find \(a_{2018} \cdot a_{2020} \cdot a_{2022}\).\par \vspace{0.5em}\item Find the sum of all positive integers \(b < 1000\) such that the base-\(b\) integer \(36_{b}\) is a perfect square and the base-\(b\) integer \(27_{b}\) is a perfect cube.\par \vspace{0.5em}\item In equiangular octagon \(CAROLINE\), \(CA = RO = LI = NE =\) \(\sqrt{2}\) and \(AR = OL = IN = EC = 1\). The self-intersecting octagon \(CORNELIA\) encloses six non-overlapping triangular regions. Let \(K\) be the area enclosed by \(CORNELIA\), that is, the total area of the six triangular regions. Then \(K = \frac{a}{b}\), where \(a\) and \(b\) are relatively prime positive integers. Find \(a + b\).\par \vspace{0.5em}\item Suppose that \(x\), \(y\), and \(z\) are complex numbers such that \(xy = -80 - 320i\), \(yz = 60\), and \(zx = -96 + 24i\), where \(i\) \(=\) \(\sqrt{-1}\). Then there are real numbers \(a\) and \(b\) such that \(x + y + z = a + bi\). Find \(a^2 + b^2\).\par \vspace{0.5em}\item A real number \(a\) is chosen randomly and uniformly from the interval \([-20, 18]\). The probability that the roots of the polynomial


\begin{equation*}
x^4 + 2ax^3 + (2a - 2)x^2 + (-4a + 3)x - 2
\end{equation*}


are all real can be written in the form \(\dfrac{m}{n}\), where \(m\) and \(n\) are relatively prime positive integers. Find \(m + n\).\par \vspace{0.5em}\item Triangle \(ABC\) has side lengths \(AB = 9\), \(BC =\) \(5\sqrt{3}\), and \(AC = 12\). Points \(A = P_{0}, P_{1}, P_{2}, ... , P_{2450} = B\) are on segment \(\overline{AB}\) with \(P_{k}\) between \(P_{k-1}\) and \(P_{k+1}\) for \(k = 1, 2, ..., 2449\), and points \(A = Q_{0}, Q_{1}, Q_{2}, ... , Q_{2450} = C\) are on segment \(\overline{AC}\) with \(Q_{k}\) between \(Q_{k-1}\) and \(Q_{k+1}\) for \(k = 1, 2, ..., 2449\). Furthermore, each segment \(\overline{P_{k}Q_{k}}\), \(k = 1, 2, ..., 2449\), is parallel to \(\overline{BC}\). The segments cut the triangle into \(2450\) regions, consisting of \(2449\) trapezoids and \(1\) triangle. Each of the \(2450\) regions has the same area. Find the number of segments \(\overline{P_{k}Q_{k}}\), \(k = 1, 2, ..., 2450\), that have rational length.\par \vspace{0.5em}\item A frog is positioned at the origin of the coordinate plane. From the point \((x, y)\), the frog can jump to any of the points \((x + 1, y)\), \((x + 2, y)\), \((x, y + 1)\), or \((x, y + 2)\). Find the number of distinct sequences of jumps in which the frog begins at \((0, 0)\) and ends at \((4, 4)\).



==Problem 9== 
Octagon \(ABCDEFGH\) with side lengths \(AB = CD = EF = GH = 10\) and \(BC = DE = FG = HA = 11\) is formed by removing 6-8-10 triangles from the corners of a \(23\) \(\times\) \(27\) rectangle with side \(\overline{AH}\) on a short side of the rectangle, as shown. Let \(J\) be the midpoint of \(\overline{AH}\), and partition the octagon into 7 triangles by drawing segments \(\overline{JB}\), \(\overline{JC}\), \(\overline{JD}\), \(\overline{JE}\), \(\overline{JF}\), and \(\overline{JG}\). Find the area of the convex polygon whose vertices are the centroids of these 7 triangles.


\begin{center}
\begin{asy}
import olympiad;
import cse5;
unitsize(6);
pair P = (0, 0), Q = (0, 23), R = (27, 23), SS = (27, 0);
pair A = (0, 6), B = (8, 0), C = (19, 0), D = (27, 6), EE = (27, 17), F = (19, 23),  G = (8, 23), J = (0, 23/2), H = (0, 17);
draw(P--Q--R--SS--cycle);
draw(J--B);
draw(J--C);
draw(J--D);
draw(J--EE);
draw(J--F);
draw(J--G);
draw(A--B);
draw(H--G);
real dark = 0.6;
filldraw(A--B--P--cycle, gray(dark));
filldraw(H--G--Q--cycle, gray(dark));
filldraw(F--EE--R--cycle, gray(dark));
filldraw(D--C--SS--cycle, gray(dark));
dot(A);
dot(B);
dot(C);
dot(D);
dot(EE);
dot(F);
dot(G);
dot(H);
dot(J);
dot(H);
defaultpen(fontsize(10pt));
real r = 1.3;
label("$A$", A, W*r);
label("$B$", B, S*r);
label("$C$", C, S*r);
label("$D$", D, E*r);
label("$E$", EE, E*r);
label("$F$", F, N*r);
label("$G$", G, N*r);
label("$H$", H, W*r);
label("$J$", J, W*r);
\end{asy}
\end{center}
\par \vspace{0.5em}\item Find the number of functions \(f(x)\) from \(\{1, 2, 3, 4, 5\}\) to \(\{1, 2, 3, 4, 5\}\) that satisfy \(f(f(x)) = f(f(f(x)))\) for all \(x\) in \(\{1, 2, 3, 4, 5\}\).\par \vspace{0.5em}\item Find the number of permutations of \(1, 2, 3, 4, 5, 6\) such that for each \(k\) with \(1\) \(\leq\) \(k\) \(\leq\) \(5\), at least one of the first \(k\) terms of the permutation is greater than \(k\).\par \vspace{0.5em}\item Let \(ABCD\) be a convex quadrilateral with \(AB = CD = 10\), \(BC = 14\), and \(AD = 2\sqrt{65}\). Assume that the diagonals of \(ABCD\) intersect at point \(P\), and that the sum of the areas of triangles \(APB\) and \(CPD\) equals the sum of the areas of triangles \(BPC\) and \(APD\). Find the area of quadrilateral \(ABCD\).\par \vspace{0.5em}\item Misha rolls a standard, fair six-sided die until she rolls 1-2-3 in that order on three consecutive rolls. The probability that she will roll the die an odd number of times is \(\dfrac{m}{n}\) where \(m\) and \(n\) are relatively prime positive integers. Find \(m+n\).\par \vspace{0.5em}\item The incircle \(\omega\) of triangle \(ABC\) is tangent to \(\overline{BC}\) at \(X\). Let \(Y \neq X\) be the other intersection of \(\overline{AX}\) with \(\omega\). Points \(P\) and \(Q\) lie on \(\overline{AB}\) and \(\overline{AC}\), respectively, so that \(\overline{PQ}\) is tangent to \(\omega\) at \(Y\). Assume that \(AP = 3\), \(PB = 4\), \(AC = 8\), and \(AQ = \dfrac{m}{n}\), where \(m\) and \(n\) are relatively prime positive integers. Find \(m+n\).\par \vspace{0.5em}\item Find the number of functions \(f\) from \(\{0, 1, 2, 3, 4, 5, 6\}\) to the integers such that \(f(0) = 0\), \(f(6) = 12\), and


\begin{equation*}
|x - y|  \leq  |f(x) - f(y)|  \leq  3|x - y|
\end{equation*}


for all \(x\) and \(y\) in \(\{0, 1, 2, 3, 4, 5, 6\}\).



\{\{AIME box|year=2018|n=II|before=|after=\}\}
\{\{MAA Notice\}\}\par \vspace{0.5em}
\end{enumerate}

\end{document}
