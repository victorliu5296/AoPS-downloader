
\documentclass{article}
\usepackage{amsmath, amssymb}
\usepackage{geometry}
\geometry{a4paper, margin=0.75in}
\usepackage{enumitem}
\usepackage[hypertexnames=true, linktoc=all]{hyperref}
\usepackage{fancyhdr}
\usepackage{tikz}
\usepackage{graphicx}
\usepackage{asymptote}
\usepackage{arcs}
\usepackage{xwatermark}
\begin{asydef}
  // Global Asymptote settings
  settings.outformat = "pdf";
  settings.render = 0;
  settings.prc = false;
  import olympiad;
  import cse5;
  size(8cm);
\end{asydef}
\pagestyle{fancy}
\fancyhead[L]{\textbf{AIME Problems}}
\fancyhead[R]{\textbf{2010}}
\fancyfoot[C]{\thepage}
\renewcommand{\headrulewidth}{0.4pt}
\renewcommand{\footrulewidth}{0.4pt}

\title{AIME Problems \\ 2010}
\date{}
\begin{document}\maketitle\thispagestyle{fancy}\newpage\section*{2010 AIME II}\begin{enumerate}[label=\arabic*., itemsep=0.5em]\item Let \(N\) be the greatest integer multiple of \(36\) all of whose digits are even and no two of whose digits are the same. Find the remainder when \(N\) is divided by \(1000\).\par \vspace{0.5em}\item A point \(P\) is chosen at random in the interior of a unit square \(S\). Let \(d(P)\) denote the distance from \(P\) to the closest side of \(S\). The probability that \(\frac{1}{5}\le d(P)\le\frac{1}{3}\) is equal to \(\frac{m}{n}\), where \(m\) and \(n\) are relatively prime positive integers. Find \(m+n\).\par \vspace{0.5em}\item Let \(K\) be the product of all factors \((b-a)\) (not necessarily distinct) where \(a\) and \(b\) are integers satisfying \(1\le a < b \le 20\). Find the greatest positive integer \(n\) such that \(2^n\) divides \(K\).\par \vspace{0.5em}\item Dave arrives at an airport which has twelve gates arranged in a straight line with exactly \(100\) feet between adjacent gates. His departure gate is assigned at random. After waiting at that gate, Dave is told the departure gate has been changed to a different gate, again at random. Let the probability that Dave walks \(400\) feet or less to the new gate be a fraction \(\frac{m}{n}\), where \(m\) and \(n\) are relatively prime positive integers. Find \(m+n\).\par \vspace{0.5em}\item Positive numbers \(x\), \(y\), and \(z\) satisfy \(xyz = 10^{81}\) and \((\log_{10}x)(\log_{10} yz) + (\log_{10}y) (\log_{10}z) = 468\). Find \(\sqrt {(\log_{10}x)^2 + (\log_{10}y)^2 + (\log_{10}z)^2}\).\par \vspace{0.5em}\item Find the smallest positive integer \(n\) with the property that the polynomial \(x^4 - nx + 63\) can be written as a product of two nonconstant polynomials with integer coefficients.\par \vspace{0.5em}\item Let \(P(z)=z^3+az^2+bz+c\), where \(a\), \(b\), and \(c\) are real. There exists a complex number \(w\) such that the three roots of \(P(z)\) are \(w+3i\), \(w+9i\), and \(2w-4\), where \(i^2=-1\). Find \(|a+b+c|\).\par \vspace{0.5em}\item Let \(N\) be the number of ordered pairs of nonempty sets \(\mathcal{A}\) and \(\mathcal{B}\) that have the following properties:

<UL>
<LI> \(\mathcal{A} \cup \mathcal{B} = \{1,2,3,4,5,6,7,8,9,10,11,12\}\),</LI>
<LI> \(\mathcal{A} \cap \mathcal{B} = \emptyset\), </LI>
<LI> The number of elements of \(\mathcal{A}\) is not an element of \(\mathcal{A}\),</LI>
<LI> The number of elements of \(\mathcal{B}\) is not an element of \(\mathcal{B}\).
</UL>

Find \(N\).\par \vspace{0.5em}\item Let \(ABCDEF\) be a regular hexagon. Let \(G\), \(H\), \(I\), \(J\), \(K\), and \(L\) be the midpoints of sides \(AB\), \(BC\), \(CD\), \(DE\), \(EF\), and \(AF\), respectively. The segments \(\overline{AH}\), \(\overline{BI}\), \(\overline{CJ}\), \(\overline{DK}\), \(\overline{EL}\), and \(\overline{FG}\) bound a smaller regular hexagon. Let the ratio of the area of the smaller hexagon to the area of \(ABCDEF\) be expressed as a fraction \(\frac {m}{n}\) where \(m\) and \(n\) are relatively prime positive integers. Find \(m + n\).\par \vspace{0.5em}\item Find the number of second-degree polynomials \(f(x)\) with integer coefficients and integer zeros for which \(f(0)=2010\).\par \vspace{0.5em}\item Define a <i>T-grid</i> to be a \(3\times3\) matrix which satisfies the following two properties:

<OL>
<LI>Exactly five of the entries are \(1\)'s, and the remaining four entries are \(0\)'s.</LI>
<LI>Among the eight rows, columns, and long diagonals (the long diagonals are \(\{a_{13},a_{22},a_{31}\}\) and \(\{a_{11},a_{22},a_{33}\})\), no more than one of the eight has all three entries equal.</LI></OL>

Find the number of distinct <i>T-grids</i>.\par \vspace{0.5em}\item Two noncongruent integer-sided isosceles triangles have the same perimeter and the same area. The ratio of the lengths of the bases of the two triangles is \(8: 7\). Find the minimum possible value of their common perimeter.\par \vspace{0.5em}\item The \(52\) cards in a deck are numbered \(1, 2, \cdots, 52\). Alex, Blair, Corey, and Dylan each pick a card from the deck randomly and without replacement. The two people with lower numbered cards form a team, and the two people with higher numbered cards form another team. Let \(p(a)\) be the probability that Alex and Dylan are on the same team, given that Alex picks one of the cards \(a\) and \(a+9\), and Dylan picks the other of these two cards. The minimum value of \(p(a)\) for which \(p(a)\ge\frac{1}{2}\) can be written as \(\frac{m}{n}\), where \(m\) and \(n\) are relatively prime positive integers. Find \(m+n\).\par \vspace{0.5em}\item Triangle \(ABC\) with right angle at \(C\), \(\angle BAC < 45^\circ\) and \(AB = 4\). Point \(P\) on \(\overline{AB}\) is chosen such that \(\angle APC = 2\angle ACP\) and \(CP = 1\). The ratio \(\frac{AP}{BP}\) can be represented in the form \(p + q\sqrt{r}\), where \(p\), \(q\), \(r\) are positive integers and \(r\) is not divisible by the square of any prime. Find \(p+q+r\).\par \vspace{0.5em}\item In triangle \(ABC\), \(AC=13\), \(BC=14\), and \(AB=15\). Points \(M\) and \(D\) lie on \(AC\) with \(AM=MC\) and \(\angle ABD = \angle DBC\). Points \(N\) and \(E\) lie on \(AB\) with \(AN=NB\) and \(\angle ACE = \angle ECB\). Let \(P\) be the point, other than \(A\), of intersection of the circumcircles of \(\triangle AMN\) and \(\triangle ADE\). Ray \(AP\) meets \(BC\) at \(Q\). The ratio \(\frac{BQ}{CQ}\) can be written in the form \(\frac{m}{n}\), where \(m\) and \(n\) are relatively prime positive integers. Find \(m-n\).\par \vspace{0.5em}\end{enumerate}
\end{document}
