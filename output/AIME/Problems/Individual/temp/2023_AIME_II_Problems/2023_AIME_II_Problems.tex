
\documentclass{article}
\usepackage{amsmath, amssymb}
\usepackage{geometry}
\geometry{a4paper, margin=0.75in}
\usepackage{enumitem}
\usepackage[hypertexnames=true, linktoc=all]{hyperref}
\usepackage{fancyhdr}
\usepackage{tikz}
\usepackage{graphicx}
\usepackage{asymptote}
\usepackage{arcs}
\usepackage{xwatermark}
\begin{asydef}
  // Global Asymptote settings
  settings.outformat = "pdf";
  settings.render = 0;
  settings.prc = false;
  import olympiad;
  import cse5;
  size(8cm);
\end{asydef}
\pagestyle{fancy}
\fancyhead[L]{\textbf{AIME Problems}}
\fancyhead[R]{\textbf{2023}}
\fancyfoot[C]{\thepage}
\renewcommand{\headrulewidth}{0.4pt}
\renewcommand{\footrulewidth}{0.4pt}

\title{AIME Problems \\ 2023}
\date{}
\begin{document}\maketitle\thispagestyle{fancy}\newpage\section*{2023 AIME II}
\begin{enumerate}[label=\arabic*., itemsep=0.5em]
\item The numbers of apples growing on each of six apple trees form an arithmetic sequence where the greatest number of apples growing on any of the six trees is double the least number of apples growing on any of the six trees. The total number of apples growing on all six trees is \(990.\) Find the greatest number of apples growing on any of the six trees.\par \vspace{0.5em}\item Recall that a palindrome is a number that reads the same forward and backward. Find the greatest integer less than \(1000\) that is a palindrome both when written in base ten and when written in base eight, such as \(292 = 444_{\text{eight}}.\)\par \vspace{0.5em}\item Let \(\triangle ABC\) be an isosceles triangle with \(\angle A = 90^\circ.\) There exists a point \(P\) inside \(\triangle ABC\) such that \(\angle PAB = \angle PBC = \angle PCA\) and \(AP = 10.\) Find the area of \(\triangle ABC.\)\par \vspace{0.5em}\item Let \(x,y,\) and \(z\) be real numbers satisfying the system of equations

\begin{align*}
xy + 4z &= 60 \\
yz + 4x &= 60 \\
zx + 4y &= 60.
\end{align*}

Let \(S\) be the set of possible values of \(x.\) Find the sum of the squares of the elements of \(S.\)\par \vspace{0.5em}\item Let \(S\) be the set of all positive rational numbers \(r\) such that when the two numbers \(r\) and \(55r\) are written as fractions in lowest terms, the sum of the numerator and denominator of one fraction is the same as the sum of the numerator and denominator of the other fraction. The sum of all the elements of \(S\) can be expressed in the form \(\frac{p}{q},\) where \(p\) and \(q\) are relatively prime positive integers. Find \(p+q.\)\par \vspace{0.5em}\item Consider the L-shaped region formed by three unit squares joined at their sides, as shown below. Two points \(A\) and \(B\) are chosen independently and uniformly at random from inside the region. The probability that the midpoint of \(\overline{AB}\) also lies inside this L-shaped region can be expressed as \(\frac{m}{n},\) where \(m\) and \(n\) are relatively prime positive integers. Find \(m+n.\)

\begin{center}
\begin{asy}
import olympiad;
import cse5;
unitsize(2cm);
draw((0,0)--(2,0)--(2,1)--(1,1)--(1,2)--(0,2)--cycle);
draw((0,1)--(1,1)--(1,0),dashed);
\end{asy}
\end{center}
\par \vspace{0.5em}\item Each vertex of a regular dodecagon (\(12\)-gon) is to be colored either red or blue, and thus there are \(2^{12}\) possible colorings. Find the number of these colorings with the property that no four vertices colored the same color are the four vertices of a rectangle.\par \vspace{0.5em}\item Let \(\omega = \cos\frac{2\pi}{7} + i \cdot \sin\frac{2\pi}{7},\) where \(i = \sqrt{-1}.\) Find the value of the product 
\begin{equation*}
\prod_{k=0}^6 \left(\omega^{3k} + \omega^k + 1\right).
\end{equation*}
\par \vspace{0.5em}\item Circles \(\omega_1\) and \(\omega_2\) intersect at two points \(P\) and \(Q,\) and their common tangent line closer to \(P\) intersects \(\omega_1\) and \(\omega_2\) at points \(A\) and \(B,\) respectively. The line parallel to \(AB\) that passes through \(P\) intersects \(\omega_1\) and \(\omega_2\) for the second time at points \(X\) and \(Y,\) respectively. Suppose \(PX=10,\) \(PY=14,\) and \(PQ=5.\) Then the area of trapezoid \(XABY\) is \(m\sqrt{n},\) where \(m\) and \(n\) are positive integers and \(n\) is not divisible by the square of any prime. Find \(m+n.\)\par \vspace{0.5em}\item Let \(N\) be the number of ways to place the integers \(1\) through \(12\) in the \(12\) cells of a \(2 \times 6\) grid so that for any two cells sharing a side, the difference between the numbers in those cells is not divisible by \(3.\) One way to do this is shown below. Find the number of positive integer divisors of \(N.\)

\begin{equation*}
\begin{array}{|c|c|c|c|c|c|} \hline
\,1\, & \,3\, & \,5\, & \,7\, & \,9\, & 11 \\ \hline
\,2\, & \,4\, & \,6\, & \,8\, & 10 & 12 \\ \hline
\end{array}
\end{equation*}
\par \vspace{0.5em}\item Find the number of collections of \(16\) distinct subsets of \(\{1,2,3,4,5\}\) with the property that for any two subsets \(X\) and \(Y\) in the collection, \(X \cap Y \not= \emptyset.\)\par \vspace{0.5em}\item In \(\triangle ABC\) with side lengths \(AB = 13,\) \(BC = 14,\) and \(CA = 15,\) let \(M\) be the midpoint of \(\overline{BC}.\) Let \(P\) be the point on the circumcircle of \(\triangle ABC\) such that \(M\) is on \(\overline{AP}.\) There exists a unique point \(Q\) on segment \(\overline{AM}\) such that \(\angle PBQ = \angle PCQ.\) Then \(AQ\) can be written as \(\frac{m}{\sqrt{n}},\) where \(m\) and \(n\) are relatively prime positive integers. Find \(m+n.\)\par \vspace{0.5em}\item Let \(A\) be an acute angle such that \(\tan A = 2 \cos A.\) Find the number of positive integers \(n\) less than or equal to \(1000\) such that \(\sec^n A + \tan^n A\) is a positive integer whose units digit is \(9.\)\par \vspace{0.5em}\item A cube-shaped container has vertices \(A,\) \(B,\) \(C,\) and \(D,\) where \(\overline{AB}\) and \(\overline{CD}\) are parallel edges of the cube, and \(\overline{AC}\) and \(\overline{BD}\) are diagonals of faces of the cube, as shown. Vertex \(A\) of the cube is set on a horizontal plane \(\mathcal{P}\) so that the plane of the rectangle \(ABDC\) is perpendicular to \(\mathcal{P},\) vertex \(B\) is \(2\) meters above \(\mathcal{P},\) vertex \(C\) is \(8\) meters above \(\mathcal{P},\) and vertex \(D\) is \(10\) meters above \(\mathcal{P}.\) The cube contains water whose surface is parallel to \(\mathcal{P}\) at a height of \(7\) meters above \(\mathcal{P}.\) The volume of water is \(\frac{m}{n}\) cubic meters, where \(m\) and \(n\) are relatively prime positive integers. Find \(m+n.\)

\begin{center}
\begin{asy}
import olympiad;
import cse5;
//Made by Djmathman (orz)
size(250);
defaultpen(linewidth(0.6));
pair A = origin, B = (6,3), X = rotate(40)*B, Y = rotate(70)*X, C = X+Y, Z = X+B, D = B+C, W = B+Y;
pair P1 = 0.8*C+0.2*Y, P2 = 2/3*C+1/3*X, P3 = 0.2*D+0.8*Z, P4 = 0.63*D+0.37*W;
pair E = (-20,6), F = (-6,-5), G = (18,-2), H = (9,8);
filldraw(E--F--G--H--cycle,rgb(0.98,0.98,0.2));
fill(A--Y--P1--P4--P3--Z--B--cycle,rgb(0.35,0.7,0.9));
draw(A--B--Z--X--A--Y--C--X^^C--D--Z);
draw(P1--P2--P3--P4--cycle^^D--P4);
dot("$A$",A,S);
dot("$B$",B,S);
dot("$C$",C,N);
dot("$D$",D,N);
label("$\mathcal P$",(-13,4.5));
\end{asy}
\end{center}
\par \vspace{0.5em}\item For each positive integer \(n\) let \(a_n\) be the least positive integer multiple of \(23\) such that \(a_n \equiv 1 \pmod{2^n}.\) Find the number of positive integers \(n\) less than or equal to \(1000\) that satisfy \(a_n = a_{n+1}.\)\par \vspace{0.5em}
\end{enumerate}

\end{document}
