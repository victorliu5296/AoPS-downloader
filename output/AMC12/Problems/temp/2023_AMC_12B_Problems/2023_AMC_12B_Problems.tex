
\documentclass{article}
\usepackage{amsmath, amssymb}
\usepackage{geometry}
\geometry{a4paper, margin=0.75in}
\usepackage{enumitem}
\usepackage{hyperref}
\usepackage{fancyhdr}
\usepackage{tikz}
\usepackage{graphicx}
\usepackage{asymptote}
\begin{asydef}
  // Global Asymptote settings
  settings.outformat = "pdf";
  settings.render = 0;
  settings.prc = false;
  import olympiad;
  import cse5;
  size(8cm);
\end{asydef}
\pagestyle{fancy}
\fancyhead[L]{\textbf{AMC12 Problems}}
\fancyhead[R]{\textbf{2023}}
\fancyfoot[C]{\thepage}
\renewcommand{\headrulewidth}{0.4pt}
\renewcommand{\footrulewidth}{0.4pt}

\title{AMC12 Problems \\ 2023}
\date{}
\begin{document}\maketitle\thispagestyle{fancy}\newpage\section*{Problems}\begin{enumerate}[label=\arabic*., itemsep=0.5em]\item Mrs. Jones is pouring orange juice into four identical glasses for her four sons. She fills the first three glasses completely but runs out of juice when the fourth glass is only $\frac{1}{3}$ full. What fraction of a glass must Mrs. Jones pour from each of the first three glasses into the fourth glass so that all four glasses will have the same amount of juice?

$\textbf{(A) }\frac{1}{12}\qquad\textbf{(B) }\frac{1}{4}\qquad\textbf{(C) }\frac{1}{6}\qquad\textbf{(D) }\frac{1}{8}\qquad\textbf{(E) }\frac{2}{9}$\par \vspace{0.5em}\item Carlos went to a sports store to buy running shoes. Running shoes were on sale, with prices reduced by $20\%$on every pair of shoes. Carlos also knew that he had to pay a $7.5\%$ sales tax on the discounted price. He had $43$ dollars. What is the original (before discount) price of the most expensive shoes he could afford to buy?

$\textbf{(A) }46\qquad\textbf{(B) }50\qquad\textbf{(C) }48\qquad\textbf{(D) }47\qquad\textbf{(E) }49$\par \vspace{0.5em}\item A $3-4-5$ right triangle is inscribed in circle $A$, and a $5-12-13$ right triangle is inscribed in circle $B$. What is the ratio of the area of circle $A$ to the area of circle $B$?

$\textbf{(A)}~\frac{9}{25}\qquad\textbf{(B)}~\frac{1}{9}\qquad\textbf{(C)}~\frac{1}{5}\qquad\textbf{(D)}~\frac{25}{169}\qquad\textbf{(E)}~\frac{4}{25}$\par \vspace{0.5em}\item Jackson's paintbrush makes a narrow strip with a width of $6.5$ millimeters. Jackson has enough paint to make a strip $25$ meters long. How many square centimeters of paper could Jackson cover with paint?

$\textbf{(A) }162,500\qquad\textbf{(B) }162.5\qquad\textbf{(C) }1,625\qquad\textbf{(D) }1,625,000\qquad\textbf{(E) }16,250$\par \vspace{0.5em}\item You are playing a game. A $2 \times 1$ rectangle covers two adjacent squares (oriented either horizontally or vertically) of a $3 \times 3$ grid of squares, but you are not told which two squares are covered. Your goal is to find at least one square that is covered by the rectangle. A "turn" consists of you guessing a square, after which you are told whether that square is covered by the hidden rectangle. What is the minimum number of turns you need to ensure that at least one of your guessed squares is covered by the rectangle?

$\textbf{(A)}~3\qquad\textbf{(B)}~5\qquad\textbf{(C)}~4\qquad\textbf{(D)}~8\qquad\textbf{(E)}~6$\par \vspace{0.5em}\item When the roots of the polynomial


\begin{equation*}
P(x)  = (x-1)^1 (x-2)^2 (x-3)^3 \cdot \cdot \cdot (x-10)^{10}
\end{equation*}


are removed from the number line, what remains is the union of 11 disjoint open intervals. On how many of these intervals is $P(x)$ positive?

$\textbf{(A)}~3\qquad\textbf{(B)}~7\qquad\textbf{(C)}~6\qquad\textbf{(D)}~4\qquad\textbf{(E)}~5$\par \vspace{0.5em}\item For how many integers $n$ does the expression
\begin{equation*}
\sqrt{\frac{\log (n^2) - (\log n)^2}{\log n - 3}}
\end{equation*}
represent a real number, where log denotes the base $10$ logarithm?

$\textbf{(A) }900 \qquad \textbf{(B) }3\qquad \textbf{(C) }902 \qquad \textbf{(D) } 2  \qquad \textbf{(E) }901$\par \vspace{0.5em}\item How many nonempty subsets $B$ of $\{0, 1, 2, 3, \dots, 12\}$ have the property that the number of elements in $B$ is equal to the least element of $B$? For example, $B = \{4, 6, 8, 11\}$ satisfies the condition.

$\textbf{(A)}\ 256 \qquad\textbf{(B)}\ 136 \qquad\textbf{(C)}\ 108 \qquad\textbf{(D)}\ 144 \qquad\textbf{(E)}\ 156$\par \vspace{0.5em}\item What is the area of the region in the coordinate plane defined by

$\left||x|-1\right|+\left||y|-1\right|\leq 1?$

$\textbf{(A)}~2\qquad\textbf{(B)}~8\qquad\textbf{(C)}~4\qquad\textbf{(D)}~15\qquad\textbf{(E)}~12$\par \vspace{0.5em}\item In the $xy$-plane, a circle of radius $4$ with center on the positive $x$-axis is tangent to the $y$-axis at the origin, and a circle with radius $10$ with center on the positive $y$-axis is tangent to the $x$-axis at the origin. What is the slope of the line passing through the two points at which these circles intersect?

$\textbf{(A)}\ \dfrac{2}{7} \qquad\textbf{(B)}\ \dfrac{3}{7}  \qquad\textbf{(C)}\ \dfrac{2}{\sqrt{29}}  \qquad\textbf{(D)}\ \dfrac{1}{\sqrt{29}}  \qquad\textbf{(E)}\ \dfrac{2}{5}$\par \vspace{0.5em}\item What is the maximum area of an isosceles trapezoid that has legs of length $1$ and one base twice as long as the other?

$\textbf{(A) }\frac 54 \qquad \textbf{(B) } \frac 87 \qquad \textbf{(C)} \frac{5\sqrt2}4 \qquad \textbf{(D) } \frac 32  \qquad \textbf{(E) } \frac{3\sqrt3}4$\par \vspace{0.5em}\item For complex numbers $u=a+bi$ and $v=c+di$, define the binary operation $\otimes$ by
\begin{equation*}
u\otimes v=ac+bdi.
\end{equation*}
Suppose $z$ is a complex number such that $z\otimes z=z^{2}+40$. What is $|z|$?

$\textbf{(A) }2\qquad\textbf{(B) }5\qquad\textbf{(C) }\sqrt{5}\qquad\textbf{(D) }\sqrt{10}\qquad\textbf{(E) }5\sqrt{2}$\par \vspace{0.5em}\item A rectangular box $P$ has distinct edge lengths $a$, $b$, and $c$. The sum of the lengths of all $12$ edges of $P$ is $13$, the sum of the areas of all $6$ faces of $P$ is $\frac{11}{2}$, and the volume of $P$ is $\frac{1}{2}$. What is the length of the longest interior diagonal connecting two vertices of $P$?

$\textbf{(A)}~2\qquad\textbf{(B)}~\frac{3}{8}\qquad\textbf{(C)}~\frac{9}{8}\qquad\textbf{(D)}~\frac{9}{4}\qquad\textbf{(E)}~\frac{3}{2}$\par \vspace{0.5em}\item For how many ordered pairs $(a,b)$ of integers does the polynomial $x^3+ax^2+bx+6$ have $3$ distinct integer roots?

$\textbf{(A)}\ 5 \qquad\textbf{(B)}\ 6 \qquad\textbf{(C)}\ 8 \qquad\textbf{(D)}\ 7 \qquad\textbf{(E)}\ 4$\par \vspace{0.5em}\item Suppose $a$, $b$, and $c$ are positive integers such that
\begin{equation*}
\frac{a}{14}+\frac{b}{15}=\frac{c}{210}.
\end{equation*}
Which of the following statements are necessarily true?

I. If $\gcd(a,14)=1$ or $\gcd(b,15)=1$ or both, then $\gcd(c,210)=1$.

II. If $\gcd(c,210)=1$, then $\gcd(a,14)=1$ or $\gcd(b,15)=1$ or both.

III. $\gcd(c,210)=1$ if and only if $\gcd(a,14)=\gcd(b,15)=1$.

$\textbf{(A)}~\text{I, II, and III}\qquad\textbf{(B)}~\text{I only}\qquad\textbf{(C)}~\text{I and II only}\qquad\textbf{(D)}~\text{III only}\qquad\textbf{(E)}~\text{II and III only}$\par \vspace{0.5em}\item In Coinland, there are three types of coins, each worth $6, 10,$ and $15.$ What is the sum of the digits of the maximum amount of money that is impossible to have?

$\textbf{(A) }8\qquad\textbf{(B) }10\qquad\textbf{(C) }7\qquad\textbf{(D) }11\qquad\textbf{(E) }9$\par \vspace{0.5em}\item Triangle $ABC$ has side lengths in arithmetic progression, and the smallest side has length $6$. If the triangle has an angle of $120^\circ,$ what is the area of $ABC$?

$\textbf{(A) }12\sqrt{3}\qquad\textbf{(B) }8\sqrt{6}\qquad\textbf{(C) }14\sqrt{2}\qquad\textbf{(D) }20\sqrt{2}\qquad\textbf{(E) }15\sqrt{3}$\par \vspace{0.5em}\item Last academic year Yolanda and Zelda took different courses that did not necessarily administer the same number of quizzes during each of the two semesters. Yolanda's average on all the quizzes she took during the first semester was $3$ points higher than Zelda's average on all the quizzes she took during the first semester. Yolanda's average on all the quizzes she took during the second semester was $18$ points higher than her average for the first semester and was again $3$ points higher than Zelda's average on all the quizzes Zelda took during her second semester. Which one of the following statements cannot possibly be true?

$\textbf{(A)}$ Yolanda's quiz average for the academic year was $22$ points higher than Zelda's.

$\textbf{(B)}$ Zelda's quiz average for the academic year was higher than Yolanda's.

$\textbf{(C)}$ Yolanda's quiz average for the academic year was $3$ points higher than Zelda's.

$\textbf{(D)}$ Zelda's quiz average for the academic year equaled Yolanda's.

$\textbf{(E)}$ If Zelda had scored $3$ points higher on each quiz she took, then she would have had the same average for the academic year as Yolanda.\par \vspace{0.5em}\item Each of $2023$ balls is placed in one of $3$ bins. Which of the following is closest to the probability that each of the bins will contain an odd number of balls?

$\textbf{(A) } \frac{2}{3} \qquad \textbf{(B) } \frac{3}{10} \qquad \textbf{(C) } \frac{1}{2} \qquad \textbf{(D) } \frac{1}{3} \qquad \textbf{(E) } \frac{1}{4}$\par \vspace{0.5em}\item Cyrus the frog jumps $2$ units in a direction, then $2$ more in another direction. What is the probability that he lands less than $1$ unit away from his starting position?

$\textbf{(A)}~\frac{1}{6}\qquad\textbf{(B)}~\frac{1}{5}\qquad\textbf{(C)}~\frac{\sqrt{3}}{8}\qquad\textbf{(D)}~\frac{\arctan \frac{1}{2}}{\pi}\qquad\textbf{(E)}~\frac{2\arcsin \frac{1}{4}}{\pi}$\par \vspace{0.5em}\item A lampshade is made in the form of the lateral surface of the frustum of a right circular cone. The height of the frustum is $3\sqrt3$ inches, its top diameter is $6$ inches, and its bottom diameter is $12$ inches. A bug is at the bottom of the lampshade and there is a glob of honey on the top edge of the lampshade at the spot farthest from the bug. The bug wants to crawl to the honey, but it must stay on the surface of the lampshade. What is the length in inches of its shortest path to the honey?

$\textbf{(A) } 6 + 3\pi\qquad \textbf{(B) }6 + 6\pi\qquad \textbf{(C) } 6\sqrt3 \qquad \textbf{(D) } 6\sqrt5 \qquad \textbf{(E) } 6\sqrt3 + \pi$\par \vspace{0.5em}\item A real-valued function $f$ has the property that for all real numbers $a$ and $b,$
\begin{equation*}
f(a + b)  + f(a - b) = 2f(a) f(b).
\end{equation*}
Which one of the following cannot be the value of $f(1)?$

$\textbf{(A) } 0 \qquad \textbf{(B) } 1 \qquad \textbf{(C) } -1 \qquad \textbf{(D) } 2 \qquad \textbf{(E) } -2$\par \vspace{0.5em}\item When $n$ standard six-sided dice are rolled, the product of the numbers rolled can be any of $936$ possible values. What is $n$?

$\textbf{(A)}~11\qquad\textbf{(B)}~6\qquad\textbf{(C)}~8\qquad\textbf{(D)}~10\qquad\textbf{(E)}~9$\par \vspace{0.5em}\item Suppose that $a$, $b$, $c$ and $d$ are positive integers satisfying all of the following relations.


\begin{equation*}
abcd=2^6\cdot 3^9\cdot 5^7
\end{equation*}


\begin{equation*}
\text{lcm}(a,b)=2^3\cdot 3^2\cdot 5^3
\end{equation*}


\begin{equation*}
\text{lcm}(a,c)=2^3\cdot 3^3\cdot 5^3
\end{equation*}


\begin{equation*}
\text{lcm}(a,d)=2^3\cdot 3^3\cdot 5^3
\end{equation*}


\begin{equation*}
\text{lcm}(b,c)=2^1\cdot 3^3\cdot 5^2
\end{equation*}


\begin{equation*}
\text{lcm}(b,d)=2^2\cdot 3^3\cdot 5^2
\end{equation*}


\begin{equation*}
\text{lcm}(c,d)=2^2\cdot 3^3\cdot 5^2
\end{equation*}


What is $\text{gcd}(a,b,c,d)$?

$\textbf{(A)}~30\qquad\textbf{(B)}~45\qquad\textbf{(C)}~3\qquad\textbf{(D)}~15\qquad\textbf{(E)}~6$\par \vspace{0.5em}\item A regular pentagon with area $\sqrt{5}+1$ is printed on paper and cut out. The five vertices of the pentagon are folded into the center of the pentagon, creating a smaller pentagon. What is the area of the new pentagon?

$\textbf{(A)}~4-\sqrt{5}\qquad\textbf{(B)}~\sqrt{5}-1\qquad\textbf{(C)}~8-3\sqrt{5}\qquad\textbf{(D)}~\frac{\sqrt{5}+1}{2}\qquad\textbf{(E)}~\frac{2+\sqrt{5}}{3}$\par \vspace{0.5em}\end{enumerate}
\end{document}
