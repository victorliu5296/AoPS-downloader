
\documentclass{article}
\usepackage{amsmath, amssymb}
\usepackage{geometry}
\geometry{a4paper, margin=0.75in}
\usepackage{enumitem}
\usepackage{hyperref}
\usepackage{fancyhdr}
\usepackage{tikz}
\usepackage{graphicx}
\usepackage{asymptote}
\usepackage{arcs}
\usepackage{xwatermark}
\begin{asydef}
  // Global Asymptote settings
  settings.outformat = "pdf";
  settings.render = 0;
  settings.prc = false;
  import olympiad;
  import cse5;
  size(8cm);
\end{asydef}
\pagestyle{fancy}
\fancyhead[L]{\textbf{AMC12 Problems}}
\fancyhead[R]{\textbf{2024}}
\fancyfoot[C]{\thepage}
\renewcommand{\headrulewidth}{0.4pt}
\renewcommand{\footrulewidth}{0.4pt}

\title{AMC12 Problems \\ 2024}
\date{}
\begin{document}\maketitle\thispagestyle{fancy}\newpage\section*{2020 AMC 12B}\begin{enumerate}[label=\arabic*., itemsep=0.5em]\item What is the value in simplest form of the following expression?
\begin{equation*}
\sqrt{1} + \sqrt{1+3} + \sqrt{1+3+5} + \sqrt{1+3+5+7}
\end{equation*}


\(\textbf{(A) }5 \qquad \textbf{(B) }4 + \sqrt{7} + \sqrt{10} \qquad \textbf{(C) } 10 \qquad \textbf{(D) } 15 \qquad \textbf{(E) } 4 + 3\sqrt{3} + 2\sqrt{5} + \sqrt{7}\)\par \vspace{0.5em}\item What is the value of the following expression?

\begin{equation*}
\frac{100^2-7^2}{70^2-11^2} \cdot \frac{(70-11)(70+11)}{(100-7)(100+7)}
\end{equation*}
\(\textbf{(A) } 1 \qquad \textbf{(B) } \frac{9951}{9950} \qquad \textbf{(C) } \frac{4780}{4779} \qquad \textbf{(D) } \frac{108}{107} \qquad \textbf{(E) } \frac{81}{80} \)\par \vspace{0.5em}\item The ratio of \(w\) to \(x\) is \(4 : 3\), the ratio of \(y\) to \(z\) is \(3 : 2\), and the ratio of \(z\) to \(x\) is \(1 : 6\). What is the ratio of \(w\) to \(y\)?

\(\textbf{(A) }4:3 \qquad \textbf{(B) }3:2 \qquad \textbf{(C) } 8:3 \qquad \textbf{(D) } 4:1 \qquad \textbf{(E) } 16:3 \)\par \vspace{0.5em}\item The acute angles of a right triangle are \(a^{\circ}\) and \(b^{\circ}\), where \(a>b\) and both \(a\) and \(b\) are prime numbers. What is the least possible value of \(b\)?

\(\textbf{(A) }2\qquad\textbf{(B) }3\qquad\textbf{(C) }5\qquad\textbf{(D) }7\qquad\textbf{(E) }11\)\par \vspace{0.5em}\item Teams \(A\) and \(B\) are playing in a basketball league where each game results in a win for one team and a loss for the other team. Team \(A\) has won \(\tfrac{2}{3}\) of its games and team \(B\) has won \(\tfrac{5}{8}\) of its games. Also, team \(B\) has won \(7\) more games and lost \(7\) more games than team \(A.\) How many games has team \(A\) played?

\(\textbf{(A) } 21 \qquad \textbf{(B) } 27 \qquad \textbf{(C) } 42 \qquad \textbf{(D) } 48 \qquad \textbf{(E) } 63\)\par \vspace{0.5em}\item For all integers \(n \geq 9,\) the value of

\begin{equation*}
\frac{(n+2)!-(n+1)!}{n!}
\end{equation*}
is always which of the following?

\(\textbf{(A) } \text{a multiple of 4} \qquad \textbf{(B) } \text{a multiple of 10} \qquad \textbf{(C) } \text{a prime number} \qquad \textbf{(D) } \text{a perfect square} \qquad \textbf{(E) } \text{a perfect cube}\)\par \vspace{0.5em}\item Two nonhorizontal, non vertical lines in the \(xy\)-coordinate plane intersect to form a \(45^{\circ}\) angle. One line has slope equal to \(6\) times the slope of the other line. What is the greatest possible value of the product of the slopes of the two lines?

\(\textbf{(A)}\ \frac16 \qquad\textbf{(B)}\ \frac23 \qquad\textbf{(C)}\  \frac32 \qquad\textbf{(D)}\ 3 \qquad\textbf{(E)}\ 6\)\par \vspace{0.5em}\item How many ordered pairs of integers \((x, y)\) satisfy the equation
\begin{equation*}
x^{2020}+y^2=2y?
\end{equation*}

\(\textbf{(A) } 1 \qquad\textbf{(B) } 2 \qquad\textbf{(C) } 3 \qquad\textbf{(D) } 4 \qquad\textbf{(E) } \text{infinitely many}\)\par \vspace{0.5em}\item A three-quarter sector of a circle of radius \(4\) inches together with its interior can be rolled up to form the lateral surface of a right circular cone by taping together along the two radii shown. What is the volume of the cone in cubic inches?

\begin{center}
\begin{asy}
import olympiad;
import cse5;
draw(Arc((0,0), 4, 0, 270));
draw((0,-4)--(0,0)--(4,0));

label("$4$", (2,0), S);
\end{asy}
\end{center}


\(\textbf{(A)}\ 3\pi \sqrt5 \qquad\textbf{(B)}\ 4\pi \sqrt3 \qquad\textbf{(C)}\ 3 \pi \sqrt7 \qquad\textbf{(D)}\ 6\pi \sqrt3 \qquad\textbf{(E)}\ 6\pi \sqrt7\)\par \vspace{0.5em}\item In unit square \(ABCD,\) the inscribed circle \(\omega\) intersects \(\overline{CD}\) at \(M,\) and \(\overline{AM}\) intersects \(\omega\) at a point \(P\) different from \(M.\) What is \(AP?\)

\(\textbf{(A) } \frac{\sqrt5}{12} \qquad \textbf{(B) } \frac{\sqrt5}{10} \qquad \textbf{(C) } \frac{\sqrt5}{9} \qquad \textbf{(D) } \frac{\sqrt5}{8} \qquad \textbf{(E) } \frac{2\sqrt5}{15}\)\par \vspace{0.5em}\item As shown in the figure below, six semicircles lie in the interior of a regular hexagon with side length \(2\) so that the diameters of the semicircles coincide with the sides of the hexagon. What is the area of the shaded regioninside the hexagon but outside all of the semicircles?


\begin{center}
\begin{asy}
import olympiad;
import cse5;
size(140);
fill((1,0)--(3,0)--(4,sqrt(3))--(3,2sqrt(3))--(1,2sqrt(3))--(0,sqrt(3))--cycle,gray(0.4));
fill(arc((2,0),1,180,0)--(2,0)--cycle,white);
fill(arc((3.5,sqrt(3)/2),1,60,240)--(3.5,sqrt(3)/2)--cycle,white);
fill(arc((3.5,3sqrt(3)/2),1,120,300)--(3.5,3sqrt(3)/2)--cycle,white);
fill(arc((2,2sqrt(3)),1,180,360)--(2,2sqrt(3))--cycle,white);
fill(arc((0.5,3sqrt(3)/2),1,240,420)--(0.5,3sqrt(3)/2)--cycle,white);
fill(arc((0.5,sqrt(3)/2),1,300,480)--(0.5,sqrt(3)/2)--cycle,white);
draw((1,0)--(3,0)--(4,sqrt(3))--(3,2sqrt(3))--(1,2sqrt(3))--(0,sqrt(3))--(1,0));
draw(arc((2,0),1,180,0)--(2,0)--cycle);
draw(arc((3.5,sqrt(3)/2),1,60,240)--(3.5,sqrt(3)/2)--cycle);
draw(arc((3.5,3sqrt(3)/2),1,120,300)--(3.5,3sqrt(3)/2)--cycle);
draw(arc((2,2sqrt(3)),1,180,360)--(2,2sqrt(3))--cycle);
draw(arc((0.5,3sqrt(3)/2),1,240,420)--(0.5,3sqrt(3)/2)--cycle);
draw(arc((0.5,sqrt(3)/2),1,300,480)--(0.5,sqrt(3)/2)--cycle);
label("$2$",(3.5,3sqrt(3)/2),NE);
\end{asy}
\end{center}


\( \textbf {(A) } 6\sqrt{3}-3\pi \qquad \textbf {(B) } \frac{9\sqrt{3}}{2} - 2\pi\ \qquad \textbf {(C) } \frac{3\sqrt{3}}{2} - \frac{\pi}{3} \qquad \textbf {(D) } 3\sqrt{3} - \pi \qquad \textbf {(E) } \frac{9\sqrt{3}}{2} - \pi \)\par \vspace{0.5em}\item Let \(\overline{AB}\) be a diameter in a circle of radius \(5\sqrt2.\) Let \(\overline{CD}\) be a chord in the circle that intersects \(\overline{AB}\) at a point \(E\) such that \(BE=2\sqrt5\) and \(\angle AEC = 45^{\circ}.\) What is \(CE^2+DE^2?\)

\(\textbf{(A)}\ 96 \qquad\textbf{(B)}\ 98 \qquad\textbf{(C)}\  44\sqrt5 \qquad\textbf{(D)}\ 70\sqrt2 \qquad\textbf{(E)}\ 100\)\par \vspace{0.5em}\item Which of the following is the value of \(\sqrt{\log_2{6}+\log_3{6}}?\)

\(\textbf{(A) } 1 \qquad\textbf{(B) } \sqrt{\log_5{6}} \qquad\textbf{(C) } 2 \qquad\textbf{(D) } \sqrt{\log_2{3}}+\sqrt{\log_3{2}} \qquad\textbf{(E) } \sqrt{\log_2{6}}+\sqrt{\log_3{6}}\)\par \vspace{0.5em}\item Bela and Jenn play the following game on the closed interval \([0, n]\) of the real number line, where \(n\) is a fixed integer greater than \(4\). They take turns playing, with Bela going first. At his first turn, Bela chooses any real number in the interval \([0, n]\). Thereafter, the player whose turn it is chooses a real number that is more than one unit away from all numbers previously chosen by either player. A player unable to choose such a number loses. Using optimal strategy, which player will win the game?

\(\textbf{(A)} \text{ Bela will always win.} \qquad \textbf{(B)} \text{ Jenn will always win.} \qquad \textbf{(C)} \text{ Bela will win if and only if }n \text{ is odd.}\)
\(\textbf{(D)} \text{ Jenn will win if and only if }n \text{ is odd.} \qquad \textbf{(E)} \text { Jenn will win if and only if } n>8.\)\par \vspace{0.5em}\item There are 10 people standing equally spaced around a circle. Each person knows exactly 3 of the other 9 people: the 2 people standing next to her or him, as well as the person directly across the circle. How many ways are there for the 10 people to split up into 5 pairs so that the members of each pair know each other?

\(\textbf{(A) } 11 \qquad \textbf{(B) } 12 \qquad \textbf{(C) } 13 \qquad \textbf{(D) } 14 \qquad \textbf{(E) } 15\)\par \vspace{0.5em}\item An urn contains one red ball and one blue ball. A box of extra red and blue balls lie nearby. George performs the following operation four times: he draws a ball from the urn at random and then takes a ball of the same color from the box and returns those two matching balls to the urn. After the four iterations the urn contains six balls. What is the probability that the urn contains three balls of each color?

\(\textbf{(A) } \frac16 \qquad \textbf{(B) }\frac15 \qquad \textbf{(C) } \frac14 \qquad \textbf{(D) } \frac13 \qquad \textbf{(E) } \frac12\)\par \vspace{0.5em}\item How many polynomials of the form \(x^5 + ax^4 + bx^3 + cx^2 + dx + 2020\), where \(a\), \(b\), \(c\), and \(d\) are real numbers, have the property that whenever \(r\) is a root, so is \(\frac{-1+i\sqrt{3}}{2} \cdot r\)? (Note that \(i=\sqrt{-1}\))

\(\textbf{(A) } 0 \qquad \textbf{(B) }1 \qquad \textbf{(C) } 2 \qquad \textbf{(D) } 3 \qquad \textbf{(E) } 4\)\par \vspace{0.5em}\item In square \(ABCD\), points \(E\) and \(H\) lie on \(\overline{AB}\) and \(\overline{DA}\), respectively, so that \(AE=AH.\) Points \(F\) and \(G\) lie on \(\overline{BC}\) and \(\overline{CD}\), respectively, and points \(I\) and \(J\) lie on \(\overline{EH}\) so that \(\overline{FI} \perp \overline{EH}\) and \(\overline{GJ} \perp \overline{EH}\). See the figure below. Triangle \(AEH\), quadrilateral \(BFIE\), quadrilateral \(DHJG\), and pentagon \(FCGJI\) each has area \(1.\) What is \(FI^2\)?

\begin{center}
\begin{asy}
import olympiad;
import cse5;
real x=2sqrt(2);
real y=2sqrt(16-8sqrt(2))-4+2sqrt(2);
real z=2sqrt(8-4sqrt(2));
pair A, B, C, D, E, F, G, H, I, J;
A = (0,0);
B = (4,0);
C = (4,4);
D = (0,4);
E = (x,0);
F = (4,y);
G = (y,4);
H = (0,x);
I = F + z * dir(225);
J = G + z * dir(225);

draw(A--B--C--D--A);
draw(H--E);
draw(J--G^^F--I);
draw(rightanglemark(G, J, I), linewidth(.5));
draw(rightanglemark(F, I, E), linewidth(.5));

dot("$A$", A, S);
dot("$B$", B, S);
dot("$C$", C, dir(90));
dot("$D$", D, dir(90));
dot("$E$", E, S);
dot("$F$", F, dir(0));
dot("$G$", G, N);
dot("$H$", H, W);
dot("$I$", I, SW);
dot("$J$", J, SW);
\end{asy}
\end{center}


\(\textbf{(A) } \frac{7}{3} \qquad \textbf{(B) } 8-4\sqrt2 \qquad \textbf{(C) } 1+\sqrt2 \qquad \textbf{(D) } \frac{7}{4}\sqrt2 \qquad \textbf{(E) } 2\sqrt2\)\par \vspace{0.5em}\item Square \(ABCD\) in the coordinate plane has vertices at the points \(A(1,1), B(-1,1), C(-1,-1),\) and \(D(1,-1).\) Consider the following four transformations:

\(\quad\bullet\qquad\) \(L,\) a rotation of \(90^{\circ}\) counterclockwise around the origin;

\(\quad\bullet\qquad\) \(R,\) a rotation of \(90^{\circ}\) clockwise around the origin;

\(\quad\bullet\qquad\) \(H,\) a reflection across the \(x\)-axis; and

\(\quad\bullet\qquad\) \(V,\) a reflection across the \(y\)-axis.

Each of these transformations maps the squares onto itself, but the positions of the labeled vertices will change. For example, applying \(R\) and then \(V\) would send the vertex \(A\) at \((1,1)\) to \((-1,-1)\) and would send the vertex \(B\) at \((-1,1)\) to itself. How many sequences of \(20\) transformations chosen from \(\{L, R, H, V\}\) will send all of the labeled vertices back to their original positions? (For example, \(R, R, V, H\) is one sequence of \(4\) transformations that will send the vertices back to their original positions.)

\(\textbf{(A)}\ 2^{37} \qquad\textbf{(B)}\ 3\cdot 2^{36} \qquad\textbf{(C)}\  2^{38} \qquad\textbf{(D)}\ 3\cdot 2^{37} \qquad\textbf{(E)}\ 2^{39}\)\par \vspace{0.5em}\item Two different cubes of the same size are to be painted, with the color of each face being chosen independently and at random to be either black or white. What is the probability that after they are painted, the cubes can be rotated to be identical in appearance?

\(\textbf{(A)}\ \frac{9}{64} \qquad\textbf{(B)}\ \frac{289}{2048} \qquad\textbf{(C)}\  \frac{73}{512} \qquad\textbf{(D)}\ \frac{147}{1024} \qquad\textbf{(E)}\ \frac{589}{4096}\)\par \vspace{0.5em}\item How many positive integers \(n\) satisfy
\begin{equation*}
\frac{n+1000}{70} = \lfloor \sqrt{n} \rfloor?
\end{equation*}
(Recall that \(\lfloor x\rfloor\) is the greatest integer not exceeding \(x\).)

\(\textbf{(A) } 2 \qquad\textbf{(B) } 4 \qquad\textbf{(C) } 6 \qquad\textbf{(D) } 30 \qquad\textbf{(E) } 32\)\par \vspace{0.5em}\item What is the maximum value of \(\frac{(2^t-3t)t}{4^t}\) for real values of \(t?\)

\(\textbf{(A)}\ \frac{1}{16} \qquad\textbf{(B)}\ \frac{1}{15} \qquad\textbf{(C)}\ \frac{1}{12} \qquad\textbf{(D)}\ \frac{1}{10} \qquad\textbf{(E)}\ \frac{1}{9}\)\par \vspace{0.5em}\item How many integers \(n \geq 2\) are there such that whenever \(z_1, z_2, ..., z_n\) are complex numbers such that


\begin{equation*}
|z_1| = |z_2| = ... = |z_n| = 1 \text{    and    } z_1 + z_2 + ... + z_n = 0,
\end{equation*}

then the numbers \(z_1, z_2, ..., z_n\) are equally spaced on the unit circle in the complex plane?

\(\textbf{(A)}\ 1 \qquad\textbf{(B)}\ 2 \qquad\textbf{(C)}\ 3 \qquad\textbf{(D)}\ 4 \qquad\textbf{(E)}\ 5\)\par \vspace{0.5em}\item Let \(D(n)\) denote the number of ways of writing the positive integer \(n\) as a product
\begin{equation*}
n = f_1\cdot f_2\cdots f_k,
\end{equation*}
where \(k\ge1\), the \(f_i\) are integers strictly greater than \(1\), and the order in which the factors are listed matters (that is, two representations that differ only in the order of the factors are counted as distinct). For example, the number \(6\) can be written as \(6\), \(2\cdot 3\), and \(3\cdot2\), so \(D(6) = 3\). What is \(D(96)\)?

\(\textbf{(A) } 112 \qquad\textbf{(B) } 128 \qquad\textbf{(C) } 144 \qquad\textbf{(D) } 172 \qquad\textbf{(E) } 184\)\par \vspace{0.5em}\item For each real number \(a\) with \(0 \leq a \leq 1\), let numbers \(x\) and \(y\) be chosen independently at random from the intervals \([0, a]\) and \([0, 1]\), respectively, and let \(P(a)\) be the probability that


\begin{equation*}
\sin^2{(\pi x)} + \sin^2{(\pi y)} > 1
\end{equation*}

What is the maximum value of \(P(a)?\)

\(\textbf{(A)}\ \frac{7}{12} \qquad\textbf{(B)}\ 2 - \sqrt{2} \qquad\textbf{(C)}\ \frac{1+\sqrt{2}}{4} \qquad\textbf{(D)}\ \frac{\sqrt{5}-1}{2} \qquad\textbf{(E)}\ \frac{5}{8}\)\par \vspace{0.5em}\end{enumerate}
\end{document}
