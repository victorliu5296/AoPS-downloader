
\documentclass{article}
\usepackage{amsmath, amssymb}
\usepackage{geometry}
\geometry{a4paper, margin=0.75in}
\usepackage{enumitem}
\usepackage{hyperref}
\usepackage{fancyhdr}
\usepackage{tikz}
\usepackage{graphicx}
\usepackage{asymptote}
\begin{asydef}
  // Global Asymptote settings
  settings.outformat = "pdf";
  settings.render = 0;
  settings.prc = false;
  import olympiad;
  import cse5;
  size(8cm);
\end{asydef}
\pagestyle{fancy}
\fancyhead[L]{\textbf{AMC12 Problems}}
\fancyhead[R]{\textbf{2010-2024}}
\fancyfoot[C]{\thepage}
\renewcommand{\headrulewidth}{0.4pt}
\renewcommand{\footrulewidth}{0.4pt}

\title{AMC12 Problems \\ 2010-2024}
\date{}
\begin{document}\maketitle\thispagestyle{fancy}\tableofcontents\newpage\section*{2010 AMC1212A}\begin{enumerate}[label=\arabic*., itemsep=0.5em]\item What is $\left(20-\left(2010-201\right)\right)+\left(2010-\left(201-20\right)\right)$?

$\textbf{(A)}\ -4020 \qquad \textbf{(B)}\ 0 \qquad \textbf{(C)}\ 40 \qquad \textbf{(D)}\ 401 \qquad \textbf{(E)}\ 4020$\par \vspace{0.5em}\item A ferry boat shuttles tourists to an island every hour starting at 10 AM until its last trip, which starts at 3 PM. One day the boat captain notes that on the 10 AM trip there were 100 tourists on the ferry boat, and that on each successive trip, the number of tourists was 1 fewer than on the previous trip. How many tourists did the ferry take to the island that day?

$\textbf{(A)}\ 585 \qquad \textbf{(B)}\ 594 \qquad \textbf{(C)}\ 672 \qquad \textbf{(D)}\ 679 \qquad \textbf{(E)}\ 694$\par \vspace{0.5em}\item Rectangle $ABCD$, pictured below, shares $50\%$ of its area with square $EFGH$. Square $EFGH$ shares $20\%$ of its area with rectangle $ABCD$. What is $\frac{AB}{AD}$?

<center>
\begin{center}
\begin{asy}
import olympiad;
import cse5;
unitsize(1mm);
defaultpen(linewidth(.8pt)+fontsize(8pt));

draw((0,0)--(0,25)--(25,25)--(25,0)--cycle);
fill((0,20)--(0,15)--(25,15)--(25,20)--cycle,gray);
draw((0,15)--(0,20)--(25,20)--(25,15)--cycle);
draw((25,15)--(25,20)--(50,20)--(50,15)--cycle);

label("$A$",(0,20),W);
label("$B$",(50,20),E);
label("$C$",(50,15),E);
label("$D$",(0,15),W);
label("$E$",(0,25),NW);
label("$F$",(25,25),NE);
label("$G$",(25,0),SE);
label("$H$",(0,0),SW);
\end{asy}
\end{center}
</center>


$\textbf{(A)}\ 4 \qquad \textbf{(B)}\ 5 \qquad \textbf{(C)}\ 6 \qquad \textbf{(D)}\ 8 \qquad \textbf{(E)}\ 10$\par \vspace{0.5em}\item If $x<0$, then which of the following must be positive?

$\textbf{(A)}\ \frac{x}{\left|x\right|} \qquad \textbf{(B)}\ -x^2 \qquad \textbf{(C)}\ -2^x \qquad \textbf{(D)}\ -x^{-1} \qquad \textbf{(E)}\ \sqrt[3]{x}$\par \vspace{0.5em}\item Halfway through a 100-shot archery tournament, Chelsea leads by 50 points. For each shot a bullseye scores 10 points, with other possible scores being 8, 4, 2, and 0 points. Chelsea always scores at least 4 points on each shot. If Chelsea's next $n$ shots are bullseyes she will be guaranteed victory. What is the minimum value for $n$?

$\textbf{(A)}\ 38 \qquad \textbf{(B)}\ 40 \qquad \textbf{(C)}\ 42 \qquad \textbf{(D)}\ 44 \qquad \textbf{(E)}\ 46$\par \vspace{0.5em}\item A $\textit{palindrome}$, such as 83438, is a number that remains the same when its digits are reversed. The numbers $x$ and $x+32$ are three-digit and four-digit palindromes, respectively. What is the sum of the digits of $x$?

$\textbf{(A)}\ 20 \qquad \textbf{(B)}\ 21 \qquad \textbf{(C)}\ 22 \qquad \textbf{(D)}\ 23 \qquad \textbf{(E)}\ 24$\par \vspace{0.5em}\item Logan is constructing a scaled model of his town. The city's water tower stands 40 meters high, and the top portion is a sphere that holds 100,000 liters of water. Logan's miniature water tower holds 0.1 liters. How tall, in meters, should Logan make his tower?

$\textbf{(A)}\ 0.04 \qquad \textbf{(B)}\ \frac{0.4}{\pi} \qquad \textbf{(C)}\ 0.4 \qquad \textbf{(D)}\ \frac{4}{\pi} \qquad \textbf{(E)}\ 4$\par \vspace{0.5em}\item Triangle $ABC$ has $AB=2 \cdot AC$. Let $D$ and $E$ be on $\overline{AB}$ and $\overline{BC}$, respectively, such that $\angle BAE = \angle ACD$. Let $F$ be the intersection of segments $AE$ and $CD$, and suppose that $\triangle CFE$ is equilateral. What is $\angle ACB$?

$\textbf{(A)}\ 60^\circ \qquad \textbf{(B)}\ 75^\circ \qquad \textbf{(C)}\ 90^\circ \qquad \textbf{(D)}\ 105^\circ \qquad \textbf{(E)}\ 120^\circ$\par \vspace{0.5em}\item A solid cube has side length $3$ inches. A $2$-inch by $2$-inch square hole is cut into the center of each face. The edges of each cut are parallel to the edges of the cube, and each hole goes all the way through the cube. What is the volume, in cubic inches, of the remaining solid?

$\textbf{(A)}\ 7 \qquad \textbf{(B)}\ 8 \qquad \textbf{(C)}\ 10 \qquad \textbf{(D)}\ 12 \qquad \textbf{(E)}\ 15$\par \vspace{0.5em}\item The first four terms of an arithmetic sequence are $p$, $9$, $3p-q$, and $3p+q$. What is the $2010^\text{th}$ term of this sequence?

$\textbf{(A)}\ 8041 \qquad \textbf{(B)}\ 8043 \qquad \textbf{(C)}\ 8045 \qquad \textbf{(D)}\ 8047 \qquad \textbf{(E)}\ 8049$\par \vspace{0.5em}\item The solution of the equation $7^{x+7} = 8^x$ can be expressed in the form $x = \log_b 7^7$. What is $b$?

$\textbf{(A)}\ \frac{7}{15} \qquad \textbf{(B)}\ \frac{7}{8} \qquad \textbf{(C)}\ \frac{8}{7} \qquad \textbf{(D)}\ \frac{15}{8} \qquad \textbf{(E)}\ \frac{15}{7}$\par \vspace{0.5em}\item In a magical swamp there are two species of talking amphibians: toads, whose statements are always true, and frogs, whose statements are always false. Four amphibians, Brian, Chris, LeRoy, and Mike live together in this swamp, and they make the following statements.

Brian: "Mike and I are different species."

Chris: "LeRoy is a frog."

LeRoy: "Chris is a frog."

Mike: "Of the four of us, at least two are toads."

How many of these amphibians are frogs?

$\textbf{(A)}\ 0 \qquad \textbf{(B)}\ 1 \qquad \textbf{(C)}\ 2 \qquad \textbf{(D)}\ 3 \qquad \textbf{(E)}\ 4$\par \vspace{0.5em}\item For how many integer values of $k$ do the graphs of $x^2+y^2=k^2$ and $xy = k$ not intersect?

$\textbf{(A)}\ 0 \qquad \textbf{(B)}\ 1 \qquad \textbf{(C)}\ 2 \qquad \textbf{(D)}\ 4 \qquad \textbf{(E)}\ 8$\par \vspace{0.5em}\item Nondegenerate $\triangle ABC$ has integer side lengths, $\overline{BD}$ is an angle bisector, $AD = 3$, and $DC=8$. What is the smallest possible value of the perimeter?

$\textbf{(A)}\ 30 \qquad \textbf{(B)}\ 33 \qquad \textbf{(C)}\ 35 \qquad \textbf{(D)}\ 36 \qquad \textbf{(E)}\ 37$\par \vspace{0.5em}\item A coin is altered so that the probability that it lands on heads is less than $\frac{1}{2}$ and when the coin is flipped four times, the probability of an equal number of heads and tails is $\frac{1}{6}$. What is the probability that the coin lands on heads?

$\textbf{(A)}\ \frac{\sqrt{15}-3}{6} \qquad \textbf{(B)}\ \frac{6-\sqrt{6\sqrt{6}+2}}{12} \qquad \textbf{(C)}\ \frac{\sqrt{2}-1}{2} \qquad \textbf{(D)}\ \frac{3-\sqrt{3}}{6} \qquad \textbf{(E)}\ \frac{\sqrt{3}-1}{2}$\par \vspace{0.5em}\item Bernardo randomly picks 3 distinct numbers from the set $\{1,2,3,...,7,8,9\}$ and arranges them in descending order to form a 3-digit number. Silvia randomly picks 3 distinct numbers from the set $\{1,2,3,...,6,7,8\}$ and also arranges them in descending order to form a 3-digit number. What is the probability that Bernardo's number is larger than Silvia's number?

$\textbf{(A)}\ \frac{47}{72} \qquad \textbf{(B)}\ \frac{37}{56} \qquad \textbf{(C)}\ \frac{2}{3} \qquad \textbf{(D)}\ \frac{49}{72} \qquad \textbf{(E)}\ \frac{39}{56}$\par \vspace{0.5em}\item Equiangular hexagon $ABCDEF$ has side lengths $AB=CD=EF=1$ and $BC=DE=FA=r$. The area of $\triangle ACE$ is $70\%$ of the area of the hexagon. What is the sum of all possible values of $r$?

$\textbf{(A)}\ \frac{4\sqrt{3}}{3} \qquad \textbf{(B)} \frac{10}{3} \qquad \textbf{(C)}\ 4 \qquad \textbf{(D)}\ \frac{17}{4} \qquad \textbf{(E)}\ 6$\par \vspace{0.5em}\item A 16-step path is to go from $(-4,-4)$ to $(4,4)$ with each step increasing either the $x$-coordinate or the $y$-coordinate by 1. How many such paths stay outside or on the boundary of the square $-2 \le x \le 2$, $-2 \le y \le 2$ at each step?

$\textbf{(A)}\ 92 \qquad \textbf{(B)}\ 144 \qquad \textbf{(C)}\ 1568 \qquad \textbf{(D)}\ 1698 \qquad \textbf{(E)}\ 12,800$\par \vspace{0.5em}\item Each of 2010 boxes in a line contains a single red marble, and for $1 \le k \le 2010$, the box in the $k\text{th}$ position also contains $k$ white marbles. Isabella begins at the first box and successively draws a single marble at random from each box, in order. She stops when she first draws a red marble. Let $P(n)$ be the probability that Isabella stops after drawing exactly $n$ marbles. What is the smallest value of $n$ for which $P(n) < \frac{1}{2010}$?

$\textbf{(A)}\ 45 \qquad \textbf{(B)}\ 63 \qquad \textbf{(C)}\ 64 \qquad \textbf{(D)}\ 201 \qquad \textbf{(E)}\ 1005$\par \vspace{0.5em}\item Arithmetic sequences $\left(a_n\right)$ and $\left(b_n\right)$ have integer terms with $a_1=b_1=1<a_2 \le b_2$ and $a_n b_n = 2010$ for some $n$. What is the largest possible value of $n$?

$\textbf{(A)}\ 2 \qquad \textbf{(B)}\ 3 \qquad \textbf{(C)}\ 8 \qquad \textbf{(D)}\ 288 \qquad \textbf{(E)}\ 2009$\par \vspace{0.5em}\item The graph of $y=x^6-10x^5+29x^4-4x^3+ax^2$ lies above the line $y=bx+c$ except at three values of $x$, where the graph and the line intersect. What is the largest of these values?

$\textbf{(A)}\ 4 \qquad \textbf{(B)}\ 5 \qquad \textbf{(C)}\ 6 \qquad \textbf{(D)}\ 7 \qquad \textbf{(E)}\ 8$\par \vspace{0.5em}\item What is the minimum value of $\left|x-1\right| + \left|2x-1\right| + \left|3x-1\right| + \cdots + \left|119x - 1 \right|$?

$\textbf{(A)}\ 49 \qquad \textbf{(B)}\ 50 \qquad \textbf{(C)}\ 51 \qquad \textbf{(D)}\ 52 \qquad \textbf{(E)}\ 53$\par \vspace{0.5em}\item The number obtained from the last two nonzero digits of $90!$ is equal to $n$. What is $n$?

$\textbf{(A)}\ 12 \qquad \textbf{(B)}\ 32 \qquad \textbf{(C)}\ 48 \qquad \textbf{(D)}\ 52 \qquad \textbf{(E)}\ 68$\par \vspace{0.5em}\item Let $f(x) = \log_{10} \left(\sin(\pi x) \cdot \sin(2 \pi x) \cdot \sin (3 \pi x) \cdots \sin(8 \pi x)\right)$. The intersection of the domain of $f(x)$ with the interval $[0,1]$ is a union of $n$ disjoint open intervals. What is $n$?

$\textbf{(A)}\ 2 \qquad \textbf{(B)}\ 12 \qquad \textbf{(C)}\ 18 \qquad \textbf{(D)}\ 22 \qquad \textbf{(E)}\ 36$\par \vspace{0.5em}\item Two quadrilaterals are considered the same if one can be obtained from the other by a rotation and a translation. How many different convex cyclic quadrilaterals are there with integer sides and perimeter equal to 32?

$\textbf{(A)}\ 560 \qquad \textbf{(B)}\ 564 \qquad \textbf{(C)}\ 568 \qquad \textbf{(D)}\ 1498 \qquad \textbf{(E)}\ 2255$\par \vspace{0.5em}\end{enumerate}\newpage\section*{2010 AMC1212B}\begin{enumerate}[label=\arabic*., itemsep=0.5em]\item Makarla attended two meetings during her $9$-hour work day. The first meeting took $45$ minutes and the second meeting took twice as long. What percent of her work day was spent attending meetings?

$\textbf{(A)}\ 15 \qquad \textbf{(B)}\ 20 \qquad \textbf{(C)}\ 25 \qquad \textbf{(D)}\ 30 \qquad \textbf{(E)}\ 35$\par \vspace{0.5em}\item A big $L$ is formed as shown. What is its area?

<center>
\begin{center}
\begin{asy}
import olympiad;
import cse5;
unitsize(4mm);
defaultpen(linewidth(.8pt));

draw((0,0)--(5,0)--(5,2)--(2,2)--(2,8)--(0,8)--cycle);
label("8",(0,4),W);
label("5",(5/2,0),S);
label("2",(5,1),E);
label("2",(1,8),N);
\end{asy}
\end{center}
</center>

$\textbf{(A)}\ 22 \qquad \textbf{(B)}\ 24 \qquad \textbf{(C)}\ 26 \qquad \textbf{(D)}\ 28 \qquad \textbf{(E)}\ 30$\par \vspace{0.5em}\item A ticket to a school play cost $x$ dollars, where $x$ is a whole number. A group of 9<sup>th</sup> graders buys tickets costing a total of $$48$, and a group of 10<sup>th</sup> graders buys tickets costing a total of $$64$. How many values for $x$ are possible?

$\textbf{(A)}\ 1 \qquad \textbf{(B)}\ 2 \qquad \textbf{(C)}\ 3 \qquad \textbf{(D)}\ 4 \qquad \textbf{(E)}\ 5$\par \vspace{0.5em}\item A month with $31$ days has the same number of Mondays and Wednesdays. How many of the seven days of the week could be the first day of this month?

$\textbf{(A)}\ 2 \qquad \textbf{(B)}\ 3 \qquad \textbf{(C)}\ 4 \qquad \textbf{(D)}\ 5 \qquad \textbf{(E)}\ 6$\par \vspace{0.5em}\item Lucky Larry's teacher asked him to substitute numbers for $a$, $b$, $c$, $d$, and $e$ in the expression $a-(b-(c-(d+e)))$ and evaluate the result. Larry ignored the parentheses but added and subtracted correctly and obtained the correct result by coincidence. The numbers Larry substituted for $a$, $b$, $c$, and $d$ were $1$, $2$, $3$, and $4$, respectively. What number did Larry substitute for $e$?

$\textbf{(A)}\ -5 \qquad \textbf{(B)}\ -3 \qquad \textbf{(C)}\ 0 \qquad \textbf{(D)}\ 3 \qquad \textbf{(E)}\ 5$\par \vspace{0.5em}\item At the beginning of the school year, $50\%$ of all students in Mr. Wells' math class answered "Yes" to the question "Do you love math", and $50\%$ answered "No." At the end of the school year, $70\%$ answered "Yes" and $30\%$ answered "No." Altogether, $x\%$ of the students gave a different answer at the beginning and end of the school year. What is the difference between the maximum and the minimum possible values of $x$?

$\textbf{(A)}\ 0 \qquad \textbf{(B)}\ 20 \qquad \textbf{(C)}\ 40 \qquad \textbf{(D)}\ 60 \qquad \textbf{(E)}\ 80$\par \vspace{0.5em}\item Shelby drives her scooter at a speed of $30$ miles per hour if it is not raining, and $20$ miles per hour if it is raining. Today she drove in the sun in the morning and in the rain in the evening, for a total of $16$ miles in $40$ minutes. How many minutes did she drive in the rain?

$\textbf{(A)}\ 18 \qquad \textbf{(B)}\ 21 \qquad \textbf{(C)}\ 24 \qquad \textbf{(D)}\ 27 \qquad \textbf{(E)}\ 30$\par \vspace{0.5em}\item Every high school in the city of Euclid sent a team of $3$ students to a math contest. Each participant in the contest received a different score. Andrea's score was the median among all students, and hers was the highest score on her team. Andrea's teammates Beth and Carla placed $37$<sup>th</sup> and $64$<sup>th</sup>, respectively. How many schools are in the city?

$\textbf{(A)}\ 22 \qquad \textbf{(B)}\ 23 \qquad \textbf{(C)}\ 24 \qquad \textbf{(D)}\ 25 \qquad \textbf{(E)}\ 26$\par \vspace{0.5em}\item Let $n$ be the smallest positive integer such that $n$ is divisible by $20$, $n^2$ is a perfect cube, and $n^3$ is a perfect square. What is the number of digits of $n$?

$\textbf{(A)}\ 3 \qquad \textbf{(B)}\ 4 \qquad \textbf{(C)}\ 5 \qquad \textbf{(D)}\ 6 \qquad \textbf{(E)}\ 7$\par \vspace{0.5em}\item The average of the numbers $1, 2, 3,\cdots, 98, 99,$ and $x$ is $100x$. What is $x$?

$\textbf{(A)}\ \dfrac{49}{101} \qquad \textbf{(B)}\ \dfrac{50}{101} \qquad \textbf{(C)}\ \dfrac{1}{2} \qquad \textbf{(D)}\ \dfrac{51}{101} \qquad \textbf{(E)}\ \dfrac{50}{99}$\par \vspace{0.5em}\item A palindrome between $1000$ and $10,000$ is chosen at random. What is the probability that it is divisible by $7$?

$\textbf{(A)}\ \dfrac{1}{10} \qquad \textbf{(B)}\ \dfrac{1}{9} \qquad \textbf{(C)}\ \dfrac{1}{7} \qquad \textbf{(D)}\ \dfrac{1}{6} \qquad \textbf{(E)}\ \dfrac{1}{5}$\par \vspace{0.5em}\item For what value of $x$ does


\begin\{equation*\}
\log\_\{\sqrt\{2\}\}\sqrt\{x\}+\log\_\{2\}\{x\}+\log\_\{4\}\{x\^2\}+\log\_\{8\}\{x\^3\}+\log\_\{16\}\{x\^4\}=40?
\end\{equation*\}


$\textbf{(A)}\ 8 \qquad \textbf{(B)}\ 16 \qquad \textbf{(C)}\ 32 \qquad \textbf{(D)}\ 256 \qquad \textbf{(E)}\ 1024$\par \vspace{0.5em}\item In $\triangle ABC$, $\cos(2A-B)+\sin(A+B)=2$ and $AB=4$. What is $BC$?

$\textbf{(A)}\ \sqrt{2} \qquad \textbf{(B)}\ \sqrt{3} \qquad \textbf{(C)}\ 2 \qquad \textbf{(D)}\ 2\sqrt{2} \qquad \textbf{(E)}\ 2\sqrt{3}$\par \vspace{0.5em}\item Let $a$, $b$, $c$, $d$, and $e$ be positive integers with $a+b+c+d+e=2010$ and let $M$ be the largest of the sums $a+b$, $b+c$, $c+d$ and $d+e$. What is the smallest possible value of $M$?

$\textbf{(A)}\ 670 \qquad \textbf{(B)}\ 671 \qquad \textbf{(C)}\ 802 \qquad \textbf{(D)}\ 803 \qquad \textbf{(E)}\ 804$\par \vspace{0.5em}\item For how many ordered triples $(x,y,z)$ of nonnegative integers less than $20$ are there exactly two distinct elements in the set $\{i^x, (1+i)^y, z\}$, where $i=\sqrt{-1}$?

$\textbf{(A)}\ 149 \qquad \textbf{(B)}\ 205 \qquad \textbf{(C)}\ 215 \qquad \textbf{(D)}\ 225 \qquad \textbf{(E)}\ 235$\par \vspace{0.5em}\item Positive integers $a$, $b$, and $c$ are randomly and independently selected with replacement from the set $\{1, 2, 3,\dots, 2010\}$. What is the probability that $abc + ab + a$ is divisible by $3$?

$\textbf{(A)}\ \dfrac{1}{3} \qquad \textbf{(B)}\ \dfrac{29}{81} \qquad \textbf{(C)}\ \dfrac{31}{81} \qquad \textbf{(D)}\ \dfrac{11}{27} \qquad \textbf{(E)}\ \dfrac{13}{27}$\par \vspace{0.5em}\item The entries in a $3 \times 3$ array include all the digits from $1$ through $9$, arranged so that the entries in every row and column are in increasing order. How many such arrays are there?

$\textbf{(A)}\ 18 \qquad \textbf{(B)}\ 24 \qquad \textbf{(C)}\ 36 \qquad \textbf{(D)}\ 42 \qquad \textbf{(E)}\ 60$\par \vspace{0.5em}\item A frog makes $3$ jumps, each exactly $1$ meter long. The directions of the jumps are chosen independently at random. What is the probability that the frog's final position is no more than $1$ meter from its starting position?

$\textbf{(A)}\ \dfrac{1}{6} \qquad \textbf{(B)}\ \dfrac{1}{5} \qquad \textbf{(C)}\ \dfrac{1}{4} \qquad \textbf{(D)}\ \dfrac{1}{3} \qquad \textbf{(E)}\ \dfrac{1}{2}$\par \vspace{0.5em}\item A high school basketball game between the Raiders and Wildcats was tied at the end of the first quarter. The number of points scored by the Raiders in each of the four quarters formed an increasing geometric sequence, and the number of points scored by the Wildcats in each of the four quarters formed an increasing arithmetic sequence. At the end of the fourth quarter, the Raiders had won by one point. Neither team scored more than $100$ points. What was the total number of points scored by the two teams in the first half?

$\textbf{(A)}\ 30 \qquad \textbf{(B)}\ 31 \qquad \textbf{(C)}\ 32 \qquad \textbf{(D)}\ 33 \qquad \textbf{(E)}\ 34$\par \vspace{0.5em}\item A geometric sequence $(a_n)$ has $a_1=\sin x$, $a_2=\cos x$, and $a_3= \tan x$ for some real number $x$. For what value of $n$ does $a_n=1+\cos x$?


$\textbf{(A)}\ 4 \qquad \textbf{(B)}\ 5 \qquad \textbf{(C)}\ 6 \qquad \textbf{(D)}\ 7 \qquad \textbf{(E)}\ 8$\par \vspace{0.5em}\item Let $a > 0$, and let $P(x)$ be a polynomial with integer coefficients such that

<center>
$P(1) = P(3) = P(5) = P(7) = a$, and<br/>
$P(2) = P(4) = P(6) = P(8) = -a$.
</center>

What is the smallest possible value of $a$?

$\textbf{(A)}\ 105 \qquad \textbf{(B)}\ 315 \qquad \textbf{(C)}\ 945 \qquad \textbf{(D)}\ 7! \qquad \textbf{(E)}\ 8!$\par \vspace{0.5em}\item Let $ABCD$ be a cyclic quadrilateral. The side lengths of $ABCD$ are distinct integers less than $15$ such that $BC\cdot CD=AB\cdot DA$. What is the largest possible value of $BD$?

$\textbf{(A)}\ \sqrt{\dfrac{325}{2}} \qquad \textbf{(B)}\ \sqrt{185} \qquad \textbf{(C)}\ \sqrt{\dfrac{389}{2}} \qquad \textbf{(D)}\ \sqrt{\dfrac{425}{2}} \qquad \textbf{(E)}\ \sqrt{\dfrac{533}{2}}$\par \vspace{0.5em}\item Monic quadratic polynomials $P(x)$ and $Q(x)$ have the property that $P(Q(x))$ has zeros at $x=-23, -21, -17,$ and $-15$, and $Q(P(x))$ has zeros at $x=-59,-57,-51$ and $-49$. What is the sum of the minimum values of $P(x)$ and $Q(x)$? 

$\textbf{(A)}\ -100 \qquad \textbf{(B)}\ -82 \qquad \textbf{(C)}\ -73 \qquad \textbf{(D)}\ -64 \qquad \textbf{(E)}\ 0$\par \vspace{0.5em}\item The set of real numbers $x$ for which 


\begin\{equation*\}
\dfrac\{1\}\{x-2009\}+\dfrac\{1\}\{x-2010\}+\dfrac\{1\}\{x-2011\}\ge1
\end\{equation*\}


is the union of intervals of the form $a<x\le b$. What is the sum of the lengths of these intervals?

$\textbf{(A)}\ \dfrac{1003}{335} \qquad \textbf{(B)}\ \dfrac{1004}{335} \qquad \textbf{(C)}\ 3 \qquad \textbf{(D)}\ \dfrac{403}{134} \qquad \textbf{(E)}\ \dfrac{202}{67}$\par \vspace{0.5em}\item For every integer $n\ge2$, let $\text{pow}(n)$ be the largest power of the largest prime that divides $n$. For example $\text{pow}(144)=\text{pow}(2^4\cdot3^2)=3^2$. What is the largest integer $m$ such that $2010^m$ divides

<center>
$\prod_{n=2}^{5300}\text{pow}(n)$?
</center>


$\textbf{(A)}\ 74 \qquad \textbf{(B)}\ 75 \qquad \textbf{(C)}\ 76 \qquad \textbf{(D)}\ 77 \qquad \textbf{(E)}\ 78$\par \vspace{0.5em}\end{enumerate}\newpage\section*{2011 AMC1212A}\begin{enumerate}[label=\arabic*., itemsep=0.5em]\item A cell phone plan costs $\$20$ dollars each month, plus $5$ cents per text message sent, plus $10$ cents for each minute used over $30$ hours. In January Michelle sent $100$ text messages and talked for $30.5\$ hours. How much did she have to pay?
\$
\textbf\{(A)\}\ 24.00 \qquad
\textbf\{(B)\}\ 24.50 \qquad
\textbf\{(C)\}\ 25.50 \qquad
\textbf\{(D)\}\ 28.00 \qquad
\textbf\{(E)\}\ 30.00 \$\par \vspace{0.5em}\item There are $5$ coins placed flat on a table according to the figure. What is the order of the coins from top to bottom?

\begin{center}
\begin{asy}
import olympiad;
import cse5;
size(100); defaultpen(linewidth(.8pt)+fontsize(8pt));
draw(arc((0,1), 1.2, 25, 214));
draw(arc((.951,.309), 1.2, 0, 360));
draw(arc((.588,-.809), 1.2, 132, 370));
draw(arc((-.588,-.809), 1.2, 75, 300));
draw(arc((-.951,.309), 1.2, 96, 228));
label("$A$",(0,1),NW); label("$B$",(-1.1,.309),NW); label("$C$",(.951,.309),E); label("$D$",(-.588,-.809),W); label("$E$",(.588,-.809),S);
\end{asy}
\end{center}

\$
\textbf\{(A)\}\ (C, A, E, D, B) \qquad
\textbf\{(B)\}\ (C, A, D, E, B) \qquad
\textbf\{(C)\}\ (C, D, E, A, B) \qquad
\textbf\{(D)\}\ (C, E, A, D, B) \qquad \\
\textbf\{(E)\}\ (C, E, D, A, B) \$\par \vspace{0.5em}\item A small bottle of shampoo can hold $35$ milliliters of shampoo, whereas a large bottle can hold $500$ milliliters of shampoo. Jasmine wants to buy the minimum number of small bottles necessary to completely fill a large bottle. How many bottles must she buy?

\$
\textbf\{(A)\}\ 11 \qquad
\textbf\{(B)\}\ 12 \qquad
\textbf\{(C)\}\ 13 \qquad
\textbf\{(D)\}\ 14 \qquad
\textbf\{(E)\}\ 15 \$\par \vspace{0.5em}\item At an elementary school, the students in third grade, fourth grade, and fifth grade run an average of $12$, $15$, and $10$ minutes per day, respectively. There are twice as many third graders as fourth graders, and twice as many fourth graders as fifth graders. What is the average number of minutes run per day by these students?

\$
\textbf\{(A)\}\ 12 \qquad
\textbf\{(B)\}\ \frac\{37\}\{3\} \qquad
\textbf\{(C)\}\ \frac\{88\}\{7\} \qquad
\textbf\{(D)\}\ 13 \qquad
\textbf\{(E)\}\ 14 \$\par \vspace{0.5em}\item Last summer $30\%$ of the birds living on Town Lake were geese, $25\%$ were swans, $10\%$ were herons, and $35\%$ were ducks. What percent of the birds that were not swans were geese?
 
\$
\textbf\{(A)\}\ 20 \qquad
\textbf\{(B)\}\ 30 \qquad
\textbf\{(C)\}\ 40 \qquad
\textbf\{(D)\}\ 50 \qquad
\textbf\{(E)\}\ 60\$\par \vspace{0.5em}\item The players on a basketball team made some three-point shots, some two-point shots, and some one-point free throws. They scored as many points with two-point shots as with three-point shots. Their number of successful free throws was one more than their number of successful two-point shots. The team's total score was $61$ points. How many free throws did they make?
 
\$
\textbf\{(A)\}\ 13 \qquad
\textbf\{(B)\}\ 14 \qquad
\textbf\{(C)\}\ 15 \qquad
\textbf\{(D)\}\ 16 \qquad
\textbf\{(E)\}\ 17 \$\par \vspace{0.5em}\item A majority of the $30$ students in Ms. Demeanor's class bought pencils at the school bookstore. Each of these students bought the same number of pencils, and this number was greater than $1$. The cost of a pencil in cents was greater than the number of pencils each student bought, and the total cost of all the pencils was $17.71$. What was the cost of a pencil in cents?

\$
\textbf\{(A)\}\ 7 \qquad
\textbf\{(B)\}\ 11 \qquad
\textbf\{(C)\}\ 17 \qquad
\textbf\{(D)\}\ 23 \qquad
\textbf\{(E)\}\ 77 \$\par \vspace{0.5em}\item In the eight term sequence $A$, $B$, $C$, $D$, $E$, $F$, $G$, $H$, the value of $C$ is $5$ and the sum of any three consecutive terms is $30$. What is $A+H$?

\$
\textbf\{(A)\}\ 17 \qquad
\textbf\{(B)\}\ 18 \qquad
\textbf\{(C)\}\ 25 \qquad
\textbf\{(D)\}\ 26 \qquad
\textbf\{(E)\}\ 43 \$\par \vspace{0.5em}\item At a twins and triplets convention, there were $9$ sets of twins and $6$ sets of triplets, all from different families. Each twin shook hands with all the twins except his/her siblings and with half the triplets. Each triplet shook hands with all the triplets except his/her siblings and with half the twins. How many handshakes took place?

\$
\textbf\{(A)\}\ 324 \qquad
\textbf\{(B)\}\ 441 \qquad
\textbf\{(C)\}\ 630 \qquad
\textbf\{(D)\}\ 648 \qquad
\textbf\{(E)\}\ 882 \$\par \vspace{0.5em}\item A pair of standard $6$-sided dice is rolled once. The sum of the numbers rolled determines the diameter of a circle. What is the probability that the numerical value of the area of the circle is less than the numerical value of the circle's circumference?

\$
\textbf\{(A)\}\ \frac\{1\}\{36\} \qquad
\textbf\{(B)\}\ \frac\{1\}\{12\} \qquad
\textbf\{(C)\}\ \frac\{1\}\{6\} \qquad
\textbf\{(D)\}\ \frac\{1\}\{4\} \qquad
\textbf\{(E)\}\ \frac\{5\}\{18\} \$\par \vspace{0.5em}\item Circles $A, B,$ and $C$ each have radius 1. Circles $A$ and $B$ share one point of tangency. Circle $C$ has a point of tangency with the midpoint of $\overline{AB}.$ What is the area inside circle $C$ but outside circle $A$ and circle $B?$

\$
\textbf\{(A)\}\ 3 - \frac\{\pi\}\{2\} \qquad
\textbf\{(B)\}\ \frac\{\pi\}\{2\} \qquad
\textbf\{(C)\}\  2 \qquad
\textbf\{(D)\}\ \frac\{3\pi\}\{4\} \qquad
\textbf\{(E)\}\ 1+\frac\{\pi\}\{2\} \$\par \vspace{0.5em}\item A power boat and a raft both left dock $A$ on a river and headed downstream. The raft drifted at the speed of the river current. The power boat maintained a constant speed with respect to the river. The power boat reached dock $B$ downriver, then immediately turned and traveled back upriver. It eventually met the raft on the river 9 hours after leaving dock $A.$ How many hours did it take the power boat to go from $A$ to $B?$

\$
\textbf\{(A)\}\ 3 \qquad
\textbf\{(B)\}\ 3.5 \qquad
\textbf\{(C)\}\  4 \qquad
\textbf\{(D)\}\ 4.5 \qquad
\textbf\{(E)\}\ 5 \$\par \vspace{0.5em}\item Triangle $ABC$ has side-lengths $AB = 12, BC = 24,$ and $AC = 18.$ The line through the incenter of $\triangle ABC$ parallel to $\overline{BC}$ intersects $\overline{AB}$ at $M$ and $\overline{AC}$ at $N.$ What is the perimeter of $\triangle AMN?$

\$
\textbf\{(A)\}\ 27 \qquad
\textbf\{(B)\}\ 30 \qquad
\textbf\{(C)\}\  33 \qquad
\textbf\{(D)\}\ 36 \qquad
\textbf\{(E)\}\ 42 \$\par \vspace{0.5em}\item Suppose $a$ and $b$ are single-digit positive integers chosen independently and at random. What is the probability that the point $(a,b)$ lies above the parabola $y=ax^2-bx$?

\$
\textbf\{(A)\}\ \frac\{11\}\{81\} \qquad
\textbf\{(B)\}\ \frac\{13\}\{81\} \qquad
\textbf\{(C)\}\ \frac\{5\}\{27\} \qquad
\textbf\{(D)\}\ \frac\{17\}\{81\} \qquad
\textbf\{(E)\}\ \frac\{19\}\{81\} \$\par \vspace{0.5em}\item The circular base of a hemisphere of radius $2$ rests on the base of a square pyramid of height $6$. The hemisphere is tangent to the other four faces of the pyramid. What is the edge-length of the base of the pyramid?

\$
\textbf\{(A)\}\ 3\sqrt\{2\} \qquad
\textbf\{(B)\}\ \frac\{13\}\{3\} \qquad
\textbf\{(C)\}\ 4\sqrt\{2\} \qquad
\textbf\{(D)\}\ 6 \qquad
\textbf\{(E)\}\ \frac\{13\}\{2\} \$\par \vspace{0.5em}\item Each vertex of convex polygon $ABCDE$ is to be assigned a color. There are $6$ colors to choose from, and the ends of each diagonal must have different colors. How many different colorings are possible?

\$
\textbf\{(A)\}\ 2520 \qquad
\textbf\{(B)\}\ 2880 \qquad
\textbf\{(C)\}\ 3120 \qquad
\textbf\{(D)\}\ 3250 \qquad
\textbf\{(E)\}\ 3750 \$\par \vspace{0.5em}\item Circles with radii $1$, $2$, and $3$ are mutually externally tangent. What is the area of the triangle determined by the points of tangency?

\$
\textbf\{(A)\}\ \frac\{3\}\{5\} \qquad
\textbf\{(B)\}\ \frac\{4\}\{5\} \qquad
\textbf\{(C)\}\ 1 \qquad
\textbf\{(D)\}\ \frac\{6\}\{5\} \qquad
\textbf\{(E)\}\ \frac\{4\}\{3\} \$\par \vspace{0.5em}\item Suppose that $\left|x+y\right|+\left|x-y\right|=2$. What is the maximum possible value of $x^2-6x+y^2$?

\$
\textbf\{(A)\}\ 5 \qquad
\textbf\{(B)\}\ 6 \qquad
\textbf\{(C)\}\ 7 \qquad
\textbf\{(D)\}\ 8 \qquad
\textbf\{(E)\}\ 9 \$\par \vspace{0.5em}\item At a competition with $N$ players, the number of players given elite status is equal to $2^{1+\lfloor \log_{2} (N-1) \rfloor}-N$. Suppose that $19$ players are given elite status. What is the sum of the two smallest possible values of $N$?

\$
\textbf\{(A)\}\ 38 \qquad
\textbf\{(B)\}\ 90 \qquad
\textbf\{(C)\}\ 154 \qquad
\textbf\{(D)\}\ 406 \qquad
\textbf\{(E)\}\ 1024 \$\par \vspace{0.5em}\item Let $f(x)=ax^2+bx+c$, where $a$, $b$, and $c$ are integers. Suppose that $f(1)=0$, $50<f(7)<60$, $70<f(8)<80$, $5000k<f(100)<5000(k+1)$ for some integer $k$. What is $k$?

\$
\textbf\{(A)\}\ 1 \qquad
\textbf\{(B)\}\ 2 \qquad
\textbf\{(C)\}\ 3 \qquad
\textbf\{(D)\}\ 4 \qquad
\textbf\{(E)\}\ 5 \$\par \vspace{0.5em}\item Let $f_{1}(x)=\sqrt{1-x}$, and for integers $n \geq 2$, let $f_{n}(x)=f_{n-1}(\sqrt{n^2 - x})$. If $N$ is the largest value of $n$ for which the domain of $f_{n}$ is nonempty, the domain of $f_{N}$ is $\{ c\}$. What is $N+c$?

\$
\textbf\{(A)\}\ -226 \qquad
\textbf\{(B)\}\ -144 \qquad
\textbf\{(C)\}\ -20 \qquad
\textbf\{(D)\}\ 20 \qquad
\textbf\{(E)\}\ 144 \$\par \vspace{0.5em}\item Let $R$ be a unit square region and $n \geq 4$ an integer. A point $X$ in the interior of $R$ is called ''n-ray partitional'' if there are $n$ rays emanating from $X$ that divide $R$ into $n$ triangles of equal area. How many points are $100$-ray partitional but not $60$-ray partitional?

\$
\textbf\{(A)\}\ 1500 \qquad
\textbf\{(B)\}\ 1560 \qquad
\textbf\{(C)\}\ 2320 \qquad
\textbf\{(D)\}\ 2480 \qquad
\textbf\{(E)\}\ 2500 \$\par \vspace{0.5em}\item Let $f(z)= \frac{z+a}{z+b}$ and $g(z)=f(f(z))$, where $a$ and $b$ are complex numbers. Suppose that $\left| a \right| = 1$ and $g(g(z))=z$ for all $z$ for which $g(g(z))$ is defined. What is the difference between the largest and smallest possible values of $\left| b \right|$?

\$
\textbf\{(A)\}\ 0 \qquad
\textbf\{(B)\}\ \sqrt\{2\}-1 \qquad
\textbf\{(C)\}\ \sqrt\{3\}-1 \qquad
\textbf\{(D)\}\ 1 \qquad
\textbf\{(E)\}\ 2 \$\par \vspace{0.5em}\item Consider all quadrilaterals $ABCD$ such that $AB=14$, $BC=9$, $CD=7$, and $DA=12$. What is the radius of the largest possible circle that fits inside or on the boundary of such a quadrilateral?

\$
\textbf\{(A)\}\ \sqrt\{15\} \qquad
\textbf\{(B)\}\ \sqrt\{21\} \qquad
\textbf\{(C)\}\ 2\sqrt\{6\} \qquad
\textbf\{(D)\}\ 5 \qquad
\textbf\{(E)\}\ 2\sqrt\{7\} \$\par \vspace{0.5em}\item Triangle $ABC$ has $\angle BAC = 60^{\circ}$, $\angle CBA \leq 90^{\circ}$, $BC=1$, and $AC \geq AB$. Let $H$, $I$, and $O$ be the orthocenter, incenter, and circumcenter of $\triangle ABC$, respectively. Assume that the area of pentagon $BCOIH$ is the maximum possible. What is $\angle CBA$?

\$
\textbf\{(A)\}\ 60\^\{\circ\} \qquad
\textbf\{(B)\}\ 72\^\{\circ\} \qquad
\textbf\{(C)\}\ 75\^\{\circ\} \qquad
\textbf\{(D)\}\ 80\^\{\circ\} \qquad
\textbf\{(E)\}\ 90\^\{\circ\} \$\par \vspace{0.5em}\end{enumerate}\newpage\section*{2011 AMC1212B}\begin{enumerate}[label=\arabic*., itemsep=0.5em]\item What is <center>$ \frac{2+4+6}{1+3+5}-\frac{1+3+5}{2+4+6}? $</center>


$\textbf{(A)}\ -1 \qquad \textbf{(B)}\ \frac{5}{36} \qquad \textbf{(C)}\ \frac{7}{12} \qquad \textbf{(D)}\ \frac{147}{60} \qquad \textbf{(E)}\ \frac{43}{3}$\par \vspace{0.5em}\item Josanna's test scores to date are $90$, $80$, $70$, $60$, and $85$.  Her goal is to raise her test average at least $3$ points with her next test.  What is the minimum test score she would need to accomplish this goal?

$\textbf{(A)}\ 80 \qquad \textbf{(B)}\ 82 \qquad \textbf{(C)}\ 85 \qquad \textbf{(D)}\ 90 \qquad \textbf{(E)}\ 95$\par \vspace{0.5em}\item LeRoy and Bernardo went on a week-long trip together and agreed to share the costs equally.  Over the week, each of them paid for various joint expenses such as gasoline and car rental.  At the end of the trip it turned out that LeRoy had paid $A$ dollars and Bernardo had paid $B$ dollars, where $A<B$.  How many dollars must LeRoy give to Bernardo so that they share the costs equally?

$\textbf{(A)}\ \frac{A+B}{2} \qquad \textbf{(B)}\ \frac{A-B}{2} \qquad \textbf{(C)}\ \frac{B-A}{2} \qquad \textbf{(D)}\ B-A \qquad \textbf{(E)}\ A+B$\par \vspace{0.5em}\item In multiplying two positive integers $a$ and $b$, Ron reversed the digits of the two-digit number $a$.  His erroneous product was 161.  What is the correct value of the product of $a$ and $b$?

$\textbf{(A)}\ 116 \qquad \textbf{(B)}\ 161 \qquad \textbf{(C)}\ 204 \qquad \textbf{(D)}\ 214 \qquad \textbf{(E)}\ 224$\par \vspace{0.5em}\item Let $N$ be the second smallest positive integer that is divisible by every positive integer less than $7$.  What is the sum of the digits of $N$?

$\textbf{(A)}\ 3 \qquad \textbf{(B)}\ 4 \qquad \textbf{(C)}\ 5 \qquad \textbf{(D)}\ 6 \qquad \textbf{(E)}\ 9$\par \vspace{0.5em}\item Two tangents to a circle are drawn from a point $A$.  The points of contact $B$ and $C$ divide the circle into arcs with lengths in the ratio $2 : 3$.  What is the degree measure of $\angle{BAC}$?

$\textbf{(A)}\ 24 \qquad \textbf{(B)}\ 30 \qquad \textbf{(C)}\ 36 \qquad \textbf{(D)}\ 48 \qquad \textbf{(E)}\ 60$\par \vspace{0.5em}\item Let $x$ and $y$ be two-digit positive integers with mean $60$.  What is the maximum value of the ratio $\frac{x}{y}$?

$\textbf{(A)}\ 3 \qquad \textbf{(B)}\ \frac{33}{7} \qquad \textbf{(C)}\ \frac{39}{7} \qquad \textbf{(D)}\ 9 \qquad \textbf{(E)}\ \frac{99}{10}$\par \vspace{0.5em}\item Keiko walks once around a track at exactly the same constant speed every day. The sides of the track are straight, and the ends are semicircles. The track has width $6$ meters, and it takes her $36$ seconds longer to walk around the outside edge of the track than around the inside edge. What is Keiko's speed in meters per second?

$\textbf{(A)}\ \frac{\pi}{3} \qquad \textbf{(B)}\ \frac{2\pi}{3} \qquad \textbf{(C)}\ \pi \qquad \textbf{(D)}\ \frac{4\pi}{3} \qquad \textbf{(E)}\ \frac{5\pi}{3}$\par \vspace{0.5em}\item Two real numbers are selected independently and at random from the interval $[-20,10]$.  What is the probability that the product of those numbers is greater than zero?

$\textbf{(A)}\ \frac{1}{9} \qquad \textbf{(B)}\ \frac{1}{3} \qquad \textbf{(C)}\ \frac{4}{9} \qquad \textbf{(D)}\ \frac{5}{9} \qquad \textbf{(E)}\ \frac{2}{3}$\par \vspace{0.5em}\item Rectangle $ABCD$ has $AB=6$ and $BC=3$. Point $M$ is chosen on side $AB$ so that $\angle AMD=\angle CMD$. What is the degree measure of $\angle AMD$?

$\textbf{(A)}\ 15 \qquad \textbf{(B)}\ 30 \qquad \textbf{(C)}\ 45 \qquad \textbf{(D)}\ 60 \qquad \textbf{(E)}\ 75$\par \vspace{0.5em}\item A frog located at $(x,y)$, with both $x$ and $y$ integers, makes successive jumps of length $5$ and always lands on points with integer coordinates. Suppose that the frog starts at $(0,0)$ and ends at $(1,0)$. What is the smallest possible number of jumps the frog makes?

$\textbf{(A)}\ 2 \qquad \textbf{(B)}\ 3 \qquad \textbf{(C)}\ 4 \qquad \textbf{(D)}\ 5 \qquad \textbf{(E)}\ 6$\par \vspace{0.5em}\item A dart board is a regular octagon divided into regions as shown below. Suppose that a dart thrown at the board is equally likely to land anywhere on the board. What is the probability that the dart lands within the center square?


\begin{center}
\begin{asy}
import olympiad;
import cse5;
unitsize(10mm);
defaultpen(linewidth(.8pt)+fontsize(10pt));
dotfactor=4;
pair A=(0,1), B=(1,0), C=(1+sqrt(2),0), D=(2+sqrt(2),1), E=(2+sqrt(2),1+sqrt(2)), F=(1+sqrt(2),2+sqrt(2)), G=(1,2+sqrt(2)), H=(0,1+sqrt(2));
draw(A--B--C--D--E--F--G--H--cycle);
draw(A--D);
draw(B--G);
draw(C--F);
draw(E--H);
\end{asy}
\end{center}


$\textbf{(A)}\ \frac{\sqrt{2} - 1}{2} \qquad \textbf{(B)}\ \frac{1}{4} \qquad \textbf{(C)}\ \frac{2 - \sqrt{2}}{2} \qquad \textbf{(D)}\ \frac{\sqrt{2}}{4} \qquad \textbf{(E)}\ 2 - \sqrt{2}$\par \vspace{0.5em}\item Brian writes down four integers $w > x > y > z$ whose sum is $44$. The pairwise positive differences of these numbers are $1, 3, 4, 5, 6$ and $9$. What is the sum of the possible values of $w$?

$\textbf{(A)}\ 16 \qquad \textbf{(B)}\ 31 \qquad \textbf{(C)}\ 48 \qquad \textbf{(D)}\ 62 \qquad \textbf{(E)}\ 93$\par \vspace{0.5em}\item A segment through the focus $F$ of a parabola with vertex $V$ is perpendicular to $\overline{FV}$ and intersects the parabola in points $A$ and $B$. What is $\cos\left(\angle AVB\right)$?

$\textbf{(A)}\ -\frac{3\sqrt{5}}{7} \qquad \textbf{(B)}\ -\frac{2\sqrt{5}}{5} \qquad \textbf{(C)}\ -\frac{4}{5} \qquad \textbf{(D)}\ -\frac{3}{5} \qquad \textbf{(E)}\ -\frac{1}{2}$\par \vspace{0.5em}\item How many positive two-digit integers are factors of $2^{24}-1$?

$\textbf{(A)}\ 4 \qquad \textbf{(B)}\ 8 \qquad \textbf{(C)}\ 10 \qquad \textbf{(D)}\ 12 \qquad \textbf{(E)}\ 14$\par \vspace{0.5em}\item Rhombus $ABCD$ has side length $2$ and $\angle B = 120^{\circ}$. Region $R$ consists of all points inside of the rhombus that are closer to vertex $B$ than any of the other three vertices. What is the area of $R$?

$\textbf{(A)}\ \frac{\sqrt{3}}{3} \qquad \textbf{(B)}\ \frac{\sqrt{3}}{2} \qquad \textbf{(C)}\ \frac{2\sqrt{3}}{3} \qquad \textbf{(D)}\ 1 + \frac{\sqrt{3}}{3} \qquad \textbf{(E)}\ 2$\par \vspace{0.5em}\item Let $f(x) = 10^{10x}, g(x) = \log_{10}\left(\frac{x}{10}\right), h_1(x) = g(f(x))$, and $h_n(x) = h_1(h_{n-1}(x))$ for integers $n \geq 2$. What is the sum of the digits of $h_{2011}(1)$?

$\textbf{(A)}\ 16081 \qquad \textbf{(B)}\ 16089 \qquad \textbf{(C)}\ 18089 \qquad \textbf{(D)}\ 18098 \qquad \textbf{(E)}\ 18099$\par \vspace{0.5em}\item A pyramid has a square base with side of length 1 and has lateral faces that are equilateral triangles. A cube is placed within the pyramid so that one face is on the base of the pyramid and its opposite face has all its edges on the lateral faces of the pyramid. What is the volume of this cube?

$\textbf{(A)}\ 5\sqrt{2} - 7 \qquad \textbf{(B)}\ 7 - 4\sqrt{3} \qquad \textbf{(C)}\ \frac{2\sqrt{2}}{27} \qquad \textbf{(D)}\ \frac{\sqrt{2}}{9} \qquad \textbf{(E)}\ \frac{\sqrt{3}}{9}$\par \vspace{0.5em}\item A lattice point in an $xy$-coordinate system is any point $(x, y)$ where both $x$ and $y$ are integers. The graph of $y = mx + 2$ passes through no lattice point with $0 < x \leq 100$ for all $m$ such that $\frac{1}{2} < m < a$. What is the maximum possible value of $a$?

$\textbf{(A)}\ \frac{51}{101} \qquad \textbf{(B)}\ \frac{50}{99} \qquad \textbf{(C)}\ \frac{51}{100} \qquad \textbf{(D)}\ \frac{52}{101} \qquad \textbf{(E)}\ \frac{13}{25}$\par \vspace{0.5em}\item Triangle $ABC$ has $AB = 13, BC = 14$, and $AC = 15$. The points $D, E$, and $F$ are the midpoints of $\overline{AB}, \overline{BC}$, and $\overline{AC}$ respectively. Let $X \not= E$ be the intersection of the circumcircles of $\Delta BDE$ and $\Delta CEF$. What is $XA + XB + XC$?

$\textbf{(A)}\ 24 \qquad \textbf{(B)}\ 14\sqrt{3} \qquad \textbf{(C)}\ \frac{195}{8} \qquad \textbf{(D)}\ \frac{129\sqrt{7}}{14} \qquad \textbf{(E)}\ \frac{69\sqrt{2}}{4}$\par \vspace{0.5em}\item The arithmetic mean of two distinct positive integers $x$ and $y$ is a two-digit integer. The geometric mean of $x$ and $y$ is obtained by reversing the digits of the arithmetic mean. What is $|x - y|$?

$\textbf{(A)}\ 24 \qquad \textbf{(B)}\ 48 \qquad \textbf{(C)}\ 54 \qquad \textbf{(D)}\ 66 \qquad \textbf{(E)}\ 70$\par \vspace{0.5em}\item Let $T_1$ be a triangle with side lengths $2011, 2012$, and $2013$. For $n \geq 1$, if $T_n = \Delta ABC$ and $D, E$, and $F$ are the points of tangency of the incircle of $\Delta ABC$ to the sides $AB, BC$, and $AC$, respectively, then $T_{n+1}$ is a triangle with side lengths $AD, BE$, and $CF$, if it exists. What is the perimeter of the last triangle in the sequence $\left(T_n\right)$?

$\textbf{(A)}\ \frac{1509}{8} \qquad \textbf{(B)}\  \frac{1509}{32} \qquad \textbf{(C)}\  \frac{1509}{64} \qquad \textbf{(D)}\  \frac{1509}{128} \qquad \textbf{(E)}\  \frac{1509}{256}$\par \vspace{0.5em}\item A bug travels in the coordinate plane, moving only along the lines that are parallel to the $x$-axis or $y$-axis. Let $A = (-3, 2)$ and $B = (3, -2)$. Consider all possible paths of the bug from $A$ to $B$ of length at most $20$. How many points with integer coordinates lie on at least one of these paths?

$\textbf{(A)}\ 161 \qquad \textbf{(B)}\ 185 \qquad \textbf{(C)}\  195 \qquad \textbf{(D)}\  227 \qquad \textbf{(E)}\  255$\par \vspace{0.5em}\item Let $P(z) = z^8 + \left(4\sqrt{3} + 6\right)z^4 - \left(4\sqrt{3} + 7\right)$. What is the minimum perimeter among all the $8$-sided polygons in the complex plane whose vertices are precisely the zeros of $P(z)$?

$\textbf{(A)}\ 4\sqrt{3} + 4 \qquad \textbf{(B)}\ 8\sqrt{2} \qquad \textbf{(C)}\  3\sqrt{2} + 3\sqrt{6} \qquad \textbf{(D)}\  4\sqrt{2} + 4\sqrt{3} \qquad \textbf{(E)}\  4\sqrt{3} + 6$\par \vspace{0.5em}\item For every $m$ and $k$ integers with $k$ odd, denote by $\left[\frac{m}{k}\right]$ the integer closest to $\frac{m}{k}$. For every odd integer $k$, let $P(k)$ be the probability that


\begin\{equation*\}
\left[\frac\{n\}\{k\}\right] + \left[\frac\{100 - n\}\{k\}\right] = \left[\frac\{100\}\{k\}\right]
\end\{equation*\}


for an integer $n$ randomly chosen from the interval $1 \leq n \leq 99!$. What is the minimum possible value of $P(k)$ over the odd integers $k$ in the interval $1 \leq k \leq 99$?

$\textbf{(A)}\ \frac{1}{2} \qquad \textbf{(B)}\ \frac{50}{99} \qquad \textbf{(C)}\ \frac{44}{87} \qquad \textbf{(D)}\  \frac{34}{67} \qquad \textbf{(E)}\  \frac{7}{13}$\par \vspace{0.5em}\end{enumerate}\newpage\section*{2012 AMC1212A}\begin{enumerate}[label=\arabic*., itemsep=0.5em]\item A bug crawls along a number line, starting at $-2$. It crawls to $-6$, then turns around and crawls to $5$. How many units does the bug crawl altogether?

$ \textbf{(A)}\ 9\qquad\textbf{(B)}\ 11\qquad\textbf{(C)}\ 13\qquad\textbf{(D)}\ 14\qquad\textbf{(E)}\ 15 $\par \vspace{0.5em}\item Cagney can frost a cupcake every $20$ seconds and Lacey can frost a cupcake every $30$ seconds. Working together, how many cupcakes can they frost in $5$ minutes?

$ \textbf{(A)}\ 10\qquad\textbf{(B)}\ 15\qquad\textbf{(C)}\ 20\qquad\textbf{(D)}\ 25\qquad\textbf{(E)}\ 30 $\par \vspace{0.5em}\item A box $2$ centimeters high, $3$ centimeters wide, and $5$ centimeters long can hold $40$ grams of clay.  A second box with twice the height, three times the width, and the same length as the first box can hold $n$ grams of clay.  What is $n$?

$\textbf{(A)}\ 120\qquad\textbf{(B)}\ 160\qquad\textbf{(C)}\ 200\qquad\textbf{(D)}\ 240\qquad\textbf{(E)}\ 280$\par \vspace{0.5em}\item In a bag of marbles, $\tfrac{3}{5}$ of the marbles are blue and the rest are red.  If the number of red marbles is doubled and the number of blue marbles stays the same, what fraction of the marbles will be red?

\$ \textbf\{(A)\}\ \dfrac\{2\}\{5\}
\qquad\textbf\{(B)\}\ \dfrac\{3\}\{7\}
\qquad\textbf\{(C)\}\ \dfrac\{4\}\{7\}
\qquad\textbf\{(D)\}\ \dfrac\{3\}\{5\}
\qquad\textbf\{(E)\}\ \dfrac\{4\}\{5\}
 \$\par \vspace{0.5em}\item A fruit salad consists of blueberries, raspberries, grapes, and cherries.  The fruit salad has a total of $280$ pieces of fruit.  There are twice as many raspberries as blueberries, three times as many grapes as cherries, and four times as many cherries as raspberries.  How many cherries are there in the fruit salad?

$ \textbf{(A)}\ 8\qquad\textbf{(B)}\ 16\qquad\textbf{(C)}\ 25\qquad\textbf{(D)}\ 64\qquad\textbf{(E)}\ 96 $\par \vspace{0.5em}\item The sums of three whole numbers taken in pairs are $12$, $17$, and $19$.  What is the middle number?

$ \textbf{(A)}\ 4\qquad\textbf{(B)}\ 5\qquad\textbf{(C)}\ 6\qquad\textbf{(D)}\ 7\qquad\textbf{(E)}\ 8 $\par \vspace{0.5em}\item Mary divides a circle into $12$ sectors.  The central angles of these sectors, measured in degrees, are all integers and they form an arithmetic sequence.  What is the degree measure of the smallest possible sector angle?

$ \textbf{(A)}\ 5\qquad\textbf{(B)}\ 6\qquad\textbf{(C)}\ 8\qquad\textbf{(D)}\ 10\qquad\textbf{(E)}\ 12 $\par \vspace{0.5em}\item An ''iterative average'' of the numbers $1$, $2$, $3$, $4$, and $5$ is computed in the following way.  Arrange the five numbers in some order.  Find the mean of the first two numbers, then find the mean of that with the third number, then the mean of that with the fourth number, and finally the mean of that with the fifth number.  What is the difference between the largest and smallest possible values that can be obtained using this procedure?

$ \textbf{(A)}\ \frac{31}{16}\qquad\textbf{(B)}\ 2\qquad\textbf{(C)}\ \frac{17}{8}\qquad\textbf{(D)}\ 3\qquad\textbf{(E)}\ \frac{65}{16} $\par \vspace{0.5em}\item A year is a leap year if and only if the year number is divisible by $400$ (such as $2000$) or is divisible by $4$ but not by $100$ (such as $2012$).  The $200\text{th}$ anniversary of the birth of novelist Charles Dickens was celebrated on February $7$, $2012$, a Tuesday.  On what day of the week was Dickens born?

\$ \textbf\{(A)\}\ \text\{Friday\}
\qquad\textbf\{(B)\}\ \text\{Saturday\}
\qquad\textbf\{(C)\}\ \text\{Sunday\}
\qquad\textbf\{(D)\}\ \text\{Monday\}
\qquad\textbf\{(E)\}\ \text\{Tuesday\}
 \$\par \vspace{0.5em}\item A triangle has area $30$, one side of length $10$, and the median to that side of length $9$.  Let $\theta$ be the acute angle formed by that side and the median.  What is $\sin{\theta}$?

$ \textbf{(A)}\ \frac{3}{10}\qquad\textbf{(B)}\ \frac{1}{3}\qquad\textbf{(C)}\ \frac{9}{20}\qquad\textbf{(D)}\ \frac{2}{3}\qquad\textbf{(E)}\ \frac{9}{10} $\par \vspace{0.5em}\item Alex, Mel, and Chelsea play a game that has $6$ rounds.  In each round there is a single winner, and the outcomes of the rounds are independent.  For each round the probability that Alex wins is $\frac{1}{2}$, and Mel is twice as likely to win as Chelsea.  What is the probability that Alex wins three rounds, Mel wins two rounds, and Chelsea wins one round?

$ \textbf{(A)}\ \frac{5}{72}\qquad\textbf{(B)}\ \frac{5}{36}\qquad\textbf{(C)}\ \frac{1}{6}\qquad\textbf{(D)}\ \frac{1}{3}\qquad\textbf{(E)}\ 1 $\par \vspace{0.5em}\item A square region $ABCD$ is externally tangent to the circle with equation $x^2+y^2=1$ at the point $(0,1)$ on the side $CD$.  Vertices $A$ and $B$ are on the circle with equation $x^2+y^2=4$.  What is the side length of this square?

$ \textbf{(A)}\ \frac{\sqrt{10}+5}{10}\qquad\textbf{(B)}\ \frac{2\sqrt{5}}{5}\qquad\textbf{(C)}\ \frac{2\sqrt{2}}{3}\qquad\textbf{(D)}\ \frac{2\sqrt{19}-4}{5}\qquad\textbf{(E)}\ \frac{9-\sqrt{17}}{5} $\par \vspace{0.5em}\item Paula the painter and her two helpers each paint at constant, but different, rates.  They always start at $\text{8:00 AM}$, and all three always take the same amount of time to eat lunch.  On Monday the three of them painted $50\%$ of a house, quitting at $\text{4:00 PM}$.  On Tuesday, when Paula wasn't there, the two helpers painted only $24\%$ of the house and quit at $\text{2:12 PM}$.  On Wednesday Paula worked by herself and finished the house by working until $\text{7:12 PM}$.  How long, in minutes, was each day's lunch break?

\$ \textbf\{(A)\}\ 30
\qquad\textbf\{(B)\}\ 36
\qquad\textbf\{(C)\}\ 42
\qquad\textbf\{(D)\}\ 48
\qquad\textbf\{(E)\}\ 60
 \$\par \vspace{0.5em}\item The closed curve in the figure is made up of $9$ congruent circular arcs each of length $\frac{2\pi}{3}$, where each of the centers of the corresponding circles is among the vertices of a regular hexagon of side $2$. What is the area enclosed by the curve? 


\begin{center}
\begin{asy}
import olympiad;
import cse5;
size(6cm);
defaultpen(fontsize(6pt));
dotfactor=4;
label("$\circ$",(0,1));
label("$\circ$",(0.865,0.5));
label("$\circ$",(-0.865,0.5));
label("$\circ$",(0.865,-0.5));
label("$\circ$",(-0.865,-0.5));
label("$\circ$",(0,-1));
dot((0,1.5));
dot((-0.4325,0.75));
dot((0.4325,0.75));
dot((-0.4325,-0.75));
dot((0.4325,-0.75));
dot((-0.865,0));
dot((0.865,0));
dot((-1.2975,-0.75));
dot((1.2975,-0.75));
draw(Arc((0,1),0.5,210,-30));
draw(Arc((0.865,0.5),0.5,150,270));
draw(Arc((0.865,-0.5),0.5,90,-150));
draw(Arc((0.865,-0.5),0.5,90,-150));
draw(Arc((0,-1),0.5,30,150));
draw(Arc((-0.865,-0.5),0.5,330,90));
draw(Arc((-0.865,0.5),0.5,-90,30));
\end{asy}
\end{center}


$ \textbf{(A)}\ 2\pi+6\qquad\textbf{(B)}\ 2\pi+4\sqrt3 \qquad\textbf{(C)}\ 3\pi+4 \qquad\textbf{(D)}\ 2\pi+3\sqrt3+2 \qquad\textbf{(E)}\ \pi+6\sqrt3 $\par \vspace{0.5em}\item A $3\times3$ square is partitioned into $9$ unit squares.  Each unit square is painted either white or black with each color being equally likely, chosen independently and at random.  The square is then rotated $90^\circ$ clockwise about its center, and every white square in a position formerly occupied by a black square is painted black.  The colors of all other squares are left unchanged.  What is the probability that the grid is now entirely black?

\$ \textbf\{(A)\}\ \dfrac\{49\}\{512\}
\qquad\textbf\{(B)\}\ \dfrac\{7\}\{64\}
\qquad\textbf\{(C)\}\ \dfrac\{121\}\{1024\}
\qquad\textbf\{(D)\}\ \dfrac\{81\}\{512\}
\qquad\textbf\{(E)\}\ \dfrac\{9\}\{32\}
 \$\par \vspace{0.5em}\item Circle $C_1$ has its center $O$ lying on circle $C_2$.  The two circles meet at $X$ and $Y$.  Point $Z$ in the exterior of $C_1$ lies on circle $C_2$ and $XZ=13$, $OZ=11$, and $YZ=7$.  What is the radius of circle $C_1$?

$ \textbf{(A)}\ 5\qquad\textbf{(B)}\ \sqrt{26}\qquad\textbf{(C)}\ 3\sqrt{3}\qquad\textbf{(D)}\ 2\sqrt{7}\qquad\textbf{(E)}\ \sqrt{30} $\par \vspace{0.5em}\item Let $S$ be a subset of $\{1,2,3,\dots,30\}$ with the property that no pair of distinct elements in $S$ has a sum divisible by $5$.  What is the largest possible size of $S$?

$ \textbf{(A)}\ 10\qquad\textbf{(B)}\ 13\qquad\textbf{(C)}\ 15\qquad\textbf{(D)}\ 16\qquad\textbf{(E)}\ 18 $\par \vspace{0.5em}\item Triangle $ABC$ has $AB=27$, $BC=25$, and $CA=26$.  Let $I$ denote the intersection of the internal angle bisectors of $\triangle ABC$.  What is $BI$?

$ \textbf{(A)}\ 15\qquad\textbf{(B)}\ 5+\sqrt{26}+3\sqrt{3}\qquad\textbf{(C)}\ 3\sqrt{26}\qquad\textbf{(D)}\ \frac{2}{3}\sqrt{546}\qquad\textbf{(E)}\ 9\sqrt{3} $\par \vspace{0.5em}\item Adam, Benin, Chiang, Deshawn, Esther, and Fiona have internet accounts.  Some, but not all, of them are internet friends with each other, and none of them has an internet friend outside this group.  Each of them has the same number of internet friends.  In how many different ways can this happen?

\$ \textbf\{(A)\}\ 60
\qquad\textbf\{(B)\}\ 170
\qquad\textbf\{(C)\}\ 290
\qquad\textbf\{(D)\}\ 320
\qquad\textbf\{(E)\}\ 660
 \$\par \vspace{0.5em}\item Consider the polynomial


\begin\{equation*\}
P(x)=\prod\_\{k=0\}\^\{10\}(x\^\{2\^k\}+2\^k)=(x+1)(x\^2+2)(x\^4+4)\cdots (x\^\{1024\}+1024)
\end\{equation*\}


The coefficient of $x^{2012}$ is equal to $2^a$.  What is $a$?

\$ \textbf\{(A)\}\ 5
\qquad\textbf\{(B)\}\ 6
\qquad\textbf\{(C)\}\ 7
\qquad\textbf\{(D)\}\ 10
\qquad\textbf\{(E)\}\ 24
 \$\par \vspace{0.5em}\item Let $a$, $b$, and $c$ be positive integers with $a\ge$ $b\ge$ $c$ such that

\begin\{align*\}a\^\{2\}-b\^\{2\}-c\^\{2\}+ab\&=2011\text\{ and\}\\
a\^\{2\}+3b\^\{2\}+3c\^\{2\}-3ab-2ac-2bc\&=-1997
\end\{align*\}

What is $a$?

$ \textbf{(A)}\ 249\qquad\textbf{(B)}\ 250\qquad\textbf{(C)}\ 251\qquad\textbf{(D)}\ 252\qquad\textbf{(E)}\ 253 $\par \vspace{0.5em}\item Distinct planes $p_1,p_2,....,p_k$ intersect the interior of a cube $Q$. Let $S$ be the union of the faces of $Q$ and let $ P =\bigcup_{j=1}^{k}p_{j} $. The intersection of $P$ and $S$ consists of the union of all segments joining the midpoints of every pair of edges belonging to the same face of $Q$. What is the difference between the maximum and minimum possible values of $k$?

$ \textbf{(A)}\ 8\qquad\textbf{(B)}\ 12\qquad\textbf{(C)}\ 20\qquad\textbf{(D)}\ 23\qquad\textbf{(E)}\ 24 $\par \vspace{0.5em}\item Let $S$ be the square one of whose diagonals has endpoints $(0.1,0.7)$ and $(-0.1,-0.7)$.  A point $v=(x,y)$ is chosen uniformly at random over all pairs of real numbers $x$ and $y$ such that $0 \le x \le 2012$ and $0\le y\le 2012$.  Let $T(v)$ be a translated copy of $S$ centered at $v$.  What is the probability that the square region determined by $T(v)$ contains exactly two points with integer coordinates in its interior?

$ \textbf{(A)}\ \frac{1}{8}\qquad\textbf{(B) }\frac{7}{50}\qquad\textbf{(C) }\frac{4}{25}\qquad\textbf{(D) }\frac{1}{4}\qquad\textbf{(E) }\frac{8}{25} $\par \vspace{0.5em}\item Let $\{a_k\}_{k=1}^{2011}$ be the sequence of real numbers defined by $a_1=0.201,$ $a_2=(0.2011)^{a_1},$ $a_3=(0.20101)^{a_2},$ $a_4=(0.201011)^{a_3}$, and in general, 


\begin\{equation*\}
a\_k=\begin\{cases\}
(0.\underbrace\{20101\cdots 0101\}\_\{k+2\text\{ digits\}\})\^\{a\_\{k-1\}\} \& \text\{if \}k\text\{ is odd,\}\\
(0.\underbrace\{20101\cdots 01011\}\_\{k+2\text\{ digits\}\})\^\{a\_\{k-1\}\}\& \text\{if \}k\text\{ is even.\}
\end\{cases\}
\end\{equation*\}


Rearranging the numbers in the sequence  $\{a_k\}_{k=1}^{2011}$ in decreasing order produces a new sequence  $\{b_k\}_{k=1}^{2011}$.  What is the sum of all integers $k$, $1\le k \le 2011$, such that $a_k=b_k?$

$ \textbf{(A)}\ 671\qquad\textbf{(B)}\ 1006\qquad\textbf{(C)}\ 1341\qquad\textbf{(D)}\ 2011\qquad\textbf{(E)}\ 2012 $\par \vspace{0.5em}\item Let $f(x)=|2\{x\}-1|$ where $\{x\}$ denotes the fractional part of $x$.  The number $n$ is the smallest positive integer such that the equation 
\begin\{equation*\}
nf(xf(x))=x
\end\{equation*\}
 has at least $2012$ real solutions.  What is $n$?  '''Note:''' the fractional part of $x$ is a real number $y=\{x\}$ such that $0\le y<1$ and $x-y$ is an integer.

$ \textbf{(A)}\ 30\qquad\textbf{(B)}\ 31\qquad\textbf{(C)}\ 32\qquad\textbf{(D)}\ 62\qquad\textbf{(E)}\ 64 $\par \vspace{0.5em}\end{enumerate}\newpage\section*{2012 AMC1212B}\begin{enumerate}[label=\arabic*., itemsep=0.5em]\item Each third-grade classroom at Pearl Creek Elementary has 18 students and 2 pet rabbits. How many more students than rabbits are there in all 4 of the third-grade classrooms?

$ \textbf{(A)}\ 48\qquad\textbf{(B)}\ 56\qquad\textbf{(C)}\ 64\qquad\textbf{(D)}\ 72\qquad\textbf{(E)}\ 80 $\par \vspace{0.5em}\item A circle of radius 5 is inscribed in a rectangle as shown. The ratio of the length of the rectangle to its width is 2:1. What is the area of the rectangle?

\begin{center}
\begin{asy}
import olympiad;
import cse5;
draw((0,0)--(0,10)--(20,10)--(20,0)--cycle); 
draw(circle((10,5),5));
\end{asy}
\end{center}

$\textbf{(A)}\ 50\qquad\textbf{(B)}\ 100\qquad\textbf{(C)}\ 125\qquad\textbf{(D)}\ 150\qquad\textbf{(E)}\ 200$\par \vspace{0.5em}\item For a science project, Sammy observed a chipmunk and squirrel stashing acorns in holes. The chipmunk hid 3 acorns in each of the holes it dug. The squirrel hid 4 acorns in each of the holes it dug. They each hid the same number of acorns, although the squirrel needed 4 fewer holes. How many acorns did the chipmunk hide? 

$\textbf{(A)}\ 30\qquad\textbf{(B)}\ 36\qquad\textbf{(C)}\ 42\qquad\textbf{(D)}\ 48\qquad\textbf{(E)}\ 54$\par \vspace{0.5em}\item Suppose that the euro is worth $1.30$ dollars. If Diana has $500$ dollars and Etienne has $400$ euros, by what percent is the value of Etienne's money greater than the value of Diana's money?

$\textbf{(A)}\ 2\qquad\textbf{(B)}\ 4\qquad\textbf{(C)}\ 6.5\qquad\textbf{(D)}\ 8\qquad\textbf{(E)}\ 13$\par \vspace{0.5em}\item Two integers have a sum of 26. When two more integers are added to the first two, the sum is 41. Finally, when two more integers are added to the sum of the previous 4 integers, the sum is 57. What is the minimum number of even integers among the 6 integers? 

$\textbf{(A)}\ 1\qquad\textbf{(B)}\ 2\qquad\textbf{(C)}\ 3\qquad\textbf{(D)}\ 4\qquad\textbf{(E)}\ 5$\par \vspace{0.5em}\item In order to estimate the value of $x-y$ where $x$ and $y$ are real numbers with $x > y > 0$, Xiaoli rounded $x$ up by a small amount, rounded $y$ down by the same amount, and then subtracted her rounded values. Which of the following statements is necessarily correct?

$\textbf{(A)}\ \text{Her estimate is larger than }x-y$

$\textbf{(B)}\ \text{Her estimate is smaller than }x-y$

$\textbf{(C)}\ \text{Her estimate equals }x-y$

$\textbf{(D)}\ \text{Her estimate equals }y - x$

$\textbf{(E)}\ \text{Her estimate is 0}$\par \vspace{0.5em}\item Small lights are hung on a string 6 inches apart in the order red, red, green, green, green, red, red, green, green, green, and so on continuing this pattern of 2 red lights followed by 3 green lights. How many feet separate the 3rd red light and the 21st red light?

'''Note:''' 1 foot is equal to 12 inches.

$\textbf{(A)}\ 18\qquad\textbf{(B)}\ 18.5\qquad\textbf{(C)}\ 20\qquad\textbf{(D)}\ 20.5\qquad\textbf{(E)}\ 22.5 $\par \vspace{0.5em}\item A dessert chef prepares the dessert for every day of a week starting with Sunday. The dessert each day is either cake, pie, ice cream, or pudding. The same dessert may not be served two days in a row. There must be cake on Friday because of a birthday. How many different dessert menus for the week are possible?

$\textbf{(A)}\ 729\qquad\textbf{(B)}\ 972\qquad\textbf{(C)}\ 1024\qquad\textbf{(D)}\ 2187\qquad\textbf{(E)}\ 2304 $\par \vspace{0.5em}\item It takes Clea 60 seconds to walk down an escalator when it is not moving, and 24 seconds when it is moving. How many seconds would it take Clea to ride the escalator down when she is not walking?

$\textbf{(A)}\ 36\qquad\textbf{(B)}\ 40\qquad\textbf{(C)}\ 42\qquad\textbf{(D)}\ 48\qquad\textbf{(E)}\ 52 $\par \vspace{0.5em}\item What is the area of the polygon whose vertices are the points of intersection of the curves $x^2 + y^2 =25$ and $(x-4)^2 + 9y^2 = 81$?

$\textbf{(A)}\ 24\qquad\textbf{(B)}\ 27\qquad\textbf{(C)}\ 36\qquad\textbf{(D)}\ 37.5\qquad\textbf{(E)}\ 42$\par \vspace{0.5em}\item In the equation below, $A$ and $B$ are consecutive positive integers, and $A$, $B$, and $A+B$ represent number bases: 
\begin\{equation*\}
132\_A+43\_B=69\_\{A+B\}.
\end\{equation*\}

What is $A+B$?

$\textbf{(A)}\ 9\qquad\textbf{(B)}\ 11\qquad\textbf{(C)}\ 13\qquad\textbf{(D)}\ 15\qquad\textbf{(E)}\ 17 $\par \vspace{0.5em}\item How many sequences of zeros and ones of length 20 have all the zeros consecutive, or all the ones consecutive, or both?

$\textbf{(A)}\ 190\qquad\textbf{(B)}\ 192\qquad\textbf{(C)}\ 211\qquad\textbf{(D)}\ 380\qquad\textbf{(E)}\ 382 $\par \vspace{0.5em}\item Two parabolas have equations $y= x^2 + ax +b$ and $y= x^2 + cx +d$, where $a$, $b$, $c$, and $d$ are integers, each chosen independently by rolling a fair six-sided die. What is the probability that the parabolas will have at least one point in common?

$\textbf{(A)}\ \frac{1}{2}\qquad\textbf{(B)}\ \frac{25}{36}\qquad\textbf{(C)}\ \frac{5}{6}\qquad\textbf{(D)}\ \frac{31}{36}\qquad\textbf{(E)}\ 1 $\par \vspace{0.5em}\item Bernardo and Silvia play the following game. An integer between $0$ and $999$ inclusive is selected and given to Bernardo. Whenever Bernardo receives a number, he doubles it and passes the result to Silvia. Whenever Silvia receives a number, she adds $50$ to it and passes the result to Bernardo. The winner is the last person who produces a number less than $1000$. Let $N$ be the smallest initial number that results in a win for Bernardo. What is the sum of the digits of $N$?

$ \textbf{(A)}\ 7\qquad\textbf{(B)}\ 8\qquad\textbf{(C)}\ 9\qquad\textbf{(D)}\ 10\qquad\textbf{(E)}\ 11 $\par \vspace{0.5em}\item Jesse cuts a circular paper disk of radius $12$ along two radii to form two sectors, the smaller having a central angle of $120$ degrees. He makes two circular cones, using each sector to form the lateral surface of a cone. What is the ratio of the volume of the smaller cone to that of the larger one?

$\textbf{(A)}\ \frac{1}{8}\qquad\textbf{(B)}\ \frac{1}{4}\qquad\textbf{(C)}\ \frac{\sqrt{10}}{10}\qquad\textbf{(D)}\ \frac{\sqrt{5}}{6}\qquad\textbf{(E)}\ \frac{\sqrt{5}}{5}$\par \vspace{0.5em}\item Amy, Beth, and Jo listen to four different songs and discuss which ones they like. No song is liked by all three. Furthermore, for each of the three pairs of the girls, there is at least one song liked by those girls but disliked by the third. In how many different ways is this possible?

$\textbf{(A)}\ 108\qquad\textbf{(B)}\ 132\qquad\textbf{(C)}\ 671\qquad\textbf{(D)}\ 846\qquad\textbf{(E)}\ 1105 $\par \vspace{0.5em}\item Square $PQRS$ lies in the first quadrant. Points $(3,0), (5,0), (7,0),$ and $(13,0)$ lie on lines $SP, RQ, PQ,$ and $SR$, respectively. What is the sum of the coordinates of the center of the square $PQRS$?

$\textbf{(A)}\ 6\qquad\textbf{(B)}\ \frac{31}{5}\qquad\textbf{(C)}\ \frac{32}{5}\qquad\textbf{(D)}\ \frac{33}{5}\qquad\textbf{(E)}\ \frac{34}{5} $\par \vspace{0.5em}\item Let $(a_1,a_2, \dots ,a_{10})$ be a list of the first $10$ positive integers such that for each $2 \le i \le 10$ either $a_i+1$ or $a_i-1$ or both appear somewhere before $a_i$ in the list. How many such lists are there?

$\textbf{(A)}\ 120\qquad\textbf{(B)}\ 512\qquad\textbf{(C)}\ 1024\qquad\textbf{(D)}\ 181,440\qquad\textbf{(E)}\ 362,880$\par \vspace{0.5em}\item A unit cube has vertices $P_1,P_2,P_3,P_4,P_1',P_2',P_3',$ and $P_4'$. Vertices $P_2$, $P_3$, and $P_4$ are adjacent to $P_1$, and for $1\le i\le 4,$ vertices $P_i$ and $P_i'$ are opposite to each other. A regular octahedron has one vertex in each of the segments $P_1P_2$, $P_1P_3$, $P_1P_4$, $P_1'P_2'$, $P_1'P_3'$, and $P_1'P_4'$. What is the octahedron's side length?


\begin{center}
\begin{asy}
import olympiad;
import cse5;
import three;

size(7.5cm);
triple eye = (-4, -8, 3);
currentprojection = perspective(eye);

triple[] P = \{(1, -1, -1), (-1, -1, -1), (-1, 1, -1), (-1, -1, 1), (1, -1, -1)\}; // P[0] = P[4] for convenience
triple[] Pp = \{-P[0], -P[1], -P[2], -P[3], -P[4]\};

// draw octahedron
triple pt(int k)\{ return (3*P[k] + P[1])/4; \}
triple ptp(int k)\{ return (3*Pp[k] + Pp[1])/4; \}
draw(pt(2)--pt(3)--pt(4)--cycle, gray(0.6));
draw(ptp(2)--pt(3)--ptp(4)--cycle, gray(0.6));
draw(ptp(2)--pt(4), gray(0.6));
draw(pt(2)--ptp(4), gray(0.6));
draw(pt(4)--ptp(3)--pt(2), gray(0.6) + linetype("4 4"));
draw(ptp(4)--ptp(3)--ptp(2), gray(0.6) + linetype("4 4"));

// draw cube
for(int i = 0; i < 4; ++i)\{
	draw(P[1]--P[i]); draw(Pp[1]--Pp[i]);
	for(int j = 0; j < 4; ++j)\{
		if(i == 1 || j == 1 || i == j) continue;
		draw(P[i]--Pp[j]); draw(Pp[i]--P[j]);
	\}
	dot(P[i]); dot(Pp[i]);
	dot(pt(i)); dot(ptp(i));
\}

label("$P_1$", P[1], dir(P[1]));
label("$P_2$", P[2], dir(P[2]));
label("$P_3$", P[3], dir(-45));
label("$P_4$", P[4], dir(P[4]));
label("$P'_1$", Pp[1], dir(Pp[1]));
label("$P'_2$", Pp[2], dir(Pp[2]));
label("$P'_3$", Pp[3], dir(-100));
label("$P'_4$", Pp[4], dir(Pp[4]));
\end{asy}
\end{center}


$\textbf{(A)}\ \frac{3\sqrt{2}}{4}\qquad\textbf{(B)}\ \frac{7\sqrt{6}}{16}\qquad\textbf{(C)}\ \frac{\sqrt{5}}{2}\qquad\textbf{(D)}\ \frac{2\sqrt{3}}{3}\qquad\textbf{(E)}\ \frac{\sqrt{6}}{2} $\par \vspace{0.5em}\item A trapezoid has side lengths $3$, $5$, $7$, and $11$. The sums of all the possible areas of the trapezoid can be written in the form of $r_1\sqrt{n_1}+r_2\sqrt{n_2}+r_3$, where $r_1$, $r_2$, and $r_3$ are rational numbers and $n_1$ and $n_2$ are positive integers not divisible by the square of any prime. What is the greatest integer less than or equal to $r_1+r_2+r_3+n_1+n_2$?

$\textbf{(A)}\ 57\qquad\textbf{(B)}\ 59\qquad\textbf{(C)}\ 61\qquad\textbf{(D)}\ 63\qquad\textbf{(E)}\ 65$\par \vspace{0.5em}\item Square $AXYZ$ is inscribed in equiangular hexagon $ABCDEF$ with $X$ on $\overline{BC}$, $Y$ on $\overline{DE}$, and $Z$ on $\overline{EF}$. Suppose that $AB=40$, and $EF=41(\sqrt{3}-1)$. What is the side-length of the square?


\begin{center}
\begin{asy}
import olympiad;
import cse5;
size(200);
defaultpen(linewidth(1));
pair A=origin,B=(2.5,0),C=B+2.5*dir(60), D=C+1.75*dir(120),E=D-(3.19,0),F=E-1.8*dir(60);
pair X=waypoint(B--C,0.345),Z=rotate(90,A)*X,Y=rotate(90,Z)*A;
draw(A--B--C--D--E--F--cycle);
draw(A--X--Y--Z--cycle,linewidth(0.9)+linetype("2 2"));
dot("$A$",A,W,linewidth(4));
dot("$B$",B,dir(0),linewidth(4));
dot("$C$",C,dir(0),linewidth(4));
dot("$D$",D,dir(20),linewidth(4));
dot("$E$",E,dir(100),linewidth(4));
dot("$F$",F,W,linewidth(4));
dot("$X$",X,dir(0),linewidth(4));
dot("$Y$",Y,N,linewidth(4));
dot("$Z$",Z,W,linewidth(4));
\end{asy}
\end{center}


$\textbf{(A)}\ 29\sqrt{3} \qquad\textbf{(B)}\ \frac{21}{2}\sqrt{2}+\frac{41}{2}\sqrt{3}\qquad\textbf{(C)}\ 20\sqrt{3}+16$

$\textbf{(D)}\ 20\sqrt{2}+13\sqrt{3} \qquad\textbf{(E)}\ 21\sqrt{6} $\par \vspace{0.5em}\item A bug travels from $A$ to $B$ along the segments in the hexagonal lattice pictured below. The segments marked with an arrow can be traveled only in the direction of the arrow, and the bug never travels the same segment more than once. How many different paths are there?


\begin{center}
\begin{asy}
import olympiad;
import cse5;
size(10cm);
draw((0.0,0.0)--(1.0,1.7320508075688772)--(3.0,1.7320508075688772)--(4.0,3.4641016151377544)--(6.0,3.4641016151377544)--(7.0,5.196152422706632)--(9.0,5.196152422706632)--(10.0,6.928203230275509)--(12.0,6.928203230275509));
draw((0.0,0.0)--(1.0,1.7320508075688772)--(3.0,1.7320508075688772)--(4.0,3.4641016151377544)--(6.0,3.4641016151377544)--(7.0,5.196152422706632)--(9.0,5.196152422706632)--(10.0,6.928203230275509)--(12.0,6.928203230275509));
draw((3.0,-1.7320508075688772)--(4.0,0.0)--(6.0,0.0)--(7.0,1.7320508075688772)--(9.0,1.7320508075688772)--(10.0,3.4641016151377544)--(12.0,3.464101615137755)--(13.0,5.196152422706632)--(15.0,5.196152422706632));
draw((6.0,-3.4641016151377544)--(7.0,-1.7320508075688772)--(9.0,-1.7320508075688772)--(10.0,0.0)--(12.0,0.0)--(13.0,1.7320508075688772)--(15.0,1.7320508075688776)--(16.0,3.464101615137755)--(18.0,3.4641016151377544));
draw((9.0,-5.196152422706632)--(10.0,-3.464101615137755)--(12.0,-3.464101615137755)--(13.0,-1.7320508075688776)--(15.0,-1.7320508075688776)--(16.0,0)--(18.0,0.0)--(19.0,1.7320508075688772)--(21.0,1.7320508075688767));
draw((12.0,-6.928203230275509)--(13.0,-5.196152422706632)--(15.0,-5.196152422706632)--(16.0,-3.464101615137755)--(18.0,-3.4641016151377544)--(19.0,-1.7320508075688772)--(21.0,-1.7320508075688767)--(22.0,0));
draw((0.0,-0.0)--(1.0,-1.7320508075688772)--(3.0,-1.7320508075688772)--(4.0,-3.4641016151377544)--(6.0,-3.4641016151377544)--(7.0,-5.196152422706632)--(9.0,-5.196152422706632)--(10.0,-6.928203230275509)--(12.0,-6.928203230275509));
draw((3.0,1.7320508075688772)--(4.0,-0.0)--(6.0,-0.0)--(7.0,-1.7320508075688772)--(9.0,-1.7320508075688772)--(10.0,-3.4641016151377544)--(12.0,-3.464101615137755)--(13.0,-5.196152422706632)--(15.0,-5.196152422706632));
draw((6.0,3.4641016151377544)--(7.0,1.7320508075688772)--(9.0,1.7320508075688772)--(10.0,-0.0)--(12.0,-0.0)--(13.0,-1.7320508075688772)--(15.0,-1.7320508075688776)--(16.0,-3.464101615137755)--(18.0,-3.4641016151377544));
draw((9.0,5.1961524)--(10.0,3.464101)--(12.0,3.46410)--(13.0,1.73205)--(15.0,1.732050)--(16.0,0)--(18.0,-0.0)--(19.0,-1.7320)--(21.0,-1.73205080));
draw((12.0,6.928203)--(13.0,5.1961524)--(15.0,5.1961524)--(16.0,3.464101615)--(18.0,3.4641016)--(19.0,1.7320508)--(21.0,1.732050)--(22.0,0));
dot((0,0));
dot((22,0));
label("$A$",(0,0),WNW);
label("$B$",(22,0),E);
filldraw((2.0,1.7320508075688772)--(1.6,1.2320508075688772)--(1.75,1.7320508075688772)--(1.6,2.232050807568877)--cycle,black);
filldraw((5.0,3.4641016151377544)--(4.6,2.9641016151377544)--(4.75,3.4641016151377544)--(4.6,3.9641016151377544)--cycle,black);
filldraw((8.0,5.196152422706632)--(7.6,4.696152422706632)--(7.75,5.196152422706632)--(7.6,5.696152422706632)--cycle,black);
filldraw((11.0,6.928203230275509)--(10.6,6.428203230275509)--(10.75,6.928203230275509)--(10.6,7.428203230275509)--cycle,black);
filldraw((4.6,0.0)--(5.0,-0.5)--(4.85,0.0)--(5.0,0.5)--cycle,white);
filldraw((8.0,1.732050)--(7.6,1.2320)--(7.75,1.73205)--(7.6,2.2320)--cycle,black);
filldraw((11.0,3.4641016)--(10.6,2.9641016)--(10.75,3.46410161)--(10.6,3.964101)--cycle,black);
filldraw((14.0,5.196152422706632)--(13.6,4.696152422706632)--(13.75,5.196152422706632)--(13.6,5.696152422706632)--cycle,black);
filldraw((8.0,-1.732050)--(7.6,-2.232050)--(7.75,-1.7320508)--(7.6,-1.2320)--cycle,black);
filldraw((10.6,0.0)--(11,-0.5)--(10.85,0.0)--(11,0.5)--cycle,white);
filldraw((14.0,1.7320508075688772)--(13.6,1.2320508075688772)--(13.75,1.7320508075688772)--(13.6,2.232050807568877)--cycle,black);
filldraw((17.0,3.464101615137755)--(16.6,2.964101615137755)--(16.75,3.464101615137755)--(16.6,3.964101615137755)--cycle,black);
filldraw((11.0,-3.464101615137755)--(10.6,-3.964101615137755)--(10.75,-3.464101615137755)--(10.6,-2.964101615137755)--cycle,black);
filldraw((14.0,-1.7320508075688776)--(13.6,-2.2320508075688776)--(13.75,-1.7320508075688776)--(13.6,-1.2320508075688776)--cycle,black);
filldraw((16.6,0)--(17,-0.5)--(16.85,0)--(17,0.5)--cycle,white);
filldraw((20.0,1.7320508075688772)--(19.6,1.2320508075688772)--(19.75,1.7320508075688772)--(19.6,2.232050807568877)--cycle,black);
filldraw((14.0,-5.196152422706632)--(13.6,-5.696152422706632)--(13.75,-5.196152422706632)--(13.6,-4.696152422706632)--cycle,black);
filldraw((17.0,-3.464101615137755)--(16.6,-3.964101615137755)--(16.75,-3.464101615137755)--(16.6,-2.964101615137755)--cycle,black);
filldraw((20.0,-1.7320508075688772)--(19.6,-2.232050807568877)--(19.75,-1.7320508075688772)--(19.6,-1.2320508075688772)--cycle,black);
filldraw((2.0,-1.7320508075688772)--(1.6,-1.2320508075688772)--(1.75,-1.7320508075688772)--(1.6,-2.232050807568877)--cycle,black);
filldraw((5.0,-3.4641016)--(4.6,-2.964101)--(4.75,-3.4641)--(4.6,-3.9641016)--cycle,black);
filldraw((8.0,-5.1961524)--(7.6,-4.6961524)--(7.75,-5.19615242)--(7.6,-5.696152422)--cycle,black);
filldraw((11.0,-6.9282032)--(10.6,-6.4282032)--(10.75,-6.928203)--(10.6,-7.428203)--cycle,black);
\end{asy}
\end{center}


$\textbf{(A)}\ 2112\qquad\textbf{(B)}\ 2304\qquad\textbf{(C)}\ 2368\qquad\textbf{(D)}\ 2384\qquad\textbf{(E)}\ 2400$\par \vspace{0.5em}\item Consider all polynomials of a complex variable, $P(z)=4z^4+az^3+bz^2+cz+d$, where $a,b,c,$ and $d$ are integers, $0\le d\le c\le b\le a\le 4$, and the polynomial has a zero $z_0$ with $|z_0|=1.$ What is the sum of all values $P(1)$ over all the polynomials with these properties?

$\textbf{(A)}\ 84\qquad\textbf{(B)}\ 92\qquad\textbf{(C)}\ 100\qquad\textbf{(D)}\ 108\qquad\textbf{(E)}\ 120 $\par \vspace{0.5em}\item Define the function $f_1$ on the positive integers by setting $f_1(1)=1$ and if $n=p_1^{e_1}p_2^{e_2}\cdots p_k^{e_k}$ is the prime factorization of $n>1$, then 
\begin\{equation*\}
f\_1(n)=(p\_1+1)\^\{e\_1-1\}(p\_2+1)\^\{e\_2-1\}\cdots (p\_k+1)\^\{e\_k-1\}.
\end\{equation*\}

For every $m\ge 2$, let $f_m(n)=f_1(f_{m-1}(n))$. For how many $N$s in the range $1\le N\le 400$ is the sequence $(f_1(N),f_2(N),f_3(N),\dots )$ unbounded?

'''Note:''' A sequence of positive numbers is unbounded if for every integer $B$, there is a member of the sequence greater than $B$.

$\textbf{(A)}\ 15\qquad\textbf{(B)}\ 16\qquad\textbf{(C)}\ 17\qquad\textbf{(D)}\ 18\qquad\textbf{(E)}\ 19 $\par \vspace{0.5em}\item Let $S=\{(x,y) : x\in \{0,1,2,3,4\}, y\in \{0,1,2,3,4,5\},\text{ and } (x,y)\ne (0,0)\}$. 
Let $T$ be the set of all right triangles whose vertices are in $S$. For every right triangle $t=\triangle{ABC}$ with vertices $A$, $B$, and $C$ in counter-clockwise order and right angle at $A$, let $f(t)=\tan(\angle{CBA})$. What is 
\begin\{equation*\}
\prod\_\{t\in T\} f(t)?
\end\{equation*\}


$\textbf{(A)}\ 1\qquad\textbf{(B)}\ \frac{625}{144}\qquad\textbf{(C)}\ \frac{125}{24}\qquad\textbf{(D)}\ 6\qquad\textbf{(E)}\ \frac{625}{24} $\par \vspace{0.5em}\end{enumerate}\newpage\section*{2013 AMC1212A}\begin{enumerate}[label=\arabic*., itemsep=0.5em]\item Square $ ABCD $ has side length $ 10 $. Point $ E $ is on $ \overline{BC} $, and the area of $ \bigtriangleup ABE $ is $ 40 $. What is $ BE $?

\begin{center}
\begin{asy}
import olympiad;
import cse5;
pair A,B,C,D,E;
A=(0,0);
B=(0,50);
C=(50,50);
D=(50,0);
E = (40,50);
   draw(A--B);
   draw(B--E);
   draw(E--C);
draw(C--D);
draw(D--A);
draw(A--E);
dot(A);
dot(B);
dot(C);
dot(D);
dot(E);
label("A",A,SW);
label("B",B,NW);
label("C",C,NE);
label("D",D,SE);
label("E",E,N);
\end{asy}
\end{center}

$\textbf{(A)} \ 4 \qquad \textbf{(B)} \ 5 \qquad \textbf{(C)} \ 6 \qquad \textbf{(D)} \ 7 \qquad \textbf{(E)} \ 8 \qquad $\par \vspace{0.5em}\item A softball team played ten games, scoring $1,2,3,4,5,6,7,8,9$, and $10$ runs. They lost by one run in exactly five games. In each of the other games, they scored twice as many runs as their opponent. How many total runs did their opponents score? 

$ \textbf {(A) } 35 \qquad \textbf {(B) } 40 \qquad \textbf {(C) } 45 \qquad \textbf {(D) } 50 \qquad \textbf {(E) } 55 $\par \vspace{0.5em}\item A flower bouquet contains pink roses, red roses, pink carnations, and red carnations. One third of the pink flowers are roses, three fourths of the red flowers are carnations, and six tenths of the flowers are pink. What percent of the flowers are carnations?

$ \textbf{(A)}\ 15\qquad\textbf{(B)}\ 30\qquad\textbf{(C)}\ 40\qquad\textbf{(D)}\ 60\qquad\textbf{(E)}\ 70 $\par \vspace{0.5em}\item What is the value of 
\begin\{equation*\}
\frac\{2\^\{2014\}+2\^\{2012\}\}\{2\^\{2014\}-2\^\{2012\}\}?
\end\{equation*\}


$ \textbf{(A)}\ -1\qquad\textbf{(B)}\ 1\qquad\textbf{(C)}\ \frac{5}{3}\qquad\textbf{(D)}\ 2013\qquad\textbf{(E)}\ 2^{4024} $\par \vspace{0.5em}\item Tom, Dorothy, and Sammy went on a vacation and agreed to split the costs evenly. During their trip Tom paid $$105$, Dorothy paid $$125$, and Sammy paid $$175$. In order to share the costs equally, Tom gave Sammy $t$ dollars, and Dorothy gave Sammy $d$ dollars. What is $t-d\$?

$ \textbf{(A)}\ 15\qquad\textbf{(B)}\ 20\qquad\textbf{(C)}\ 25\qquad\textbf{(D)}\ 30\qquad\textbf{(E)}\ 35 $\par \vspace{0.5em}\item In a recent basketball game, Shenille attempted only three-point shots and two-point shots. She was successful on $20\%$ of her three-point shots and $30\%$ of her two-point shots. Shenille attempted $30$ shots. How many points did she score?

$ \textbf{(A)}\ 12\qquad\textbf{(B)}\ 18\qquad\textbf{(C)}\ 24\qquad\textbf{(D)}\ 30\qquad\textbf{(E)}\ 36 $\par \vspace{0.5em}\item The sequence $S_1, S_2, S_3, \cdots, S_{10}$ has the property that every term beginning with the third is the sum of the previous two.  That is, 
\begin\{equation*\}
S\_n = S\_\{n-2\} + S\_\{n-1\} \text\{ for \} n \ge 3.
\end\{equation*\}
 Suppose that $S_9 = 110$ and $S_7 = 42$.  What is $S_4$?

$ \textbf{(A)}\ 4\qquad\textbf{(B)}\ 6\qquad\textbf{(C)}\ 10\qquad\textbf{(D)}\ 12\qquad\textbf{(E)}\ 16\qquad $\par \vspace{0.5em}\item Given that $x$ and $y$ are distinct nonzero real numbers such that $x+\tfrac{2}{x} = y + \tfrac{2}{y}$, what is $xy$?

$ \textbf{(A)}\ \frac{1}{4}\qquad\textbf{(B)}\ \frac{1}{2}\qquad\textbf{(C)}\ 1\qquad\textbf{(D)}\ 2\qquad\textbf{(E)}\ 4\qquad $\par \vspace{0.5em}\item In $\triangle ABC$, $AB=AC=28$ and $BC=20$.  Points $D,E,$ and $F$ are on sides $\overline{AB}$, $\overline{BC}$, and $\overline{AC}$, respectively, such that $\overline{DE}$ and $\overline{EF}$ are parallel to $\overline{AC}$ and $\overline{AB}$, respectively.  What is the perimeter of parallelogram $ADEF$?


\begin{center}
\begin{asy}
import olympiad;
import cse5;
size(180);
pen dps = linewidth(0.7) + fontsize(10); defaultpen(dps);
real r=5/7;
pair A=(10,sqrt(28\^2-100)),B=origin,C=(20,0),D=(A.x*r,A.y*r);
pair bottom=(C.x+(D.x-A.x),C.y+(D.y-A.y));
pair E=extension(D,bottom,B,C);
pair top=(E.x+D.x,E.y+D.y);
pair F=extension(E,top,A,C);
draw(A--B--C--cycle\^\^D--E--F);
dot(A\^\^B\^\^C\^\^D\^\^E\^\^F);
label("$A$",A,NW);
label("$B$",B,SW);
label("$C$",C,SE);
label("$D$",D,W);
label("$E$",E,S);
label("$F$",F,dir(0));
\end{asy}
\end{center}


\$\textbf\{(A) \}48\qquad
\textbf\{(B) \}52\qquad
\textbf\{(C) \}56\qquad
\textbf\{(D) \}60\qquad
\textbf\{(E) \}72\qquad\$\par \vspace{0.5em}\item Let $S$ be the set of positive integers $n$ for which $\tfrac{1}{n}$ has the repeating decimal representation $0.\overline{ab} = 0.ababab\cdots,$ with $a$ and $b$ different digits.  What is the sum of the elements of $S$?

$ \textbf{(A)}\ 11\qquad\textbf{(B)}\ 44\qquad\textbf{(C)}\ 110\qquad\textbf{(D)}\ 143\qquad\textbf{(E)}\ 155\qquad $\par \vspace{0.5em}\item Triangle $ABC$ is equilateral with $AB=1$. Points $E$ and $G$ are on $\overline{AC}$ and points $D$ and $F$ are on $\overline{AB}$ such that both $\overline{DE}$ and $\overline{FG}$ are parallel to $\overline{BC}$. Furthermore, triangle $ADE$ and trapezoids $DFGE$ and $FBCG$ all have the same perimeter. What is $DE+FG$?


\begin{center}
\begin{asy}
import olympiad;
import cse5;
size(180);
pen dps = linewidth(0.7) + fontsize(10); defaultpen(dps);
real s=1/2,m=5/6,l=1;
pair A=origin,B=(l,0),C=rotate(60)*l,D=(s,0),E=rotate(60)*s,F=m,G=rotate(60)*m;
draw(A--B--C--cycle\^\^D--E\^\^F--G);
dot(A\^\^B\^\^C\^\^D\^\^E\^\^F\^\^G);
label("$A$",A,SW);
label("$B$",B,SE);
label("$C$",C,N);
label("$D$",D,S);
label("$E$",E,NW);
label("$F$",F,S);
label("$G$",G,NW);
\end{asy}
\end{center}


\$\textbf\{(A) \}1\qquad
\textbf\{(B) \}\dfrac\{3\}\{2\}\qquad
\textbf\{(C) \}\dfrac\{21\}\{13\}\qquad
\textbf\{(D) \}\dfrac\{13\}\{8\}\qquad
\textbf\{(E) \}\dfrac\{5\}\{3\}\qquad\$\par \vspace{0.5em}\item The angles in a particular triangle are in arithmetic progression, and the side lengths are $4,5,x$. The sum of the possible values of $x$ equals $a+\sqrt{b}+\sqrt{c}$ where $a, b$, and $c$ are positive integers. What is $a+b+c$?

$ \textbf{(A)}\ 36\qquad\textbf{(B)}\ 38\qquad\textbf{(C)}\ 40\qquad\textbf{(D)}\ 42\qquad\textbf{(E)}\ 44$\par \vspace{0.5em}\item Let points $ A = (0,0) , \ B = (1,2), \ C = (3,3), $ and $ D = (4,0) $. Quadrilateral $ ABCD $ is cut into equal area pieces by a line passing through $ A $. This line intersects $ \overline{CD} $ at point $ \left (\frac{p}{q}, \frac{r}{s} \right ) $, where these fractions are in lowest terms. What is $ p + q + r + s $?

$ \textbf{(A)} \ 54 \qquad \textbf{(B)} \ 58 \qquad  \textbf{(C)} \ 62 \qquad \textbf{(D)} \ 70 \qquad \textbf{(E)} \ 75 $\par \vspace{0.5em}\item The sequence

$\log_{12}{162}$, $\log_{12}{x}$, $\log_{12}{y}$, $\log_{12}{z}$, $\log_{12}{1250}$

is an arithmetic progression. What is $x$?

$ \textbf{(A)} \ 125\sqrt{3} \qquad \textbf{(B)} \ 270 \qquad \textbf{(C)} \ 162\sqrt{5} \qquad \textbf{(D)} \ 434 \qquad \textbf{(E)} \ 225\sqrt{6}$\par \vspace{0.5em}\item Rabbits Peter and Pauline have three offspringFlopsie, Mopsie, and Cotton-tail. These five rabbits are to be distributed to four different pet stores so that no store gets both a parent and a child. It is not required that every store gets a rabbit. In how many different ways can this be done?

$\textbf{(A)} \ 96 \qquad  \textbf{(B)} \ 108 \qquad  \textbf{(C)} \ 156 \qquad  \textbf{(D)} \ 204 \qquad  \textbf{(E)} \ 372 $\par \vspace{0.5em}\item $A$, $B$, $C$ are three piles of rocks. The mean weight of the rocks in $A$ is $40$ pounds, the mean weight of the rocks in $B$ is $50$ pounds, the mean weight of the rocks in the combined piles $A$ and $B$ is $43$ pounds, and the mean weight of the rocks in the combined piles $A$ and $C$ is $44$ pounds. What is the greatest possible integer value for the mean in pounds of the rocks in the combined piles $B$ and $C$?

$ \textbf{(A)} \ 55 \qquad \textbf{(B)} \ 56 \qquad \textbf{(C)} \ 57 \qquad \textbf{(D)} \ 58 \qquad \textbf{(E)} \ 59$\par \vspace{0.5em}\item A group of $ 12 $ pirates agree to divide a treasure chest of gold coins among themselves as follows. The $ k^\text{th} $ pirate to take a share takes $ \frac{k}{12} $ of the coins that remain in the chest. The number of coins initially in the chest is the smallest number for which this arrangement will allow each pirate to receive a positive whole number of coins. How many coins does the $ 12^{\text{th}} $ pirate receive?

$ \textbf{(A)} \ 720 \qquad  \textbf{(B)} \ 1296 \qquad  \textbf{(C)} \ 1728 \qquad  \textbf{(D)} \ 1925 \qquad  \textbf{(E)} \ 3850 $\par \vspace{0.5em}\item Six spheres of radius $1$ are positioned so that their centers are at the vertices of a regular hexagon of side length $2$. The six spheres are internally tangent to a larger sphere whose center is the center of the hexagon. An eighth sphere is externally tangent to the six smaller spheres and internally tangent to the larger sphere. What is the radius of this eighth sphere?

$ \textbf{(A)} \ \sqrt{2} \qquad \textbf{(B)} \ \frac{3}{2} \qquad \textbf{(C)} \ \frac{5}{3} \qquad \textbf{(D)} \ \sqrt{3} \qquad \textbf{(E)} \ 2$\par \vspace{0.5em}\item In $ \bigtriangleup ABC $, $ AB = 86 $, and $ AC = 97 $. A circle with center $ A $ and radius $ AB $ intersects $ \overline{BC} $ at points $ B $ and $ X $. Moreover $ \overline{BX} $ and $ \overline{CX} $ have integer lengths. What is $ BC $?

$ \textbf{(A)} \ 11 \qquad  \textbf{(B)} \ 28 \qquad  \textbf{(C)} \ 33 \qquad  \textbf{(D)} \ 61 \qquad  \textbf{(E)} \ 72 $\par \vspace{0.5em}\item Let $S$ be the set $\{1,2,3,...,19\}$. For $a,b \in S$, define $a \succ b$ to mean that either $0 < a - b \le 9$ or $b - a > 9$. How many ordered triples $(x,y,z)$ of elements of $S$ have the property that $x \succ y$, $y \succ z$, and $z \succ x$?

$ \textbf{(A)} \ 810 \qquad  \textbf{(B)} \ 855 \qquad  \textbf{(C)} \ 900 \qquad  \textbf{(D)} \ 950 \qquad  \textbf{(E)} \ 988 $\par \vspace{0.5em}\item Consider $ A = \log (2013 + \log (2012 + \log (2011 + \log (\cdots + \log (3 + \log 2) \cdots )))) $. Which of the following intervals contains $ A $?

$ \textbf{(A)} \ (\log 2016, \log 2017) $
$ \textbf{(B)} \ (\log 2017, \log 2018) $
$ \textbf{(C)} \ (\log 2018, \log 2019) $
$ \textbf{(D)} \ (\log 2019, \log 2020) $
$ \textbf{(E)} \ (\log 2020, \log 2021) $\par \vspace{0.5em}\item A palindrome is a nonnegative integer number that reads the same forwards and backwards when written in base 10 with no leading zeros. A 6-digit palindrome $n$ is chosen uniformly at random. What is the probability that $\frac{n}{11}$ is also a palindrome?

$ \textbf{(A)} \ \frac{8}{25} \qquad \textbf{(B)} \ \frac{33}{100} \qquad \textbf{(C)} \ \frac{7}{20} \qquad \textbf{(D)} \ \frac{9}{25} \qquad \textbf{(E)} \ \frac{11}{30}$\par \vspace{0.5em}\item $ ABCD$ is a square of side length $ \sqrt{3} + 1 $. Point $ P $ is on $ \overline{AC} $ such that $ AP = \sqrt{2} $. The square region bounded by $ ABCD $ is rotated $ 90^{\circ} $ counterclockwise with center $ P $, sweeping out a region whose area is $ \frac{1}{c} (a \pi + b) $, where $a $, $b$, and $ c $ are positive integers and $ \text{gcd}(a,b,c) = 1 $. What is $ a + b + c $?

$\textbf{(A)} \ 15 \qquad \textbf{(B)} \ 17 \qquad \textbf{(C)} \ 19 \qquad \textbf{(D)} \ 21 \qquad \textbf{(E)} \ 23 $\par \vspace{0.5em}\item Three distinct segments are chosen at random among the segments whose end-points are the vertices of a regular $12$-gon. What is the probability that the lengths of these three segments are the three side lengths of a triangle with positive area?

$ \textbf{(A)} \ \frac{553}{715} \qquad \textbf{(B)} \ \frac{443}{572} \qquad \textbf{(C)} \ \frac{111}{143} \qquad \textbf{(D)} \ \frac{81}{104} \qquad \textbf{(E)} \ \frac{223}{286}$\par \vspace{0.5em}\item Let $f : \mathbb{C} \to \mathbb{C} $ be defined by $ f(z) = z^2 + iz + 1 $. How many complex numbers $z $ are there such that $ \text{Im}(z) > 0 $ and both the real and the imaginary parts of $f(z)$ are integers with absolute value at most $ 10 $?

$ \textbf{(A)} \ 399 \qquad \textbf{(B)} \ 401 \qquad \textbf{(C)} \ 413 \qquad \textbf{(D)} \ 431 \qquad \textbf{(E)} \ 441 $\par \vspace{0.5em}\end{enumerate}\newpage\section*{2013 AMC1212B}\begin{enumerate}[label=\arabic*., itemsep=0.5em]\item On a particular January day, the high temperature in Lincoln, Nebraska, was $16$ degrees higher than the low temperature, and the average of the high and low temperatures was $3$. In degrees, what was the low temperature in Lincoln that day?

$\textbf{(A)}\ -13 \qquad \textbf{(B)}\ -8 \qquad \textbf{(C)}\ -5 \qquad \textbf{(D)}\ -3 \qquad \textbf{(E)}\ 11$\par \vspace{0.5em}\item Mr. Green measures his rectangular garden by walking two of the sides and finds that it is $15$ steps by $20$ steps. Each of Mr. Greens steps is $2$ feet long. Mr. Green expects a half a pound of potatoes per square foot from his garden. How many pounds of potatoes does Mr. Green expect from his garden?

$\textbf{(A)}\ 600 \qquad \textbf{(B)}\ 800 \qquad \textbf{(C)}\ 1000 \qquad \textbf{(D)}\ 1200 \qquad \textbf{(E)}\ 1400$\par \vspace{0.5em}\item When counting from $3$ to $201$, $53$ is the $51^{\text{st}}$ number counted. When counting backwards from $201$ to $3$, $53$ is the $n^{\text{th}}$ number counted. What is $n$?

$\textbf{(A)}\ 146 \qquad \textbf{(B)}\ 147 \qquad \textbf{(C)}\ 148 \qquad \textbf{(D)}\ 149 \qquad \textbf{(E)}\ 150$\par \vspace{0.5em}\item Ray's car averages $40$ miles per gallon of gasoline, and Tom's car averages $10$ miles per gallon of gasoline. Ray and Tom each drive the same number of miles. What is the cars' combined rate of miles per gallon of gasoline?<br \>

$\textbf{(A)}\ 10 \qquad \textbf{(B)}\ 16 \qquad \textbf{(C)}\ 25 \qquad \textbf{(D)}\ 30 \qquad \textbf{(E)}\ 40$\par \vspace{0.5em}\item The average age of $33$ fifth-graders is $11$. The average age of $55$ of their parents is $33$. What is the average age of all of these parents and fifth-graders?

$\textbf{(A)}\ 22 \qquad \textbf{(B)}\ 23.25 \qquad \textbf{(C)}\ 24.75 \qquad \textbf{(D)}\ 26.25 \qquad \textbf{(E)}\ 28$\par \vspace{0.5em}\item Real numbers $x$ and $y$ satisfy the equation $x^2 + y^2 = 10x - 6y - 34$. What is $x + y$?

$\textbf{(A)}\ 1 \qquad \textbf{(B)}\ 2 \qquad \textbf{(C)}\ 3 \qquad \textbf{(D)}\ 6 \qquad \textbf{(E)}\ 8$\par \vspace{0.5em}\item Jo and Blair take turns counting from $1$ to one more than the last number said by the other person. Jo starts by saying $``1"$, so Blair follows by saying $``1, 2"$. Jo then says $``1, 2, 3"$, and so on. What is the $53^{\text{rd}}$ number said?

$\textbf{(A)}\ 2 \qquad \textbf{(B)}\ 3 \qquad \textbf{(C)}\ 5 \qquad \textbf{(D)}\ 6 \qquad \textbf{(E)}\ 8$\par \vspace{0.5em}\item Line $l_1$ has equation $3x - 2y = 1$ and goes through $A = (-1, -2)$. Line $l_2$ has equation $y = 1$ and meets line $l_1$ at point $B$. Line $l_3$ has positive slope, goes through point $A$, and meets $l_2$ at point $C$. The area of $\triangle ABC$ is $3$. What is the slope of $l_3$?

$\textbf{(A)}\ \frac{2}{3} \qquad \textbf{(B)}\ \frac{3}{4} \qquad \textbf{(C)}\ 1 \qquad \textbf{(D)}\ \frac{4}{3} \qquad \textbf{(E)}\ \frac{3}{2}$\par \vspace{0.5em}\item What is the sum of the exponents of the prime factors of the square root of the largest perfect square that divides $12!$ ?

$\textbf{(A)}\ 5 \qquad \textbf{(B)}\ 7 \qquad \textbf{(C)}\ 8 \qquad \textbf{(D)}\ 10 \qquad \textbf{(E)}\ 12 $\par \vspace{0.5em}\item Alex has $75$ red tokens and $75$ blue tokens. There is a booth where Alex can give two red tokens and receive in return a silver token and a blue token, and another booth where Alex can give three blue tokens and receive in return a silver token and a red token. Alex continues to exchange tokens until no more exchanges are possible. How many silver tokens will Alex have at the end?

$\textbf{(A)}\ 62 \qquad \textbf{(B)}\ 82 \qquad \textbf{(C)}\ 83 \qquad \textbf{(D)}\ 102 \qquad \textbf{(E)}\ 103$\par \vspace{0.5em}\item Two bees start at the same spot and fly at the same rate in the following directions. Bee $A$ travels $1$ foot north, then $1$ foot east, then $1$ foot upwards, and then continues to repeat this pattern. Bee $B$ travels $1$ foot south, then $1$ foot west, and then continues to repeat this pattern. In what directions are the bees traveling when they are exactly $10$ feet away from each other?

$\textbf{(A)}\ A$ east, $B$ west<br \>$\textbf{(B)}\ A$ north, $B$ south<br \>$\textbf{(C)}\ A$ north, $B$ west<br \>$\textbf{(D)}\ A$ up, $B$ south<br \>$\textbf{(E)}\ A$ up, $B$ west<br \>\par \vspace{0.5em}\item Cities $A$, $B$, $C$, $D$, and $E$ are connected by roads $\widetilde{AB}$, $\widetilde{AD}$, $\widetilde{AE}$, $\widetilde{BC}$, $\widetilde{BD}$, $\widetilde{CD}$, and $\widetilde{DE}$. How many different routes are there from $A$ to $B$ that use each road exactly once? (Such a route will necessarily visit some cities more than once.)

\begin{center}
\begin{asy}
import olympiad;
import cse5;
unitsize(10mm);
defaultpen(linewidth(1.2pt)+fontsize(10pt));
dotfactor=4;
pair A=(1,0), B=(4.24,0), C=(5.24,3.08), D=(2.62,4.98), E=(0,3.08);
dot (A);
dot (B);
dot (C);
dot (D);
dot (E);
label("$A$",A,S);
label("$B$",B,SE);
label("$C$",C,E);
label("$D$",D,N);
label("$E$",E,W);
guide squiggly(path g, real stepsize, real slope=45)
\{
 real len = arclength(g);
 real step = len / round(len / stepsize);
 guide squig;
 for (real u = 0; u < len; u += step)\{
 real a = arctime(g, u);
 real b = arctime(g, u + step / 2);
 pair p = point(g, a);
 pair q = point(g, b);
 pair np = unit( rotate(slope) * dir(g,a));
 pair nq = unit( rotate(0 - slope) * dir(g,b));
 squig = squig .. p\{np\} .. q\{nq\};
 \}
 squig = squig .. point(g, length(g))\{unit(rotate(slope)*dir(g,length(g)))\};
 return squig;
\}
pen pp = defaultpen + 2.718;
draw(squiggly(A--B, 4.04, 30), pp);
draw(squiggly(A--D, 7.777, 20), pp);
draw(squiggly(A--E, 5.050, 15), pp);
draw(squiggly(B--C, 5.050, 15), pp);
draw(squiggly(B--D, 4.04, 20), pp);
draw(squiggly(C--D, 2.718, 20), pp);
draw(squiggly(D--E, 2.718, -60), pp);
\end{asy}
\end{center}


$\textbf{(A)}\ 7 \qquad \textbf{(B)}\ 9 \qquad \textbf{(C)}\ 12 \qquad \textbf{(D)}\ 16 \qquad \textbf{(E)}\ 18$\par \vspace{0.5em}\item The internal angles of quadrilateral $ABCD$ form an arithmetic progression. Triangles $ABD$ and $DCB$ are similar with $\angle DBA = \angle DCB$ and $\angle ADB = \angle CBD$. Moreover, the angles in each of these two triangles also form an arithmetic progression. In degrees, what is the largest possible sum of the two largest angles of $ABCD$?

$\textbf{(A)}\ 210 \qquad \textbf{(B)}\ 220 \qquad \textbf{(C)}\ 230 \qquad \textbf{(D)}\ 240 \qquad \textbf{(E)}\ 250$\par \vspace{0.5em}\item Two non-decreasing sequences of nonnegative integers have different first terms. Each sequence has the property that each term beginning with the third is the sum of the previous two terms, and the seventh term of each sequence is $N$. What is the smallest possible value of $N$ ?

$\textbf{(A)}\ 55 \qquad \textbf{(B)}\ 89 \qquad \textbf{(C)}\ 104 \qquad \textbf{(D)}\ 144 \qquad \textbf{(E)}\ 273$\par \vspace{0.5em}\item The number $2013$ is expressed in the form <br \> <center> $2013 = \frac {a_1!a_2!...a_m!}{b_1!b_2!...b_n!}$,</center><br />where $a_1 \ge a_2 \ge ... \ge a_m$ and $b_1 \ge b_2 \ge ... \ge b_n$ are positive integers and $a_1 + b_1$ is as small as possible. What is $|a_1 - b_1|$?
$\textbf{(A)}\ 1 \qquad \textbf{(B)}\ 2 \qquad \textbf{(C)}\ 3 \qquad \textbf{(D)}\ 4 \qquad \textbf{(E)}\ 5$\par \vspace{0.5em}\item Let $ABCDE$ be an equiangular convex pentagon of perimeter $1$. The pairwise intersections of the lines that extend the sides of the pentagon determine a five-pointed star polygon. Let $s$ be the perimeter of this star. What is the difference between the maximum and the minimum possible values of $s$?

$\textbf{(A)}\ 0 \qquad \textbf{(B)}\ \frac{1}{2} \qquad \textbf{(C)}\ \frac{\sqrt{5}-1}{2} \qquad \textbf{(D)}\  \frac{\sqrt{5}+1}{2} \qquad \textbf{(E)}\ \sqrt{5}$\par \vspace{0.5em}\item Let $a,b,$ and $c$ be real numbers such that 


\begin\{equation*\}
a+b+c=2, \text\{ and\}
\end\{equation*\}


\begin\{equation*\}
a\^2+b\^2+c\^2=12
\end\{equation*\}


What is the difference between the maximum and minimum possible values of $c$?

$ \textbf{(A) }2\qquad \textbf{ (B) }\frac{10}{3}\qquad \textbf{ (C) }4 \qquad \textbf{ (D) }\frac{16}{3}\qquad \textbf{ (E) }\frac{20}{3} $\par \vspace{0.5em}\item Barbara and Jenna play the following game, in which they take turns. A number of coins lie on a table. When it is Barbaras turn, she must remove $2$ or $4$ coins, unless only one coin remains, in which case she loses her turn. When it is Jennas turn, she must remove $1$ or $3$ coins. A coin flip determines who goes first. Whoever removes the last coin wins the game. Assume both players use their best strategy. Who will win when the game starts with $2013$ coins and when the game starts with $2014$ coins?

$ \textbf{(A)}$ Barbara will win with $2013$ coins and Jenna will win with $2014$ coins. 

$\textbf{(B)}$ Jenna will win with $2013$ coins, and whoever goes first will win with $2014$ coins. 

$\textbf{(C)}$ Barbara will win with $2013$ coins, and whoever goes second will win with $2014$ coins.

$\textbf{(D)}$ Jenna will win with $2013$ coins, and Barbara will win with $2014$ coins.

$\textbf{(E)}$ Whoever goes first will win with $2013$ coins, and whoever goes second will win with $2014$ coins.\par \vspace{0.5em}\item In triangle $ABC$, $AB=13$, $BC=14$, and $CA=15$. Distinct points $D$, $E$, and $F$ lie on segments $\overline{BC}$, $\overline{CA}$, and $\overline{DE}$, respectively, such that $\overline{AD}\perp\overline{BC}$, $\overline{DE}\perp\overline{AC}$, and $\overline{AF}\perp\overline{BF}$. The length of segment $\overline{DF}$ can be written as $\frac{m}{n}$, where $m$ and $n$ are relatively prime positive integers. What is $m+n$?

$ \textbf{(A)}\ 18\qquad\textbf{(B)}\ 21\qquad\textbf{(C)}\ 24\qquad\textbf{(D)}\ 27\qquad\textbf{(E)}\ 30 $\par \vspace{0.5em}\item For $135^\circ < x < 180^\circ$, points $P=(\cos x, \cos^2 x), Q=(\cot x, \cot^2 x), R=(\sin x, \sin^2 x)$ and $S =(\tan x, \tan^2 x)$ are the vertices of a trapezoid. What is $\sin(2x)$?

$ \textbf{(A)}\ 2-2\sqrt{2}\qquad\textbf{(B)} 3\sqrt{3}-6\qquad\textbf{(C)}\ 3\sqrt{2}-5\qquad\textbf{(D)}\ -\frac{3}{4}\qquad\textbf{(E)}\ 1-\sqrt{3}$\par \vspace{0.5em}\item Consider the set of $30$ parabolas defined as follows: all parabolas have as focus the point $(0,0)$ and the directrix lines have the form $y=ax+b$ with $a$ and $b$ integers such that $a\in \{-2,-1,0,1,2\}$ and $b\in \{-3,-2,-1,1,2,3\}$. No three of these parabolas have a common point. How many points in the plane are on two of these parabolas?

$ \textbf{(A)}\ 720\qquad\textbf{(B)}\ 760\qquad\textbf{(C)}\ 810\qquad\textbf{(D)}\ 840\qquad\textbf{(E)}\ 870 $\par \vspace{0.5em}\item Let $m>1$ and $n>1$ be integers. Suppose that the product of the solutions for $x$ of the equation

\begin\{equation*\}
8(\log\_n x)(\log\_m x)-7\log\_n x-6 \log\_m x-2013 = 0
\end\{equation*\}

is the smallest possible integer. What is $m+n$?

$ \textbf{(A)}\ 12\qquad\textbf{(B)}\ 20\qquad\textbf{(C)}\ 24\qquad\textbf{(D)}\ 48\qquad\textbf{(E)}\ 272 $\par \vspace{0.5em}\item Bernardo chooses a three-digit positive integer $N$ and writes both its base-$5$ and base-$6$ representations on a blackboard. Later LeRoy sees the two numbers Bernardo has written. Treating the two numbers as base-$10$ integers, he adds them to obtain an integer $S$. For example, if $N=749$, Bernardo writes the numbers $10444$ and $3245$, and LeRoy obtains the sum $S=13,689$. For how many choices of $N$ are the two rightmost digits of $S$, in order, the same as those of $2N$?

$ \textbf{(A)}\ 5\qquad\textbf{(B)}\ 10\qquad\textbf{(C)}\ 15\qquad\textbf{(D)}\ 20\qquad\textbf{(E)}\ 25 $\par \vspace{0.5em}\item Let $ABC$ be a triangle where $M$ is the midpoint of $\overline{AC}$, and $\overline{CN}$ is the angle bisector of $\angle{ACB}$ with $N$ on $\overline{AB}$. Let $X$ be the intersection of the median $\overline{BM}$ and the bisector $\overline{CN}$. In addition $\triangle BXN$ is equilateral with $AC=2$. What is $BX^2$?

$\textbf{(A)}\  \frac{10-6\sqrt{2}}{7} \qquad \textbf{(B)}\ \frac{2}{9} \qquad \textbf{(C)}\ \frac{5\sqrt{2}-3\sqrt{3}}{8} \qquad \textbf{(D)}\ \frac{\sqrt{2}}{6} \qquad \textbf{(E)}\ \frac{3\sqrt{3}-4}{5}$\par \vspace{0.5em}\item Let $G$ be the set of polynomials of the form

\begin\{equation*\}
P(z)=z\^n+c\_\{n-1\}z\^\{n-1\}+\cdots+c\_2z\^2+c\_1z+50,
\end\{equation*\}

where $ c_1,c_2,\cdots, c_{n-1} $ are integers and $P(z)$ has distinct roots of the form $a+ib$ with $a$ and $b$ integers. How many polynomials are in $G$?

$ \textbf{(A)}\ 288\qquad\textbf{(B)}\ 528\qquad\textbf{(C)}\ 576\qquad\textbf{(D)}\ 992\qquad\textbf{(E)}\ 1056 $\par \vspace{0.5em}\end{enumerate}\newpage\section*{2014 AMC1212A}\begin{enumerate}[label=\arabic*., itemsep=0.5em]\item What is $10 \cdot \left(\tfrac{1}{2} + \tfrac{1}{5} + \tfrac{1}{10}\right)^{-1}?$

$ \textbf{(A)}\ 3\qquad\textbf{(B)}\ 8\qquad\textbf{(C)}\ \frac{25}{2}\qquad\textbf{(D)}\ \frac{170}{3}\qquad\textbf{(E)}\ 170$\par \vspace{0.5em}\item At the theater children get in for half price.  The price for $5$ adult tickets and $4$ child tickets is $\$24.50$.  How much would $8$ adult tickets and $6\$ child tickets cost?

$\textbf{(A) }\$35\qquad
\textbf\{(B) \}\\$38.50\qquad
\textbf\{(C) \}\\$40\qquad
\textbf\{(D) \}\\$42\qquad
\textbf\{(E) \}\$42.50$\par \vspace{0.5em}\item Walking down Jane Street, Ralph passed four houses in a row, each painted a different color. He passed the orange house before the red house, and he passed the blue house before the yellow house. The blue house was not next to the yellow house. How many orderings of the colored houses are possible?

$ \textbf{(A)}\ 2\qquad\textbf{(B)}\ 3\qquad\textbf{(C)}\ 4\qquad\textbf{(D)}\ 5\qquad\textbf{(E)}\ 6$\par \vspace{0.5em}\item Suppose that $a$ cows give $b$ gallons of milk in $c$ days. At this rate, how many gallons of milk will $d$ cows give in $e$ days?

$ \textbf{(A)}\ \frac{bde}{ac}\qquad\textbf{(B)}\ \frac{ac}{bde}\qquad\textbf{(C)}\ \frac{abde}{c}\qquad\textbf{(D)}\ \frac{bcde}{a}\qquad\textbf{(E)}\ \frac{abc}{de}$\par \vspace{0.5em}\item On an algebra quiz, $10\%$ of the students scored $70$ points, $35\%$ scored $80$ points, $30\%$ scored $90$ points, and the rest scored $100$ points. What is the difference between the mean and median score of the students' scores on this quiz?

$ \textbf{(A)}\ 1\qquad\textbf{(B)}\ 2\qquad\textbf{(C)}\ 3\qquad\textbf{(D)}\ 4\qquad\textbf{(E)}\ 5$\par \vspace{0.5em}\item The difference between a two-digit number and the number obtained by reversing its digits is $5$ times the sum of the digits of either number.  What is the sum of the two digit number and its reverse?

\$\textbf\{(A) \}44\qquad
\textbf\{(B) \}55\qquad
\textbf\{(C) \}77\qquad
\textbf\{(D) \}99\qquad
\textbf\{(E) \}110\$\par \vspace{0.5em}\item The first three terms of a geometric progression are $\sqrt 3$, $\sqrt[3]3$, and $\sqrt[6]3$.  What is the fourth term?

\$\textbf\{(A) \}1\qquad
\textbf\{(B) \}\sqrt[7]3\qquad
\textbf\{(C) \}\sqrt[8]3\qquad
\textbf\{(D) \}\sqrt[9]3\qquad
\textbf\{(E) \}\sqrt[10]3\qquad\$\par \vspace{0.5em}\item A customer who intends to purchase an appliance has three coupons, only one of which may be used:

Coupon 1: $10\%$ off the listed price if the listed price is at least $\$50\$

Coupon 2: $\$20$ off the listed price if the listed price is at least $\$100$

Coupon 3: $18\%$ off the amount by which the listed price exceeds $\$100\$

For which of the following listed prices will coupon $1$ offer a greater price reduction than either coupon $2$ or coupon $3$?

$\textbf{(A) }\$179.95\qquad
\textbf\{(B) \}\\$199.95\qquad
\textbf\{(C) \}\\$219.95\qquad
\textbf\{(D) \}\\$239.95\qquad
\textbf\{(E) \}\$259.95\qquad$\par \vspace{0.5em}\item Five positive consecutive integers starting with $a$ have average $b$. What is the average of $5$ consecutive integers that start with $b$?

$ \textbf{(A)}\ a+3\qquad\textbf{(B)}\ a+4\qquad\textbf{(C)}\ a+5\qquad\textbf{(D)}\ a+6\qquad\textbf{(E)}\ a+7$\par \vspace{0.5em}\item Three congruent isosceles triangles are constructed with their bases on the sides of an equilateral triangle of side length $1$.  The sum of the areas of the three isosceles triangles is the same as the area of the equilateral triangle.  What is the length of one of the two congruent sides of one of the isosceles triangles?

\$\textbf\{(A) \}\dfrac\{\sqrt3\}4\qquad
\textbf\{(B) \}\dfrac\{\sqrt3\}3\qquad
\textbf\{(C) \}\dfrac23\qquad
\textbf\{(D) \}\dfrac\{\sqrt2\}2\qquad
\textbf\{(E) \}\dfrac\{\sqrt3\}2\$\par \vspace{0.5em}\item David drives from his home to the airport to catch a flight.  He drives $35$ miles in the first hour, but realizes that he will be $1$ hour late if he continues at this speed.  He increases his speed by $15$ miles per hour for the rest of the way to the airport and arrives $30$ minutes early.  How many miles is the airport from his home?

\$\textbf\{(A) \}140\qquad
\textbf\{(B) \}175\qquad
\textbf\{(C) \}210\qquad
\textbf\{(D) \}245\qquad
\textbf\{(E) \}280\qquad\$\par \vspace{0.5em}\item Two circles intersect at points $A$ and $B$.  The minor arcs $AB$ measure $30^\circ$ on one circle and $60^\circ$ on the other circle.  What is the ratio of the area of the larger circle to the area of the smaller circle?

\$\textbf\{(A) \}2\qquad
\textbf\{(B) \}1+\sqrt3\qquad
\textbf\{(C) \}3\qquad
\textbf\{(D) \}2+\sqrt3\qquad
\textbf\{(E) \}4\qquad\$\par \vspace{0.5em}\item A fancy bed and breakfast inn has $5$ rooms, each with a distinctive color-coded decor.  One day $5$ friends arrive to spend the night.  There are no other guests that night.  The friends can room in any combination they wish, but with no more than $2$ friends per room.  In how many ways can the innkeeper assign the guests to the rooms?

\$\textbf\{(A) \}2100\qquad
\textbf\{(B) \}2220\qquad
\textbf\{(C) \}3000\qquad
\textbf\{(D) \}3120\qquad
\textbf\{(E) \}3125\qquad\$\par \vspace{0.5em}\item Let $a<b<c$ be three integers such that $a,b,c$ is an arithmetic progression and $a,c,b$ is a geometric progression.  What is the smallest possible value of $c$?

\$\textbf\{(A) \}-2\qquad
\textbf\{(B) \}1\qquad
\textbf\{(C) \}2\qquad
\textbf\{(D) \}4\qquad
\textbf\{(E) \}6\qquad\$\par \vspace{0.5em}\item A five-digit palindrome is a positive integer with respective digits $abcba$, where $a$ is non-zero.  Let $S$ be the sum of all five-digit palindromes.  What is the sum of the digits of $S$?

\$\textbf\{(A) \}9\qquad
\textbf\{(B) \}18\qquad
\textbf\{(C) \}27\qquad
\textbf\{(D) \}36\qquad
\textbf\{(E) \}45\qquad\$\par \vspace{0.5em}\item The product $(8)(888\ldots 8)$, where the second factor has $k$ digits, is an integer whose digits have a sum of $1000$.  What is $k$?

\$\textbf\{(A) \}901\qquad
\textbf\{(B) \}911\qquad
\textbf\{(C) \}919\qquad
\textbf\{(D) \}991\qquad
\textbf\{(E) \}999\qquad\$\par \vspace{0.5em}\item A $4\times 4\times h$ rectangular box contains a sphere of radius $2$ and eight smaller spheres of radius $1$.  The smaller spheres are each tangent to three sides of the box, and the larger sphere is tangent to each of the smaller spheres.  What is $h$?

<center>
\begin{center}
\begin{asy}
import olympiad;
import cse5;
import graph3;
import solids;
real h=2+2*sqrt(7);
currentprojection=orthographic((0.75,-5,h/2+1),target=(2,2,h/2));
currentlight=light(4,-4,4);
draw((0,0,0)--(4,0,0)--(4,4,0)--(0,4,0)--(0,0,0)\^\^(4,0,0)--(4,0,h)--(4,4,h)--(0,4,h)--(0,4,0));
draw(shift((1,3,1))*unitsphere,gray(0.85));
draw(shift((3,3,1))*unitsphere,gray(0.85));
draw(shift((3,1,1))*unitsphere,gray(0.85));
draw(shift((1,1,1))*unitsphere,gray(0.85));
draw(shift((2,2,h/2))*scale(2,2,2)*unitsphere,gray(0.85));
draw(shift((1,3,h-1))*unitsphere,gray(0.85));
draw(shift((3,3,h-1))*unitsphere,gray(0.85));
draw(shift((3,1,h-1))*unitsphere,gray(0.85));
draw(shift((1,1,h-1))*unitsphere,gray(0.85));
draw((0,0,0)--(0,0,h)--(4,0,h)\^\^(0,0,h)--(0,4,h));
\end{asy}
\end{center}
</center>

\$\textbf\{(A) \}2+2\sqrt 7\qquad
\textbf\{(B) \}3+2\sqrt 5\qquad
\textbf\{(C) \}4+2\sqrt 7\qquad
\textbf\{(D) \}4\sqrt 5\qquad
\textbf\{(E) \}4\sqrt 7\qquad\$\par \vspace{0.5em}\item The domain of the function $f(x)=\log_{\frac12}(\log_4(\log_{\frac14}(\log_{16}(\log_{\frac1{16}}x))))$ is an interval of length $\tfrac mn$, where $m$ and $n$ are relatively prime positive integers.  What is $m+n$?

\$\textbf\{(A) \}19\qquad
\textbf\{(B) \}31\qquad
\textbf\{(C) \}271\qquad
\textbf\{(D) \}319\qquad
\textbf\{(E) \}511\qquad\$\par \vspace{0.5em}\item There are exactly $N$ distinct rational numbers $k$ such that $|k|<200$ and 
\begin\{equation*\}
5x\^2+kx+12=0
\end\{equation*\}
 has at least one integer solution for $x$.  What is $N$?

\$\textbf\{(A) \}6\qquad
\textbf\{(B) \}12\qquad
\textbf\{(C) \}24\qquad
\textbf\{(D) \}48\qquad
\textbf\{(E) \}78\qquad\$\par \vspace{0.5em}\item In $\triangle BAC$, $\angle BAC=40^\circ$, $AB=10$, and $AC=6$.  Points $D$ and $E$ lie on $\overline{AB}$ and $\overline{AC}$ respectively.  What is the minimum possible value of $BE+DE+CD$?

\$\textbf\{(A) \}6\sqrt 3+3\qquad
\textbf\{(B) \}\dfrac\{27\}2\qquad
\textbf\{(C) \}8\sqrt 3\qquad
\textbf\{(D) \}14\qquad
\textbf\{(E) \}3\sqrt 3+9\qquad\$\par \vspace{0.5em}\item For every real number $x$, let $\lfloor x\rfloor$ denote the greatest integer not exceeding $x$, and let 
\begin\{equation*\}
f(x)=\lfloor x\rfloor(2014\^\{x-\lfloor x\rfloor\}-1).
\end\{equation*\}
  The set of all numbers $x$ such that $1\leq x<2014$ and $f(x)\leq 1$ is a union of disjoint intervals.  What is the sum of the lengths of those intervals?

\$\textbf\{(A) \}1\qquad
\textbf\{(B) \}\dfrac\{\log 2015\}\{\log 2014\}\qquad
\textbf\{(C) \}\dfrac\{\log 2014\}\{\log 2013\}\qquad
\textbf\{(D) \}\dfrac\{2014\}\{2013\}\qquad
\textbf\{(E) \}2014\^\{\frac1\{2014\}\}\qquad\$\par \vspace{0.5em}\item The number $5^{867}$ is between $2^{2013}$ and $2^{2014}$.  How many pairs of integers $(m,n)$ are there such that $1\leq m\leq 2012$ and 
\begin\{equation*\}
5\^n<2\^m<2\^\{m+2\}<5\^\{n+1\}?
\end\{equation*\}

\$\textbf\{(A) \}278\qquad
\textbf\{(B) \}279\qquad
\textbf\{(C) \}280\qquad
\textbf\{(D) \}281\qquad
\textbf\{(E) \}282\qquad\$\par \vspace{0.5em}\item The fraction 
\begin\{equation*\}
\dfrac1\{99\^2\}=0.\overline\{b\_\{n-1\}b\_\{n-2\}\ldots b\_2b\_1b\_0\},
\end\{equation*\}
 where $n$ is the length of the period of the repeating decimal expansion.  What is the sum $b_0+b_1+\cdots+b_{n-1}$?

\$\textbf\{(A) \}874\qquad
\textbf\{(B) \}883\qquad
\textbf\{(C) \}887\qquad
\textbf\{(D) \}891\qquad
\textbf\{(E) \}892\qquad\$\par \vspace{0.5em}\item Let $f_0(x)=x+|x-100|-|x+100|$, and for $n\geq 1$, let $f_n(x)=|f_{n-1}(x)|-1$.  For how many values of $x$ is $f_{100}(x)=0$?

\$\textbf\{(A) \}299\qquad
\textbf\{(B) \}300\qquad
\textbf\{(C) \}301\qquad
\textbf\{(D) \}302\qquad
\textbf\{(E) \}303\qquad\$\par \vspace{0.5em}\item The parabola $P$ has focus $(0,0)$ and goes through the points $(4,3)$ and $(-4,-3)$.  For how many points $(x,y)\in P$ with integer coordinates is it true that $|4x+3y|\leq 1000$?

\$\textbf\{(A) \}38\qquad
\textbf\{(B) \}40\qquad
\textbf\{(C) \}42\qquad
\textbf\{(D) \}44\qquad
\textbf\{(E) \}46\qquad\$\par \vspace{0.5em}\end{enumerate}\newpage\section*{2014 AMC1212B}\begin{enumerate}[label=\arabic*., itemsep=0.5em]\item Leah has $ 13 $ coins, all of which are pennies and nickels. If she had one more nickel than she has now, then she would have the same number of pennies and nickels. In cents, how much are Leah's coins worth?

$ \textbf{(A)}\ 33\qquad\textbf{(B)}\ 35\qquad\textbf{(C)}\ 37\qquad\textbf{(D)}\ 39\qquad\textbf{(E)}\ 41 $\par \vspace{0.5em}\item Orvin went to the store with just enough money to buy $ 30 $ balloons. When he arrived he discovered that the store had a special sale on balloons: buy $ 1 $ balloon at the regular price and get a second at $ \frac{1}{3} $ off the regular price. What is the greatest number of balloons Orvin could buy?

$ \textbf{(A)}\ 33\qquad\textbf{(B)}\ 34\qquad\textbf{(C)}\ 36\qquad\textbf{(D)}\ 38\qquad\textbf{(E)}\ 39 $\par \vspace{0.5em}\item Randy drove the first third of his trip on a gravel road, the next $ 20 $ miles on pavement, and the remaining one-fifth on a dirt road. In miles, how long was Randy's trip?

$ \textbf{(A)}\ 30\qquad\textbf{(B)}\ \frac{400}{11}\qquad\textbf{(C)}\ \frac{75}{2}\qquad\textbf{(D)}\ 40\qquad\textbf{(E)}\ \frac{300}{7} $\par \vspace{0.5em}\item Susie pays for $ 4 $ muffins and $ 3 $ bananas. Calvin spends twice as much paying for $ 2 $ muffins and $ 16 $ bananas. A muffin is how many times as expensive as a banana?

$ \textbf{(A)}\ \frac{3}{2}\qquad\textbf{(B)}\ \frac{5}{3}\qquad\textbf{(C)}\ \frac{7}{4}\qquad\textbf{(D)}\ 2\qquad\textbf{(E)}\ \frac{13}{4} $\par \vspace{0.5em}\item Doug constructs a square window using $ 8 $ equal-size panes of glass, as shown. The ratio of the height to width for each pane is $ 5 : 2 $, and the borders around and between the panes are $ 2 $ inches wide. In inches, what is the side length of the square window?

\begin{center}
\begin{asy}
import olympiad;
import cse5;
fill((0,0)--(2,0)--(2,26)--(0,26)--cycle,gray);
fill((6,0)--(8,0)--(8,26)--(6,26)--cycle,gray);
fill((12,0)--(14,0)--(14,26)--(12,26)--cycle,gray);
fill((18,0)--(20,0)--(20,26)--(18,26)--cycle,gray);
fill((24,0)--(26,0)--(26,26)--(24,26)--cycle,gray);
fill((0,0)--(26,0)--(26,2)--(0,2)--cycle,gray);
fill((0,12)--(26,12)--(26,14)--(0,14)--cycle,gray);
fill((0,24)--(26,24)--(26,26)--(0,26)--cycle,gray);
\end{asy}
\end{center}

$ \textbf{(A)}\ 26\qquad\textbf{(B)}\ 28\qquad\textbf{(C)}\ 30\qquad\textbf{(D)}\ 32\qquad\textbf{(E)}\ 34 $\par \vspace{0.5em}\item Ed and Ann both have lemonade with their lunch. Ed orders the regular size. Ann gets the large lemonade, which is 50\% more than the regular. After both consume $\frac{3}{4}$ of their drinks, Ann gives Ed a third of what she has left, and 2 additional ounces. When they finish their lemonades they realize that they both drank the same amount. How many ounces of lemonade did they drink together?

$ \textbf{(A)}\ 30\qquad\textbf{(B)}\ 32\qquad\textbf{(C)}\ 36\qquad\textbf{(D)}\ 40\qquad\textbf{(E)}\ 50 $\par \vspace{0.5em}\item For how many positive integers $n$ is $\frac{n}{30-n}$ also a positive integer?

$ \textbf{(A)}\ 4\qquad\textbf{(B)}\ 5\qquad\textbf{(C)}\ 6\qquad\textbf{(D)}\ 7\qquad\textbf{(E)}\ 8 $\par \vspace{0.5em}\item In the addition shown below $ A $, $ B $, $ C $, and $ D $ are distinct digits. How many different values are possible for $ D $?


\begin\{equation*\}
\begin\{tabular\}\{cccccc\}\&A\&B\&B\&C\&B\\ +\&B\&C\&A\&D\&A\\ \hline \&D\&B\&D\&D\&D\end\{tabular\}
\end\{equation*\}


$ \textbf{(A)}\ 2\qquad\textbf{(B)}\ 4\qquad\textbf{(C)}\ 7\qquad\textbf{(D)}\ 8\qquad\textbf{(E)}\ 9 $\par \vspace{0.5em}\item Convex quadrilateral $ ABCD $ has $ AB=3 $, $ BC=4 $, $ CD=13 $, $ AD=12 $, and $ \angle ABC=90^{\circ} $, as shown. What is the area of the quadrilateral?


\begin{center}
\begin{asy}
import olympiad;
import cse5;
pair A=(0,0), B=(-3,0), C=(-3,-4), D=(48/5,-36/5);
draw(A--B--C--D--A); 
label("$A$",A,N); label("$B$",B,NW); label("$C$",C,SW); label("$D$",D,E);
draw(rightanglemark(A,B,C,25));
\end{asy}
\end{center}


$ \textbf{(A)}\ 30\qquad\textbf{(B)}\ 36\qquad\textbf{(C)}\ 40\qquad\textbf{(D)}\ 48\qquad\textbf{(E)}\ 58.5 $\par \vspace{0.5em}\item Danica drove her new car on a trip for a whole number of hours, averaging 55 miles per hour. At the beginning of the trip, $abc$ miles was displayed on the odometer, where $abc$ is a 3-digit number with $a \geq{1}$ and $a+b+c \leq{7}$. At the end of the trip, the odometer showed $cba$ miles. What is $a^2+b^2+c^2?$.

$ \textbf{(A)}\ 26\qquad\textbf{(B)}\ 27\qquad\textbf{(C)}\ 36\qquad\textbf{(D)}\ 37\qquad\textbf{(E)}\ 41 $\par \vspace{0.5em}\item A list of 11 positive integers has a mean of 10, a median of 9, and a unique mode of 8. What is the largest possible value of an integer in the list?

$ \textbf{(A)}\ 24\qquad\textbf{(B)}\ 30\qquad\textbf{(C)}\ 31\qquad\textbf{(D)}\ 33\qquad\textbf{(E)}\ 35 $\par \vspace{0.5em}\item A set $S$ consists of triangles whose sides have integer lengths less than 5, and no two elements of $S$ are congruent or similar. What is the largest number of elements that $S$ can have?

$ \textbf{(A)}\ 8\qquad\textbf{(B)}\ 9\qquad\textbf{(C)}\ 10\qquad\textbf{(D)}\ 11\qquad\textbf{(E)}\ 12 $\par \vspace{0.5em}\item Real numbers $a$ and $b$ are chosen with $1<a<b$ such that no triangles with positive area has side lengths $1, a,$ and $b$ or $\tfrac{1}{b}, \tfrac{1}{a},$ and $1$. What is the smallest possible value of $b$?

$ \textbf{(A)}\ \frac{3+\sqrt{3}}{2}\qquad\textbf{(B)}\ \frac{5}{2}\qquad\textbf{(C)}\ \frac{3+\sqrt{5}}{2}\qquad\textbf{(D)}\ \frac{3+\sqrt{6}}{2}\qquad\textbf{(E)}\ 3 $\par \vspace{0.5em}\item A rectangular box has a total surface area of 94 square inches. The sum of the lengths of all its edges is 48 inches. What is the sum of the lengths in inches of all of its interior diagonals?

$ \textbf{(A)}\ 8\sqrt{3}\qquad\textbf{(B)}\ 10\sqrt{2}\qquad\textbf{(C)}\ 16\sqrt{3}\qquad\textbf{(D)}\ 20\sqrt{2}\qquad\textbf{(E)}\ 40\sqrt{2} $\par \vspace{0.5em}\item When $p = \sum\limits_{k=1}^{6} k \ln{k}$, the number $e^p$ is an integer.  What is the largest power of 2 that is a factor of $e^p$ ?

$ \textbf{(A)}\ 2^{12}\qquad\textbf{(B)}\ 2^{14}\qquad\textbf{(C)}\ 2^{16}\qquad\textbf{(D)}\ 2^{18}\qquad\textbf{(E)}\ 2^{20} $\par \vspace{0.5em}\item Let $P$ be a cubic polynomial with $P(0) = k$, $P(1) = 2k$, and $P(-1) = 3k$.  What is $P(2) + P(-2)$ ?

$ \textbf{(A)}\ 0\qquad\textbf{(B)}\ k\qquad\textbf{(C)}\ 6k\qquad\textbf{(D)}\ 7k\qquad\textbf{(E)}\ 14k $\par \vspace{0.5em}\item Let $P$ be the parabola with equation $y=x^2$ and let $Q = (20, 14)$. There are real numbers $r$ and $s$ such that the line through $Q$ with slope $m$ does not intersect $P$ if and only if $r < m < s$. What is $r + s$?

$ \textbf{(A)}\ 1\qquad\textbf{(B)}\ 26\qquad\textbf{(C)}\ 40\qquad\textbf{(D)}\ 52\qquad\textbf{(E)}\ 80 $\par \vspace{0.5em}\item The numbers $1$, $2$, $3$, $4$, $5$, are to be arranged in a circle.  An arrangement is $\textit{bad}$ if it is not true that for every $n$ from $1$ to $15$ one can find a subset of the numbers that appear consecutively on the circle that sum to $n$.  Arrangements that differ only by a rotation or a reflection are considered the same.  How many different bad arrangements are there?

$ \textbf{(A) }1\qquad\textbf{(B) }2\qquad\textbf{(C) }3\qquad\textbf{(D) }4\qquad\textbf{(E) }5 $\par \vspace{0.5em}\item A sphere is inscribed in a truncated right circular cone as shown. The volume of the truncated cone is twice that of the sphere. What is the ratio of the radius of the bottom base of the truncated cone to the radius of the top base of the truncated cone?

\begin{center}
\begin{asy}
import olympiad;
import cse5;
real r=(3+sqrt(5))/2;
real s=sqrt(r);
real Brad=r;
real brad=1;
real Fht = 2*s;
import graph3;
import solids;
currentprojection=orthographic(1,0,.2);
currentlight=(10,10,5);
revolution sph=sphere((0,0,Fht/2),Fht/2);
//draw(surface(sph),green+white+opacity(0.5));
//triple f(pair t) \{return (t.x*cos(t.y),t.x*sin(t.y),t.x\^(1/n)*sin(t.y/n));\}
triple f(pair t) \{
triple v0 = Brad*(cos(t.x),sin(t.x),0);
triple v1 = brad*(cos(t.x),sin(t.x),0)+(0,0,Fht);
return (v0 + t.y*(v1-v0));
\}
triple g(pair t) \{
return (t.y*cos(t.x),t.y*sin(t.x),0);
\}
surface sback=surface(f,(3pi/4,0),(7pi/4,1),80,2);
surface sfront=surface(f,(7pi/4,0),(11pi/4,1),80,2);
surface base = surface(g,(0,0),(2pi,Brad),80,2);
draw(sback,gray(0.9));
draw(sfront,gray(0.5));
draw(base,gray(0.9));
draw(surface(sph),gray(0.4));
\end{asy}
\end{center}

$\text{(A) } \dfrac32 \quad \text{(B) } \dfrac{1+\sqrt5}2 \quad \text{(C) } \sqrt3 \quad \text{(D) } 2 \quad \text{(E) } \dfrac{3+\sqrt5}2$\par \vspace{0.5em}\item For how many positive integers $x$ is $\log_{10}(x-40) + \log_{10}(60-x) < 2$ ?

\$\textbf\{(A) \}10\qquad
\textbf\{(B) \}18\qquad
\textbf\{(C) \}19\qquad
\textbf\{(D) \}20\qquad
\textbf\{(E) \}\text\{infinitely many\}\qquad\$\par \vspace{0.5em}\item In the figure, $ ABCD $ is a square of side length $ 1 $. The rectangles $ JKHG $ and $ EBCF $ are congruent. What is $ BE $?

\begin{center}
\begin{asy}
import olympiad;
import cse5;
pair A=(1,0), B=(0,0), C=(0,1), D=(1,1), E=(2-sqrt(3),0), F=(2-sqrt(3),1), G=(1,sqrt(3)/2), H=(2.5-sqrt(3),1), J=(.5,0), K=(2-sqrt(3),1-sqrt(3)/2);
draw(A--B--C--D--cycle);
draw(K--H--G--J--cycle);
draw(F--E);
label("$A$",A,SE); label("$B$",B,SW); label("$C$",C,NW); label("$D$",D,NE); label("$E$",E,S); label("$F$",F,N);
label("$G$",G,E); label("$H$",H,N); label("$J$",J,S); label("$K$",K,W);
\end{asy}
\end{center}

$ \textbf{(A) }\frac{1}{2}(\sqrt{6}-2)\qquad\textbf{(B) }\frac{1}{4}\qquad\textbf{(C) }2-\sqrt{3}\qquad\textbf{(D) }\frac{\sqrt{3}}{6}\qquad\textbf{(E) } 1-\frac{\sqrt{2}}{2}$\par \vspace{0.5em}\item In a small pond there are eleven lily pads in a row labeled 0 through 10.  A frog is sitting on pad 1.  When the frog is on pad $N$, $0<N<10$, it will jump to pad $N-1$ with probability $\frac{N}{10}$ and to pad $N+1$ with probability $1-\frac{N}{10}$.  Each jump is independent of the previous jumps.  If the frog reaches pad 0 it will be eaten by a patiently waiting snake.  If the frog reaches pad 10 it will exit the pond, never to return.  What is the probability that the frog will escape without being eaten by the snake?

\$\textbf\{(A) \}\frac\{32\}\{79\}\qquad
\textbf\{(B) \}\frac\{161\}\{384\}\qquad
\textbf\{(C) \}\frac\{63\}\{146\}\qquad
\textbf\{(D) \}\frac\{7\}\{16\}\qquad
\textbf\{(E) \}\frac\{1\}\{2\}\qquad\$\par \vspace{0.5em}\item The number 2017 is prime.  Let $S = \sum \limits_{k=0}^{62} \dbinom{2014}{k}$.  What is the remainder when $S$ is divided by 2017?

\$\textbf\{(A) \}32\qquad
\textbf\{(B) \}684\qquad
\textbf\{(C) \}1024\qquad
\textbf\{(D) \}1576\qquad
\textbf\{(E) \}2016\qquad\$\par \vspace{0.5em}\item Let $ABCDE$ be a pentagon inscribed in a circle such that $AB = CD = 3$, $BC = DE = 10$, and $AE= 14$.  The sum of the lengths of all diagonals of $ABCDE$ is equal to $\frac{m}{n}$, where $m$ and $n$ are relatively prime positive integers.  What is $m+n$ ?

\$\textbf\{(A) \}129\qquad
\textbf\{(B) \}247\qquad
\textbf\{(C) \}353\qquad
\textbf\{(D) \}391\qquad
\textbf\{(E) \}421\qquad\$\par \vspace{0.5em}\item Find the sum of all the positive solutions of 
\begin\{equation*\}
2\cos(2x) \left(\cos(2x) - \cos\left( \frac\{2014\pi\^2\}\{x\} \right)\right) = \cos(4x) - 1
\end\{equation*\}


$ \textbf{(A)}\ \pi \qquad\textbf{(B)}\ 810\pi  \qquad\textbf{(C)}\ 1008\pi \qquad\textbf{(D)}\ 1080 \pi \qquad\textbf{(E)}\ 1800\pi $\par \vspace{0.5em}\end{enumerate}\newpage\section*{2015 AMC1212A}\begin{enumerate}[label=\arabic*., itemsep=0.5em]\item What is the value of $(2^0-1+5^2-0)^{-1}\times5?$

$ \textbf{(A)}\ -125\qquad\textbf{(B)}\ -120\qquad\textbf{(C)}\ \frac{1}{5}\qquad\textbf{(D)}\ \frac{5}{24}\qquad\textbf{(E)}\ 25 $\par \vspace{0.5em}\item Two of the three sides of a triangle are 20 and 15. Which of the following numbers is not a possible perimeter of the triangle?

$ \textbf{(A)}\ 52\qquad\textbf{(B)}\ 57\qquad\textbf{(C)}\ 62\qquad\textbf{(D)}\ 67\qquad\textbf{(E)}\ 72 $\par \vspace{0.5em}\item Mr. Patrick teaches math to 15 students. He was grading tests and found that when he graded everyone's test except Payton's, the average grade for the class was 80. After he graded Payton's test, the class average became 81. What was Payton's score on the test?

$ \textbf{(A)}\ 81\qquad\textbf{(B)}\ 85\qquad\textbf{(C)}\ 91\qquad\textbf{(D)}\ 94\qquad\textbf{(E)}\ 95 $\par \vspace{0.5em}\item The sum of two positive numbers is 5 times their difference. What is the ratio of the larger number to the smaller?

$ \textbf{(A)}\ \frac54 \qquad\textbf{(B)}\ \frac32 \qquad\textbf{(C)}\ \frac95 \qquad\textbf{(D)}\ 2 \qquad\textbf{(E)}\ \frac52 $\par \vspace{0.5em}\item Amelia needs to estimate the quantity $\frac{a}{b} - c$, where $a, b,$ and $c$ are large positive integers. She rounds each of the integers so that the calculation will be easier to do mentally. In which of these situations will her answer necessarily be greater than the exact value of $\frac{a}{b} - c$?

\$ \textbf\{(A)\}\ \text\{She rounds all three numbers up.\}\\
\qquad\textbf\{(B)\}\ \text\{She rounds \} a \text\{ and \} b \text\{ up, and she rounds \} c \text\{ down.\}\\
\qquad\textbf\{(C)\}\ \text\{She rounds \} a \text\{ and \} c \text\{ up, and she rounds \} b \text\{ down.\} \\
\qquad\textbf\{(D)\}\ \text\{She rounds \} a \text\{ up, and she rounds \} b \text\{ and \} c \text\{ down.\}\\
\qquad\textbf\{(E)\}\ \text\{She rounds \} c \text\{ up, and she rounds \} a \text\{ and \} b \text\{ down.\} \$\par \vspace{0.5em}\item Two years ago Pete was three times as old as his cousin Claire. Two years before that, Pete was four times as old as Claire. In how many years will the ratio of their ages be $2 : 1$?

$ \textbf{(A)}\ 2 \qquad\textbf{(B)}\ 4 \qquad\textbf{(C)}\ 5 \qquad\textbf{(D)}\ 6 \qquad\textbf{(E)}\ 8$\par \vspace{0.5em}\item Two right circular cylinders have the same volume. The radius of the second cylinder is $10\%$ more than the radius of the first. What is the relationship between the heights of the two cylinders?

\$ \textbf\{(A)\}\ \text\{The second height is \} 10\\% \text\{ less than the first.\} \\
\qquad\textbf\{(B)\}\ \text\{The first height is \} 10\\% \text\{ more than the second.\} \\
\qquad\textbf\{(C)\}\ \text\{The second height is \} 21\\% \text\{ less than the first.\} \\
\qquad\textbf\{(D)\}\ \text\{The first height is \} 21\\% \text\{ more than the second.\} \\
\qquad\textbf\{(E)\}\ \text\{The second height is \} 80\\% \text\{ of the first.\} \$\par \vspace{0.5em}\item The ratio of the length to the width of a rectangle is $4$ : $3$. If the rectangle has diagonal of length $d$, then the area may be expressed as $kd^2$ for some constant $k$. What is $k$?

$ \textbf{(A)}\ \frac27 \qquad\textbf{(B)}\ \frac37 \qquad\textbf{(C)}\ \frac{12}{25} \qquad\textbf{(D)}\ \frac{16}{25} \qquad\textbf{(E)}\ \frac34$\par \vspace{0.5em}\item A box contains 2 red marbles, 2 green marbles, and 2 yellow marbles. Carol takes 2 marbles from the box at random; then Claudia takes 2 of the remaining marbles at random; and then Cheryl takes the last 2 marbles. What is the probability that Cheryl gets 2 marbles of the same color?

$ \textbf{(A)}\ \frac{1}{10} \qquad\textbf{(B)}\ \frac16 \qquad\textbf{(C)}\ \frac15 \qquad\textbf{(D)}\ \frac13 \qquad\textbf{(E)}\ \frac12$\par \vspace{0.5em}\item Integers $x$ and $y$ with $x>y>0$ satisfy $x+y+xy=80$. What is $x$?

$ \textbf{(A)}\ 8 \qquad\textbf{(B)}\ 10 \qquad\textbf{(C)}\ 15 \qquad\textbf{(D)}\ 18 \qquad\textbf{(E)}\ 26$\par \vspace{0.5em}\item On a sheet of paper, Isabella draws a circle of radius $2$, a circle of radius $3$, and all possible lines simultaneously tangent to both circles. Isabella notices that she has drawn exactly $k \ge 0$ lines. How many different values of $k$ are possible?

$ \textbf{(A)}\ 2 \qquad\textbf{(B)}\ 3 \qquad\textbf{(C)}\ 4 \qquad\textbf{(D)}\ 5\qquad\textbf{(E)}\ 6$\par \vspace{0.5em}\item The parabolas $y=ax^2 - 2$ and $y=4 - bx^2$ intersect the coordinate axes in exactly four points, and these four points are the vertices of a kite of area $12$. What is $a+b$?

$ \textbf{(A)}\ 1\qquad\textbf{(B)}\ 1.5\qquad\textbf{(C)}\ 2\qquad\textbf{(D)}\ 2.5\qquad\textbf{(E)}\ 3$\par \vspace{0.5em}\item A league with 12 teams holds a round-robin tournament, with each team playing every other team exactly once. Games either end with one team victorious or else end in a draw. A team scores 2 points for every game it wins and 1 point for every game it draws. Which of the following is NOT a true statement about the list of 12 scores?

\$ \textbf\{(A)\}\ \text\{There must be an even number of odd scores.\}\\
\qquad\textbf\{(B)\}\ \text\{There must be an even number of even scores.\}\\
\qquad\textbf\{(C)\}\ \text\{There cannot be two scores of \}0\text\{.\}\\
\qquad\textbf\{(D)\}\ \text\{The sum of the scores must be at least \}100\text\{.\}\\
\qquad\textbf\{(E)\}\ \text\{The highest score must be at least \}12\text\{.\}\$\par \vspace{0.5em}\item What is the value of $a$ for which $\frac{1}{\log_2 a} + \frac{1}{\log_3 a} + \frac{1}{\log_4 a} = 1$?

$\textbf{(A)}\ 9\qquad\textbf{(B)}\ 12\qquad\textbf{(C)}\ 18\qquad\textbf{(D)}\ 24\qquad\textbf{(E)}\ 36$\par \vspace{0.5em}\item What is the minimum number of digits to the right of the decimal point needed to express the fraction $\frac{123456789}{2^{26}\cdot 5^4}$ as a decimal?

$ \textbf{(A)}\ 4\qquad\textbf{(B)}\ 22\qquad\textbf{(C)}\ 26\qquad\textbf{(D)}\ 30\qquad\textbf{(E)}\ 104$\par \vspace{0.5em}\item Tetrahedron $ABCD$ has $AB=5,AC=3,BC=4,BD=4,AD=3,$ and $CD=\frac{12}{5}\sqrt{2}$. What is the volume of the tetrahedron?

$ \textbf{(A)}\ 3\sqrt{2}\qquad\textbf{(B)}\ 2\sqrt{5}\qquad\textbf{(C)}\ \frac{24}{5}\qquad\textbf{(D)}\ 3\sqrt{3}\qquad\textbf{(E)}\ \frac{24}{5}\sqrt{2}$\par \vspace{0.5em}\item Eight people are sitting around a circular table, each holding a fair coin. All eight people flip their coins and those who flip heads stand while those who flip tails remain seated. What is the probability that no two adjacent people will stand?

$ \textbf{(A)}\ \frac{47}{256} \qquad\textbf{(B)}\ \frac{3}{16} \qquad\textbf{(C)}\ \frac{49}{256} \qquad\textbf{(D)}\ \frac{25}{128} \qquad\textbf{(E)}\ \frac{51}{256}$\par \vspace{0.5em}\item The zeros of the function $f(x) = x^2-ax+2a$ are integers. What is the sum of the possible values of $a$?

$ \textbf{(A)}\ 7 \qquad\textbf{(B)}\ 8 \qquad\textbf{(C)}\ 16 \qquad\textbf{(D)}\ 17 \qquad\textbf{(E)}\ 18$\par \vspace{0.5em}\item For some positive integers $p$, there is a quadrilateral $ABCD$ with positive integer side lengths, perimeter $p$, right angles at $B$ and $C$, $AB=2$, and $CD=AD$. How many different values of $p<2015$ are possible?

$ \textbf{(A)}\ 30 \qquad\textbf{(B)}\ 31 \qquad\textbf{(C)}\ 61 \qquad\textbf{(D)}\ 62 \qquad\textbf{(E)}\ 63$\par \vspace{0.5em}\item Isosceles triangles $T$ and $T'$ are not congruent but have the same area and the same perimeter. The sides of $T$ have lengths of $5,5,$ and $8$, while those of $T'$ have lengths of $a,a,$ and $b$. Which of the following numbers is closest to $b$?

$ \textbf{(A)}\ 3 \qquad\textbf{(B)}\ 4 \qquad\textbf{(C)}\ 5 \qquad\textbf{(D)}\ 6 \qquad\textbf{(E)}\ 8$\par \vspace{0.5em}\item A circle of radius $r$ passes through both foci of, and exactly four points on, the ellipse with equation $x^2+16y^2=16$. The set of all possible values of $r$ is an interval $[a,b)$. What is $a+b$?

$ \textbf{(A)}\ 5\sqrt{2}+4 \qquad\textbf{(B)}\ \sqrt{17}+7 \qquad\textbf{(C)}\ 6\sqrt{2}+3 \qquad\textbf{(D)}\ \sqrt{15}+8 \qquad\textbf{(E)}\ 12$\par \vspace{0.5em}\item For each positive integer $n$, let $S(n)$ be the number of sequences of length $n$ consisting solely of the letters $A$ and $B$, with no more than three $A$s in a row and no more than three $B$s in a row. What is the remainder when $S(2015)$ is divided by 12?

$ \textbf{(A)}\ 0 \qquad\textbf{(B)}\ 4 \qquad\textbf{(C)}\ 6 \qquad\textbf{(D)}\ 8 \qquad\textbf{(E)}\ 10$\par \vspace{0.5em}\item Let $S$ be a square of side length 1. Two points are chosen independently at random on the sides of $S$. The probability that the straight-line distance between the points is at least $\frac12$ is $\frac{a-b\pi}{c}$, where $a,b,$ and $c$ are positive integers and $\text{gcd}(a,b,c) = 1$. What is $a+b+c$?

$ \textbf{(A)}\ 59 \qquad\textbf{(B)}\ 60 \qquad\textbf{(C)}\ 61 \qquad\textbf{(D)}\ 62 \qquad\textbf{(E)}\ 63$\par \vspace{0.5em}\item Rational numbers $a$ and $b$ are chosen at random among all rational numbers in the interval $[0,2)$ that can be written as fractions $\frac{n}{d}$ where $n$ and $d$ are integers with $1 \le d \le 5$. What is the probability that

\begin\{equation*\}
(\text\{cos\}(a\pi)+i\text\{sin\}(b\pi))\^4
\end\{equation*\}

is a real number?

$ \textbf{(A)}\ \frac{3}{50} \qquad\textbf{(B)}\ \frac{4}{25} \qquad\textbf{(C)}\ \frac{41}{200} \qquad\textbf{(D)}\ \frac{6}{25} \qquad\textbf{(E)}\ \frac{13}{50}$\par \vspace{0.5em}\item A collection of circles in the upper half-plane, all tangent to the $x$-axis, is constructed in layers as follows. Layer $L_0$ consists of two circles of radii $70^2$ and $73^2$ that are externally tangent. For $k\ge1$, the circles in $\bigcup_{j=0}^{k-1}L_j$ are ordered according to their points of tangency with the $x$-axis. For every pair of consecutive circles in this order, a new circle is constructed externally tangent to each of the two circles in the pair. Layer $L_k$ consists of the $2^{k-1}$ circles constructed in this way. Let $S=\bigcup_{j=0}^{6}L_j$, and for every circle $C$ denote by $r(C)$ its radius. What is

\begin\{equation*\}
\sum\_\{C\in S\} \frac\{1\}\{\sqrt\{r(C)\}\}?
\end\{equation*\}



\begin{center}
\begin{asy}
import olympiad;
import cse5;
import olympiad;
size(350);
defaultpen(linewidth(0.7));
// define a bunch of arrays and starting points
pair[] coord = new pair[65];
int[] trav = \{32,16,8,4,2,1\};
coord[0] = (0,73\^2); coord[64] = (2*73*70,70\^2);
// draw the big circles and the bottom line
path arc1 = arc(coord[0],coord[0].y,260,360);
path arc2 = arc(coord[64],coord[64].y,175,280);
fill((coord[0].x-910,coord[0].y)--arc1--cycle,gray(0.75));
fill((coord[64].x+870,coord[64].y+425)--arc2--cycle,gray(0.75));
draw(arc1\^\^arc2);
draw((-930,0)--(70\^2+73\^2+850,0));
// We now apply the findCenter function 63 times to get
// the location of the centers of all 63 constructed circles.
// The complicated array setup ensures that all the circles
// will be taken in the right order
for(int i = 0;i<=5;i=i+1)
\{
int skip = trav[i];
for(int k=skip;k<=64 - skip; k = k + 2*skip)
\{
pair cent1 = coord[k-skip], cent2 = coord[k+skip];
real r1 = cent1.y, r2 = cent2.y, rn=r1*r2/((sqrt(r1)+sqrt(r2))\^2);
real shiftx = cent1.x + sqrt(4*r1*rn);
coord[k] = (shiftx,rn);
\}
// Draw the remaining 63 circles
\}
for(int i=1;i<=63;i=i+1)
\{
filldraw(circle(coord[i],coord[i].y),gray(0.75));
\}
\end{asy}
\end{center}



$ \textbf{(A)}\ \frac{286}{35} \qquad\textbf{(B)}\ \frac{583}{70} \qquad\textbf{(C)}\ \frac{715}{73}\qquad\textbf{(D)}\ \frac{143}{14} \qquad\textbf{(E)}\ \frac{1573}{146}$\par \vspace{0.5em}\end{enumerate}\newpage\section*{2015 AMC1212B}\begin{enumerate}[label=\arabic*., itemsep=0.5em]\item What is the value of $2-(-2)^{-2}$ ?

$\textbf{(A)}\; -2 \qquad\textbf{(B)}\; \dfrac{1}{16} \qquad\textbf{(C)}\; \dfrac{7}{4} \qquad\textbf{(D)}\; \dfrac{9}{4} \qquad\textbf{(E)}\; 6$\par \vspace{0.5em}\item Marie does three equally time-consuming tasks in a row without taking breaks. She begins the first task at 1:00 PM and finishes the second task at 2:40 PM. When does she finish the third task?

$\textbf{(A)}\; \text{3:10 PM} \qquad\textbf{(B)}\; \text{3:30 PM} \qquad\textbf{(C)}\; \text{4:00 PM} \qquad\textbf{(D)}\; \text{4:10 PM} \qquad\textbf{(E)}\; \text{4:30 PM}$\par \vspace{0.5em}\item Isaac has written down one integer two times and another integer three times. The sum of the five numbers is 100, and one of the numbers is 28. What is the other number?

$\textbf{(A)}\; 8 \qquad\textbf{(B)}\; 11 \qquad\textbf{(C)}\; 14 \qquad\textbf{(D)}\; 15 \qquad\textbf{(E)}\; 18$\par \vspace{0.5em}\item David, Hikmet, Jack, Marta, Rand, and Todd were in a 12-person race with 6 other people. Rand finished 6 places ahead of Hikmet. Marta finished 1 place behind Jack. David finished 2 places behind Hikmet. Jack finished 2 places behind Todd. Todd finished 1 place behind Rand. Marta finished in 6th place. Who finished in 8th place?

$\textbf{(A)}\; \text{David} \qquad\textbf{(B)}\; \text{Hikmet} \qquad\textbf{(C)}\; \text{Jack} \qquad\textbf{(D)}\; \text{Rand} \qquad\textbf{(E)}\; \text{Todd}$\par \vspace{0.5em}\item The Tigers beat the Sharks 2 out of the 3 times they played. They then played $N$ more times, and the Sharks ended up winning at least 95\% of all the games played. What is the minimum possible value for $N$?

$\textbf{(A)}\; 35 \qquad  \textbf{(B)}\; 37 \qquad \textbf{(C)}\; 39 \qquad \textbf{(D)}\; 41 \qquad \textbf{(E)}\; 43$\par \vspace{0.5em}\item Back in 1930, Tillie had to memorize her multiplication facts from $0 \times 0$ to $12 \times 12$. The multiplication table she was given had rows and columns labeled with the factors, and the products formed the body of the table. To the nearest hundredth, what fraction of the numbers in the body of the table are odd?

$\textbf{(A)}\; 0.21 \qquad\textbf{(B)}\; 0.25 \qquad\textbf{(C)}\; 0.46 \qquad\textbf{(D)}\; 0.50 \qquad\textbf{(E)}\; 0.75$\par \vspace{0.5em}\item A regular 15-gon has $L$ lines of symmetry, and the smallest positive angle for which it has rotational symmetry is $R$ degrees. What is $L+R$ ?

$\textbf{(A)}\; 24 \qquad\textbf{(B)}\; 27 \qquad\textbf{(C)}\; 32 \qquad\textbf{(D)}\; 39 \qquad\textbf{(E)}\; 54$\par \vspace{0.5em}\item What is the value of $(625^{\log_5 2015})^{\frac{1}{4}}$ ?

$\textbf{(A)}\; 5 \qquad\textbf{(B)}\; \sqrt[4]{2015} \qquad\textbf{(C)}\; 625 \qquad\textbf{(D)}\; 2015 \qquad\textbf{(E)}\; \sqrt[4]{5^{2015}}$\par \vspace{0.5em}\item Larry and Julius are playing a game, taking turns throwing a ball at a bottle sitting on a ledge. Larry throws first. The winner is the first person to knock the bottle off the ledge. At each turn the probability that a player knocks the bottle off the ledge is $\tfrac{1}{2}$, independently of what has happened before. What is the probability that Larry wins the game?

$\textbf{(A)}\; \dfrac{1}{2} \qquad\textbf{(B)}\; \dfrac{3}{5} \qquad\textbf{(C)}\; \dfrac{2}{3} \qquad\textbf{(D)}\; \dfrac{3}{4} \qquad\textbf{(E)}\; \dfrac{4}{5}$\par \vspace{0.5em}\item How many noncongruent integer-sided triangles with positive area and perimeter less than 15 are neither equilateral, isosceles, nor right triangles?

$\textbf{(A)}\; 3 \qquad\textbf{(B)}\; 4 \qquad\textbf{(C)}\; 5 \qquad\textbf{(D)}\; 6 \qquad\textbf{(E)}\; 7$\par \vspace{0.5em}\item The line $12x+5y=60$ forms a triangle with the coordinate axes. What is the sum of the lengths of the altitudes of this triangle?

$\textbf{(A)}\; 20 \qquad\textbf{(B)}\; \dfrac{360}{17} \qquad\textbf{(C)}\; \dfrac{107}{5} \qquad\textbf{(D)}\; \dfrac{43}{2} \qquad\textbf{(E)}\; \dfrac{281}{13}$\par \vspace{0.5em}\item Let $a$, $b$, and $c$ be three distinct one-digit numbers. What is the maximum value of the sum of the roots of the equation $(x-a)(x-b)+(x-b)(x-c)=0$ ?

$\textbf{(A)}\; 15 \qquad\textbf{(B)}\; 15.5 \qquad\textbf{(C)}\; 16 \qquad\textbf{(D)}\; 16.5 \qquad\textbf{(E)}\; 17$\par \vspace{0.5em}\item Quadrilateral $ABCD$ is inscribed in a circle with $\angle BAC=70^{\circ}, \angle ADB=40^{\circ}, AD=4,$ and $BC=6$. What is $AC$?

$\textbf{(A)}\; 3+\sqrt{5} \qquad\textbf{(B)}\; 6 \qquad\textbf{(C)}\; \dfrac{9}{2}\sqrt{2} \qquad\textbf{(D)}\; 8-\sqrt{2} \qquad\textbf{(E)}\; 7$\par \vspace{0.5em}\item A circle of radius 2 is centered at $A$. An equilateral triangle with side 4 has a vertex at $A$. What is the difference between the area of the region that lies inside the circle but outside the triangle and the area of the region that lies inside the triangle but outside the circle?

$\textbf{(A)}\; 8-\pi \qquad\textbf{(B)}\; \pi+2 \qquad\textbf{(C)}\; 2\pi-\dfrac{\sqrt{2}}{2} \qquad\textbf{(D)}\; 4(\pi-\sqrt{3}) \qquad\textbf{(E)}\; 2\pi-\dfrac{\sqrt{3}}{2}$\par \vspace{0.5em}\item At Rachelle's school an A counts 4 points, a B 3 points, a C 2 points, and a D 1 point. Her GPA on the four classes she is taking is computed as the total sum of points divided by 4. She is certain that she will get As in both Mathematics and Science, and at least a C in each of English and History. She thinks she has a $\tfrac{1}{6}$ chance of getting an A in English, and a $\tfrac{1}{4}$ chance of getting a B. In History, she has a $\tfrac{1}{4}$ chance of getting an A, and a $\tfrac{1}{3}$ chance of getting a B, independently of what she gets in English. What is the probability that Rachelle will get a GPA of at least 3.5?

$\textbf{(A)}\; \frac{11}{72} \qquad\textbf{(B)}\; \frac{1}{6} \qquad\textbf{(C)}\; \frac{3}{16} \qquad\textbf{(D)}\; \frac{11}{24} \qquad\textbf{(E)}\; \frac{1}{2}$\par \vspace{0.5em}\item A regular hexagon with sides of length 6 has an isosceles triangle attached to each side. Each of these triangles has two sides of length 8. The isosceles triangles are folded to make a pyramid with the hexagon as the base of the pyramid. What is the volume of the pyramid?

$\textbf{(A)}\; 18 \qquad\textbf{(B)}\; 162 \qquad\textbf{(C)}\; 36\sqrt{21} \qquad\textbf{(D)}\; 18\sqrt{138} \qquad\textbf{(E)}\; 54\sqrt{21}$\par \vspace{0.5em}\item An unfair coin lands on heads with a probability of $\tfrac{1}{4}$. When tossed $n$ times, the probability of exactly two heads is the same as the probability of exactly three heads. What is the value of $n$ ?

$\textbf{(A)}\; 5 \qquad\textbf{(B)}\; 8 \qquad\textbf{(C)}\; 10 \qquad\textbf{(D)}\; 11 \qquad\textbf{(E)}\; 13$\par \vspace{0.5em}\item For every composite positive integer $n$, define $r(n)$ to be the sum of the factors in the prime factorization of $n$. For example, $r(50) = 12$ because the prime factorization of $50$ is $2 \times 5^{2}$, and $2 + 5 + 5 = 12$. What is the range of the function $r$, $\{r(n): n \text{ is a composite positive integer}\}$ ?

\$\textbf\{(A)\}\; \text\{the set of positive integers\} \\
\textbf\{(B)\}\; \text\{the set of composite positive integers\} \\
\textbf\{(C)\}\; \text\{the set of even positive integers\} \\
\textbf\{(D)\}\; \text\{the set of integers greater than 3\} \\
\textbf\{(E)\}\; \text\{the set of integers greater than 4\}\$\par \vspace{0.5em}\item In $\triangle ABC$, $\angle C = 90^\circ$ and $AB = 12$. Squares $ABXY$ and $CBWZ$ are constructed outside of the triangle. The points $X$, $Y$, $Z$, and $W$ lie on a circle. What is the perimeter of the triangle?

$\textbf{(A)}\; 12+9\sqrt{3} \qquad\textbf{(B)}\; 18+6\sqrt{3} \qquad\textbf{(C)}\; 12+12\sqrt{2} \qquad\textbf{(D)}\; 30 \qquad\textbf{(E)}\; 32$\par \vspace{0.5em}\item For every positive integer $n$, let $\text{mod}_5 (n)$ be the remainder obtained when $n$ is divided by 5. Define a function $f: \{0,1,2,3,\dots\} \times \{0,1,2,3,4\} \to \{0,1,2,3,4\}$ recursively as follows:


\begin\{equation*\}
f(i,j) = \begin\{cases\}\text\{mod\}\_5 (j+1) \& \text\{ if \} i = 0 \text\{ and \} 0 \le j \le 4 \text\{,\}\\
f(i-1,1) \& \text\{ if \} i \ge 1 \text\{ and \} j = 0 \text\{, and\} \\
f(i-1, f(i,j-1)) \& \text\{ if \} i \ge 1 \text\{ and \} 1 \le j \le 4.
\end\{cases\}
\end\{equation*\}
  

What is $f(2015,2)$?

$\textbf{(A)}\; 0 \qquad\textbf{(B)}\; 1 \qquad\textbf{(C)}\; 2 \qquad\textbf{(D)}\; 3 \qquad\textbf{(E)}\; 4$\par \vspace{0.5em}\item Cozy the Cat and Dash the Dog are going up a staircase with a certain number of steps. However, instead of walking up the steps one at a time, both Cozy and Dash jump. Cozy goes two steps up with each jump (though if necessary, he will just jump the last step). Dash goes five steps up with each jump (though if necessary, he will just jump the last steps if there are fewer than 5 steps left). Suppose that Dash takes 19 fewer jumps than Cozy to reach the top of the staircase. Let $s$ denote the sum of all possible numbers of steps this staircase can have. What is the sum of the digits of $s$?

$\textbf{(A)}\; 9 \qquad\textbf{(B)}\; 11 \qquad\textbf{(C)}\; 12 \qquad\textbf{(D)}\; 13 \qquad\textbf{(E)}\; 15$\par \vspace{0.5em}\item Six chairs are evenly spaced around a circular table. One person is seated in each chair. Each person gets up and sits down in a chair that is not the same chair and is not adjacent to the chair he or she originally occupied, so that again one person is seated in each chair. In how many ways can this be done?

$\textbf{(A)}\; 14 \qquad\textbf{(B)}\; 16 \qquad\textbf{(C)}\; 18 \qquad\textbf{(D)}\; 20 \qquad\textbf{(E)}\; 24$\par \vspace{0.5em}\item A rectangular box measures $a \times b \times c$, where $a$, $b$, and $c$ are integers and $1\leq a \leq b \leq c$. The volume and the surface area of the box are numerically equal. How many ordered triples $(a,b,c)$ are possible?

$\textbf{(A)}\; 4 \qquad\textbf{(B)}\; 10 \qquad\textbf{(C)}\; 12 \qquad\textbf{(D)}\; 21 \qquad\textbf{(E)}\; 26$\par \vspace{0.5em}\item Four circles, no two of which are congruent, have centers at $A$, $B$, $C$, and $D$, and points $P$ and $Q$ lie on all four circles. The radius of circle $A$ is $\tfrac{5}{8}$ times the radius of circle $B$, and the radius of circle $C$ is $\tfrac{5}{8}$ times the radius of circle $D$. Furthermore, $AB = CD = 39$ and $PQ = 48$. Let $R$ be the midpoint of $\overline{PQ}$. What is $AR+BR+CR+DR$ ?

$\textbf{(A)}\; 180 \qquad\textbf{(B)}\; 184 \qquad\textbf{(C)}\; 188 \qquad\textbf{(D)}\; 192\qquad\textbf{(E)}\; 196$\par \vspace{0.5em}\item A bee starts flying from point $P_0$. She flies $1$ inch due east to point $P_1$. For $j \ge 1$, once the bee reaches point $P_j$, she turns $30^{\circ}$ counterclockwise and then flies $j+1$ inches straight to point $P_{j+1}$. When the bee reaches $P_{2015}$ she is exactly $a \sqrt{b} + c \sqrt{d}$ inches away from $P_0$, where $a$, $b$, $c$ and $d$ are positive integers and $b$ and $d$ are not divisible by the square of any prime. What is $a+b+c+d$ ?

$\textbf{(A)}\; 2016 \qquad\textbf{(B)}\; 2024 \qquad\textbf{(C)}\; 2032 \qquad\textbf{(D)}\; 2040 \qquad\textbf{(E)}\; 2048$\par \vspace{0.5em}\end{enumerate}\newpage\section*{2016 AMC1212A}\begin{enumerate}[label=\arabic*., itemsep=0.5em]\item What is the value of $\frac{11!-10!}{9!}$?

$\textbf{(A)}\ 99\qquad\textbf{(B)}\ 100\qquad\textbf{(C)}\ 110\qquad\textbf{(D)}\ 121\qquad\textbf{(E)}\ 132$\par \vspace{0.5em}\item For what value of $x$ does $10^x \cdot 100^{2x} = 1000^5$?

$\textbf{(A)}\ 1\qquad\textbf{(B)}\ 2\qquad\textbf{(C)}\ 3\qquad\textbf{(D)}\ 4\qquad\textbf{(E)}\ 5$\par \vspace{0.5em}\item The remainder can be defined for all real numbers $x$ and $y$ with $y \neq 0$ by 
\begin\{equation*\}
\text\{rem\} (x ,y)=x-y\left \lfloor \frac\{x\}\{y\} \right \rfloor
\end\{equation*\}
where $\left \lfloor \tfrac{x}{y} \right \rfloor$ denotes the greatest integer less than or equal to $\tfrac{x}{y}$. What is the value of $\text{rem} (\tfrac{3}{8}, -\tfrac{2}{5} )$?

$\textbf{(A) } -\frac{3}{8} \qquad \textbf{(B) } -\frac{1}{40} \qquad \textbf{(C) } 0 \qquad \textbf{(D) } \frac{3}{8} \qquad \textbf{(E) } \frac{31}{40}$\par \vspace{0.5em}\item The mean, median, and mode of the $7$ data values $60, 100, x, 40, 50, 200, 90$ are all equal to $x$. What is the value of $x$?

$\textbf{(A)}\ 50\qquad\textbf{(B)}\ 60\qquad\textbf{(C)}\ 75\qquad\textbf{(D)}\ 90\qquad\textbf{(E)}\ 100$\par \vspace{0.5em}\item Goldbach's conjecture states that every even integer greater than 2 can be written as the sum of two prime numbers (for example, $2016=13+2003$). So far, no one has been able to prove that the conjecture is true, and no one has found a counterexample to show that the conjecture is false. What would a counterexample consist of?

\$ \textbf\{(A)\}\ \text\{an odd integer greater than \} 2 \text\{ that can be written as the sum of two prime numbers\}\\
\qquad\textbf\{(B)\}\ \text\{an odd integer greater than \} 2 \text\{ that cannot be written as the sum of two prime numbers\}\\
\qquad\textbf\{(C)\}\ \text\{an even integer greater than \} 2 \text\{ that can be written as the sum of two numbers that are not prime\}\\
\qquad\textbf\{(D)\}\ \text\{an even integer greater than \} 2 \text\{ that can be written as the sum of two prime numbers\}\\
\qquad\textbf\{(E)\}\ \text\{an even integer greater than \} 2 \text\{ that cannot be written as the sum of two prime numbers\}\$\par \vspace{0.5em}\item A triangular array of $2016$ coins has $1$ coin in the first row, $2$ coins in the second row, $3$ coins in the third row, and so on up to $N$ coins in the $N$th row. What is the sum of the digits of $N$ ?

$\textbf{(A)}\ 6\qquad\textbf{(B)}\ 7\qquad\textbf{(C)}\ 8\qquad\textbf{(D)}\ 9\qquad\textbf{(E)}\ 10$\par \vspace{0.5em}\item Which of these describes the graph of $x^2(x+y+1)=y^2(x+y+1)$ ?

\$ \textbf\{(A)\}\ \text\{two parallel lines\}\\
\qquad\textbf\{(B)\}\ \text\{two intersecting lines\}\\
\qquad\textbf\{(C)\}\ \text\{three lines that all pass through a common point\}\\
\qquad\textbf\{(D)\}\ \text\{three lines that do not all pass through a common point\}\\
\qquad\textbf\{(E)\}\ \text\{a line and a parabola\}\$\par \vspace{0.5em}\item What is the area of the shaded region of the given $8\times 5$ rectangle?


\begin{center}
\begin{asy}
import olympiad;
import cse5;
size(6cm);
defaultpen(fontsize(9pt));
draw((0,0)--(8,0)--(8,5)--(0,5)--cycle);
filldraw((7,0)--(8,0)--(8,1)--(0,4)--(0,5)--(1,5)--cycle,gray(0.8));

label("$1$",(1/2,5),dir(90));
label("$7$",(9/2,5),dir(90));

label("$1$",(8,1/2),dir(0));
label("$4$",(8,3),dir(0));

label("$1$",(15/2,0),dir(270));
label("$7$",(7/2,0),dir(270));

label("$1$",(0,9/2),dir(180));
label("$4$",(0,2),dir(180));
\end{asy}
\end{center}


$\textbf{(A)}\ 4.75\qquad\textbf{(B)}\ 5\qquad\textbf{(C)}\ 5.25\qquad\textbf{(D)}\ 6.5\qquad\textbf{(E)}\ 8$\par \vspace{0.5em}\item The five small shaded squares inside this unit square are congruent and have disjoint interiors. The midpoint of each side of the middle square coincides with one of the vertices of the other four small squares as shown. The common side length is $\tfrac{a-\sqrt{2}}{b}$, where $a$ and $b$ are positive integers. What is $a+b$ ?


\begin{center}
\begin{asy}
import olympiad;
import cse5;
real x=.369;
draw((0,0)--(0,1)--(1,1)--(1,0)--cycle);
filldraw((0,0)--(0,x)--(x,x)--(x,0)--cycle, gray);
filldraw((0,1)--(0,1-x)--(x,1-x)--(x,1)--cycle, gray);
filldraw((1,1)--(1,1-x)--(1-x,1-x)--(1-x,1)--cycle, gray);
filldraw((1,0)--(1,x)--(1-x,x)--(1-x,0)--cycle, gray);
filldraw((.5,.5-x*sqrt(2)/2)--(.5+x*sqrt(2)/2,.5)--(.5,.5+x*sqrt(2)/2)--(.5-x*sqrt(2)/2,.5)--cycle, gray);
\end{asy}
\end{center}


$\textbf{(A)}\ 7\qquad\textbf{(B)}\ 8\qquad\textbf{(C)}\ 9\qquad\textbf{(D)}\ 10\qquad\textbf{(E)}\ 11$\par \vspace{0.5em}\item Five friends sat in a movie theater in a row containing $5$ seats, numbered $1$ to $5$ from left to right. (The directions "left" and "right" are from the point of view of the people as they sit in the seats.) During the movie Ada went to the lobby to get some popcorn. When she returned, she found that Bea had moved two seats to the right, Ceci had moved one seat to the left, and Dee and Edie had switched seats, leaving an end seat for Ada. In which seat had Ada been sitting before she got up?

$\textbf{(A)}\ 1\qquad\textbf{(B)}\ 2\qquad\textbf{(C)}\ 3\qquad\textbf{(D)}\ 4\qquad\textbf{(E)}\ 5$\par \vspace{0.5em}\item Each of the $100$ students in a certain summer camp can either sing, dance, or act. Some students have more than one talent, but no student has all three talents. There are $42$ students who cannot sing, $65$ students who cannot dance, and $29$ students who cannot act. How many students have two of these talents?

$\textbf{(A)}\ 16\qquad\textbf{(B)}\ 25\qquad\textbf{(C)}\ 36\qquad\textbf{(D)}\ 49\qquad\textbf{(E)}\ 64$\par \vspace{0.5em}\item In $\triangle ABC$, $AB = 6$, $BC = 7$, and $CA = 8$. Point $D$ lies on $\overline{BC}$, and $\overline{AD}$ bisects $\angle BAC$. Point $E$ lies on $\overline{AC}$, and $\overline{BE}$ bisects $\angle ABC$. The bisectors intersect at $F$. What is the ratio $AF$ : $FD$?


\begin{center}
\begin{asy}
import olympiad;
import cse5;
pair A = (0,0), B=(6,0), C=intersectionpoints(Circle(A,8),Circle(B,7))[0], F=incenter(A,B,C), D=extension(A,F,B,C),E=extension(B,F,A,C);
draw(A--B--C--A--D\^\^B--E);
label("$A$",A,SW);
label("$B$",B,SE);
label("$C$",C,N);
label("$D$",D,NE);
label("$E$",E,NW);
label("$F$",F,1.5*N);
\end{asy}
\end{center}


$\textbf{(A)}\ 3:2\qquad\textbf{(B)}\ 5:3\qquad\textbf{(C)}\ 2:1\qquad\textbf{(D)}\ 7:3\qquad\textbf{(E)}\ 5:2$\par \vspace{0.5em}\item Let $N$ be a positive multiple of $5$. One red ball and $N$ green balls are arranged in a line in random order. Let $P(N)$ be the probability that at least $\tfrac{3}{5}$ of the green balls are on the same side of the red ball. Observe that $P(5)=1$ and that $P(N)$ approaches $\tfrac{4}{5}$ as $N$ grows large. What is the sum of the digits of the least value of $N$ such that $P(N) < \tfrac{321}{400}$?

$\textbf{(A)}\ 12\qquad\textbf{(B)}\ 14\qquad\textbf{(C)}\ 16\qquad\textbf{(D)}\ 18\qquad\textbf{(E)}\ 20$\par \vspace{0.5em}\item Each vertex of a cube is to be labeled with an integer from $1$ through $8$, with each integer being used once, in such a way that the sum of the four numbers on the vertices of a face is the same for each face.  Arrangements that can be obtained from each other through rotations of the cube are considered to be the same.  How many different arrangements are possible?

$\textbf{(A)}\ 1\qquad\textbf{(B)}\ 3\qquad\textbf{(C)}\ 6\qquad\textbf{(D)}\ 12\qquad\textbf{(E)}\ 24$\par \vspace{0.5em}\item Circles with centers $P, Q$ and $R$, having radii $1, 2$ and $3$, respectively, lie on the same side of line $l$ and are tangent to $l$ at $P', Q'$ and $R'$, respectively, with $Q'$ between $P'$ and $R'$. The circle with center $Q$ is externally tangent to each of the other two circles. What is the area of triangle $PQR$?

$\textbf{(A) } 0\qquad \textbf{(B) } \frac{\sqrt{6}}{3}\qquad\textbf{(C) } 1\qquad\textbf{(D) } \sqrt{6}-\sqrt{2}\qquad\textbf{(E) }\frac{\sqrt{6}}{2}$\par \vspace{0.5em}\item The graphs of $y=\log_3 x, y=\log_x 3, y=\log_\frac{1}{3} x,$ and $y=\log_x \dfrac{1}{3}$ are plotted on the same set of axes. How many points in the plane with positive $x$-coordinates lie on two or more of the graphs? 

$\textbf{(A)}\ 2\qquad\textbf{(B)}\ 3\qquad\textbf{(C)}\ 4\qquad\textbf{(D)}\ 5\qquad\textbf{(E)}\ 6$\par \vspace{0.5em}\item Let $ABCD$ be a square. Let $E, F, G$ and $H$ be the centers, respectively, of equilateral triangles with bases $\overline{AB}, \overline{BC}, \overline{CD},$ and $\overline{DA},$ each exterior to the square. What is the ratio of the area of square $EFGH$ to the area of square $ABCD$? 

$\textbf{(A)}\ 1\qquad\textbf{(B)}\ \frac{2+\sqrt{3}}{3} \qquad\textbf{(C)}\ \sqrt{2} \qquad\textbf{(D)}\ \frac{\sqrt{2}+\sqrt{3}}{2} \qquad\textbf{(E)}\ \sqrt{3}$\par \vspace{0.5em}\item For some positive integer $n,$ the number $110n^3$ has $110$ positive integer divisors, including $1$ and the number $110n^3.$ How many positive integer divisors does the number $81n^4$ have? 

$\textbf{(A)}\ 110\qquad\textbf{(B)}\ 191\qquad\textbf{(C)}\ 261\qquad\textbf{(D)}\ 325\qquad\textbf{(E)}\ 425$\par \vspace{0.5em}\item Jerry starts at $0$ on the real number line. He tosses a fair coin $8$ times. When he gets heads, he moves $1$ unit in the positive direction; when he gets tails, he moves $1$ unit in the negative direction. The probability that he reaches $4$ at some time during this process is $\frac{a}{b},$ where $a$ and $b$ are relatively prime positive integers. What is $a + b?$ (For example, he succeeds if his sequence of tosses is $HTHHHHHH.$)

$\textbf{(A)}\ 69\qquad\textbf{(B)}\ 151\qquad\textbf{(C)}\ 257\qquad\textbf{(D)}\ 293\qquad\textbf{(E)}\ 313$\par \vspace{0.5em}\item A binary operation $\diamondsuit $ has the properties that $a\ \diamondsuit\ (b\ \diamondsuit\ c) = (a\ \diamondsuit\ b)\cdot c$ and that $a\ \diamondsuit\ a = 1$ for all nonzero real numbers $a, b$ and $c.$ (Here the dot  $\cdot$  represents the usual multiplication operation.) The solution to the equation $2016\ \diamondsuit\ (6\ \diamondsuit\ x) = 100$ can be written as $\frac{p}{q},$ where $p$ and $q$ are relatively prime positive integers. What is $p + q?$ 

$\textbf{(A)}\ 109\qquad\textbf{(B)}\ 201\qquad\textbf{(C)}\ 301\qquad\textbf{(D)}\ 3049\qquad\textbf{(E)}\ 33,601$\par \vspace{0.5em}\item A quadrilateral is inscribed in a circle of radius $200\sqrt{2}.$ Three of the sides of this quadrilateral have length $200.$ What is the length of its fourth side? 

$\textbf{(A)}\ 200\qquad\textbf{(B)}\ 200\sqrt{2} \qquad\textbf{(C)}\ 200\sqrt{3} \qquad\textbf{(D)}\ 300\sqrt{2} \qquad\textbf{(E)}\ 500$\par \vspace{0.5em}\item How many ordered triples $(x,y,z)$ of positive integers satisfy $\text{lcm}(x,y) = 72, \text{lcm}(x,z) = 600$ and $\text{lcm}(y,z)=900$?

$\textbf{(A)}\ 15\qquad\textbf{(B)}\ 16\qquad\textbf{(C)}\ 24\qquad\textbf{(D)}\ 27\qquad\textbf{(E)}\ 64$\par \vspace{0.5em}\item Three numbers in the interval $\left[0,1\right]$ are chosen independently and at random. What is the probability that the chosen numbers are the side lengths of a triangle with positive area?

$\textbf{(A)}\ \dfrac{1}{6}\qquad\textbf{(B)}\ \dfrac{1}{3}\qquad\textbf{(C)}\ \dfrac{1}{2}\qquad\textbf{(D)}\ \dfrac{2}{3}\qquad\textbf{(E)}\ \dfrac{5}{6}$\par \vspace{0.5em}\item There is a smallest positive real number $a$ such that there exists a positive real number $b$ such that all the roots of the polynomial $x^3-ax^2+bx-a$ are real. In fact, for this value of $a$ the value of $b$ is unique. What is the value of $b?$

$\textbf{(A)}\ 8\qquad\textbf{(B)}\ 9\qquad\textbf{(C)}\ 10\qquad\textbf{(D)}\ 11\qquad\textbf{(E)}\ 12$\par \vspace{0.5em}\item Let $k$ be a positive integer. Bernardo and Silvia take turns writing and erasing numbers on a blackboard as follows: Bernardo starts by writing the smallest perfect square with $k+1$ digits. Every time Bernardo writes a number, Silvia erases the last $k$ digits of it. Bernardo then writes the next perfect square, Silvia erases the last $k$ digits of it, and this process continues until the last two numbers that remain on the board differ by at least 2. Let $f(k)$ be the smallest positive integer not written on the board. For example, if $k = 1$, then the numbers that Bernardo writes are $16, 25, 36, 49, 64$, and the numbers showing on the board after Silvia erases are $1, 2, 3, 4,$ and $6$, and thus $f(1) = 5$. What is the sum of the digits of $f(2) + f(4)+ f(6) + ... + f(2016)$?

$\textbf{(A)}\ 7986\qquad\textbf{(B)}\ 8002\qquad\textbf{(C)}\ 8030\qquad\textbf{(D)}\ 8048\qquad\textbf{(E)}\ 8064$\par \vspace{0.5em}\end{enumerate}\newpage\section*{2016 AMC1212B}\begin{enumerate}[label=\arabic*., itemsep=0.5em]\item What is the value of $\frac{2a^{-1}+\frac{a^{-1}}{2}}{a}$ when $a= \frac{1}{2}$?

$\textbf{(A)}\ 1\qquad\textbf{(B)}\ 2\qquad\textbf{(C)}\ \frac{5}{2}\qquad\textbf{(D)}\ 10\qquad\textbf{(E)}\ 20$\par \vspace{0.5em}\item The harmonic mean of two numbers can be calculated as twice their product divided by their sum. The harmonic mean of $1$ and $2016$ is closest to which integer?

\$\textbf\{(A)\}\ 2 \qquad
\textbf\{(B)\}\ 45 \qquad
\textbf\{(C)\}\ 504 \qquad
\textbf\{(D)\}\ 1008 \qquad
\textbf\{(E)\}\ 2015 \$\par \vspace{0.5em}\item Let $x=-2016$. What is the value of $\bigg|$ $||x|-x|-|x|$ $\bigg|$ $-x$?

$\textbf{(A)}\ -2016\qquad\textbf{(B)}\ 0\qquad\textbf{(C)}\ 2016\qquad\textbf{(D)}\ 4032\qquad\textbf{(E)}\ 6048$\par \vspace{0.5em}\item The ratio of the measures of two acute angles is $5:4$, and the complement of one of these two angles is twice as large as the complement of the other. What is the sum of the degree measures of the two angles?

$\textbf{(A)}\ 75\qquad\textbf{(B)}\ 90\qquad\textbf{(C)}\ 135\qquad\textbf{(D)}\ 150\qquad\textbf{(E)}\ 270$\par \vspace{0.5em}\item The War of $1812$ started with a declaration of war on Thursday, June $18$, $1812$. The peace treaty to end the war was signed $919$ days later, on December $24$, $1814$. On what day of the week was the treaty signed? 

\$\textbf\{(A)\}\ \text\{Friday\} \qquad
\textbf\{(B)\}\ \text\{Saturday\} \qquad
\textbf\{(C)\}\ \text\{Sunday\} \qquad
\textbf\{(D)\}\ \text\{Monday\} \qquad
\textbf\{(E)\}\ \text\{Tuesday\} \$\par \vspace{0.5em}\item All three vertices of $\bigtriangleup ABC$ lie on the parabola defined by $y=x^2$, with $A$ at the origin and $\overline{BC}$ parallel to the $x$-axis. The area of the triangle is $64$. What is the length of $BC$?  

$\textbf{(A)}\ 4\qquad\textbf{(B)}\ 6\qquad\textbf{(C)}\ 8\qquad\textbf{(D)}\ 10\qquad\textbf{(E)}\ 16$\par \vspace{0.5em}\item Josh writes the numbers $1,2,3,\dots,99,100$. He marks out $1$, skips the next number $(2)$, marks out $3$, and continues skipping and marking out the next number to the end of the list. Then he goes back to the start of his list, marks out the first remaining number $(2)$, skips the next number $(4)$, marks out $6$, skips $8$, marks out $10$, and so on to the end. Josh continues in this manner until only one number remains. What is that number?

\$\textbf\{(A)\}\ 13 \qquad
\textbf\{(B)\}\ 32 \qquad
\textbf\{(C)\}\ 56 \qquad
\textbf\{(D)\}\ 64 \qquad
\textbf\{(E)\}\ 96\$\par \vspace{0.5em}\item A thin piece of wood of uniform density in the shape of an equilateral triangle with side length $3$ inches weighs $12$ ounces. A second piece of the same type of wood, with the same thickness, also in the shape of an equilateral triangle, has side length of $5$ inches. Which of the following is closest to the weight, in ounces, of the second piece?

$\textbf{(A)}\ 14.0\qquad\textbf{(B)}\ 16.0\qquad\textbf{(C)}\ 20.0\qquad\textbf{(D)}\ 33.3\qquad\textbf{(E)}\ 55.6$\par \vspace{0.5em}\item Carl decided to fence in his rectangular garden. He bought $20$ fence posts, placed one on each of the four corners, and spaced out the rest evenly along the edges of the garden, leaving exactly $4$ yards between neighboring posts. The longer side of his garden, including the corners, has twice as many posts as the shorter side, including the corners. What is the area, in square yards, of Carls garden?

$\textbf{(A)}\ 256\qquad\textbf{(B)}\ 336\qquad\textbf{(C)}\ 384\qquad\textbf{(D)}\ 448\qquad\textbf{(E)}\ 512$\par \vspace{0.5em}\item A quadrilateral has vertices $P(a,b)$, $Q(b,a)$, $R(-a, -b)$, and $S(-b, -a)$, where $a$ and $b$ are integers with $a>b>0$. The area of $PQRS$ is $16$. What is $a+b$?

$\textbf{(A)}\ 4 \qquad\textbf{(B)}\ 5 \qquad\textbf{(C)}\ 6 \qquad\textbf{(D)}\ 12  \qquad\textbf{(E)}\ 13$\par \vspace{0.5em}\item How many squares whose sides are parallel to the axes and whose vertices have coordinates that are integers lie entirely within the region bounded by the line $y=\pi x$, the line $y=-0.1$ and the line $x=5.1?$

\$\textbf\{(A)\}\ 30 \qquad
\textbf\{(B)\}\ 41 \qquad
\textbf\{(C)\}\ 45 \qquad
\textbf\{(D)\}\ 50 \qquad
\textbf\{(E)\}\ 57\$\par \vspace{0.5em}\item All the numbers $1, 2, 3, 4, 5, 6, 7, 8, 9$ are written in a $3\times3$ array of squares, one number in each square, in such a way that if two numbers are consecutive then they occupy squares that share an edge. The numbers in the four corners add up to $18$. What is the number in the center?

$\textbf{(A)}\ 5\qquad\textbf{(B)}\ 6\qquad\textbf{(C)}\ 7\qquad\textbf{(D)}\ 8\qquad\textbf{(E)}\ 9$\par \vspace{0.5em}\item Alice and Bob live $10$ miles apart. One day Alice looks due north from her house and sees an airplane. At the same time Bob looks due west from his house and sees the same airplane. The angle of elevation of the airplane is $30^\circ$ from Alice's position and $60^\circ$ from Bob's position. Which of the following is closest to the airplane's altitude, in miles?

$\textbf{(A)}\ 3.5 \qquad\textbf{(B)}\ 4 \qquad\textbf{(C)}\ 4.5 \qquad\textbf{(D)}\ 5 \qquad\textbf{(E)}\ 5.5$\par \vspace{0.5em}\item The sum of an infinite geometric series is a positive number $S$, and the second term in the series is $1$. What is the smallest possible value of $S?$

\$\textbf\{(A)\}\ \frac\{1+\sqrt\{5\}\}\{2\} \qquad
\textbf\{(B)\}\ 2 \qquad
\textbf\{(C)\}\ \sqrt\{5\} \qquad
\textbf\{(D)\}\ 3 \qquad
\textbf\{(E)\}\ 4\$\par \vspace{0.5em}\item All the numbers $2, 3, 4, 5, 6, 7$ are assigned to the six faces of a cube, one number to each face. For each of the eight vertices of the cube, a product of three numbers is computed, where the three numbers are the numbers assigned to the three faces that include that vertex. What is the greatest possible value of the sum of these eight products?

\$\textbf\{(A)\}\ 312 \qquad
\textbf\{(B)\}\ 343 \qquad
\textbf\{(C)\}\ 625 \qquad
\textbf\{(D)\}\ 729 \qquad
\textbf\{(E)\}\ 1680\$\par \vspace{0.5em}\item In how many ways can $345$ be written as the sum of an increasing sequence of two or more consecutive positive integers?

$\textbf{(A)}\ 1\qquad\textbf{(B)}\ 3\qquad\textbf{(C)}\ 5\qquad\textbf{(D)}\ 6\qquad\textbf{(E)}\ 7$\par \vspace{0.5em}\item In $\triangle ABC$ shown in the figure, $AB=7$, $BC=8$, $CA=9$, and $\overline{AH}$ is an altitude. Points $D$ and $E$ lie on sides $\overline{AC}$ and $\overline{AB}$, respectively, so that $\overline{BD}$ and $\overline{CE}$ are angle bisectors, intersecting $\overline{AH}$ at $Q$ and $P$, respectively. What is $PQ$?


\begin{center}
\begin{asy}
import olympiad;
import cse5;
import graph; size(9cm); 
real labelscalefactor = 0.5; /* changes label-to-point distance */
pen dps = linewidth(0.7) + fontsize(10); defaultpen(dps); /* default pen style */ 
pen dotstyle = black; /* point style */ 
real xmin = -4.381056062031275, xmax = 15.020004395092375, ymin = -4.051697595316909, ymax = 10.663513514111651;  /* image dimensions */


draw((0.,0.)--(4.714285714285714,7.666518779999279)--(7.,0.)--cycle); 
 /* draw figures */
draw((0.,0.)--(4.714285714285714,7.666518779999279)); 
draw((4.714285714285714,7.666518779999279)--(7.,0.)); 
draw((7.,0.)--(0.,0.)); 
label("7",(3.2916797119724284,-0.07831656949355523),SE*labelscalefactor); 
label("9",(2.0037562070503783,4.196493361737088),SE*labelscalefactor); 
label("8",(6.114150371695219,3.785453945272603),SE*labelscalefactor); 
draw((0.,0.)--(6.428571428571427,1.9166296949998194)); 
draw((7.,0.)--(2.2,3.5777087639996634)); 
draw((4.714285714285714,7.666518779999279)--(3.7058823529411766,0.)); 
 /* dots and labels */
dot((0.,0.),dotstyle); 
label("$A$", (-0.2432592696221352,-0.5715638692509372), NE * labelscalefactor); 
dot((7.,0.),dotstyle); 
label("$B$", (7.0458397156813835,-0.48935598595804014), NE * labelscalefactor); 
dot((3.7058823529411766,0.),dotstyle); 
label("$E$", (3.8123296394941084,0.16830708038513573), NE * labelscalefactor); 
dot((4.714285714285714,7.666518779999279),dotstyle); 
label("$C$", (4.579603216894479,7.895848109917452), NE * labelscalefactor); 
dot((2.2,3.5777087639996634),linewidth(3.pt) + dotstyle); 
label("$D$", (2.1407693458718726,3.127790878929427), NE * labelscalefactor); 
dot((6.428571428571427,1.9166296949998194),linewidth(3.pt) + dotstyle); 
label("$H$", (6.004539860638023,1.9494778850645704), NE * labelscalefactor); 
dot((5.,1.49071198499986),linewidth(3.pt) + dotstyle); 
label("$Q$", (4.935837377830365,1.7302568629501784), NE * labelscalefactor); 
dot((3.857142857142857,1.1499778169998918),linewidth(3.pt) + dotstyle); 
label("$P$", (3.538303361851119,1.2370095631927964), NE * labelscalefactor); 
clip((xmin,ymin)--(xmin,ymax)--(xmax,ymax)--(xmax,ymin)--cycle); 
 /* end of picture */
\end{asy}
\end{center}


\$\textbf\{(A)\}\ 1 \qquad
\textbf\{(B)\}\ \frac\{5\}\{8\}\sqrt\{3\} \qquad
\textbf\{(C)\}\ \frac\{4\}\{5\}\sqrt\{2\} \qquad
\textbf\{(D)\}\ \frac\{8\}\{15\}\sqrt\{5\} \qquad
\textbf\{(E)\}\ \frac\{6\}\{5\}\$\par \vspace{0.5em}\item What is the area of the region enclosed by the graph of the equation $x^2+y^2=|x|+|y|?$

$\textbf{(A)}\ \pi+\sqrt{2} \qquad\textbf{(B)}\ \pi+2 \qquad\textbf{(C)}\ \pi+2\sqrt{2} \qquad\textbf{(D)}\ 2\pi+\sqrt{2} \qquad\textbf{(E)}\ 2\pi+2\sqrt{2}$\par \vspace{0.5em}\item Tom, Dick, and Harry are playing a game. Starting at the same time, each of them flips a fair coin repeatedly until he gets his first head, at which point he stops. What is the probability that all three flip their coins the same number of times?

\$\textbf\{(A)\}\ \frac\{1\}\{8\} \qquad
\textbf\{(B)\}\ \frac\{1\}\{7\} \qquad
\textbf\{(C)\}\ \frac\{1\}\{6\} \qquad
\textbf\{(D)\}\ \frac\{1\}\{4\} \qquad
\textbf\{(E)\}\ \frac\{1\}\{3\}\$\par \vspace{0.5em}\item A set of teams held a round-robin tournament in which every team played every other team exactly once. Every team won $10$ games and lost $10$ games; there were no ties. How many sets of three teams $\{A, B, C\}$ were there in which $A$ beat $B$, $B$ beat $C$, and $C$ beat $A?$

\$\textbf\{(A)\}\ 385 \qquad
\textbf\{(B)\}\ 665 \qquad
\textbf\{(C)\}\ 945 \qquad
\textbf\{(D)\}\ 1140 \qquad
\textbf\{(E)\}\ 1330\$\par \vspace{0.5em}\item Let $ABCD$ be a unit square. Let $Q_1$ be the midpoint of $\overline{CD}$. For $i=1,2,\dots,$ let $P_i$ be the intersection of $\overline{AQ_i}$ and $\overline{BD}$, and let $Q_{i+1}$ be the foot of the perpendicular from $P_i$ to $\overline{CD}$. What is 

\begin\{equation*\}
\sum\_\{i=1\}\^\{\infty\} \text\{Area of \} \triangle DQ\_i P\_i \, ?
\end\{equation*\}


\$\textbf\{(A)\}\ \frac\{1\}\{6\} \qquad
\textbf\{(B)\}\ \frac\{1\}\{4\} \qquad
\textbf\{(C)\}\ \frac\{1\}\{3\} \qquad
\textbf\{(D)\}\ \frac\{1\}\{2\} \qquad
\textbf\{(E)\}\ 1\$\par \vspace{0.5em}\item For a certain positive integer $n$ less than $1000$, the decimal equivalent of $\frac{1}{n}$ is $0.\overline{abcdef}$, a repeating decimal of period of $6$, and the decimal equivalent of $\frac{1}{n+6}$ is $0.\overline{wxyz}$, a repeating decimal of period $4$. In which interval does $n$ lie?

$\textbf{(A)}\ [1,200]\qquad\textbf{(B)}\ [201,400]\qquad\textbf{(C)}\ [401,600]\qquad\textbf{(D)}\ [601,800]\qquad\textbf{(E)}\ [801,999]$\par \vspace{0.5em}\item What is the volume of the region in three-dimensional space defined by the inequalities $|x|+|y|+|z|\le1$ and $|x|+|y|+|z-1|\le1$?

$\textbf{(A)}\ \frac{1}{6}\qquad\textbf{(B)}\ \frac{1}{4}\qquad\textbf{(C)}\ \frac{1}{3}\qquad\textbf{(D)}\ \frac{1}{2}\qquad\textbf{(E)}\ 1$\par \vspace{0.5em}\item There are exactly $77,000$ ordered quadruplets $(a, b, c, d)$ such that $\gcd(a, b, c, d) = 77$ and $\operatorname{lcm}(a, b, c, d) = n$. What is the smallest possible value for $n$?

$\textbf{(A)}\ 13,860\qquad\textbf{(B)}\ 20,790\qquad\textbf{(C)}\ 21,560 \qquad\textbf{(D)}\ 27,720 \qquad\textbf{(E)}\ 41,580$\par \vspace{0.5em}\item The sequence $(a_n)$ is defined recursively by $a_0=1$, $a_1=\sqrt[19]{2}$, and $a_n=a_{n-1}a_{n-2}^2$ for $n\geq 2$. What is the smallest positive integer $k$ such that the product $a_1a_2\cdots a_k$ is an integer?

$\textbf{(A)}\ 17\qquad\textbf{(B)}\ 18\qquad\textbf{(C)}\ 19\qquad\textbf{(D)}\ 20\qquad\textbf{(E)}\ 21$\par \vspace{0.5em}\end{enumerate}\newpage\section*{2017 AMC1212A}\begin{enumerate}[label=\arabic*., itemsep=0.5em]\item Pablo buys popsicles for his friends. The store sells single popsicles for \$1 each, 3-popsicle boxes for \$2, and 5-popsicle boxes for \$3. What is the greatest number of popsicles that Pablo can buy with \$8?

$\textbf{(A)}\ 8\qquad\textbf{(B)}\ 11\qquad\textbf{(C)}\ 12\qquad\textbf{(D)}\ 13\qquad\textbf{(E)}\ 15$\par \vspace{0.5em}\item The sum of two nonzero real numbers is 4 times their product. What is the sum of the reciprocals of the two numbers?

$\textbf{(A)}\ 1\qquad\textbf{(B)}\ 2\qquad\textbf{(C)}\ 4\qquad\textbf{(D)}\ 8\qquad\textbf{(E)}\ 12$\par \vspace{0.5em}\item Ms. Carroll promised that anyone who got all the multiple choice questions right on the upcoming exam would receive an A on the exam. Which one of these statements necessarily follows logically?

$ \textbf{(A)}\ \text{ If Lewis did not receive an A, then he got all of the multiple choice questions wrong.} \\ \qquad\textbf{(B)}\ \text{ If Lewis did not receive an A, then he got at least one of the multiple choice questions wrong.} \\ \qquad\textbf{(C)}\ \text{ If Lewis got at least one of the multiple choice questions wrong, then he did not receive an A.} \\ \qquad\textbf{(D)}\ \text{ If Lewis received an A, then he got all of the multiple choice questions right.} \\ \qquad\textbf{(E)}\ \text{ If Lewis received an A, then he got at least one of the multiple choice questions right.} $\par \vspace{0.5em}\item Jerry and Silvia wanted to go from the southwest corner of a square field to the northeast corner. Jerry walked due east and then due north to reach the goal, but Silvia headed northeast and reached the goal walking in a straight line. Which of the following is closest to how much shorter Silvia's trip was, compared to Jerry's trip?

$\textbf{(A)}\ 30\%\qquad\textbf{(B)}\ 40\%\qquad\textbf{(C)}\ 50\%\qquad\textbf{(D)}\ 60\%\qquad\textbf{(E)}\ 70\%$\par \vspace{0.5em}\item At a gathering of $30$ people, there are $20$ people who all know each other and $10$ people who know no one. People who know each other hug, and people who do not know each other shake hands. How many handshakes occur?

$\textbf{(A)}\ 240\qquad\textbf{(B)}\ 245\qquad\textbf{(C)}\ 290\qquad\textbf{(D)}\ 480\qquad\textbf{(E)}\ 490$\par \vspace{0.5em}\item Joy has $30$ thin rods, one each of every integer length from $1 \text{ cm}$ through $30 \text{ cm}$. She places the rods with lengths $3 \text{ cm}$, $7 \text{ cm}$, and $15 \text{cm}$ on a table. She then wants to choose a fourth rod that she can put with these three to form a quadrilateral with positive area. How many of the remaining rods can she choose as the fourth rod?

$\textbf{(A)}\ 16 \qquad\textbf{(B)}\ 17 \qquad\textbf{(C)}\ 18 \qquad\textbf{(D)}\ 19  \qquad\textbf{(E)}\ 20$\par \vspace{0.5em}\item Define a function on the positive integers recursively by $f(1) = 2$, $f(n) = f(n-1) + 1$ if $n$ is even, and $f(n) = f(n-2) + 2$ if $n$ is odd and greater than $1$. What is $f(2017)$?

$ \textbf{(A)}\ 2017 \qquad\textbf{(B)}\ 2018 \qquad\textbf{(C)}\ 4034 \qquad\textbf{(D)}\ 4035 \qquad\textbf{(E)}\ 4036$\par \vspace{0.5em}\item The region consisting of all points in three-dimensional space within $3$ units of line segment $\overline{AB}$ has volume $216 \pi$. What is the length $AB$?

$ \textbf{(A)}\ 6 \qquad\textbf{(B)}\ 12 \qquad\textbf{(C)}\ 18 \qquad\textbf{(D)}\ 20 \qquad\textbf{(E)}\ 24$\par \vspace{0.5em}\item Let $S$ be the set of points $(x,y)$ in the coordinate plane such that two of the three quantities $3$, $x+2$, and $y-4$ are equal and the third of the three quantities is no greater than the common value. Which of the following is a correct description of $S$?

$ \textbf{(A)}\ \text{a single point} \qquad\textbf{(B)}\ \text{two intersecting lines} \\ \qquad\textbf{(C)}\ \text{three lines whose pairwise intersections are three distinct points} \\ \qquad\textbf{(D)}\ \text{a triangle}\qquad\textbf{(E)}\ \text{three rays with a common point} $\par \vspace{0.5em}\item Chlo chooses a real number uniformly at random from the interval $ [ 0,2017 ]$. Independently, Laurent chooses a real number uniformly at random from the interval $[ 0 , 4034 ]$. What is the probability that Laurent's number is greater than Chloe's number?  

$ \textbf{(A)}\ \dfrac{1}{2} \qquad\textbf{(B)}\ \dfrac{2}{3} \qquad\textbf{(C)}\ \dfrac{3}{4} \qquad\textbf{(D)}\ \dfrac{5}{6} \qquad\textbf{(E)}\ \dfrac{7}{8} $\par \vspace{0.5em}\item Claire adds the degree measures of the interior angles of a convex polygon and arrives at a sum of $2017$. She then discovers that she forgot to include one angle. What is the degree measure of the forgotten angle?

$\textbf{(A)}\ 37\qquad\textbf{(B)}\ 63\qquad\textbf{(C)}\ 117\qquad\textbf{(D)}\ 143\qquad\textbf{(E)}\ 163$\par \vspace{0.5em}\item There are $10$ horses, named Horse 1, Horse 2, $\ldots$, Horse 10. They get their names from how many minutes it takes them to run one lap around a circular race track: Horse $k$ runs one lap in exactly $k$ minutes. At time 0 all the horses are together at the starting point on the track. The horses start running in the same direction, and they keep running around the circular track at their constant speeds. The least time $S > 0$, in minutes, at which all $10$ horses will again simultaneously be at the starting point is $S = 2520$. Let  $T>0$ be the least time, in minutes, such that at least $5$ of the horses are again at the starting point. What is the sum of the digits of  $T$?

$\textbf{(A)}\ 2\qquad\textbf{(B)}\ 3\qquad\textbf{(C)}\ 4\qquad\textbf{(D)}\ 5\qquad\textbf{(E)}\ 6$\par \vspace{0.5em}\item Driving at a constant speed, Sharon usually takes $180$ minutes to drive from her house to her mother's house. One day Sharon begins the drive at her usual speed, but after driving $\frac{1}{3}$ of the way, she hits a bad snowstorm and reduces her speed by $20$ miles per hour. This time the trip takes her a total of $276$ minutes. How many miles is the drive from Sharon's house to her mother's house?

$\textbf{(A)}\ 132 \qquad\textbf{(B)}\ 135 \qquad\textbf{(C)}\ 138 \qquad\textbf{(D)}\ 141 \qquad\textbf{(E)}\ 144$\par \vspace{0.5em}\item Alice refuses to sit next to either Bob or Carla. Derek refuses to sit next to Eric. How many ways are there for the five of them to sit in a row of $5$ chairs under these conditions?

$\textbf{(A)}\ 12  \qquad \textbf{(B)}\ 16 \qquad\textbf{(C)}\ 28 \qquad\textbf{(D)}\ 32 \qquad\textbf{(E)}\ 40$\par \vspace{0.5em}\item Let $f(x) = \sin{x} + 2\cos{x} + 3\tan{x}$, using radian measure for the variable $x$. In what interval does the smallest positive value of $x$ for which $f(x) = 0$ lie?

$\textbf{(A)}\ (0,1)  \qquad \textbf{(B)}\ (1, 2) \qquad\textbf{(C)}\ (2, 3) \qquad\textbf{(D)}\ (3, 4) \qquad\textbf{(E)}\ (4,5)$\par \vspace{0.5em}\item In the figure below, semicircles with centers at $A$ and $B$ and with radii 2 and 1, respectively, are drawn in the interior of, and sharing bases with, a semicircle with diameter $JK$. The two smaller semicircles are externally tangent to each other and internally tangent to the largest semicircle. A circle centered at $P$ is drawn externally tangent to the two smaller semicircles and internally tangent to the largest semicircle. What is the radius of the circle centered at $P$?


\begin{center}
\begin{asy}
import olympiad;
import cse5;
size(5cm);
draw(arc((0,0),3,0,180));
draw(arc((2,0),1,0,180));
draw(arc((-1,0),2,0,180));
draw((-3,0)--(3,0));
pair P = (-1,0)+(2+6/7)*dir(36.86989);
draw(circle(P,6/7));
dot((-1,0)); dot((2,0)); dot(P);
\end{asy}
\end{center}


\$ \textbf\{(A)\}\ \frac\{3\}\{4\}
\qquad \textbf\{(B)\}\ \frac\{6\}\{7\}
\qquad\textbf\{(C)\}\ \frac\{\sqrt\{3\}\}\{2\}
\qquad\textbf\{(D)\}\ \frac\{5\}\{8\}\sqrt\{2\}
\qquad\textbf\{(E)\}\ \frac\{11\}\{12\} \$\par \vspace{0.5em}\item There are $24$ different complex numbers $z$ such that $z^{24}=1$. For how many of these is $z^6$ a real number?

$\textbf{(A)}\ 0 \qquad\textbf{(B)}\ 4 \qquad\textbf{(C)}\ 6 \qquad\textbf{(D)}\ 12 \qquad\textbf{(E)}\ 24$\par \vspace{0.5em}\item Let $S(n)$ equal the sum of the digits of positive integer $n$. For example, $S(1507) = 13$. For a particular positive integer $n$, $S(n) = 1274$. Which of the following could be the value of $S(n+1)$?

$\textbf{(A)}\ 1 \qquad\textbf{(B)}\ 3\qquad\textbf{(C)}\ 12\qquad\textbf{(D)}\ 1239\qquad\textbf{(E)}\ 1265$\par \vspace{0.5em}\item A square with side length $x$ is inscribed in a right triangle with sides of length $3$, $4$, and $5$ so that one vertex of the square coincides with the right-angle vertex of the triangle. A square with side length $y$ is inscribed in another right triangle with sides of length $3$, $4$, and $5$ so that one side of the square lies on the hypotenuse of the triangle. What is $\tfrac{x}{y}$?

$\textbf{(A)}\ \frac{12}{13} \qquad \textbf{(B)}\ \frac{35}{37} \qquad\textbf{(C)}\ 1 \qquad\textbf{(D)}\ \frac{37}{35} \qquad\textbf{(E)}\ \frac{13}{12}$\par \vspace{0.5em}\item How many ordered pairs $(a,b)$ such that $a$ is a positive real number and $b$ is an integer between $2$ and $200$, inclusive, satisfy the equation $(\log_b a)^{2017}=\log_b(a^{2017})?$

$\textbf{(A)}\ 198\qquad\textbf{(B)}\ 199\qquad\textbf{(C)}\ 398\qquad\textbf{(D)}\ 399\qquad\textbf{(E)}\ 597$\par \vspace{0.5em}\item A set $S$ is constructed as follows. To begin, $S = \{0,10\}$. Repeatedly, as long as possible, if $x$ is an integer root of some polynomial $a_{n}x^n + a_{n-1}x^{n-1} + ... + a_{1}x + a_0$ for some $n\geq{1}$, all of whose coefficients $a_i$ are elements of $S$, then $x$ is put into $S$. When no more elements can be added to $S$, how many elements does $S$ have?

$\textbf{(A)}\ 4 \qquad \textbf{(B)}\ 5 \qquad\textbf{(C)}\ 7 \qquad\textbf{(D)}\ 9 \qquad\textbf{(E)}\ 11$\par \vspace{0.5em}\item A square is drawn in the Cartesian coordinate plane with vertices at $(2, 2)$, $(-2, 2)$, $(-2, -2)$, $(2, -2)$. A particle starts at $(0,0)$. Every second it moves with equal probability to one of the eight lattice points (points with integer coordinates) closest to its current position, independently of its previous moves. In other words, the probability is $1/8$ that the particle will move from $(x, y)$ to each of $(x, y + 1)$, $(x + 1, y + 1)$, $(x + 1, y)$, $(x + 1, y - 1)$, $(x, y - 1)$, $(x - 1, y - 1)$, $(x - 1, y)$, or $(x - 1, y + 1)$. The particle will eventually hit the square for the first time, either at one of the 4 corners of the square or at one of the 12 lattice points in the interior of one of the sides of the square. The probability that it will hit at a corner rather than at an interior point of a side is $m/n$, where $m$ and $n$ are relatively prime positive integers. What is $m + n$?

$\textbf{(A)}\ 4 \qquad \textbf{(B)}\ 5 \qquad\textbf{(C)}\ 7 \qquad\textbf{(D)}\ 15 \qquad\textbf{(E)}\ 39$\par \vspace{0.5em}\item For certain real numbers $a$, $b$, and $c$, the polynomial 
\begin\{equation*\}
g(x) = x\^3 + ax\^2 + x + 10
\end\{equation*\}
has three distinct roots, and each root of $g(x)$ is also a root of the polynomial 
\begin\{equation*\}
f(x) = x\^4 + x\^3 + bx\^2 + 100x + c.
\end\{equation*\}
What is $f(1)$?

$\textbf{(A)}\ -9009 \qquad\textbf{(B)}\ -8008 \qquad\textbf{(C)}\ -7007 \qquad\textbf{(D)}\ -6006 \qquad\textbf{(E)}\ -5005$\par \vspace{0.5em}\item Quadrilateral $ABCD$ is inscribed in circle $O$ and has side lengths $AB=3, BC=2, CD=6$, and $DA=8$. Let $X$ and $Y$ be points on $\overline{BD}$ such that $\frac{DX}{BD} = \frac{1}{4}$ and $\frac{BY}{BD} = \frac{11}{36}$. Let $E$ be the intersection of line $AX$ and the line through $Y$ parallel to $\overline{AD}$. Let $F$ be the intersection of line $CX$ and the line through $E$ parallel to $\overline{AC}$. Let $G$ be the point on circle $O$ other than $C$ that lies on line $CX$. What is $XF\cdot XG$?

$\textbf{(A) }17\qquad\textbf{(B) }\frac{59 - 5\sqrt{2}}{3}\qquad\textbf{(C) }\frac{91 - 12\sqrt{3}}{4}\qquad\textbf{(D) }\frac{67 - 10\sqrt{2}}{3}\qquad\textbf{(E) }18$\par \vspace{0.5em}\item The vertices $V$ of a centrally symmetric hexagon in the complex plane are given by 
\begin\{equation*\}
V=\left\\{   \sqrt\{2\}i,-\sqrt\{2\}i, \frac\{1\}\{\sqrt\{8\}\}(1+i),\frac\{1\}\{\sqrt\{8\}\}(-1+i),\frac\{1\}\{\sqrt\{8\}\}(1-i),\frac\{1\}\{\sqrt\{8\}\}(-1-i) \right\\}.
\end\{equation*\}
 For each $j$, $1\leq j\leq 12$, an element $z_j$ is chosen from $V$ at random, independently of the other choices. Let $P={\prod}_{j=1}^{12}z_j$ be the product of the $12$ numbers selected. What is the probability that $P=-1$?

$\textbf{(A) } \dfrac{5\cdot11}{3^{10}} \qquad \textbf{(B) } \dfrac{5^2\cdot11}{2\cdot3^{10}} \qquad \textbf{(C) } \dfrac{5\cdot11}{3^{9}} \qquad \textbf{(D) } \dfrac{5\cdot7\cdot11}{2\cdot3^{10}} \qquad \textbf{(E) } \dfrac{2^2\cdot5\cdot11}{3^{10}}$\par \vspace{0.5em}\end{enumerate}\newpage\section*{2017 AMC1212B}\begin{enumerate}[label=\arabic*., itemsep=0.5em]\item Kymbrea's comic book collection currently has $30$ comic books in it, and she is adding to her collection at the rate of $2$ comic books per month. LaShawn's collection currently has $10$ comic books in it, and he is adding to his collection at the rate of $6$ comic books per month. After how many months will LaShawn's collection have twice as many comic books as Kymbrea's?

$\textbf{(A)}\ 1\qquad\textbf{(B)}\ 4\qquad\textbf{(C)}\ 5\qquad\textbf{(D)}\ 20\qquad\textbf{(E)}\ 25$\par \vspace{0.5em}\item Real numbers $x$, $y$, and $z$ satify the inequalities
$0<x<1$, $-1<y<0$, and $1<z<2$.
Which of the following numbers is necessarily positive?

$\textbf{(A)}\ y+x^2\qquad\textbf{(B)}\ y+xz\qquad\textbf{(C)}\ y+y^2\qquad\textbf{(D)}\ y+2y^2\qquad\textbf{(E)}\ y+z$\par \vspace{0.5em}\item Supposed that $x$ and $y$ are nonzero real numbers such that $\frac{3x+y}{x-3y}=-2$. What is the value of $\frac{x+3y}{3x-y}$?

$\textbf{(A)}\ -3\qquad\textbf{(B)}\ -1\qquad\textbf{(C)}\ 1\qquad\textbf{(D)}\ 2\qquad\textbf{(E)}\ 3$\par \vspace{0.5em}\item Samia set off on her bicycle to visit her friend, traveling at an average speed of $17$ kilometers per hour. When she had gone half the distance to her friend's house, a tire went flat, and she walked the rest of the way at $5$ kilometers per hour. In all it took her $44$ minutes to reach her friend's house. In kilometers rounded to the nearest tenth, how far did Samia walk?

$\textbf{(A)}\ 2.0\qquad\textbf{(B)}\ 2.2\qquad\textbf{(C)}\ 2.8\qquad\textbf{(D)}\ 3.4\qquad\textbf{(E)}\ 4.4$\par \vspace{0.5em}\item The data set $[6,19,33,33,39,41,41,43,51,57]$ has median $Q_2 = 40$, first quartile $Q_1 = 33$, and third quartile $Q_3=43$. An outlier in a data set is a value that is more than $1.5$ times the interquartile range below the first quartile $(Q_1)$ or more than $1.5$ times the interquartile range above the third quartile $(Q_3)$, where the interquartile range is defined as $Q_3 - Q_1$. How many outliers does this data set have?

$\textbf{(A)}\ 0\qquad\textbf{(B)}\ 1\qquad\textbf{(C)}\ 2\qquad\textbf{(D)}\ 3\qquad\textbf{(E)}\ 4$\par \vspace{0.5em}\item The circle having $(0,0)$ and $(8,6)$ as the endpoints of a diameter intersects the $x$-axis at a second point. What is the $x$-coordinate of this point? 

$\textbf{(A)}\ 4\sqrt{2} \qquad \textbf{(B)}\ 6\qquad \textbf{(C)}\ 5\sqrt{2}\qquad \textbf{(D)}\ 8\qquad \textbf{(E)}\ 6\sqrt{2}$\par \vspace{0.5em}\item The functions $\sin(x)$ and $\cos(x)$ are periodic with least period $2\pi$. What is the least period of the function $\cos(\sin(x))$?

$\textbf{(A)}\ \frac{\pi}{2}\qquad\textbf{(B)}\ \pi\qquad\textbf{(C)}\ 2\pi \qquad\textbf{(D)}\ 4\pi \qquad\textbf{(E)}$ It's not periodic.\par \vspace{0.5em}\item The ratio of the short side of a certain rectangle to the long side is equal to the ratio of the long side to the diagonal. What is the square of the ratio of the short side to the long side of this rectangle?

$\textbf{(A)}\ \frac{\sqrt{3}-1}{2}\qquad\textbf{(B)}\ \frac{1}{2}\qquad\textbf{(C)}\ \frac{\sqrt{5}-1}{2} \qquad\textbf{(D)}\ \frac{\sqrt{2}}{2} \qquad\textbf{(E)}\ \frac{\sqrt{6}-1}{2}$\par \vspace{0.5em}\item A circle has center $(-10,-4)$ and radius $13$. Another circle has center $(3,9)$ and radius $\sqrt{65}$. The line passing through the two points of intersection of the two circles has equation $x + y = c$. What is $c$?

$\textbf{(A)}\ 3\qquad\textbf{(B)}\ 3\sqrt{3}\qquad\textbf{(C)}\ 4\sqrt{2}\qquad\textbf{(D)}\ 6\qquad\textbf{(E)}\ \frac{13}{2}$\par \vspace{0.5em}\item At Typico High School, $60\%$ of the students like dancing, and the rest dislike it. Of those who like dancing, $80\%$ say that they like it, and the rest say that they dislike it. Of those who dislike dancing, $90\%$ say that they dislike it, and the rest say that they like it. What fraction of students who say they dislike dancing actually like it?

$\textbf{(A)}\ 10\%\qquad\textbf{(B)}\ 12\%\qquad\textbf{(C)}\ 20\%\qquad\textbf{(D)}\ 25\%\qquad\textbf{(E)}\ 33\frac{1}{3}\%$\par \vspace{0.5em}\item Call a positive integer $monotonous$ if it is a one-digit number or its digits, when read from left to right, form either a strictly increasing or a strictly decreasing sequence. For example, $3$, $23578$, and $987620$ are monotonous, but $88$, $7434$, and $23557$ are not. How many monotonous positive integers are there?

$\textbf{(A)}\ 1024\qquad\textbf{(B)}\ 1524\qquad\textbf{(C)}\ 1533\qquad\textbf{(D)}\ 1536\qquad\textbf{(E)}\ 2048$\par \vspace{0.5em}\item What is the sum of the roots of $z^{12}=64$ that have a positive real part? 

$\textbf{(A)}\ 2 \qquad \textbf{(B)}\ 4 \qquad \textbf{(C)}\ \sqrt{2}+2\sqrt{3} \qquad \textbf{(D)}\ 2\sqrt{2}+\sqrt{6} \qquad \textbf{(E)}\ (1+\sqrt{3}) + (1+\sqrt{3})i$\par \vspace{0.5em}\item In the figure below, $3$ of the $6$ disks are to be painted blue, $2$ are to be painted red, and $1$ is to be painted green. Two paintings that can be obtained from one another by a rotation or a reflection of the entire figure are considered the same. How many different paintings are possible?


\begin{center}
\begin{asy}
import olympiad;
import cse5;
size(100);
pair A, B, C, D, E, F;
A = (0,0);
B = (1,0);
C = (2,0);
D = rotate(60, A)*B;
E = B + D;
F = rotate(60, A)*C;
draw(Circle(A, 0.5));
draw(Circle(B, 0.5));
draw(Circle(C, 0.5));
draw(Circle(D, 0.5));
draw(Circle(E, 0.5));
draw(Circle(F, 0.5));
\end{asy}
\end{center}


$\textbf{(A) } 6 \qquad \textbf{(B) } 8 \qquad \textbf{(C) } 9 \qquad \textbf{(D) } 12 \qquad \textbf{(E) } 15$\par \vspace{0.5em}\item An ice-cream novelty item consists of a cup in the shape of a 4-inch-tall frustum of a right circular cone, with a 2-inch-diameter base at the bottom and a 4-inch-diameter base at the top, packed solid with ice cream, together with a solid cone of ice cream of height 4 inches, whose base, at the bottom, is the top base of the frustum. What is the total volume of the ice cream, in cubic inches? 

$\textbf{(A)}\ 8\pi \qquad \textbf{(B)}\ \frac{28\pi}{3} \qquad \textbf{(C)}\ 12\pi \qquad \textbf{(D)}\ 14\pi \qquad \textbf{(E)}\ \frac{44\pi}{3}$\par \vspace{0.5em}\item Let $ABC$ be an equilateral triangle. Extend side $\overline{AB}$ beyond $B$ to a point $B'$ so that $BB'=3 \cdot AB$. Similarly, extend side $\overline{BC}$ beyond $C$ to a point $C'$ so that $CC'=3 \cdot BC$, and extend side $\overline{CA}$ beyond $A$ to a point $A'$ so that $AA'=3 \cdot CA$. What is the ratio of the area of $\triangle A'B'C'$ to the area of $\triangle ABC$?

$\textbf{(A)}\ 9\qquad\textbf{(B)}\ 16\qquad\textbf{(C)}\ 25\qquad\textbf{(D)}\ 36\qquad\textbf{(E)}\ 37$\par \vspace{0.5em}\item The number $21!=51,090,942,171,709,440,000$ has over $60,000$ positive integer divisors. One of them is chosen at random. What is the probability that it is odd?

$\textbf{(A)}\ \frac{1}{21} \qquad \textbf{(B)}\ \frac{1}{19} \qquad \textbf{(C)}\ \frac{1}{18} \qquad \textbf{(D)}\ \frac{1}{2} \qquad \textbf{(E)}\ \frac{11}{21}$\par \vspace{0.5em}\item A coin is biased in such a way that on each toss the probability of heads is $\frac{2}{3}$ and the probability of tails is $\frac{1}{3}$. The outcomes of the tosses are independent. A player has the choice of playing Game A or Game B. In Game A she tosses the coin three times and wins if all three outcomes are the same. In Game B she tosses the coin four times and wins if both the outcomes of the first and second tosses are the same and the outcomes of the third and fourth tosses are the same. How do the chances of winning Game A compare to the chances of winning Game B?

$\textbf{(A)}$ The probability of winning Game A is $\frac{4}{81}$ less than the probability of winning Game B.

$\textbf{(B)}$ The probability of winning Game A is $\frac{2}{81}$ less than the probability of winning Game B.

$\textbf{(C)}$ The probabilities are the same.

$\textbf{(D)}$ The probability of winning Game A is $\frac{2}{81}$ greater than the probability of winning Game B.

$\textbf{(E)}$ The probability of winning Game A is $\frac{4}{81}$ greater than the probability of winning Game B.\par \vspace{0.5em}\item The diameter $AB$ of a circle of radius $2$ is extended to a point $D$ outside the circle so that $BD=3$. Point $E$ is chosen so that $ED=5$ and line $ED$ is perpendicular to line $AD$. Segment $AE$ intersects the circle at a point $C$ between $A$ and $E$. What is the area of \$\triangle 
ABC\$?

$\textbf{(A)}\ \frac{120}{37}\qquad\textbf{(B)}\ \frac{140}{39}\qquad\textbf{(C)}\ \frac{145}{39}\qquad\textbf{(D)}\ \frac{140}{37}\qquad\textbf{(E)}\ \frac{120}{31}$\par \vspace{0.5em}\item Let $N=123456789101112\dots4344$ be the $79$-digit number that is formed by writing the integers from $1$ to $44$ in order, one after the other. What is the remainder when $N$ is divided by $45$?

$\textbf{(A)}\ 1\qquad\textbf{(B)}\ 4\qquad\textbf{(C)}\ 9\qquad\textbf{(D)}\ 18\qquad\textbf{(E)}\ 44$\par \vspace{0.5em}\item Real numbers $x$ and $y$ are chosen independently and uniformly at random from the interval $(0,1)$. What is the probability that $\lfloor\log_2x\rfloor=\lfloor\log_2y\rfloor$?

$\textbf{(A)}\ \frac{1}{8}\qquad\textbf{(B)}\ \frac{1}{6}\qquad\textbf{(C)}\ \frac{1}{4}\qquad\textbf{(D)}\ \frac{1}{3}\qquad\textbf{(E)}\ \frac{1}{2}$\par \vspace{0.5em}\item Last year Isabella took $7$ math tests and received $7$ different scores, each an integer between $91$ and $100$, inclusive. After each test she noticed that the average of her test scores was an integer. Her score on the seventh test was $95$. What was her score on the sixth test?

$\textbf{(A)}\ 92\qquad\textbf{(B)}\ 94\qquad\textbf{(C)}\ 96\qquad\textbf{(D)}\ 98\qquad\textbf{(E)}\ 100$\par \vspace{0.5em}\item Abby, Bernardo, Carl, and Debra play a game in which each of them starts with four coins. The game consists of four rounds. In each round, four balls are placed in an urn---one green, one red, and two white. The players each draw a ball at random without replacement. Whoever gets the green ball gives one coin to whoever gets the red ball. What is the probability that, at the end of the fourth round, each of the players has four coins?

$\textbf{(A)}\ \frac{7}{576} \qquad \textbf{(B)}\ \frac{5}{192} \qquad \textbf{(C)}\ \frac{1}{36} \qquad \textbf{(D)}\ \frac{5}{144} \qquad\textbf{(E)}\ \frac{7}{48}$\par \vspace{0.5em}\item The graph of $y=f(x)$, where $f(x)$ is a polynomial of degree $3$, contains points $A(2,4)$, $B(3,9)$, and $C(4,16)$. Lines $AB$, $AC$, and $BC$ intersect the graph again at points $D$, $E$, and $F$, respectively, and the sum of the $x$-coordinates of $D$, $E$, and $F$ is $24$. What is $f(0)$?
$\textbf{(A)}\ -2 \qquad \textbf{(B)}\ 0 \qquad \textbf{(C)}\ 2 \qquad \textbf{(D)}\ \frac{24}{5} \qquad\textbf{(E)}\ 8$\par \vspace{0.5em}\item Quadrilateral $ABCD$ has right angles at $B$ and $C$, $\triangle ABC \sim \triangle BCD$, and $AB > BC$. There is a point $E$ in the interior of $ABCD$ such that $\triangle ABC \sim \triangle CEB$ and the area of $\triangle AED$ is $17$ times the area of $\triangle CEB$. What is $\frac{AB}{BC}$?

$\textbf{(A)}\ 1 + \sqrt{2} \qquad \textbf{(B)}\ 2 + \sqrt{2} \qquad \textbf{(C)}\ \sqrt{17} \qquad \textbf{(D)}\ 2 + \sqrt{5} \qquad\textbf{(E)}\ 1 + 2\sqrt{3}$\par \vspace{0.5em}\item A set of $n$ people participate in an online video basketball tournament. Each person may be a member of any number of $5$-player teams, but no teams may have exactly the same $5$ members. The site statistics show a curious fact: The average, over all subsets of size $9$ of the set of $n$ participants, of the number of complete teams whose members are among those 9 people is equal to the reciprocal of the average, over all subsets of size $8$ of the set of $n$ participants, of the number of complete teams whose members are among those $8$ people. How many values $n$, $9 \leq n \leq 2017$, can be the number of participants?

$\textbf{(A)}\ 477 \qquad \textbf{(B)}\ 482 \qquad \textbf{(C)}\ 487 \qquad \textbf{(D)}\ 557 \qquad\textbf{(E)}\ 562$\par \vspace{0.5em}\end{enumerate}\newpage\section*{2018 AMC1212A}\begin{enumerate}[label=\arabic*., itemsep=0.5em]\item A large urn contains $100$ balls, of which $36 \%$ are red and the rest are blue. How many of the blue balls must be removed so that the percentage of red balls in the urn will be $72 \%$? (No red balls are to be removed.)

\$ \textbf\{(A)\}\ 28 \qquad\textbf\{(B)\}\  32 \qquad\textbf\{(C)\}\  36 \qquad\textbf\{(D)\}\ 
 50 \qquad\textbf\{(E)\}\ 64 \$\par \vspace{0.5em}\item While exploring a cave, Carl comes across a collection of $5$-pound rocks worth $\$14$ each, $4$-pound rocks worth $\$11$ each, and $1$-pound rocks worth $\$2$ each. There are at least $20$ of each size. He can carry at most $18\$ pounds. What is the maximum value, in dollars, of the rocks he can carry out of the cave?

$\textbf{(A) } 48 \qquad \textbf{(B) } 49 \qquad \textbf{(C) } 50 \qquad \textbf{(D) } 51 \qquad \textbf{(E) } 52 $\par \vspace{0.5em}\item How many ways can a student schedule $3$ mathematics courses -- algebra, geometry, and number theory -- in a $6$-period day if no two mathematics courses can be taken in consecutive periods? (What courses the student takes during the other $3$ periods is of no concern here.)

$\textbf{(A) }3\qquad\textbf{(B) }6\qquad\textbf{(C) }12\qquad\textbf{(D) }18\qquad\textbf{(E) }24$\par \vspace{0.5em}\item Alice, Bob, and Charlie were on a hike and were wondering how far away the nearest town was. When Alice said, "We are at least $6$ miles away," Bob replied, "We are at most $5$ miles away." Charlie then remarked, "Actually the nearest town is at most $4$ miles away." It turned out that none of the three statements were true. Let $d$ be the distance in miles to the nearest town. Which of the following intervals is the set of all possible values of $d$?

$\textbf{(A) }   (0,4)   \qquad        \textbf{(B) }   (4,5)   \qquad    \textbf{(C) }   (4,6)   \qquad   \textbf{(D) }  (5,6)  \qquad  \textbf{(E) }   (5,\infty) $\par \vspace{0.5em}\item What is the sum of all possible values of $k$ for which the polynomials $x^2 - 3x + 2$ and $x^2 - 5x + k$ have a root in common?

$\textbf{(A) }3 \qquad\textbf{(B) }4 \qquad\textbf{(C) }5 \qquad\textbf{(D) }6 \qquad\textbf{(E) }10 \qquad$\par \vspace{0.5em}\item For positive integers $m$ and $n$ such that $m+10<n+1$, both the mean and the median of the set $\{m, m+4, m+10, n+1, n+2, 2n\}$ are equal to $n$. What is $m+n$?

$\textbf{(A) }20\qquad\textbf{(B) }21\qquad\textbf{(C) }22\qquad\textbf{(D) }23\qquad\textbf{(E) }24$\par \vspace{0.5em}\item For how many (not necessarily positive) integer values of $n$ is the value of $4000\cdot \left(\tfrac{2}{5}\right)^n$ an integer?

\$
\textbf\{(A) \}3 \qquad
\textbf\{(B) \}4 \qquad
\textbf\{(C) \}6 \qquad
\textbf\{(D) \}8 \qquad
\textbf\{(E) \}9 \qquad
\$\par \vspace{0.5em}\item All of the triangles in the diagram below are similar to isosceles triangle $ABC$, in which $AB=AC$. Each of the $7$ smallest triangles has area $1,$ and $\triangle ABC$ has area $40$. What is the area of trapezoid $DBCE$?


\begin{center}
\begin{asy}
import olympiad;
import cse5;
unitsize(5);
dot((0,0));
dot((60,0));
dot((50,10));
dot((10,10));
dot((30,30));
draw((0,0)--(60,0)--(50,10)--(30,30)--(10,10)--(0,0));
draw((10,10)--(50,10));
label("$B$",(0,0),SW);
label("$C$",(60,0),SE);
label("$E$",(50,10),E);
label("$D$",(10,10),W);
label("$A$",(30,30),N);
draw((10,10)--(15,15)--(20,10)--(25,15)--(30,10)--(35,15)--(40,10)--(45,15)--(50,10));
draw((15,15)--(45,15));
\end{asy}
\end{center}


$\textbf{(A) }   16   \qquad        \textbf{(B) }   18   \qquad    \textbf{(C) }   20   \qquad   \textbf{(D) }  22 \qquad  \textbf{(E) }   24 $\par \vspace{0.5em}\item Which of the following describes the largest subset of values of $y$ within the closed interval $[0,\pi]$ for which

\begin\{equation*\}
\sin(x+y)\leq \sin(x)+\sin(y)
\end\{equation*\}
for every $x$ between $0$ and $\pi$, inclusive?

$\textbf{(A) } y=0 \qquad \textbf{(B) } 0\leq y\leq \frac{\pi}{4} \qquad \textbf{(C) } 0\leq y\leq \frac{\pi}{2} \qquad \textbf{(D) } 0\leq y\leq \frac{3\pi}{4} \qquad \textbf{(E) } 0\leq y\leq \pi $\par \vspace{0.5em}\item How many ordered pairs of real numbers $(x,y)$ satisfy the following system of equations?

\begin\{align*\}
x+3y\&=3 \\
\big||x|-|y|\big|\&=1
\end\{align*\}

\$\textbf\{(A) \} 1 \qquad 
\textbf\{(B) \} 2 \qquad 
\textbf\{(C) \} 3 \qquad 
\textbf\{(D) \} 4 \qquad 
\textbf\{(E) \} 8 \$\par \vspace{0.5em}\item A paper triangle with sides of lengths $3,4,$ and $5$ inches, as shown, is folded so that point $A$ falls on point $B$. What is the length in inches of the crease?

\begin{center}
\begin{asy}
import olympiad;
import cse5;
draw((0,0)--(4,0)--(4,3)--(0,0));
label("$A$", (0,0), SW);
label("$B$", (4,3), NE);
label("$C$", (4,0), SE);
label("$4$", (2,0), S);
label("$3$", (4,1.5), E);
label("$5$", (2,1.5), NW);
fill(origin--(0,0)--(4,3)--(4,0)--cycle, gray);
\end{asy}
\end{center}

$\textbf{(A) }   1+\frac12 \sqrt2   \qquad        \textbf{(B) }   \sqrt3   \qquad    \textbf{(C) }   \frac74   \qquad   \textbf{(D) }  \frac{15}{8} \qquad  \textbf{(E) }   2 $\par \vspace{0.5em}\item Let $S$ be a set of $6$ integers taken from $\{1,2,\dots,12\}$ with the property that if $a$ and $b$ are elements of $S$ with $a<b$, then $b$ is not a multiple of $a$. What is the least possible value of an element in $S$?

$\textbf{(A)}\ 2\qquad\textbf{(B)}\ 3\qquad\textbf{(C)}\ 4\qquad\textbf{(D)}\ 5\qquad\textbf{(E)}\ 7$\par \vspace{0.5em}\item How many nonnegative integers can be written in the form 
\begin\{equation*\}
a\_7\cdot3\^7+a\_6\cdot3\^6+a\_5\cdot3\^5+a\_4\cdot3\^4+a\_3\cdot3\^3+a\_2\cdot3\^2+a\_1\cdot3\^1+a\_0\cdot3\^0,
\end\{equation*\}

where $a_i\in \{-1,0,1\}$ for $0\le i \le 7$?

\$\textbf\{(A) \} 512 \qquad 
\textbf\{(B) \} 729 \qquad 
\textbf\{(C) \} 1094 \qquad 
\textbf\{(D) \} 3281 \qquad 
\textbf\{(E) \} 59,048 \$\par \vspace{0.5em}\item The solution to the equation $\log_{3x} 4 = \log_{2x} 8$, where $x$ is a positive real number other than $\frac{1}{3}$ or $\frac{1}{2}$, can be written as $\frac {p}{q}$ where $p$ and $q$ are relatively prime positive integers. What is $p + q$?

\$\textbf\{(A) \} 5   \qquad    
\textbf\{(B) \} 13   \qquad    
\textbf\{(C) \} 17   \qquad   
\textbf\{(D) \} 31 \qquad  
\textbf\{(E) \} 35 \$\par \vspace{0.5em}\item A scanning code consists of a $7 \times 7$ grid of squares, with some of its squares colored black and the rest colored white. There must be at least one square of each color in this grid of $49$ squares. A scanning code is called $\textit{symmetric}$ if its look does not change when the entire square is rotated by a multiple of $90 ^{\circ}$ counterclockwise around its center, nor when it is reflected across a line joining opposite corners or a line joining midpoints of opposite sides. What is the total number of possible symmetric scanning codes?

$\textbf{(A)} \text{ 510} \qquad \textbf{(B)} \text{ 1022} \qquad \textbf{(C)} \text{ 8190} \qquad \textbf{(D)} \text{ 8192} \qquad \textbf{(E)} \text{ 65,534}$\par \vspace{0.5em}\item Which of the following describes the set of values of $a$ for which the curves $x^2+y^2=a^2$ and $y=x^2-a$ in the real $xy$-plane intersect at exactly $3$ points?

\$
\textbf\{(A) \}a=\frac14 \qquad
\textbf\{(B) \}\frac14 < a < \frac12 \qquad
\textbf\{(C) \}a>\frac14 \qquad
\textbf\{(D) \}a=\frac12 \qquad
\textbf\{(E) \}a>\frac12 \qquad
\$\par \vspace{0.5em}\item Farmer Pythagoras has a field in the shape of a right triangle. The right triangle's legs have lengths $3$ and $4$ units. In the corner where those sides meet at a right angle, he leaves a small unplanted square $S$ so that from the air it looks like the right angle symbol. The rest of the field is planted. The shortest distance from $S$ to the hypotenuse is $2$ units. What fraction of the field is planted?


\begin{center}
\begin{asy}
import olympiad;
import cse5;
/* Edited by MRENTHUSIASM */
size(160);
pair A, B, C, D, F;
A = origin;
B = (4,0);
C = (0,3);
D = (2/7,2/7);
F = foot(D,B,C);
fill(A--(2/7,0)--D--(0,2/7)--cycle, lightgray);
draw(A--B--C--cycle);
draw((2/7,0)--D--(0,2/7));
label("$4$", midpoint(A--B), N);
label("$3$", midpoint(A--C), E);
label("$2$", midpoint(D--F), SE);
label("$S$", midpoint(A--D));
draw(D--F, dashed);
\end{asy}
\end{center}


$\textbf{(A) }   \frac{25}{27}   \qquad        \textbf{(B) }   \frac{26}{27}   \qquad    \textbf{(C) }   \frac{73}{75}   \qquad   \textbf{(D) } \frac{145}{147} \qquad  \textbf{(E) }   \frac{74}{75} $\par \vspace{0.5em}\item Triangle $ABC$ with $AB=50$ and $AC=10$ has area $120$. Let $D$ be the midpoint of $\overline{AB}$, and let $E$ be the midpoint of $\overline{AC}$. The angle bisector of $\angle BAC$ intersects $\overline{DE}$ and $\overline{BC}$ at $F$ and $G$, respectively. What is the area of quadrilateral $FDBG$?

\$
\textbf\{(A) \}60 \qquad
\textbf\{(B) \}65 \qquad
\textbf\{(C) \}70 \qquad
\textbf\{(D) \}75 \qquad
\textbf\{(E) \}80 \qquad
\$\par \vspace{0.5em}\item Let $A$ be the set of positive integers that have no prime factors other than $2$, $3$, or $5$. The infinite sum 
\begin\{equation*\}
\frac\{1\}\{1\} + \frac\{1\}\{2\} + \frac\{1\}\{3\} + \frac\{1\}\{4\} + \frac\{1\}\{5\} + \frac\{1\}\{6\} + \frac\{1\}\{8\} + \frac\{1\}\{9\} + \frac\{1\}\{10\} + \frac\{1\}\{12\} + \frac\{1\}\{15\} + \frac\{1\}\{16\} + \frac\{1\}\{18\} + \frac\{1\}\{20\} + \cdots
\end\{equation*\}
of the reciprocals of the elements of $A$ can be expressed as $\frac{m}{n}$, where $m$ and $n$ are relatively prime positive integers. What is $m+n$?

$\textbf{(A) } 16 \qquad \textbf{(B) } 17 \qquad \textbf{(C) } 19 \qquad \textbf{(D) } 23 \qquad \textbf{(E) } 36$\par \vspace{0.5em}\item Triangle $ABC$ is an isosceles right triangle with $AB=AC=3$. Let $M$ be the midpoint of hypotenuse $\overline{BC}$. Points $I$ and $E$ lie on sides $\overline{AC}$ and $\overline{AB}$, respectively, so that $AI>AE$ and $AIME$ is a cyclic quadrilateral. Given that triangle $EMI$ has area $2$, the length $CI$ can be written as $\frac{a-\sqrt{b}}{c}$, where $a$, $b$, and $c$ are positive integers and $b$ is not divisible by the square of any prime. What is the value of $a+b+c$?

\$
\textbf\{(A) \}9 \qquad
\textbf\{(B) \}10 \qquad
\textbf\{(C) \}11 \qquad
\textbf\{(D) \}12 \qquad
\textbf\{(E) \}13 \qquad
\$\par \vspace{0.5em}\item Which of the following polynomials has the greatest real root?

$\textbf{(A) }   x^{19}+2018x^{11}+1   \qquad        \textbf{(B) }   x^{17}+2018x^{11}+1   \qquad    \textbf{(C) }   x^{19}+2018x^{13}+1   \qquad   \textbf{(D) }  x^{17}+2018x^{13}+1 \qquad  \textbf{(E) }   2019x+2018 $\par \vspace{0.5em}\item The solutions to the equations $z^2=4+4\sqrt{15}i$ and $z^2=2+2\sqrt 3i,$ where $i=\sqrt{-1},$ form the vertices of a parallelogram in the complex plane. The area of this parallelogram can be written in the form $p\sqrt q-r\sqrt s,$ where $p,$ $q,$ $r,$ and $s$ are positive integers and neither $q$ nor $s$ is divisible by the square of any prime number. What is $p+q+r+s?$

\$\textbf\{(A) \} 20 \qquad 
\textbf\{(B) \} 21 \qquad 
\textbf\{(C) \} 22 \qquad 
\textbf\{(D) \} 23 \qquad 
\textbf\{(E) \} 24 \$\par \vspace{0.5em}\item In $\triangle PAT,$ $\angle P=36^{\circ},$ $\angle A=56^{\circ},$ and $PA=10.$ Points $U$ and $G$ lie on sides $\overline{TP}$ and $\overline{TA},$ respectively, so that $PU=AG=1.$ Let $M$ and $N$ be the midpoints of segments $\overline{PA}$ and $\overline{UG},$ respectively. What is the degree measure of the acute angle formed by lines $MN$ and $PA?$

\$\textbf\{(A) \} 76 \qquad 
\textbf\{(B) \} 77 \qquad 
\textbf\{(C) \} 78 \qquad 
\textbf\{(D) \} 79 \qquad 
\textbf\{(E) \} 80 \$\par \vspace{0.5em}\item Alice, Bob, and Carol play a game in which each of them chooses a real number between $0$ and $1.$ The winner of the game is the one whose number is between the numbers chosen by the other two players. Alice announces that she will choose her number uniformly at random from all the numbers between $0$ and $1,$ and Bob announces that he will choose his number uniformly at random from all the numbers between $\tfrac{1}{2}$ and $\tfrac{2}{3}.$ Armed with this information, what number should Carol choose to maximize her chance of winning?

\$
\textbf\{(A) \}\frac\{1\}\{2\}\qquad
\textbf\{(B) \}\frac\{13\}\{24\} \qquad
\textbf\{(C) \}\frac\{7\}\{12\} \qquad
\textbf\{(D) \}\frac\{5\}\{8\} \qquad
\textbf\{(E) \}\frac\{2\}\{3\}\qquad
\$\par \vspace{0.5em}\item For a positive integer $n$ and nonzero digits $a$, $b$, and $c$, let $A_n$ be the $n$-digit integer each of whose digits is equal to $a$; let $B_n$ be the $n$-digit integer each of whose digits is equal to $b$, and let $C_n$ be the $2n$-digit (not $n$-digit) integer each of whose digits is equal to $c$. What is the greatest possible value of $a + b + c$ for which there are at least two values of $n$ such that $C_n - B_n = A_n^2$?

$\textbf{(A) } 12 \qquad \textbf{(B) } 14 \qquad \textbf{(C) } 16 \qquad \textbf{(D) } 18 \qquad \textbf{(E) } 20$\par \vspace{0.5em}\end{enumerate}\newpage\section*{2018 AMC1212B}\begin{enumerate}[label=\arabic*., itemsep=0.5em]\item Kate bakes a $20$-inch by $18$-inch pan of cornbread. The cornbread is cut into pieces that measure $2$ inches by $2$ inches. How many pieces of cornbread does the pan contain?

$\textbf{(A) } 90 \qquad \textbf{(B) } 100 \qquad \textbf{(C) } 180 \qquad \textbf{(D) } 200 \qquad \textbf{(E) } 360$\par \vspace{0.5em}\item Sam drove $96$ miles in $90$ minutes. His average speed during the first $30$ minutes was $60$ mph (miles per hour), and his average speed during the second $30$ minutes was $65$ mph. What was his average speed, in mph, during the last $30$ minutes?

\$
\textbf\{(A) \} 64 \qquad
\textbf\{(B) \} 65 \qquad
\textbf\{(C) \} 66 \qquad
\textbf\{(D) \} 67 \qquad
\textbf\{(E) \} 68
\$\par \vspace{0.5em}\item A line with slope $2$ intersects a line with slope $6$ at the point $(40,30)$. What is the distance between the $x$-intercepts of these two lines? 

$\textbf{(A) } 5 \qquad \textbf{(B) } 10 \qquad \textbf{(C) } 20 \qquad \textbf{(D) } 25 \qquad \textbf{(E) } 50$\par \vspace{0.5em}\item A circle has a chord of length $10$, and the distance from the center of the circle to the chord is $5$. What is the area of the circle?

\$
\textbf\{(A) \}25\pi \qquad
\textbf\{(B) \}50\pi \qquad
\textbf\{(C) \}75\pi \qquad
\textbf\{(D) \}100\pi \qquad
\textbf\{(E) \}125\pi \qquad
\$\par \vspace{0.5em}\item How many subsets of $\{2,3,4,5,6,7,8,9\}$ contain at least one prime number?

\$
\textbf\{(A) \} 128 \qquad
\textbf\{(B) \} 192 \qquad
\textbf\{(C) \} 224 \qquad
\textbf\{(D) \} 240 \qquad
\textbf\{(E) \} 256
\$\par \vspace{0.5em}\item Suppose $S$ cans of soda can be purchased from a vending machine for $Q$ quarters. Which of the following expressions describes the number of cans of soda that can be purchased for $D$ dollars, where $1$ dollar is worth $4$ quarters?

$\textbf{(A) } \frac{4DQ}{S} \qquad \textbf{(B) } \frac{4DS}{Q} \qquad \textbf{(C) } \frac{4Q}{DS} \qquad \textbf{(D) } \frac{DQ}{4S} \qquad \textbf{(E) } \frac{DS}{4Q}$\par \vspace{0.5em}\item What is the value of 
\begin\{equation*\}
\log\_37\cdot\log\_59\cdot\log\_711\cdot\log\_913\cdots\log\_\{21\}25\cdot\log\_\{23\}27?
\end\{equation*\}

$\textbf{(A) } 3 \qquad \textbf{(B) } 3\log_{7}23 \qquad \textbf{(C) } 6 \qquad \textbf{(D) } 9 \qquad \textbf{(E) } 10 $\par \vspace{0.5em}\item Line segment $\overline{AB}$ is a diameter of a circle with $AB = 24$. Point $C$, not equal to $A$ or $B$, lies on the circle. As point $C$ moves around the circle, the centroid (center of mass) of $\triangle ABC$ traces out a closed curve missing two points. To the nearest positive integer, what is the area of the region bounded by this curve?

$\textbf{(A) } 25 \qquad \textbf{(B) } 38  \qquad \textbf{(C) } 50  \qquad \textbf{(D) } 63 \qquad \textbf{(E) } 75  $\par \vspace{0.5em}\item What is

\begin\{equation*\}
\sum\^\{100\}\_\{i=1\} \sum\^\{100\}\_\{j=1\} (i+j) ?
\end\{equation*\}


\$ \textbf\{(A) \}100\{,\}100 \qquad
\textbf\{(B) \}500\{,\}500\qquad
\textbf\{(C) \}505\{,\}000 \qquad
\textbf\{(D) \}1\{,\}001\{,\}000 \qquad
\textbf\{(E) \}1\{,\}010\{,\}000 \qquad \$\par \vspace{0.5em}\item A list of $2018$ positive integers has a unique mode, which occurs exactly $10$ times. What is the least number of distinct values that can occur in the list?

\$ \textbf\{(A) \}202 \qquad
\textbf\{(B) \}223 \qquad
\textbf\{(C) \}224 \qquad
\textbf\{(D) \}225 \qquad
\textbf\{(E) \}234 \qquad \$\par \vspace{0.5em}\item A closed box with a square base is to be wrapped with a square sheet of wrapping paper. The box is centered on the wrapping paper with the vertices of the base lying on the midlines of the square sheet of paper, as shown in the figure on the left. The four corners of the wrapping paper are to be folded up over the sides and brought together to meet at the center of the top of the box, point $A$ in the figure on the right. The box has base length $w$ and height $h$. What is the area of the sheet of wrapping paper?


\begin{center}
\begin{asy}
import olympiad;
import cse5;
size(270pt);
defaultpen(fontsize(10pt));
filldraw(((3,3)--(-3,3)--(-3,-3)--(3,-3)--cycle),lightgrey);
dot((-3,3));
label("$A$",(-3,3),NW);
draw((1,3)--(-3,-1),dashed+linewidth(.5));
draw((-1,3)--(3,-1),dashed+linewidth(.5));
draw((-1,-3)--(3,1),dashed+linewidth(.5));
draw((1,-3)--(-3,1),dashed+linewidth(.5));
draw((0,2)--(2,0)--(0,-2)--(-2,0)--cycle,linewidth(.5));
draw((0,3)--(0,-3),linetype("2.5 2.5")+linewidth(.5));
draw((3,0)--(-3,0),linetype("2.5 2.5")+linewidth(.5));
label('$w$',(-1,-1),SW);
label('$w$',(1,-1),SE);
draw((4.5,0)--(6.5,2)--(8.5,0)--(6.5,-2)--cycle);
draw((4.5,0)--(8.5,0));
draw((6.5,2)--(6.5,-2));
label("$A$",(6.5,0),NW);
dot((6.5,0));
\end{asy}
\end{center}


$\textbf{(A) } 2(w+h)^2 \qquad \textbf{(B) } \frac{(w+h)^2}2 \qquad \textbf{(C) } 2w^2+4wh \qquad \textbf{(D) } 2w^2 \qquad \textbf{(E) } w^2h $\par \vspace{0.5em}\item Side $\overline{AB}$ of $\triangle ABC$ has length $10$. The bisector of angle $A$ meets $\overline{BC}$ at $D$, and $CD = 3$. The set of all possible values of $AC$ is an open interval $(m,n)$. What is $m+n$?

\$\textbf\{(A) \}16 \qquad
\textbf\{(B) \}17 \qquad
\textbf\{(C) \}18 \qquad
\textbf\{(D) \}19 \qquad
\textbf\{(E) \}20 \qquad\$\par \vspace{0.5em}\item Square $ABCD$ has side length $30$. Point $P$ lies inside the square so that $AP = 12$ and $BP = 26$. The centroids of $\triangle{ABP}$, $\triangle{BCP}$, $\triangle{CDP}$, and $\triangle{DAP}$ are the vertices of a convex quadrilateral. What is the area of that quadrilateral? 


\begin{center}
\begin{asy}
import olympiad;
import cse5;
unitsize(120);
pair B = (0, 0), A = (0, 1), D = (1, 1), C = (1, 0), P = (1/4, 2/3);
draw(A--B--C--D--cycle);
dot(P);
defaultpen(fontsize(10pt));
draw(A--P--B);
draw(C--P--D);
label("$A$", A, W);
label("$B$", B, W);
label("$C$", C, E);
label("$D$", D, E);
label("$P$", P, N*1.5+E*0.5);
dot(A);
dot(B);
dot(C);
dot(D);
\end{asy}
\end{center}


$\textbf{(A) }100\sqrt{2}\qquad\textbf{(B) }100\sqrt{3}\qquad\textbf{(C) }200\qquad\textbf{(D) }200\sqrt{2}\qquad\textbf{(E) }200\sqrt{3}$\par \vspace{0.5em}\item Joey and Chloe and their daughter Zoe all have the same birthday. Joey is $1$ year older than Chloe, and Zoe is exactly $1$ year old today. Today is the first of the $9$ birthdays on which Chloe's age will be an integral multiple of Zoe's age. What will be the sum of the two digits of Joey's age the next time his age is a multiple of Zoe's age?

\$
\textbf\{(A) \}7 \qquad
\textbf\{(B) \}8 \qquad
\textbf\{(C) \}9 \qquad
\textbf\{(D) \}10 \qquad
\textbf\{(E) \}11 \qquad
\$\par \vspace{0.5em}\item How many odd positive $3$-digit integers are divisible by $3$ but do not contain the digit $3$?

$\textbf{(A) } 96 \qquad \textbf{(B) } 97 \qquad \textbf{(C) } 98 \qquad \textbf{(D) } 102 \qquad \textbf{(E) } 120 $\par \vspace{0.5em}\item The solutions to the equation $(z+6)^8=81$ are connected in the complex plane to form a convex regular polygon, three of whose vertices are labeled $A,B,$ and $C$. What is the least possible area of $\triangle ABC?$

$\textbf{(A) } \frac{1}{6}\sqrt{6} \qquad \textbf{(B) } \frac{3}{2}\sqrt{2}-\frac{3}{2} \qquad \textbf{(C) } 2\sqrt3-3\sqrt2 \qquad \textbf{(D) } \frac{1}{2}\sqrt{2} \qquad \textbf{(E) } \sqrt 3-1$\par \vspace{0.5em}\item Let $p$ and $q$ be positive integers such that 
\begin\{equation*\}
\frac\{5\}\{9\} < \frac\{p\}\{q\} < \frac\{4\}\{7\}
\end\{equation*\}
and $q$ is as small as possible. What is $q-p$?

$\textbf{(A) } 7 \qquad \textbf{(B) } 11 \qquad \textbf{(C) } 13 \qquad \textbf{(D) } 17 \qquad \textbf{(E) } 19 $\par \vspace{0.5em}\item A function $f$ is defined recursively by $f(1)=f(2)=1$ and 
\begin\{equation*\}
f(n)=f(n-1)-f(n-2)+n
\end\{equation*\}
for all integers $n \geq 3$. What is $f(2018)$?

$\textbf{(A) } 2016 \qquad \textbf{(B) } 2017 \qquad \textbf{(C) } 2018 \qquad \textbf{(D) } 2019 \qquad \textbf{(E) } 2020$\par \vspace{0.5em}\item Mary chose an even $4$-digit number $n$. She wrote down all the divisors of $n$ in increasing order from left to right: $1,2,\ldots,\dfrac{n}{2},n$. At some moment Mary wrote $323$ as a divisor of $n$. What is the smallest possible value of the next divisor written to the right of $323$?

$\textbf{(A) } 324 \qquad \textbf{(B) } 330 \qquad \textbf{(C) } 340 \qquad \textbf{(D) } 361 \qquad \textbf{(E) } 646$\par \vspace{0.5em}\item Let $ABCDEF$ be a regular hexagon with side length $1$. Denote by $X$, $Y$, and $Z$ the midpoints of sides $\overline {AB}$, $\overline{CD}$, and $\overline{EF}$, respectively. What is the area of the convex hexagon whose interior is the intersection of the interiors of $\triangle ACE$ and $\triangle XYZ$?

$\textbf{(A)}\ \frac {3}{8}\sqrt{3} \qquad \textbf{(B)}\ \frac {7}{16}\sqrt{3} \qquad \textbf{(C)}\ \frac {15}{32}\sqrt{3} \qquad  \textbf{(D)}\ \frac {1}{2}\sqrt{3} \qquad \textbf{(E)}\ \frac {9}{16}\sqrt{3} $\par \vspace{0.5em}\item In $\triangle{ABC}$ with side lengths $AB = 13$, $AC = 12$, and $BC = 5$, let $O$ and $I$ denote the circumcenter and incenter, respectively. A circle with center $M$ is tangent to the legs $AC$ and $BC$ and to the circumcircle of $\triangle{ABC}$. What is the area of $\triangle{MOI}$?

$\textbf{(A)}\ \frac52\qquad\textbf{(B)}\ \frac{11}{4}\qquad\textbf{(C)}\ 3\qquad\textbf{(D)}\ \frac{13}{4}\qquad\textbf{(E)}\ \frac72$\par \vspace{0.5em}\item Consider polynomials $P(x)$ of degree at most $3$, each of whose coefficients is an element of $\{0, 1, 2, 3, 4, 5, 6, 7, 8, 9\}$. How many such polynomials satisfy $P(-1) = -9$?

$\textbf{(A) } 110 \qquad \textbf{(B) } 143 \qquad \textbf{(C) } 165 \qquad \textbf{(D) } 220 \qquad \textbf{(E) } 286 $\par \vspace{0.5em}\item Ajay is standing at point $A$ near Pontianak, Indonesia, $0^\circ$ latitude and $110^\circ \text{ E}$ longitude. Billy is standing at point $B$ near Big Baldy Mountain, Idaho, USA, $45^\circ \text{ N}$ latitude and $115^\circ \text{ W}$ longitude. Assume that Earth is a perfect sphere with center $C.$ What is the degree measure of $\angle ACB?$

\$\textbf\{(A) \}105 \qquad
\textbf\{(B) \}112\frac\{1\}\{2\} \qquad
\textbf\{(C) \}120 \qquad
\textbf\{(D) \}135 \qquad
\textbf\{(E) \}150 \qquad\$\par \vspace{0.5em}\item Let $\lfloor x \rfloor$ denote the greatest integer less than or equal to $x$. How many real numbers $x$ satisfy the equation $x^2 + 10,000\lfloor x \rfloor = 10,000x$?

$\textbf{(A) } 197 \qquad \textbf{(B) } 198 \qquad \textbf{(C) } 199 \qquad \textbf{(D) } 200 \qquad \textbf{(E) } 201$\par \vspace{0.5em}\item Circles $\omega_1$, $\omega_2$, and $\omega_3$ each have radius $4$ and are placed in the plane so that each circle is externally tangent to the other two.  Points $P_1$, $P_2$, and $P_3$ lie on $\omega_1$, $\omega_2$, and $\omega_3$ respectively such that $P_1P_2=P_2P_3=P_3P_1$ and line $P_iP_{i+1}$ is tangent to $\omega_i$ for each $i=1,2,3$, where $P_4 = P_1$.  See the figure below.  The area of $\triangle P_1P_2P_3$ can be written in the form $\sqrt{a}+\sqrt{b}$ for positive integers $a$ and $b$.  What is $a+b$?


\begin{center}
\begin{asy}
import olympiad;
import cse5;
unitsize(12);
pair A = (0, 8/sqrt(3)), B = rotate(-120)*A, C = rotate(120)*A;
real theta = 41.5;
pair P1 = rotate(theta)*(2+2*sqrt(7/3), 0), P2 = rotate(-120)*P1, P3 = rotate(120)*P1;
filldraw(P1--P2--P3--cycle, gray(0.9));
draw(Circle(A, 4));
draw(Circle(B, 4));
draw(Circle(C, 4));
dot(P1);
dot(P2);
dot(P3);
defaultpen(fontsize(10pt));
label("$P_1$", P1, E*1.5);
label("$P_2$", P2, SW*1.5);
label("$P_3$", P3, N);
label("$\omega_1$", A, W*17);
label("$\omega_2$", B, E*17);
label("$\omega_3$", C, W*17);
\end{asy}
\end{center}


$\textbf{(A) }546\qquad\textbf{(B) }548\qquad\textbf{(C) }550\qquad\textbf{(D) }552\qquad\textbf{(E) }554$\par \vspace{0.5em}\end{enumerate}\newpage\section*{2019 AMC1212A}\begin{enumerate}[label=\arabic*., itemsep=0.5em]\item The area of a pizza with radius $4$ inches is $N$ percent larger than the area of a pizza with radius $3$ inches. What is the integer closest to $N$?

$\textbf{(A) } 25 \qquad\textbf{(B) } 33 \qquad\textbf{(C) } 44\qquad\textbf{(D) } 66 \qquad\textbf{(E) } 78$\par \vspace{0.5em}\item Suppose $a$ is $150\%$ of $b$. What percent of $a$ is $3b$?

$\textbf{(A) } 50 \qquad \textbf{(B) } 66+\frac{2}{3} \qquad \textbf{(C) } 150 \qquad \textbf{(D) } 200 \qquad \textbf{(E) } 450$\par \vspace{0.5em}\item A box contains $28$ red balls, $20$ green balls, $19$ yellow balls, $13$ blue balls, $11$ white balls, and $9$ black balls. What is the minimum number of balls that must be drawn from the box without replacement to guarantee that at least $15$ balls of a single color will be drawn?

$\textbf{(A) } 75 \qquad\textbf{(B) } 76 \qquad\textbf{(C) } 79 \qquad\textbf{(D) } 84 \qquad\textbf{(E) } 91$\par \vspace{0.5em}\item What is the greatest number of consecutive integers whose sum is $45$?

$\textbf{(A) } 9 \qquad\textbf{(B) } 25 \qquad\textbf{(C) } 45 \qquad\textbf{(D) } 90 \qquad\textbf{(E) } 120$\par \vspace{0.5em}\item Two lines with slopes $\dfrac{1}{2}$ and $2$ intersect at $(2,2)$. What is the area of the triangle enclosed by these two lines and the line $x+y=10$?

$\textbf{(A) } 4 \qquad\textbf{(B) } 4\sqrt{2} \qquad\textbf{(C) } 6 \qquad\textbf{(D) } 8 \qquad\textbf{(E) } 6\sqrt{2}$\par \vspace{0.5em}\item The figure below shows line $\ell$ with a regular, infinite, recurring pattern of squares and line segments.


\begin{center}
\begin{asy}
import olympiad;
import cse5;
size(300);
defaultpen(linewidth(0.8));
real r = 0.35;
path P = (0,0)--(0,1)--(1,1)--(1,0), Q = (1,1)--(1+r,1+r);
path Pp = (0,0)--(0,-1)--(1,-1)--(1,0), Qp = (-1,-1)--(-1-r,-1-r);
for(int i=0;i <= 4;i=i+1)
\{
draw(shift((4*i,0)) * P);
draw(shift((4*i,0)) * Q);
\}
for(int i=1;i <= 4;i=i+1)
\{
draw(shift((4*i-2,0)) * Pp);
draw(shift((4*i-1,0)) * Qp);
\}
draw((-1,0)--(18.5,0));
\end{asy}
\end{center}


How many of the following four kinds of rigid motion transformations of the plane in which this figure is drawn, other than the identity transformation, will transform this figure into itself?
*some rotation around a point of line $\ell$
*some translation in the direction parallel to line $\ell$
*the reflection across line $\ell$
*some reflection across a line perpendicular to line $\ell$
$\textbf{(A) } 0 \qquad\textbf{(B) } 1 \qquad\textbf{(C) } 2 \qquad\textbf{(D) } 3 \qquad\textbf{(E) } 4$\par \vspace{0.5em}\item Melanie computes the mean $\mu$, the median $M$, and the modes of the $365$ values that are the dates in the months of $2019$. Thus her data consist of $12$ $1\text{s}$, $12$ $2\text{s}$, . . . , $12$ $28\text{s}$, $11$ $29\text{s}$, $11$ $30\text{s}$, and $7$ $31\text{s}$. Let $d$ be the median of the modes. Which of the following statements is true?

$\textbf{(A) } \mu < d < M \qquad\textbf{(B) } M < d < \mu \qquad\textbf{(C) } d = M =\mu \qquad\textbf{(D) } d < M < \mu \qquad\textbf{(E) } d < \mu < M$\par \vspace{0.5em}\item For a set of four distinct lines in a plane, there are exactly $N$ distinct points that lie on two or more of the lines. What is the sum of all possible values of $N$?

$\textbf{(A) } 14 \qquad \textbf{(B) } 16 \qquad \textbf{(C) } 18 \qquad \textbf{(D) } 19 \qquad \textbf{(E) } 21$\par \vspace{0.5em}\item A sequence of numbers is defined recursively by $a_1 = 1$, $a_2 = \frac{3}{7}$, and

\begin\{equation*\}
a\_n=\frac\{a\_\{n-2\} \cdot a\_\{n-1\}\}\{2a\_\{n-2\} - a\_\{n-1\}\}
\end\{equation*\}
for all $n \geq 3$. Then $a_{2019}$ can be written as $\frac{p}{q}$, where $p$ and $q$ are relatively prime positive integers. What is $p+q ?$

$\textbf{(A) } 2020 \qquad\textbf{(B) } 4039 \qquad\textbf{(C) } 6057 \qquad\textbf{(D) } 6061 \qquad\textbf{(E) } 8078$\par \vspace{0.5em}\item The figure below shows $13$ circles of radius $1$ within a larger circle. All the intersections occur at points of tangency. What is the area of the region, shaded in the figure, inside the larger circle but outside all the circles of radius $1$?


\begin{center}
\begin{asy}
import olympiad;
import cse5;
unitsize(20);filldraw(circle((0,0),2*sqrt(3)+1),rgb(0.5,0.5,0.5));filldraw(circle((-2,0),1),white);filldraw(circle((0,0),1),white);filldraw(circle((2,0),1),white);filldraw(circle((1,sqrt(3)),1),white);filldraw(circle((3,sqrt(3)),1),white);filldraw(circle((-1,sqrt(3)),1),white);filldraw(circle((-3,sqrt(3)),1),white);filldraw(circle((1,-1*sqrt(3)),1),white);filldraw(circle((3,-1*sqrt(3)),1),white);filldraw(circle((-1,-1*sqrt(3)),1),white);filldraw(circle((-3,-1*sqrt(3)),1),white);filldraw(circle((0,2*sqrt(3)),1),white);filldraw(circle((0,-2*sqrt(3)),1),white);
\end{asy}
\end{center}


$\textbf{(A) } 4 \pi \sqrt{3} \qquad\textbf{(B) } 7 \pi \qquad\textbf{(C) } \pi\left(3\sqrt{3} +2\right) \qquad\textbf{(D) } 10 \pi \left(\sqrt{3} - 1\right) \qquad\textbf{(E) } \pi\left(\sqrt{3} + 6\right)$\par \vspace{0.5em}\item For some positive integer $k$, the repeating base-$k$ representation of the (base-ten) fraction $\frac{7}{51}$ is $0.\overline{23}_k = 0.232323..._k$. What is $k$?

$\textbf{(A) } 13 \qquad\textbf{(B) } 14 \qquad\textbf{(C) } 15 \qquad\textbf{(D) } 16 \qquad\textbf{(E) } 17$\par \vspace{0.5em}\item Positive real numbers $x \neq 1$ and $y \neq 1$ satisfy $\log_2{x} = \log_y{16}$ and $xy = 64$. What is $(\log_2{\tfrac{x}{y}})^2$?

$\textbf{(A) } \frac{25}{2} \qquad\textbf{(B) } 20 \qquad\textbf{(C) } \frac{45}{2} \qquad\textbf{(D) } 25 \qquad\textbf{(E) } 32$\par \vspace{0.5em}\item How many ways are there to paint each of the integers $2, 3, \dots, 9$ either red, green, or blue so that each number has a different color from each of its proper divisors?

$\textbf{(A)}\ 144\qquad\textbf{(B)}\ 216\qquad\textbf{(C)}\ 256\qquad\textbf{(D)}\ 384\qquad\textbf{(E)}\ 432$\par \vspace{0.5em}\item For a certain complex number $c$, the polynomial

\begin\{equation*\}
P(x) = (x\^2 - 2x + 2)(x\^2 - cx + 4)(x\^2 - 4x + 8)
\end\{equation*\}
has exactly 4 distinct roots. What is $|c|$?

$\textbf{(A) } 2 \qquad \textbf{(B) } \sqrt{6} \qquad \textbf{(C) } 2\sqrt{2} \qquad \textbf{(D) } 3 \qquad \textbf{(E) } \sqrt{10}$\par \vspace{0.5em}\item Positive real numbers $a$ and $b$ have the property that

\begin\{equation*\}
\sqrt\{\log\{a\}\} + \sqrt\{\log\{b\}\} + \log \sqrt\{a\} + \log \sqrt\{b\} = 100
\end\{equation*\}


and all four terms on the left are positive integers, where $\log$ denotes the base-$10$ logarithm. What is $ab$?

$\textbf{(A) }   10^{52}   \qquad        \textbf{(B) }   10^{100}   \qquad    \textbf{(C) }   10^{144}   \qquad   \textbf{(D) }  10^{164} \qquad  \textbf{(E) }   10^{200} $\par \vspace{0.5em}\item The numbers $1,2,\dots,9$ are randomly placed into the $9$ squares of a $3 \times 3$ grid. Each square gets one number, and each of the numbers is used once. What is the probability that the sum of the numbers in each row and each column is odd?

$\textbf{(A) }\frac{1}{21}\qquad\textbf{(B) }\frac{1}{14}\qquad\textbf{(C) }\frac{5}{63}\qquad\textbf{(D) }\frac{2}{21}\qquad\textbf{(E) } \frac17$\par \vspace{0.5em}\item Let $s_k$ denote the sum of the $\textit{k}$th powers of the roots of the polynomial $x^3-5x^2+8x-13$. In particular, $s_0=3$, $s_1=5$, and $s_2=9$. Let $a$, $b$, and $c$ be real numbers such that $s_{k+1} = a \, s_k + b \, s_{k-1} + c \, s_{k-2}$ for $k = 2$, $3$, $....$ What is $a+b+c$?

$\textbf{(A)} \; -6 \qquad \textbf{(B)} \; 0 \qquad \textbf{(C)} \; 6 \qquad \textbf{(D)} \; 10 \qquad \textbf{(E)} \; 26$\par \vspace{0.5em}\item A sphere with center $O$ has radius $6$. A triangle with sides of length $15, 15,$ and $24$ is situated in space so that each of its sides is tangent to the sphere. What is the distance between $O$ and the plane determined by the triangle?

\$
\textbf\{(A) \}2\sqrt\{3\}\qquad
\textbf\{(B) \}4\qquad
\textbf\{(C) \}3\sqrt\{2\}\qquad
\textbf\{(D) \}2\sqrt\{5\}\qquad
\textbf\{(E) \}5\qquad
\$\par \vspace{0.5em}\item In $\triangle ABC$ with integer side lengths,

\begin\{equation*\}
\cos A=\frac\{11\}\{16\}, \qquad \cos B= \frac\{7\}\{8\}, \qquad \text\{and\} \qquad\cos C=-\frac\{1\}\{4\}.
\end\{equation*\}

What is the least possible perimeter for $\triangle ABC$?

$\textbf{(A) } 9 \qquad \textbf{(B) } 12 \qquad \textbf{(C) } 23 \qquad \textbf{(D) } 27 \qquad \textbf{(E) } 44$\par \vspace{0.5em}\item Real numbers between $0$ and $1$, inclusive, are chosen in the following manner. A fair coin is flipped. If it lands heads, then it is flipped again and the chosen number is $0$ if the second flip is heads and $1$ if the second flip is tails. On the other hand, if the first coin flip is tails, then the number is chosen uniformly at random from the closed interval $[0,1]$. Two random numbers $x$ and $y$ are chosen independently in this manner. What is the probability that $|x-y| > \tfrac{1}{2}$?

$\textbf{(A) } \frac{1}{3} \qquad \textbf{(B) } \frac{7}{16} \qquad \textbf{(C) } \frac{1}{2} \qquad \textbf{(D) } \frac{9}{16} \qquad \textbf{(E) } \frac{2}{3}$\par \vspace{0.5em}\item Let 
\begin\{equation*\}
z=\frac\{1+i\}\{\sqrt\{2\}\}.
\end\{equation*\}
What is 
\begin\{equation*\}
\left(z\^\{1\^2\}+z\^\{2\^2\}+z\^\{3\^2\}+\dots+z\^\{\{12\}\^2\}\right) \cdot \left(\frac\{1\}\{z\^\{1\^2\}\}+\frac\{1\}\{z\^\{2\^2\}\}+\frac\{1\}\{z\^\{3\^2\}\}+\dots+\frac\{1\}\{z\^\{\{12\}\^2\}\}\right)?
\end\{equation*\}


$\textbf{(A) } 18 \qquad \textbf{(B) } 72-36\sqrt2 \qquad \textbf{(C) } 36 \qquad \textbf{(D) } 72 \qquad \textbf{(E) } 72+36\sqrt2$\par \vspace{0.5em}\item Circles $\omega$ and $\gamma$, both centered at $O$, have radii $20$ and $17$, respectively. Equilateral triangle $ABC$, whose interior lies in the interior of $\omega$ but in the exterior of $\gamma$, has vertex $A$ on $\omega$, and the line containing side $\overline{BC}$ is tangent to $\gamma$. Segments $\overline{AO}$ and $\overline{BC}$ intersect at $P$, and $\dfrac{BP}{CP} = 3$. Then $AB$ can be written in the form $\dfrac{m}{\sqrt{n}} - \dfrac{p}{\sqrt{q}}$ for positive integers $m$, $n$, $p$, $q$ with $\gcd(m,n) = \gcd(p,q) = 1$. What is $m+n+p+q$?
$\phantom{  }$

$\textbf{(A) } 42 \qquad \textbf{(B) }86 \qquad \textbf{(C) } 92 \qquad \textbf{(D) } 114 \qquad \textbf{(E) } 130$\par \vspace{0.5em}\item Define binary operations $\diamondsuit$ and $\heartsuit$ by 
\begin\{equation*\}
a \, \diamondsuit \, b = a\^\{\log\_\{7\}(b)\} \qquad \text\{and\} \qquad a  \, \heartsuit \, b = a\^\{\frac\{1\}\{\log\_\{7\}(b)\}\}
\end\{equation*\}
for all real numbers $a$ and $b$ for which these expressions are defined. The sequence $(a_n)$ is defined recursively by $a_3 = 3\, \heartsuit\, 2$ and 
\begin\{equation*\}
a\_n = (n\, \heartsuit\, (n-1)) \,\diamondsuit\, a\_\{n-1\}
\end\{equation*\}
for all integers $n \geq 4$. To the nearest integer, what is $\log_{7}(a_{2019})$?

$\textbf{(A) } 8 \qquad  \textbf{(B) } 9 \qquad \textbf{(C) } 10 \qquad \textbf{(D) } 11 \qquad \textbf{(E) } 12$\par \vspace{0.5em}\item For how many integers $n$ between $1$ and $50$, inclusive, is

\begin\{equation*\}
\frac\{(n\^2-1)!\}\{(n!)\^n\}
\end\{equation*\}

an integer? (Recall that $0! = 1$.)

$\textbf{(A) } 31 \qquad \textbf{(B) } 32 \qquad \textbf{(C) } 33 \qquad \textbf{(D) } 34 \qquad \textbf{(E) } 35$\par \vspace{0.5em}\item Let $\triangle A_0B_0C_0$ be a triangle whose angle measures are exactly $59.999^\circ$, $60^\circ$, and $60.001^\circ$. For each positive integer $n$, define $A_n$ to be the foot of the altitude from $A_{n-1}$ to line $B_{n-1}C_{n-1}$. Likewise, define $B_n$ to be the foot of the altitude from $B_{n-1}$ to line $A_{n-1}C_{n-1}$, and $C_n$ to be the foot of the altitude from $C_{n-1}$ to line $A_{n-1}B_{n-1}$. What is the least positive integer $n$ for which $\triangle A_nB_nC_n$ is obtuse?

$\textbf{(A) } 10 \qquad \textbf{(B) }11 \qquad \textbf{(C) } 13\qquad \textbf{(D) } 14 \qquad \textbf{(E) } 15$\par \vspace{0.5em}\end{enumerate}\newpage\section*{2019 AMC1212B}\begin{enumerate}[label=\arabic*., itemsep=0.5em]\item Alicia had two containers. The first was $\tfrac{5}{6}$ full of water and the second was empty. She poured all the water from the first container into the second container, at which point the second container was $\tfrac{3}{4}$ full of water. What is the ratio of the volume of the first container to the volume of the second container?

$\textbf{(A) } \frac{5}{8} \qquad \textbf{(B) } \frac{4}{5} \qquad \textbf{(C) } \frac{7}{8} \qquad \textbf{(D) } \frac{9}{10} \qquad \textbf{(E) } \frac{11}{12}$\par \vspace{0.5em}\item Consider the statement, "If $n$ is not prime, then $n-2$ is prime." Which of the following values of $n$ is a counterexample to this statement?

$\textbf{(A) } 11 \qquad \textbf{(B) } 15 \qquad \textbf{(C) } 19 \qquad \textbf{(D) } 21 \qquad \textbf{(E) } 27$\par \vspace{0.5em}\item Which one of the following rigid transformations (isometries) maps the line segment $\overline{AB}$ onto the line segment $\overline{A'B'}$ so that the image of $A(-2,1)$ is $A'(2,-1)$ and the image of $B(-1,4)$ is $B'(1,-4)?$

$\textbf{(A) } $ reflection in the $y$-axis

$\textbf{(B) } $ counterclockwise rotation around the origin by $90^{\circ}$

$\textbf{(C) } $ translation by $3$ units to the right and $5$ units down

$\textbf{(D) } $ reflection in the $x$-axis

$\textbf{(E) } $ clockwise rotation about the origin by $180^{\circ}$\par \vspace{0.5em}\item A positive integer $n$ satisfies the equation $(n+1)!+(n+2)!=440\cdot n!$. What is the sum of the digits of $n$?

$\textbf{(A) } 2 \qquad \textbf{(B) } 5 \qquad \textbf{(C) } 10\qquad \textbf{(D) } 12 \qquad \textbf{(E) } 15$\par \vspace{0.5em}\item Each piece of candy in a store costs a whole number of cents. Casper has exactly enough money to buy either $12$ pieces of red candy, $14$ pieces of green candy, $15$ pieces of blue candy, or $n$ pieces of purple candy. A piece of purple candy costs $20$ cents. What is the smallest possible value of $n$?

$\textbf{(A) } 18 \qquad \textbf{(B) } 21 \qquad \textbf{(C) } 24\qquad \textbf{(D) } 25 \qquad \textbf{(E) } 28$\par \vspace{0.5em}\item In a given plane, points $A$ and $B$ are $10$ units apart. How many points $C$ are there in the plane such that the perimeter of $\triangle ABC$ is $50$ units and the area of $\triangle ABC$ is $100$ square units?

$\textbf{(A) }0\qquad\textbf{(B) }2\qquad\textbf{(C) }4\qquad\textbf{(D) }8\qquad\textbf{(E) }\text{infinitely many}$\par \vspace{0.5em}\item What is the sum of all real numbers $x$ for which the median of the numbers $4,6,8,17,$ and $x$ is equal to the mean of those five numbers?

$\textbf{(A) } -5 \qquad\textbf{(B) } 0 \qquad\textbf{(C) } 5 \qquad\textbf{(D) } \frac{15}{4} \qquad\textbf{(E) } \frac{35}{4}$\par \vspace{0.5em}\item Let $f(x) = x^{2}(1-x)^{2}$. What is the value of the sum


\begin\{equation*\}
f \left(\frac\{1\}\{2019\} \right)-f  \left(\frac\{2\}\{2019\} \right)+f \left(\frac\{3\}\{2019\} \right)-f \left(\frac\{4\}\{2019\} \right)+\cdots + f \left(\frac\{2017\}\{2019\} \right) - f \left(\frac\{2018\}\{2019\} \right)?
\end\{equation*\}


$\textbf{(A) }0\qquad\textbf{(B) }\frac{1}{2019^{4}}\qquad\textbf{(C) }\frac{2018^{2}}{2019^{4}}\qquad\textbf{(D) }\frac{2020^{2}}{2019^{4}}\qquad\textbf{(E) }1$\par \vspace{0.5em}\item For how many integral values of $x$ can a triangle of positive area be formed having side lengths \$
\log\_\{2\} x, \log\_\{4\} x, 3\$?

$\textbf{(A) } 57\qquad \textbf{(B) } 59\qquad \textbf{(C) } 61\qquad \textbf{(D) } 62\qquad \textbf{(E) } 63$\par \vspace{0.5em}\item The figure below is a map showing $12$ cities and $17$ roads connecting certain pairs of cities. Paula wishes to travel along exactly $13$ of those roads, starting at city $A$ and ending at city $L,$ without traveling along any portion of a road more than once. (Paula is allowed to visit a city more than once.)


\begin{center}
\begin{asy}
import olympiad;
import cse5;
import olympiad;
unitsize(50);
for (int i = 0; i < 3; ++i) \{
for (int j = 0; j < 4; ++j) \{
pair A = (j,i);
dot(A);

\}
\}
for (int i = 0; i < 3; ++i) \{
for (int j = 0; j < 4; ++j) \{
if (j != 3) \{
draw((j,i)--(j+1,i));
\}
if (i != 2) \{
draw((j,i)--(j,i+1));
\}
\}
\}
label("$A$", (0,2), W); 
label("$L$", (3,0), E);
\end{asy}
\end{center}


How many different routes can Paula take?

$\textbf{(A) } 0 \qquad\textbf{(B) } 1 \qquad\textbf{(C) } 2 \qquad\textbf{(D) } 3\qquad\textbf{(E) } 4$\par \vspace{0.5em}\item How many unordered pairs of edges of a given cube determine a plane?

$\textbf{(A) } 12 \qquad \textbf{(B) } 28 \qquad \textbf{(C) } 36\qquad \textbf{(D) } 42 \qquad \textbf{(E) } 66$\par \vspace{0.5em}\item Right triangle $ACD$ with right angle at $C$ is constructed outwards on the hypotenuse $\overline{AC}$ of isosceles right triangle $ABC$ with leg length $1$, as shown, so that the two triangles have equal perimeters. What is $\sin(2\angle BAD)$?

\begin{center}
\begin{asy}
import olympiad;
import cse5;
/* Geogebra to Asymptote conversion, documentation at artofproblemsolving.com/Wiki go to User:Azjps/geogebra */
import graph; size(8.016233639805293cm); 
real labelscalefactor = 0.5; /* changes label-to-point distance */
pen dps = linewidth(0.7) + fontsize(10); defaultpen(dps); /* default pen style */ 
pen dotstyle = black; /* point style */ 
real xmin = -4.001920114613276, xmax = 4.014313525192017, ymin = -2.552570341575814, ymax = 5.6249093771911145;  /* image dimensions */


draw((-1.6742337260757447,-1.)--(-1.6742337260757445,-0.6742337260757447)--(-2.,-0.6742337260757447)--(-2.,-1.)--cycle, linewidth(2.)); 
draw((-1.7696484586262846,2.7696484586262846)--(-1.5392969172525692,3.)--(-1.7696484586262846,3.2303515413737154)--(-2.,3.)--cycle, linewidth(2.)); 
 /* draw figures */
draw((-2.,3.)--(-2.,-1.), linewidth(2.)); 
draw((-2.,-1.)--(2.,-1.), linewidth(2.)); 
draw((2.,-1.)--(-2.,3.), linewidth(2.)); 
draw((-0.6404058554606791,4.3595941445393205)--(-2.,3.), linewidth(2.)); 
draw((-0.6404058554606791,4.3595941445393205)--(2.,-1.), linewidth(2.)); 
label("$D$",(-0.9382446143428628,4.887784444795223),SE*labelscalefactor,fontsize(14)); 
label("$A$",(1.9411496528285788,-1.0783204767840298),SE*labelscalefactor,fontsize(14)); 
label("$B$",(-2.5046350956841272,-0.9861798602345433),SE*labelscalefactor,fontsize(14)); 
label("$C$",(-2.5737405580962416,3.5747806589650395),SE*labelscalefactor,fontsize(14)); 
label("$1$",(-2.665881174645728,1.2712652452278765),SE*labelscalefactor,fontsize(14)); 
label("$1$",(-0.3393306067712029,-1.3547423264324894),SE*labelscalefactor,fontsize(14)); 
 /* dots and labels */
dot((-2.,3.),linewidth(4.pt) + dotstyle); 
dot((-2.,-1.),linewidth(4.pt) + dotstyle); 
dot((2.,-1.),linewidth(4.pt) + dotstyle); 
dot((-0.6404058554606791,4.3595941445393205),linewidth(4.pt) + dotstyle); 
clip((xmin,ymin)--(xmin,ymax)--(xmax,ymax)--(xmax,ymin)--cycle); 
 /* end of picture */
\end{asy}
\end{center}


$\textbf{(A) } \dfrac{1}{3}  \qquad\textbf{(B) } \dfrac{\sqrt{2}}{2} \qquad\textbf{(C) } \dfrac{3}{4} \qquad\textbf{(D) } \dfrac{7}{9} \qquad\textbf{(E) }  \dfrac{\sqrt{3}}{2}$\par \vspace{0.5em}\item A red ball and a green ball are randomly and independently tossed into bins numbered with the positive integers so that for each ball, the probability that it is tossed into bin $k$ is $2^{-k}$ for $k = 1,2,3....$  What is the probability that the red ball is tossed into a higher-numbered bin than the green ball?<br>

$\textbf{(A) } \frac{1}{4} \qquad\textbf{(B) } \frac{2}{7} \qquad\textbf{(C) } \frac{1}{3} \qquad\textbf{(D) } \frac{3}{8} \qquad\textbf{(E) } \frac{3}{7}$\par \vspace{0.5em}\item Let $S$ be the set of all positive integer divisors of $100,000.$ How many numbers are the product of two distinct elements of $S?$

$\textbf{(A) }98\qquad\textbf{(B) }100\qquad\textbf{(C) }117\qquad\textbf{(D) }119\qquad\textbf{(E) }121$\par \vspace{0.5em}\item As shown in the figure, line segment $\overline{AD}$ is trisected by points $B$ and $C$ so that $AB=BC=CD=2.$ Three semicircles of radius $1,$ $\overarc{AEB},$ $\overarc{BFC},$ and $\overarc{CGD},$ have their diameters on $\overline{AD},$ and are tangent to line $EG$ at $E,F,$ and $G,$ respectively. A circle of radius $2$ has its center on $F. $ The area of the region inside the circle but outside the three semicircles, shaded in the figure, can be expressed in the form

\begin\{equation*\}
\frac\{a\}\{b\}\cdot\pi-\sqrt\{c\}+d,
\end\{equation*\}
where $a,b,c,$ and $d$ are positive integers and $a$ and $b$ are relatively prime. What is $a+b+c+d$?


\begin{center}
\begin{asy}
import olympiad;
import cse5;
size(6cm);
filldraw(circle((0,0),2), gray(0.7));
filldraw(arc((0,-1),1,0,180) -- cycle, gray(1.0));
filldraw(arc((-2,-1),1,0,180) -- cycle, gray(1.0));
filldraw(arc((2,-1),1,0,180) -- cycle, gray(1.0));
dot((-3,-1));
label("$A$",(-3,-1),S);
dot((-2,0));
label("$E$",(-2,0),NW);
dot((-1,-1));
label("$B$",(-1,-1),S);
dot((0,0));
label("$F$",(0,0),N);
dot((1,-1));
label("$C$",(1,-1), S);
dot((2,0));
label("$G$", (2,0),NE);
dot((3,-1));
label("$D$", (3,-1), S);
\end{asy}
\end{center}

$\textbf{(A) } 13 \qquad\textbf{(B) } 14 \qquad\textbf{(C) } 15 \qquad\textbf{(D) } 16\qquad\textbf{(E) } 17$\par \vspace{0.5em}\item There are lily pads in a row numbered $0$ to $11$, in that order. There are predators on lily pads $3$ and $6$, and a morsel of food on lily pad $10$. Fiona the frog starts on pad $0$, and from any given lily pad, has a $\frac{1}{2}$ chance to hop to the next pad, and an equal chance to jump $2$ pads. What is the probability that Fiona reaches pad $10$ without landing on either pad $3$ or pad $6$?

$\textbf{(A) } \frac{15}{256} \qquad \textbf{(B) } \frac{1}{16} \qquad \textbf{(C) } \frac{15}{128}\qquad \textbf{(D) } \frac{1}{8} \qquad \textbf{(E) } \frac14$\par \vspace{0.5em}\item How many nonzero complex numbers $z$ have the property that $0, z,$ and $z^3,$ when represented by points in the complex plane, are the three distinct vertices of an equilateral triangle?

$\textbf{(A) }0\qquad\textbf{(B) }1\qquad\textbf{(C) }2\qquad\textbf{(D) }4\qquad\textbf{(E) }\text{infinitely many}$\par \vspace{0.5em}\item Square pyramid $ABCDE$ has base $ABCD,$ which measures $3$ cm on a side, and altitude $\overline{AE}$ perpendicular to the base$,$ which measures $6$ cm. Point $P$ lies on $\overline{BE},$ one third of the way from $B$ to $E;$ point $Q$ lies on $\overline{DE},$ one third of the way from $D$ to $E;$ and point $R$ lies on $\overline{CE},$ two thirds of the way from $C$ to $E.$ What is the area, in square centimeters, of $\triangle PQR?$

$\textbf{(A) } \frac{3\sqrt2}{2} \qquad\textbf{(B) } \frac{3\sqrt3}{2} \qquad\textbf{(C) } 2\sqrt2 \qquad\textbf{(D) } 2\sqrt3 \qquad\textbf{(E) } 3\sqrt2$\par \vspace{0.5em}\item Raashan, Sylvia, and Ted play the following game. Each person starts with $\$1$. A bell rings every $15$ seconds, at which time each of the players who currently have money simultaneously chooses one of the other two players independently and at random and gives $\$1$ to that player. What is the probability that after the bell has rung $2019$ times, each player will have $\$1\$? 
(For example, Raashan and Ted may each decide to give $\$1$ to Sylvia, and Sylvia may decide to give her dollar to Ted, at which point Raashan will have $\$0$, Sylvia will have $\$2$, and Ted will have $\$1$, and that is the end of the first round of play. In the second round Rashaan has no money to give, but Sylvia and Ted might choose each other to give their $ \$1\$ to, and the holdings will be the same at the end of the second round.)

$\textbf{(A) } \frac{1}{7} \qquad\textbf{(B) } \frac{1}{4} \qquad\textbf{(C) } \frac{1}{3} \qquad\textbf{(D) } \frac{1}{2} \qquad\textbf{(E) } \frac{2}{3}$\par \vspace{0.5em}\item Points $A(6,13)$ and $B(12,11)$ lie on circle $\omega$ in the plane. Suppose that the tangent lines to $\omega$ at $A$ and $B$ intersect at a point on the $x$-axis. What is the area of $\omega$?

\$\textbf\{(A) \}\frac\{83\pi\}\{8\}\qquad\textbf\{(B) \}\frac\{21\pi\}\{2\}\qquad\textbf\{(C) \}
\frac\{85\pi\}\{8\}\qquad\textbf\{(D) \}\frac\{43\pi\}\{4\}\qquad\textbf\{(E) \}\frac\{87\pi\}\{8\}\$\par \vspace{0.5em}\item How many quadratic polynomials with real coefficients are there such that the set of roots equals the set of coefficients? (For clarification: If the polynomial is $ax^2+bx+c,a\neq 0,$ and the roots are $r$ and $s,$ then the requirement is that $\{a,b,c\}=\{r,s\}$.)

$\textbf{(A) } 3 \qquad\textbf{(B) } 4 \qquad\textbf{(C) } 5 \qquad\textbf{(D) } 6 \qquad\textbf{(E) } \text{infinitely many}$\par \vspace{0.5em}\item Define a sequence recursively by $x_0=5$ and

\begin\{equation*\}
x\_\{n+1\}=\frac\{x\_n\^2+5x\_n+4\}\{x\_n+6\}
\end\{equation*\}
for all nonnegative integers $n.$ Let $m$ be the least positive integer such that

\begin\{equation*\}
x\_m\leq 4+\frac\{1\}\{2\^\{20\}\}.
\end\{equation*\}
In which of the following intervals does $m$ lie?

$\textbf{(A) } [9,26] \qquad\textbf{(B) } [27,80] \qquad\textbf{(C) } [81,242]\qquad\textbf{(D) } [243,728] \qquad\textbf{(E) } [729,\infty]$\par \vspace{0.5em}\item How many sequences of $0$s and $1$s of length $19$ are there that begin with a $0$, end with a $0$, contain no two consecutive $0$s, and contain no three consecutive $1$s?

$\textbf{(A) }55\qquad\textbf{(B) }60\qquad\textbf{(C) }65\qquad\textbf{(D) }70\qquad\textbf{(E) }75$\par \vspace{0.5em}\item Let $\omega=-\tfrac{1}{2}+\tfrac{1}{2}i\sqrt3.$ Let $S$ denote all points in the complex plane of the form $a+b\omega+c\omega^2,$ where $0\leq a \leq 1,0\leq b\leq 1,$ and $0\leq c\leq 1.$ What is the area of $S$?

$\textbf{(A) } \frac{1}{2}\sqrt3 \qquad\textbf{(B) } \frac{3}{4}\sqrt3 \qquad\textbf{(C) } \frac{3}{2}\sqrt3\qquad\textbf{(D) } \frac{1}{2}\pi\sqrt3 \qquad\textbf{(E) } \pi$\par \vspace{0.5em}\item Let $ABCD$ be a convex quadrilateral with $BC=2$ and $CD=6.$ Suppose that the centroids of $\triangle ABC,\triangle BCD,$ and $\triangle ACD$ form the vertices of an equilateral triangle. What is the maximum possible value of the area of $ABCD$?

$\textbf{(A) } 27 \qquad\textbf{(B) } 16\sqrt3 \qquad\textbf{(C) } 12+10\sqrt3 \qquad\textbf{(D) } 9+12\sqrt3 \qquad\textbf{(E) } 30$\par \vspace{0.5em}\end{enumerate}\newpage\section*{2020 AMC1212A}\begin{enumerate}[label=\arabic*., itemsep=0.5em]\item Carlos took $70\%$ of a whole pie. Maria took one third of the remainder. What portion of the whole pie was left?

$\textbf{(A)}\ 10\%\qquad\textbf{(B)}\ 15\%\qquad\textbf{(C)}\ 20\%\qquad\textbf{(D)}\ 30\%\qquad\textbf{(E)}\ 35\%$\par \vspace{0.5em}\item The acronym AMC is shown in the rectangular grid below with grid lines spaced $1$ unit apart. In units, what is the sum of the lengths of the line segments that form the acronym AMC$?$


\begin{center}
\begin{asy}
import olympiad;
import cse5;
import olympiad;
unitsize(25);
for (int i = 0; i < 3; ++i) \{
for (int j = 0; j < 9; ++j) \{
pair A = (j,i);

\}
\}
for (int i = 0; i < 3; ++i) \{
for (int j = 0; j < 9; ++j) \{
if (j != 8) \{
draw((j,i)--(j+1,i), dashed);
\}
if (i != 2) \{
draw((j,i)--(j,i+1), dashed);
\}
\}
\}
draw((0,0)--(2,2),linewidth(2));
draw((2,0)--(2,2),linewidth(2));
draw((1,1)--(2,1),linewidth(2));
draw((3,0)--(3,2),linewidth(2));
draw((5,0)--(5,2),linewidth(2));
draw((4,1)--(3,2),linewidth(2));
draw((4,1)--(5,2),linewidth(2));
draw((6,0)--(8,0),linewidth(2));
draw((6,2)--(8,2),linewidth(2));
draw((6,0)--(6,2),linewidth(2));
\end{asy}
\end{center}


$\textbf{(A) } 17 \qquad \textbf{(B) } 15 + 2\sqrt{2} \qquad \textbf{(C) } 13 + 4\sqrt{2} \qquad \textbf{(D) } 11 + 6\sqrt{2} \qquad \textbf{(E) } 21$\par \vspace{0.5em}\item A driver travels for $2$ hours at $60$ miles per hour, during which her car gets $30$ miles per gallon of gasoline. She is paid $\$0.50$ per mile, and her only expense is gasoline at $\$2.00$ per gallon. What is her net rate of pay, in dollars per hour, after this expense?

$\textbf{(A) }20 \qquad\textbf{(B) }22 \qquad\textbf{(C) }24 \qquad\textbf{(D) } 25\qquad\textbf{(E) } 26$\par \vspace{0.5em}\item How many $4$-digit positive integers (that is, integers between $1000$ and $9999$, inclusive) having only even digits are divisible by $5?$

$\textbf{(A) } 80 \qquad \textbf{(B) } 100 \qquad \textbf{(C) } 125 \qquad \textbf{(D) } 200 \qquad \textbf{(E) } 500$\par \vspace{0.5em}\item The $25$ integers from $-10$ to $14,$ inclusive, can be arranged to form a $5$-by-$5$ square in which the sum of the numbers in each row, the sum of the numbers in each column, and the sum of the numbers along each of the main diagonals are all the same. What is the value of this common sum?

$\textbf{(A) }2 \qquad\textbf{(B) } 5\qquad\textbf{(C) } 10\qquad\textbf{(D) } 25\qquad\textbf{(E) } 50$\par \vspace{0.5em}\item In the plane figure shown below, $3$ of the unit squares have been shaded. What is the least number of additional unit squares that must be shaded so that the resulting figure has two lines of symmetry$?$


\begin{center}
\begin{asy}
import olympiad;
import cse5;
import olympiad;
unitsize(25);
filldraw((1,3)--(1,4)--(2,4)--(2,3)--cycle, gray(0.7));
filldraw((2,1)--(2,2)--(3,2)--(3,1)--cycle, gray(0.7));
filldraw((4,0)--(5,0)--(5,1)--(4,1)--cycle, gray(0.7));
for (int i = 0; i < 5; ++i) \{
for (int j = 0; j < 6; ++j) \{
pair A = (j,i);

\}
\}
for (int i = 0; i < 5; ++i) \{
for (int j = 0; j < 6; ++j) \{
if (j != 5) \{
draw((j,i)--(j+1,i));
\}
if (i != 4) \{
draw((j,i)--(j,i+1));
\}
\}
\}
\end{asy}
\end{center}


$\textbf{(A) } 4 \qquad \textbf{(B) } 5 \qquad \textbf{(C) } 6 \qquad \textbf{(D) } 7 \qquad \textbf{(E) } 8$\par \vspace{0.5em}\item Seven cubes, whose volumes are $1$, $8$, $27$, $64$, $125$, $216$, and $343$ cubic units, are stacked vertically to form a tower in which the volumes of the cubes decrease from bottom to top. Except for the bottom cube, the bottom face of each cube lies completely on top of the cube below it. What is the total surface area of the tower (including the bottom) in square units?

$\textbf{(A) } 644    \qquad \textbf{(B) } 658   \qquad \textbf{(C) } 664   \qquad \textbf{(D) } 720   \qquad \textbf{(E) } 749$\par \vspace{0.5em}\item What is the median of the following list of $4040$ numbers$?$

\begin\{equation*\}
1, 2, 3, \ldots, 2020, 1\^2, 2\^2, 3\^2, \ldots, 2020\^2
\end\{equation*\}

$ \textbf{(A)}\ 1974.5\qquad\textbf{(B)}\ 1975.5\qquad\textbf{(C)}\ 1976.5\qquad\textbf{(D)}\ 1977.5\qquad\textbf{(E)}\ 1978.5 $\par \vspace{0.5em}\item How many solutions does the equation $\tan{(2x)} = \cos{(\tfrac{x}{2})}$ have on the interval $[0, 2\pi]?$

$\textbf{(A) } 1 \qquad \textbf{(B) } 2 \qquad \textbf{(C) } 3 \qquad \textbf{(D) } 4 \qquad \textbf{(E) } 5$\par \vspace{0.5em}\item There is a unique positive integer $n$ such that
\begin\{equation*\}
\log\_2\{(\log\_\{16\}\{n\})\} = \log\_4\{(\log\_4\{n\})\}.
\end\{equation*\}
What is the sum of the digits of $n?$

$\textbf{(A) } 4 \qquad \textbf{(B) } 7 \qquad \textbf{(C) } 8 \qquad \textbf{(D) } 11 \qquad \textbf{(E) } 13$\par \vspace{0.5em}\item A frog sitting at the point $(1, 2)$ begins a sequence of jumps, where each jump is parallel to one of the coordinate axes and has length $1$, and the direction of each jump (up, down, right, or left) is chosen independently at random. The sequence ends when the frog reaches a side of the square with vertices $(0,0), (0,4), (4,4),$ and $(4,0)$. What is the probability that the sequence of jumps ends on a vertical side of the square$?$

$\textbf{(A) } \frac{1}{2} \qquad \textbf{(B) } \frac{5}{8} \qquad \textbf{(C) } \frac{2}{3} \qquad \textbf{(D) } \frac{3}{4} \qquad \textbf{(E) } \frac{7}{8}$\par \vspace{0.5em}\item Line $\ell$ in the coordinate plane has the equation $3x - 5y + 40 = 0$. This line is rotated $45^{\circ}$ counterclockwise about the point $(20, 20)$ to obtain line $k$. What is the $x$-coordinate of the $x$-intercept of line $k?$

$\textbf{(A) } 10 \qquad \textbf{(B) } 15 \qquad \textbf{(C) } 20 \qquad \textbf{(D) } 25 \qquad \textbf{(E) } 30$\par \vspace{0.5em}\item There are integers $a$, $b$, and $c$, each greater than 1, such that
\begin\{equation*\}
\sqrt[a]\{N \sqrt[b]\{N \sqrt[c]\{N\}\}\} = \sqrt[36]\{N\^\{25\}\}
\end\{equation*\}
for all $N > 1$. What is $b$?

$\textbf{(A)}\ 2\qquad\textbf{(B)}\ 3\qquad\textbf{(C)}\ 4\qquad\textbf{(D)}\ 5\qquad\textbf{(E)}\ 6$\par \vspace{0.5em}\item Regular octagon $ABCDEFGH$ has area $n$. Let $m$ be the area of quadrilateral $ACEG$. What is $\tfrac{m}{n}?$

$\textbf{(A) } \frac{\sqrt{2}}{4} \qquad \textbf{(B) } \frac{\sqrt{2}}{2} \qquad \textbf{(C) } \frac{3}{4} \qquad \textbf{(D) } \frac{3\sqrt{2}}{5} \qquad \textbf{(E) } \frac{2\sqrt{2}}{3}$\par \vspace{0.5em}\item In the complex plane, let $A$ be the set of solutions to $z^3 - 8 = 0$ and let $B$ be the set of solutions to $z^3 - 8z^2 - 8z + 64 = 0$. What is the greatest distance between a point of $A$ and a point of $B?$

$\textbf{(A) } 2\sqrt{3} \qquad \textbf{(B) } 6 \qquad \textbf{(C) } 9 \qquad \textbf{(D) } 2\sqrt{21} \qquad \textbf{(E) } 9 + \sqrt{3}$\par \vspace{0.5em}\item A point is chosen at random within the square in the coordinate plane whose vertices are $(0, 0), (2020, 0), (2020, 2020),$ and $(0, 2020)$. The probability that the point is within $d$ units of a lattice point is $\tfrac{1}{2}$. (A point $(x, y)$ is a lattice point if $x$ and $y$ are both integers.) What is $d$ to the nearest tenth$?$

$\textbf{(A) } 0.3 \qquad \textbf{(B) } 0.4 \qquad \textbf{(C) } 0.5 \qquad \textbf{(D) } 0.6 \qquad \textbf{(E) } 0.7$\par \vspace{0.5em}\item The vertices of a quadrilateral lie on the graph of $y = \ln x$, and the $x$-coordinates of these vertices are consecutive positive integers. The area of the quadrilateral is $\ln \frac{91}{90}$. What is the $x$-coordinate of the leftmost vertex?

$\textbf{(A)}\ 6\qquad\textbf{(B)}\ 7\qquad\textbf{(C)}\ 10\qquad\textbf{(D)}\ 12\qquad\textbf{(E)}\ 13$\par \vspace{0.5em}\item Quadrilateral $ABCD$ satisfies $\angle ABC = \angle ACD = 90^{\circ}, AC = 20$, and $CD = 30$. Diagonals $\overline{AC}$ and $\overline{BD}$ intersect at point $E$, and $AE = 5$. What is the area of quadrilateral $ABCD$?

$\textbf{(A) } 330 \qquad\textbf{(B) } 340 \qquad\textbf{(C) } 350 \qquad\textbf{(D) } 360 \qquad\textbf{(E) } 370$\par \vspace{0.5em}\item There exists a unique strictly increasing sequence of nonnegative integers $a_1 < a_2 <  < a_k$ such that
\begin\{equation*\}
\frac\{2\^\{289\}+1\}\{2\^\{17\}+1\} = 2\^\{a\_1\} + 2\^\{a\_2\} +  + 2\^\{a\_k\}.
\end\{equation*\}
What is $k?$

$\textbf{(A) } 117 \qquad \textbf{(B) } 136 \qquad \textbf{(C) } 137 \qquad \textbf{(D) } 273 \qquad \textbf{(E) } 306$\par \vspace{0.5em}\item Let $T$ be the triangle in the coordinate plane with vertices $\left(0,0\right)$, $\left(4,0\right)$, and $\left(0,3\right)$. Consider the following five isometries (rigid transformations) of the plane: rotations of $90^{\circ}$, $180^{\circ}$, and $270^{\circ}$ counterclockwise around the origin, reflection across the $x$-axis, and reflection across the $y$-axis. How many of the $125$ sequences of three of these transformations (not necessarily distinct) will return $T$ to its original position? (For example, a $180^{\circ}$ rotation, followed by a reflection across the $x$-axis, followed by a reflection across the $y$-axis will return $T$ to its original position, but a $90^{\circ}$ rotation, followed by a reflection across the $x$-axis, followed by another reflection across the $x$-axis will not return $T$ to its original position.)

$\textbf{(A) } 12\qquad\textbf{(B) } 15\qquad\textbf{(C) }17 \qquad\textbf{(D) }20 \qquad\textbf{(E) }25$\par \vspace{0.5em}\item How many positive integers $n$ are there such that $n$ is a multiple of $5$, and the least common multiple of $5!$ and $n$ equals $5$ times the greatest common divisor of $10!$ and $n?$

$\textbf{(A) } 12 \qquad \textbf{(B) } 24 \qquad \textbf{(C) } 36 \qquad \textbf{(D) } 48 \qquad \textbf{(E) } 72$\par \vspace{0.5em}\item Let $(a_n)$ and $(b_n)$ be the sequences of real numbers such that

\begin\{equation*\}
(2 + i)\^n = a\_n + b\_ni
\end\{equation*\}
for all integers $n\geq 0$, where $i = \sqrt{-1}$. What is
\begin\{equation*\}
\sum\_\{n=0\}\^\infty\frac\{a\_nb\_n\}\{7\^n\}\,?
\end\{equation*\}

$\textbf{(A) }\frac 38\qquad\textbf{(B) }\frac7{16}\qquad\textbf{(C) }\frac12\qquad\textbf{(D) }\frac9{16}\qquad\textbf{(E) }\frac47$\par \vspace{0.5em}\item Jason rolls three fair standard six-sided dice. Then he looks at the rolls and chooses a subset of the dice (possibly empty, possibly all three dice) to reroll. After rerolling, he wins if and only if the sum of the numbers face up on the three dice is exactly $7$. Jason always plays to optimize his chances of winning. What is the probability that he chooses to reroll exactly two of the dice?

$\textbf{(A) } \frac{7}{36} \qquad\textbf{(B) } \frac{5}{24} \qquad\textbf{(C) } \frac{2}{9} \qquad\textbf{(D) } \frac{17}{72} \qquad\textbf{(E) } \frac{1}{4}$\par \vspace{0.5em}\item Suppose that $\triangle ABC$ is an equilateral triangle of side length $s$, with the property that there is a unique point $P$ inside the triangle such that $AP = 1$, $BP = \sqrt{3}$, and $CP = 2$. What is $s?$

$\textbf{(A) } 1 + \sqrt{2} \qquad \textbf{(B) } \sqrt{7} \qquad \textbf{(C) } \frac{8}{3} \qquad \textbf{(D) } \sqrt{5 + \sqrt{5}} \qquad \textbf{(E) } 2\sqrt{2}$\par \vspace{0.5em}\item The number $a = \tfrac{p}{q}$, where $p$ and $q$ are relatively prime positive integers, has the property that the sum of all real numbers $x$ satisfying
\begin\{equation*\}
\lfloor x \rfloor \cdot \\{x\\} = a \cdot x\^2
\end\{equation*\}
is $420$, where $\lfloor x \rfloor$ denotes the greatest integer less than or equal to $x$ and $\{x\} = x - \lfloor x \rfloor$ denotes the fractional part of $x$. What is $p + q?$

$\textbf{(A) } 245 \qquad \textbf{(B) } 593 \qquad \textbf{(C) } 929 \qquad \textbf{(D) } 1331 \qquad \textbf{(E) } 1332$\par \vspace{0.5em}\end{enumerate}\newpage\section*{2020 AMC1212B}\begin{enumerate}[label=\arabic*., itemsep=0.5em]\item What is the value in simplest form of the following expression?
\begin\{equation*\}
\sqrt\{1\} + \sqrt\{1+3\} + \sqrt\{1+3+5\} + \sqrt\{1+3+5+7\}
\end\{equation*\}


$\textbf{(A) }5 \qquad \textbf{(B) }4 + \sqrt{7} + \sqrt{10} \qquad \textbf{(C) } 10 \qquad \textbf{(D) } 15 \qquad \textbf{(E) } 4 + 3\sqrt{3} + 2\sqrt{5} + \sqrt{7}$\par \vspace{0.5em}\item What is the value of the following expression?

\begin\{equation*\}
\frac\{100\^2-7\^2\}\{70\^2-11\^2\} \cdot \frac\{(70-11)(70+11)\}\{(100-7)(100+7)\}
\end\{equation*\}
$\textbf{(A) } 1 \qquad \textbf{(B) } \frac{9951}{9950} \qquad \textbf{(C) } \frac{4780}{4779} \qquad \textbf{(D) } \frac{108}{107} \qquad \textbf{(E) } \frac{81}{80} $\par \vspace{0.5em}\item The ratio of $w$ to $x$ is $4 : 3$, the ratio of $y$ to $z$ is $3 : 2$, and the ratio of $z$ to $x$ is $1 : 6$. What is the ratio of $w$ to $y$?

$\textbf{(A) }4:3 \qquad \textbf{(B) }3:2 \qquad \textbf{(C) } 8:3 \qquad \textbf{(D) } 4:1 \qquad \textbf{(E) } 16:3 $\par \vspace{0.5em}\item The acute angles of a right triangle are $a^{\circ}$ and $b^{\circ}$, where $a>b$ and both $a$ and $b$ are prime numbers. What is the least possible value of $b$?

$\textbf{(A) }2\qquad\textbf{(B) }3\qquad\textbf{(C) }5\qquad\textbf{(D) }7\qquad\textbf{(E) }11$\par \vspace{0.5em}\item Teams $A$ and $B$ are playing in a basketball league where each game results in a win for one team and a loss for the other team. Team $A$ has won $\tfrac{2}{3}$ of its games and team $B$ has won $\tfrac{5}{8}$ of its games. Also, team $B$ has won $7$ more games and lost $7$ more games than team $A.$ How many games has team $A$ played?

$\textbf{(A) } 21 \qquad \textbf{(B) } 27 \qquad \textbf{(C) } 42 \qquad \textbf{(D) } 48 \qquad \textbf{(E) } 63$\par \vspace{0.5em}\item For all integers $n \geq 9,$ the value of

\begin\{equation*\}
\frac\{(n+2)!-(n+1)!\}\{n!\}
\end\{equation*\}
is always which of the following?

$\textbf{(A) } \text{a multiple of 4} \qquad \textbf{(B) } \text{a multiple of 10} \qquad \textbf{(C) } \text{a prime number} \qquad \textbf{(D) } \text{a perfect square} \qquad \textbf{(E) } \text{a perfect cube}$\par \vspace{0.5em}\item Two nonhorizontal, non vertical lines in the $xy$-coordinate plane intersect to form a $45^{\circ}$ angle. One line has slope equal to $6$ times the slope of the other line. What is the greatest possible value of the product of the slopes of the two lines?

$\textbf{(A)}\ \frac16 \qquad\textbf{(B)}\ \frac23 \qquad\textbf{(C)}\  \frac32 \qquad\textbf{(D)}\ 3 \qquad\textbf{(E)}\ 6$\par \vspace{0.5em}\item How many ordered pairs of integers $(x, y)$ satisfy the equation
\begin\{equation*\}
x\^\{2020\}+y\^2=2y?
\end\{equation*\}

$\textbf{(A) } 1 \qquad\textbf{(B) } 2 \qquad\textbf{(C) } 3 \qquad\textbf{(D) } 4 \qquad\textbf{(E) } \text{infinitely many}$\par \vspace{0.5em}\item A three-quarter sector of a circle of radius $4$ inches together with its interior can be rolled up to form the lateral surface of a right circular cone by taping together along the two radii shown. What is the volume of the cone in cubic inches?

\begin{center}
\begin{asy}
import olympiad;
import cse5;
draw(Arc((0,0), 4, 0, 270));
draw((0,-4)--(0,0)--(4,0));

label("$4$", (2,0), S);
\end{asy}
\end{center}


$\textbf{(A)}\ 3\pi \sqrt5 \qquad\textbf{(B)}\ 4\pi \sqrt3 \qquad\textbf{(C)}\ 3 \pi \sqrt7 \qquad\textbf{(D)}\ 6\pi \sqrt3 \qquad\textbf{(E)}\ 6\pi \sqrt7$\par \vspace{0.5em}\item In unit square $ABCD,$ the inscribed circle $\omega$ intersects $\overline{CD}$ at $M,$ and $\overline{AM}$ intersects $\omega$ at a point $P$ different from $M.$ What is $AP?$

$\textbf{(A) } \frac{\sqrt5}{12} \qquad \textbf{(B) } \frac{\sqrt5}{10} \qquad \textbf{(C) } \frac{\sqrt5}{9} \qquad \textbf{(D) } \frac{\sqrt5}{8} \qquad \textbf{(E) } \frac{2\sqrt5}{15}$\par \vspace{0.5em}\item As shown in the figure below, six semicircles lie in the interior of a regular hexagon with side length $2$ so that the diameters of the semicircles coincide with the sides of the hexagon. What is the area of the shaded regioninside the hexagon but outside all of the semicircles?


\begin{center}
\begin{asy}
import olympiad;
import cse5;
size(140);
fill((1,0)--(3,0)--(4,sqrt(3))--(3,2sqrt(3))--(1,2sqrt(3))--(0,sqrt(3))--cycle,gray(0.4));
fill(arc((2,0),1,180,0)--(2,0)--cycle,white);
fill(arc((3.5,sqrt(3)/2),1,60,240)--(3.5,sqrt(3)/2)--cycle,white);
fill(arc((3.5,3sqrt(3)/2),1,120,300)--(3.5,3sqrt(3)/2)--cycle,white);
fill(arc((2,2sqrt(3)),1,180,360)--(2,2sqrt(3))--cycle,white);
fill(arc((0.5,3sqrt(3)/2),1,240,420)--(0.5,3sqrt(3)/2)--cycle,white);
fill(arc((0.5,sqrt(3)/2),1,300,480)--(0.5,sqrt(3)/2)--cycle,white);
draw((1,0)--(3,0)--(4,sqrt(3))--(3,2sqrt(3))--(1,2sqrt(3))--(0,sqrt(3))--(1,0));
draw(arc((2,0),1,180,0)--(2,0)--cycle);
draw(arc((3.5,sqrt(3)/2),1,60,240)--(3.5,sqrt(3)/2)--cycle);
draw(arc((3.5,3sqrt(3)/2),1,120,300)--(3.5,3sqrt(3)/2)--cycle);
draw(arc((2,2sqrt(3)),1,180,360)--(2,2sqrt(3))--cycle);
draw(arc((0.5,3sqrt(3)/2),1,240,420)--(0.5,3sqrt(3)/2)--cycle);
draw(arc((0.5,sqrt(3)/2),1,300,480)--(0.5,sqrt(3)/2)--cycle);
label("$2$",(3.5,3sqrt(3)/2),NE);
\end{asy}
\end{center}


$ \textbf {(A) } 6\sqrt{3}-3\pi \qquad \textbf {(B) } \frac{9\sqrt{3}}{2} - 2\pi\ \qquad \textbf {(C) } \frac{3\sqrt{3}}{2} - \frac{\pi}{3} \qquad \textbf {(D) } 3\sqrt{3} - \pi \qquad \textbf {(E) } \frac{9\sqrt{3}}{2} - \pi $\par \vspace{0.5em}\item Let $\overline{AB}$ be a diameter in a circle of radius $5\sqrt2.$ Let $\overline{CD}$ be a chord in the circle that intersects $\overline{AB}$ at a point $E$ such that $BE=2\sqrt5$ and $\angle AEC = 45^{\circ}.$ What is $CE^2+DE^2?$

$\textbf{(A)}\ 96 \qquad\textbf{(B)}\ 98 \qquad\textbf{(C)}\  44\sqrt5 \qquad\textbf{(D)}\ 70\sqrt2 \qquad\textbf{(E)}\ 100$\par \vspace{0.5em}\item Which of the following is the value of $\sqrt{\log_2{6}+\log_3{6}}?$

$\textbf{(A) } 1 \qquad\textbf{(B) } \sqrt{\log_5{6}} \qquad\textbf{(C) } 2 \qquad\textbf{(D) } \sqrt{\log_2{3}}+\sqrt{\log_3{2}} \qquad\textbf{(E) } \sqrt{\log_2{6}}+\sqrt{\log_3{6}}$\par \vspace{0.5em}\item Bela and Jenn play the following game on the closed interval $[0, n]$ of the real number line, where $n$ is a fixed integer greater than $4$. They take turns playing, with Bela going first. At his first turn, Bela chooses any real number in the interval $[0, n]$. Thereafter, the player whose turn it is chooses a real number that is more than one unit away from all numbers previously chosen by either player. A player unable to choose such a number loses. Using optimal strategy, which player will win the game?

$\textbf{(A)} \text{ Bela will always win.} \qquad \textbf{(B)} \text{ Jenn will always win.} \qquad \textbf{(C)} \text{ Bela will win if and only if }n \text{ is odd.}$
$\textbf{(D)} \text{ Jenn will win if and only if }n \text{ is odd.} \qquad \textbf{(E)} \text { Jenn will win if and only if } n>8.$\par \vspace{0.5em}\item There are 10 people standing equally spaced around a circle. Each person knows exactly 3 of the other 9 people: the 2 people standing next to her or him, as well as the person directly across the circle. How many ways are there for the 10 people to split up into 5 pairs so that the members of each pair know each other?

$\textbf{(A) } 11 \qquad \textbf{(B) } 12 \qquad \textbf{(C) } 13 \qquad \textbf{(D) } 14 \qquad \textbf{(E) } 15$\par \vspace{0.5em}\item An urn contains one red ball and one blue ball. A box of extra red and blue balls lie nearby. George performs the following operation four times: he draws a ball from the urn at random and then takes a ball of the same color from the box and returns those two matching balls to the urn. After the four iterations the urn contains six balls. What is the probability that the urn contains three balls of each color?

$\textbf{(A) } \frac16 \qquad \textbf{(B) }\frac15 \qquad \textbf{(C) } \frac14 \qquad \textbf{(D) } \frac13 \qquad \textbf{(E) } \frac12$\par \vspace{0.5em}\item How many polynomials of the form $x^5 + ax^4 + bx^3 + cx^2 + dx + 2020$, where $a$, $b$, $c$, and $d$ are real numbers, have the property that whenever $r$ is a root, so is $\frac{-1+i\sqrt{3}}{2} \cdot r$? (Note that $i=\sqrt{-1}$)

$\textbf{(A) } 0 \qquad \textbf{(B) }1 \qquad \textbf{(C) } 2 \qquad \textbf{(D) } 3 \qquad \textbf{(E) } 4$\par \vspace{0.5em}\item In square $ABCD$, points $E$ and $H$ lie on $\overline{AB}$ and $\overline{DA}$, respectively, so that $AE=AH.$ Points $F$ and $G$ lie on $\overline{BC}$ and $\overline{CD}$, respectively, and points $I$ and $J$ lie on $\overline{EH}$ so that $\overline{FI} \perp \overline{EH}$ and $\overline{GJ} \perp \overline{EH}$. See the figure below. Triangle $AEH$, quadrilateral $BFIE$, quadrilateral $DHJG$, and pentagon $FCGJI$ each has area $1.$ What is $FI^2$?

\begin{center}
\begin{asy}
import olympiad;
import cse5;
real x=2sqrt(2);
real y=2sqrt(16-8sqrt(2))-4+2sqrt(2);
real z=2sqrt(8-4sqrt(2));
pair A, B, C, D, E, F, G, H, I, J;
A = (0,0);
B = (4,0);
C = (4,4);
D = (0,4);
E = (x,0);
F = (4,y);
G = (y,4);
H = (0,x);
I = F + z * dir(225);
J = G + z * dir(225);

draw(A--B--C--D--A);
draw(H--E);
draw(J--G\^\^F--I);
draw(rightanglemark(G, J, I), linewidth(.5));
draw(rightanglemark(F, I, E), linewidth(.5));

dot("$A$", A, S);
dot("$B$", B, S);
dot("$C$", C, dir(90));
dot("$D$", D, dir(90));
dot("$E$", E, S);
dot("$F$", F, dir(0));
dot("$G$", G, N);
dot("$H$", H, W);
dot("$I$", I, SW);
dot("$J$", J, SW);
\end{asy}
\end{center}


$\textbf{(A) } \frac{7}{3} \qquad \textbf{(B) } 8-4\sqrt2 \qquad \textbf{(C) } 1+\sqrt2 \qquad \textbf{(D) } \frac{7}{4}\sqrt2 \qquad \textbf{(E) } 2\sqrt2$\par \vspace{0.5em}\item Square $ABCD$ in the coordinate plane has vertices at the points $A(1,1), B(-1,1), C(-1,-1),$ and $D(1,-1).$ Consider the following four transformations:

$\quad\bullet\qquad$ $L,$ a rotation of $90^{\circ}$ counterclockwise around the origin;

$\quad\bullet\qquad$ $R,$ a rotation of $90^{\circ}$ clockwise around the origin;

$\quad\bullet\qquad$ $H,$ a reflection across the $x$-axis; and

$\quad\bullet\qquad$ $V,$ a reflection across the $y$-axis.

Each of these transformations maps the squares onto itself, but the positions of the labeled vertices will change. For example, applying $R$ and then $V$ would send the vertex $A$ at $(1,1)$ to $(-1,-1)$ and would send the vertex $B$ at $(-1,1)$ to itself. How many sequences of $20$ transformations chosen from $\{L, R, H, V\}$ will send all of the labeled vertices back to their original positions? (For example, $R, R, V, H$ is one sequence of $4$ transformations that will send the vertices back to their original positions.)

$\textbf{(A)}\ 2^{37} \qquad\textbf{(B)}\ 3\cdot 2^{36} \qquad\textbf{(C)}\  2^{38} \qquad\textbf{(D)}\ 3\cdot 2^{37} \qquad\textbf{(E)}\ 2^{39}$\par \vspace{0.5em}\item Two different cubes of the same size are to be painted, with the color of each face being chosen independently and at random to be either black or white. What is the probability that after they are painted, the cubes can be rotated to be identical in appearance?

$\textbf{(A)}\ \frac{9}{64} \qquad\textbf{(B)}\ \frac{289}{2048} \qquad\textbf{(C)}\  \frac{73}{512} \qquad\textbf{(D)}\ \frac{147}{1024} \qquad\textbf{(E)}\ \frac{589}{4096}$\par \vspace{0.5em}\item How many positive integers $n$ satisfy
\begin\{equation*\}
\frac\{n+1000\}\{70\} = \lfloor \sqrt\{n\} \rfloor?
\end\{equation*\}
(Recall that $\lfloor x\rfloor$ is the greatest integer not exceeding $x$.)

$\textbf{(A) } 2 \qquad\textbf{(B) } 4 \qquad\textbf{(C) } 6 \qquad\textbf{(D) } 30 \qquad\textbf{(E) } 32$\par \vspace{0.5em}\item What is the maximum value of $\frac{(2^t-3t)t}{4^t}$ for real values of $t?$

$\textbf{(A)}\ \frac{1}{16} \qquad\textbf{(B)}\ \frac{1}{15} \qquad\textbf{(C)}\ \frac{1}{12} \qquad\textbf{(D)}\ \frac{1}{10} \qquad\textbf{(E)}\ \frac{1}{9}$\par \vspace{0.5em}\item How many integers $n \geq 2$ are there such that whenever $z_1, z_2, ..., z_n$ are complex numbers such that


\begin\{equation*\}
|z\_1| = |z\_2| = ... = |z\_n| = 1 \text\{    and    \} z\_1 + z\_2 + ... + z\_n = 0,
\end\{equation*\}

then the numbers $z_1, z_2, ..., z_n$ are equally spaced on the unit circle in the complex plane?

$\textbf{(A)}\ 1 \qquad\textbf{(B)}\ 2 \qquad\textbf{(C)}\ 3 \qquad\textbf{(D)}\ 4 \qquad\textbf{(E)}\ 5$\par \vspace{0.5em}\item Let $D(n)$ denote the number of ways of writing the positive integer $n$ as a product
\begin\{equation*\}
n = f\_1\cdot f\_2\cdots f\_k,
\end\{equation*\}
where $k\ge1$, the $f_i$ are integers strictly greater than $1$, and the order in which the factors are listed matters (that is, two representations that differ only in the order of the factors are counted as distinct). For example, the number $6$ can be written as $6$, $2\cdot 3$, and $3\cdot2$, so $D(6) = 3$. What is $D(96)$?

$\textbf{(A) } 112 \qquad\textbf{(B) } 128 \qquad\textbf{(C) } 144 \qquad\textbf{(D) } 172 \qquad\textbf{(E) } 184$\par \vspace{0.5em}\item For each real number $a$ with $0 \leq a \leq 1$, let numbers $x$ and $y$ be chosen independently at random from the intervals $[0, a]$ and $[0, 1]$, respectively, and let $P(a)$ be the probability that


\begin\{equation*\}
\sin\^2\{(\pi x)\} + \sin\^2\{(\pi y)\} > 1
\end\{equation*\}

What is the maximum value of $P(a)?$

$\textbf{(A)}\ \frac{7}{12} \qquad\textbf{(B)}\ 2 - \sqrt{2} \qquad\textbf{(C)}\ \frac{1+\sqrt{2}}{4} \qquad\textbf{(D)}\ \frac{\sqrt{5}-1}{2} \qquad\textbf{(E)}\ \frac{5}{8}$\par \vspace{0.5em}\end{enumerate}\newpage\section*{2021 AMC1212A}\begin{enumerate}[label=\arabic*., itemsep=0.5em]\item What is the value of $\frac{(2112-2021)^2}{169}$?

$\textbf{(A) } 7 \qquad\textbf{(B) } 21 \qquad\textbf{(C) } 49 \qquad\textbf{(D) } 64 \qquad\textbf{(E) } 91$\par \vspace{0.5em}\item Menkara has a $4 \times 6$ index card. If she shortens the length of one side of this card by $1$ inch, the card would have area $18$ square inches. What would the area of the card be in square inches if instead she shortens the length of the other side by $1$ inch?

$\textbf{(A) }16\qquad\textbf{(B) }17\qquad\textbf{(C) }18\qquad\textbf{(D) }19\qquad\textbf{(E) }20$\par \vspace{0.5em}\item Mr. Lopez has a choice of two routes to get to work. Route A is $6$ miles long, and his average speed along this route is $30$ miles per hour. Route B is $5$ miles long, and his average speed along this route is $40$ miles per hour, except for a $\frac{1}{2}$-mile stretch in a school zone where his average speed is $20$ miles per hour. By how many minutes is Route B quicker than Route A?

\$\textbf\{(A)\}\ 2 \frac\{3\}\{4\}  \qquad\textbf\{(B)\}\  3 \frac\{3\}\{4\} \qquad\textbf\{(C)\}\  4 \frac\{1\}\{2\} \qquad\textbf\{(D)\}\
 5 \frac\{1\}\{2\} \qquad\textbf\{(E)\}\ 6 \frac\{3\}\{4\}\$\par \vspace{0.5em}\item The six-digit number $\underline{2}\,\underline{0}\,\underline{2}\,\underline{1}\,\underline{0}\,\underline{A}$ is prime for only one digit $A.$ What is $A?$

$(\textbf{A})\: 1\qquad(\textbf{B}) \: 3\qquad(\textbf{C}) \: 5 \qquad(\textbf{D}) \: 7\qquad(\textbf{E}) \: 9$\par \vspace{0.5em}\item Elmer the emu takes $44$ equal strides to walk between consecutive telephone poles on a rural road. Oscar the ostrich can cover the same distance in $12$ equal leaps. The telephone poles are evenly spaced, and the $41$st pole along this road is exactly one mile ($5280$ feet) from the first pole. How much longer, in feet, is Oscar's leap than Elmer's stride?

$\textbf{(A) }6\qquad\textbf{(B) }8\qquad\textbf{(C) }10\qquad\textbf{(D) }11\qquad\textbf{(E) }15$\par \vspace{0.5em}\item As shown in the figure below, point $E$ lies on the opposite half-plane determined by line $CD$ from point $A$ so that $\angle CDE = 110^\circ$. Point $F$ lies on $\overline{AD}$ so that $DE=DF$, and $ABCD$ is a square. What is the degree measure of $\angle AFE$?


\begin{center}
\begin{asy}
import olympiad;
import cse5;
size(6cm);
pair A = (0,10);
label("$A$", A, N);
pair B = (0,0);
label("$B$", B, S);
pair C = (10,0);
label("$C$", C, S);
pair D = (10,10);
label("$D$", D, SW);
pair EE = (15,11.8);
label("$E$", EE, N);
pair F = (3,10);
label("$F$", F, N);
filldraw(D--arc(D,2.5,270,380)--cycle,lightgray);
dot(A\^\^B\^\^C\^\^D\^\^EE\^\^F);
draw(A--B--C--D--cycle);
draw(D--EE--F--cycle);
label("$110^\circ$", (15,9), SW);
\end{asy}
\end{center}


$\textbf{(A) }160\qquad\textbf{(B) }164\qquad\textbf{(C) }166\qquad\textbf{(D) }170\qquad\textbf{(E) }174$\par \vspace{0.5em}\item A school has $100$ students and $5$ teachers. In the first period, each student is taking one class, and each teacher is teaching one class. The enrollments in the classes are $50, 20, 20, 5, $ and $5$. Let $t$ be the average value obtained if a teacher is picked at random and the number of students in their class is noted. Let $s$ be the average value obtained if a student was picked at random and the number of students in their class, including the student, is noted. What is $t-s$?

$\textbf{(A)}\ {-}18.5  \qquad\textbf{(B)}\ {-}13.5 \qquad\textbf{(C)}\ 0 \qquad\textbf{(D)}\ 13.5 \qquad\textbf{(E)}\ 18.5$\par \vspace{0.5em}\item Let $M$ be the least common multiple of all the integers $10$ through $30,$ inclusive. Let $N$ be the least common multiple of $M,32,33,34,35,36,37,38,39,$ and $40.$ What is the value of $\frac{N}{M}?$

$\textbf{(A)}\ 1 \qquad\textbf{(B)}\ 2 \qquad\textbf{(C)}\ 37 \qquad\textbf{(D)}\ 74 \qquad\textbf{(E)}\ 2886$\par \vspace{0.5em}\item A right rectangular prism whose surface area and volume are numerically equal has edge lengths $\log_{2}x, \log_{3}x,$ and $\log_{4}x.$ What is $x?$

$\textbf{(A)}\ 2\sqrt{6} \qquad\textbf{(B)}\ 6\sqrt{6} \qquad\textbf{(C)}\ 24 \qquad\textbf{(D)}\ 48 \qquad\textbf{(E)}\ 576$\par \vspace{0.5em}\item The base-nine representation of the number $N$ is $27{,}006{,}000{,}052_{\text{nine}}.$ What is the remainder when $N$ is divided by $5?$

$\textbf{(A) } 0\qquad\textbf{(B) } 1\qquad\textbf{(C) } 2\qquad\textbf{(D) } 3\qquad\textbf{(E) }4$\par \vspace{0.5em}\item Consider two concentric circles of radius $17$ and $19.$ The larger circle has a chord, half of which lies inside the smaller circle. What is the length of the chord in the larger circle?

$\textbf{(A)}\ 12\sqrt{2} \qquad\textbf{(B)}\ 10\sqrt{3} \qquad\textbf{(C)}\ \sqrt{17 \cdot 19} \qquad\textbf{(D)}\ 18 \qquad\textbf{(E)}\ 8\sqrt{6}$\par \vspace{0.5em}\item What is the number of terms with rational coefficients among the $1001$ terms in the expansion of $\left(x\sqrt[3]{2}+y\sqrt{3}\right)^{1000}?$

$\textbf{(A)}\ 0 \qquad\textbf{(B)}\ 166 \qquad\textbf{(C)}\ 167 \qquad\textbf{(D)}\ 500 \qquad\textbf{(E)}\ 501$\par \vspace{0.5em}\item The angle bisector of the acute angle formed at the origin by the graphs of the lines $y = x$ and $y=3x$ has equation $y=kx.$ What is $k?$

$\textbf{(A)} \ \frac{1+\sqrt{5}}{2} \qquad \textbf{(B)} \ \frac{1+\sqrt{7}}{2} \qquad \textbf{(C)} \ \frac{2+\sqrt{3}}{2} \qquad \textbf{(D)} \ 2\qquad \textbf{(E)} \ \frac{2+\sqrt{5}}{2}$\par \vspace{0.5em}\item In the figure, equilateral hexagon $ABCDEF$ has three nonadjacent acute interior angles that each measure $30^\circ$. The enclosed area of the hexagon is $6\sqrt{3}$. What is the perimeter of the hexagon?

\begin{center}
\begin{asy}
import olympiad;
import cse5;
size(10cm);
pen p=black+linewidth(1),q=black+linewidth(5);
pair C=(0,0),D=(cos(pi/12),sin(pi/12)),E=rotate(150,D)*C,F=rotate(-30,E)*D,A=rotate(150,F)*E,B=rotate(-30,A)*F;
draw(C--D--E--F--A--B--cycle,p);
dot(A,q);
dot(B,q);
dot(C,q);
dot(D,q);
dot(E,q);
dot(F,q);
label("$C$",C,2*S);
label("$D$",D,2*S);
label("$E$",E,2*S);
label("$F$",F,2*dir(0));
label("$A$",A,2*N);
label("$B$",B,2*W);
\end{asy}
\end{center}

$\textbf{(A)} \: 4 \qquad \textbf{(B)} \: 4\sqrt3 \qquad \textbf{(C)} \: 12 \qquad \textbf{(D)} \: 18 \qquad \textbf{(E)} \: 12\sqrt3$\par \vspace{0.5em}\item Recall that the conjugate of the complex number $w = a + bi$, where $a$ and $b$ are real numbers and $i = \sqrt{-1}$, is the complex number $\overline{w} = a - bi$. For any complex number $z$, let $f(z) = 4i\hspace{1pt}\overline{z}$. The polynomial 
\begin\{equation*\}
P(z) = z\^4 + 4z\^3 + 3z\^2 + 2z + 1
\end\{equation*\}
 has four complex roots: $z_1$, $z_2$, $z_3$, and $z_4$. Let 
\begin\{equation*\}
Q(z) = z\^4 + Az\^3 + Bz\^2 + Cz + D
\end\{equation*\}
 be the polynomial whose roots are $f(z_1)$, $f(z_2)$, $f(z_3)$, and $f(z_4)$, where the coefficients $A,$ $B,$ $C,$ and $D$ are complex numbers. What is $B + D?$

$(\textbf{A})\: {-}304\qquad(\textbf{B}) \: {-}208\qquad(\textbf{C}) \: 12i\qquad(\textbf{D}) \: 208\qquad(\textbf{E}) \: 304$\par \vspace{0.5em}\item An organization has $30$ employees, $20$ of whom have a brand A computer while the other $10$ have a brand B computer. For security, the computers can only be connected to each other and only by cables. The cables can only connect a brand A computer to a brand B computer. Employees can communicate with each other if their computers are directly connected by a cable or by relaying messages through a series of connected computers. Initially, no computer is connected to any other. A technician arbitrarily selects one computer of each brand and installs a cable between them, provided there is not already a cable between that pair. The technician stops once every employee can communicate with each other. What is the maximum possible number of cables used?

\$\textbf\{(A)\}\ 190  \qquad\textbf\{(B)\}\  191 \qquad\textbf\{(C)\}\  192 \qquad\textbf\{(D)\}\
 195 \qquad\textbf\{(E)\}\ 196\$\par \vspace{0.5em}\item For how many ordered pairs $(b,c)$ of positive integers does neither $x^2+bx+c=0$ nor $x^2+cx+b=0$ have two distinct real solutions?

$\textbf{(A) } 4 \qquad \textbf{(B) } 6 \qquad \textbf{(C) } 8 \qquad \textbf{(D) } 12 \qquad \textbf{(E) } 16 \qquad$\par \vspace{0.5em}\item Each of $20$ balls is tossed independently and at random into one of $5$ bins. Let $p$ be the probability that some bin ends up with $3$ balls, another with $5$ balls, and the other three with $4$ balls each. Let $q$ be the probability that every bin ends up with $4$ balls. What is $\frac{p}{q}$?

$\textbf{(A)}\ 1 \qquad\textbf{(B)}\  4 \qquad\textbf{(C)}\  8 \qquad\textbf{(D)}\  12 \qquad\textbf{(E)}\ 16$\par \vspace{0.5em}\item Let $x$ be the least real number greater than $1$ such that $\sin(x) = \sin(x^2)$, where the arguments are in degrees. What is $x$ rounded up to the closest integer?

$\textbf{(A) } 10 \qquad \textbf{(B) } 13 \qquad \textbf{(C) } 14 \qquad \textbf{(D) } 19 \qquad \textbf{(E) } 20$\par \vspace{0.5em}\item For each positive integer $n$, let $f_1(n)$ be twice the number of positive integer divisors of $n$, and for $j \ge 2$, let $f_j(n) = f_1(f_{j-1}(n))$. For how many values of $n \le 50$ is $f_{50}(n) = 12?$

$\textbf{(A) }7\qquad\textbf{(B) }8\qquad\textbf{(C) }9\qquad\textbf{(D) }10\qquad\textbf{(E) }11$\par \vspace{0.5em}\item Let $ABCD$ be an isosceles trapezoid with $\overline{BC} \parallel \overline{AD}$ and $AB=CD$. Points $X$ and $Y$ lie on diagonal $\overline{AC}$ with $X$ between $A$ and $Y$, as shown in the figure. Suppose $\angle AXD = \angle BYC = 90^\circ$, $AX = 3$, $XY = 1$, and $YC = 2$. What is the area of $ABCD$?


\begin{center}
\begin{asy}
import olympiad;
import cse5;
size(10cm);
usepackage("mathptmx");
import geometry;
void perp(picture pic=currentpicture,
pair O, pair M, pair B, real size=5,
pen p=currentpen, filltype filltype = NoFill)\{
perpendicularmark(pic, M,unit(unit(O-M)+unit(B-M)),size,p,filltype);
\}
pen p=black+linewidth(1),q=black+linewidth(5);
pair C=(0,0),Y=(2,0),X=(3,0),A=(6,0),B=(2,sqrt(5.6)),D=(3,-sqrt(12.6));
draw(A--B--C--D--cycle,p);
draw(A--C,p);
draw(B--Y,p);
draw(D--X,p);
dot(A,q);
dot(B,q);
dot(C,q);
dot(D,q);
dot(X,q);
dot(Y,q);
label("2",C--Y,S);
label("1",Y--X,S);
label("3",X--A,S);
label("$A$",A,2*E);
label("$B$",B,2*N);
label("$C$",C,2*W);
label("$D$",D,2*S);
label("$Y$",Y,2*sqrt(2)*NE);
label("$X$",X,2*N);
perp(B,Y,C,8,p);
perp(A,X,D,8,p);
\end{asy}
\end{center}

$\textbf{(A)}\: 15\qquad\textbf{(B)} \: 5\sqrt{11}\qquad\textbf{(C)} \: 3\sqrt{35}\qquad\textbf{(D)} \: 18\qquad\textbf{(E)} \: 7\sqrt{7}$\par \vspace{0.5em}\item Azar and Carl play a game of tic-tac-toe. Azar places an $X$ in one of the boxes in a $3$-by-$3$ array of boxes, then Carl places an $O$ in one of the remaining boxes. After that, Azar places an $X$ in one of the remaining boxes, and so on until all boxes are filled or one of the players has of their symbols in a rowhorizontal, vertical, or diagonalwhichever comes first, in which case that player wins the game. Suppose the players make their moves at random, rather than trying to follow a rational strategy, and that Carl wins the game when he places his third $O$. How many ways can the board look after the game is over?

$\textbf{(A) } 36 \qquad\textbf{(B) } 112 \qquad\textbf{(C) } 120 \qquad\textbf{(D) } 148 \qquad\textbf{(E) } 160$\par \vspace{0.5em}\item A quadratic polynomial with real coefficients and leading coefficient $1$ is called $\emph{disrespectful}$ if the equation $p(p(x))=0$ is satisfied by exactly three real numbers. Among all the disrespectful quadratic polynomials, there is a unique such polynomial $\tilde{p}(x)$ for which the sum of the roots is maximized. What is $\tilde{p}(1)$?

$\textbf{(A) } \frac{5}{16} \qquad\textbf{(B) } \frac{1}{2} \qquad\textbf{(C) } \frac{5}{8} \qquad\textbf{(D) } 1 \qquad\textbf{(E) } \frac{9}{8}$\par \vspace{0.5em}\item Convex quadrilateral $ABCD$ has $AB = 18, \angle{A} = 60^\circ,$ and $\overline{AB} \parallel \overline{CD}.$ In some order, the lengths of the four sides form an arithmetic progression, and side $\overline{AB}$ is a side of maximum length. The length of another side is $a.$ What is the sum of all possible values of $a$?

$\textbf{(A) } 24 \qquad \textbf{(B) } 42 \qquad \textbf{(C) } 60 \qquad \textbf{(D) } 66 \qquad \textbf{(E) } 84$\par \vspace{0.5em}\item Let $m\ge 5$ be an odd integer, and let $D(m)$ denote the number of quadruples $(a_1, a_2, a_3, a_4)$ of distinct integers with $1\le a_i \le m$ for all $i$ such that $m$ divides $a_1+a_2+a_3+a_4$. There is a polynomial

\begin\{equation*\}
q(x) = c\_3x\^3+c\_2x\^2+c\_1x+c\_0
\end\{equation*\}
such that $D(m) = q(m)$ for all odd integers $m\ge 5$. What is $c_1?$

$\textbf{(A)}\ {-}6\qquad\textbf{(B)}\ {-}1\qquad\textbf{(C)}\ 4\qquad\textbf{(D)}\ 6\qquad\textbf{(E)}\ 11$\par \vspace{0.5em}\end{enumerate}\newpage\section*{2021 AMC1212B}\begin{enumerate}[label=\arabic*., itemsep=0.5em]\item What is the value of $1234+2341+3412+4123?$

$\textbf{(A)}\: 10{,}000\qquad\textbf{(B)} \: 10{,}010\qquad\textbf{(C)} \: 10{,}110\qquad\textbf{(D)} \: 11{,}000\qquad\textbf{(E)} \: 11{,}110$\par \vspace{0.5em}\item What is the area of the shaded figure shown below?

\begin{center}
\begin{asy}
import olympiad;
import cse5;
size(200);
defaultpen(linewidth(0.4)+fontsize(12));
pen s = linewidth(0.8)+fontsize(8);

pair O,X,Y;
O = origin;
X = (6,0);
Y = (0,5);
fill((1,0)--(3,5)--(5,0)--(3,2)--cycle, palegray+opacity(0.2));
for (int i=1; i<7; ++i)
\{
draw((i,0)--(i,5), gray+dashed);
label("${"+string(i)+"}$", (i,0), 2*S);
if (i<6)
\{
draw((0,i)--(6,i), gray+dashed);
label("${"+string(i)+"}$", (0,i), 2*W);
\}
\}
label("$0$", O, 2*SW);
draw(O--X+(0.35,0), black+1.5, EndArrow(10));
draw(O--Y+(0,0.35), black+1.5, EndArrow(10));
draw((1,0)--(3,5)--(5,0)--(3,2)--(1,0), black+1.5);
\end{asy}
\end{center}


$\textbf{(A)}\: 4\qquad\textbf{(B)} \: 6\qquad\textbf{(C)} \: 8\qquad\textbf{(D)} \: 10\qquad\textbf{(E)} \: 12$\par \vspace{0.5em}\item At noon on a certain day, Minneapolis is $N$ degrees warmer than St. Louis. At $4{:}00$ the temperature in Minneapolis has fallen by $5$ degrees while the temperature in St. Louis has risen by $3$ degrees, at which time the temperatures in the two cities differ by $2$ degrees. What is the product of all possible values of $N?$

$\textbf{(A)}\: 10\qquad\textbf{(B)} \: 30\qquad\textbf{(C)} \: 60\qquad\textbf{(D)} \: 100\qquad\textbf{(E)} \: 120$\par \vspace{0.5em}\item Let $n=8^{2022}$. Which of the following is equal to $\frac{n}{4}?$

$\textbf{(A)}\: 4^{1010}\qquad\textbf{(B)} \: 2^{2022}\qquad\textbf{(C)} \: 8^{2018}\qquad\textbf{(D)} \: 4^{3031}\qquad\textbf{(E)} \: 4^{3032}$\par \vspace{0.5em}\item Call a fraction $\frac{a}{b}$, not necessarily in the simplest form, ''special'' if $a$ and $b$ are positive integers whose sum is $15$. How many distinct integers can be written as the sum of two, not necessarily different, special fractions?

$\textbf{(A)}\ 9 \qquad\textbf{(B)}\  10 \qquad\textbf{(C)}\  11 \qquad\textbf{(D)}\ 12 \qquad\textbf{(E)}\ 13$\par \vspace{0.5em}\item The greatest prime number that is a divisor of $16{,}384$ is $2$ because $16{,}384 = 2^{14}$. What is the sum of the digits of the greatest prime number that is a divisor of $16{,}383$?

$\textbf{(A)} \: 3\qquad\textbf{(B)} \: 7\qquad\textbf{(C)} \: 10\qquad\textbf{(D)} \: 16\qquad\textbf{(E)} \: 22$\par \vspace{0.5em}\item Which of the following conditions is sufficient to guarantee that integers $x$, $y$, and $z$ satisfy the equation

\begin\{equation*\}
x(x-y)+y(y-z)+z(z-x) = 1?
\end\{equation*\}


$\textbf{(A)} \: x>y$ and $y=z$

$\textbf{(B)} \: x=y-1$ and $y=z-1$

$\textbf{(C)} \: x=z+1$ and $y=x+1$

$\textbf{(D)} \: x=z$ and $y-1=x$

$\textbf{(E)} \: x+y+z=1$\par \vspace{0.5em}\item The product of the lengths of the two congruent sides of an obtuse isosceles triangle is equal to the product of the base and twice the triangle's height to the base. What is the measure, in degrees, of the vertex angle of this triangle?

$\textbf{(A)} \: 105 \qquad\textbf{(B)} \: 120 \qquad\textbf{(C)} \: 135 \qquad\textbf{(D)} \: 150 \qquad\textbf{(E)} \: 165$\par \vspace{0.5em}\item Triangle $ABC$ is equilateral with side length $6$. Suppose that $O$ is the center of the inscribed
circle of this triangle. What is the area of the circle passing through $A$, $O$, and $C$?

$\textbf{(A)} \: 9\pi \qquad\textbf{(B)} \: 12\pi \qquad\textbf{(C)} \: 18\pi \qquad\textbf{(D)} \: 24\pi \qquad\textbf{(E)} \: 27\pi$\par \vspace{0.5em}\item What is the sum of all possible values of $t$ between $0$ and $360$ such that the triangle in the coordinate plane whose vertices are 
\begin\{equation*\}
(\cos 40\^\circ,\sin 40\^\circ), (\cos 60\^\circ,\sin 60\^\circ), \text\{ and \} (\cos t\^\circ,\sin t\^\circ)
\end\{equation*\}

is isosceles? 

$\textbf{(A)} \: 100 \qquad\textbf{(B)} \: 150 \qquad\textbf{(C)} \: 330 \qquad\textbf{(D)} \: 360 \qquad\textbf{(E)} \: 380$\par \vspace{0.5em}\item Una rolls $6$ standard $6$-sided dice simultaneously and calculates the product of the $6{ }$ numbers obtained. What is the probability that the product is divisible by $4?$

$\textbf{(A)}\: \frac34\qquad\textbf{(B)} \: \frac{57}{64}\qquad\textbf{(C)} \: \frac{59}{64}\qquad\textbf{(D)} \: \frac{187}{192}\qquad\textbf{(E)} \: \frac{63}{64}$\par \vspace{0.5em}\item For $n$ a positive integer, let $f(n)$ be the quotient obtained when the sum of all positive divisors of $n$ is divided by $n.$ For example, 
\begin\{equation*\}
f(14)=(1+2+7+14)\div 14=\frac\{12\}\{7\}
\end\{equation*\}

What is $f(768)-f(384)?$

\$\textbf\{(A)\}\ \frac\{1\}\{768\} \qquad\textbf\{(B)\}\ \frac\{1\}\{192\} \qquad\textbf\{(C)\}\ 1 \qquad\textbf\{(D)\}\
\frac\{4\}\{3\} \qquad\textbf\{(E)\}\ \frac\{8\}\{3\}\$\par \vspace{0.5em}\item Let $c = \frac{2\pi}{11}.$ What is the value of

\begin\{equation*\}
\frac\{\sin 3c \cdot \sin 6c \cdot \sin 9c \cdot \sin 12c \cdot \sin 15c\}\{\sin c \cdot \sin 2c \cdot \sin 3c \cdot \sin 4c \cdot \sin 5c\}?
\end\{equation*\}


\$\textbf\{(A)\}\ \{-\}1 \qquad\textbf\{(B)\}\ \{-\}\frac\{\sqrt\{11\}\}\{5\} \qquad\textbf\{(C)\}\ \frac\{\sqrt\{11\}\}\{5\} \qquad\textbf\{(D)\}\
\frac\{10\}\{11\} \qquad\textbf\{(E)\}\ 1\$\par \vspace{0.5em}\item Suppose that $P(z), Q(z)$, and $R(z)$ are polynomials with real coefficients, having degrees $2$, $3$, and $6$, respectively, and constant terms $1$, $2$, and $3$, respectively. Let $N$ be the number of distinct complex numbers $z$ that satisfy the equation $P(z) \cdot Q(z)=R(z)$. What is the minimum possible value of $N$?

$\textbf{(A)}\: 0\qquad\textbf{(B)} \: 1\qquad\textbf{(C)} \: 2\qquad\textbf{(D)} \: 3\qquad\textbf{(E)} \: 5$\par \vspace{0.5em}\item Three identical square sheets of paper each with side length $6$ are stacked on top of each other. The middle sheet is rotated clockwise $30^\circ$ about its center and the top sheet is rotated clockwise $60^\circ$ about its center, resulting in the $24$-sided polygon shown in the figure below. The area of this polygon can be expressed in the form $a-b\sqrt{c}$, where $a$, $b$, and $c$ are positive integers, and $c$ is not divisible by the square of any prime. What is $a+b+c$?
<center>
\begin{center}
\begin{asy}
import olympiad;
import cse5;
defaultpen(fontsize(8)+0.8); size(150);
pair O,A1,B1,C1,A2,B2,C2,A3,B3,C3,A4,B4,C4;
real x=45, y=90, z=60; O=origin; 
A1=dir(x); A2=dir(x+y); A3=dir(x+2y); A4=dir(x+3y);
B1=dir(x-z); B2=dir(x+y-z); B3=dir(x+2y-z); B4=dir(x+3y-z);
C1=dir(x-2z); C2=dir(x+y-2z); C3=dir(x+2y-2z); C4=dir(x+3y-2z);
draw(A1--A2--A3--A4--A1, gray+0.25+dashed);
filldraw(B1--B2--B3--B4--cycle, white, gray+dashed+linewidth(0.25));
filldraw(C1--C2--C3--C4--cycle, white, gray+dashed+linewidth(0.25));
dot(O);
pair P1,P2,P3,P4,Q1,Q2,Q3,Q4,R1,R2,R3,R4;
P1=extension(A1,A2,B1,B2); Q1=extension(A1,A2,C3,C4); 
P2=extension(A2,A3,B2,B3); Q2=extension(A2,A3,C4,C1); 
P3=extension(A3,A4,B3,B4); Q3=extension(A3,A4,C1,C2); 
P4=extension(A4,A1,B4,B1); Q4=extension(A4,A1,C2,C3); 
R1=extension(C2,C3,B2,B3); R2=extension(C3,C4,B3,B4); 
R3=extension(C4,C1,B4,B1); R4=extension(C1,C2,B1,B2);
draw(A1--P1--B2--R1--C3--Q1--A2);
draw(A2--P2--B3--R2--C4--Q2--A3);
draw(A3--P3--B4--R3--C1--Q3--A4);
draw(A4--P4--B1--R4--C2--Q4--A1);
\end{asy}
\end{center}
</center>
$(\textbf{A})\: 75\qquad(\textbf{B}) \: 93\qquad(\textbf{C}) \: 96\qquad(\textbf{D}) \: 129\qquad(\textbf{E}) \: 147$\par \vspace{0.5em}\item Suppose $a$, $b$, $c$ are positive integers such that 
\begin\{equation*\}
a+b+c=23
\end\{equation*\}
 and 
\begin\{equation*\}
\gcd(a,b)+\gcd(b,c)+\gcd(c,a)=9.
\end\{equation*\}
 What is the sum of all possible distinct values of $a^2+b^2+c^2$? 

$\textbf{(A)}\: 259\qquad\textbf{(B)} \: 438\qquad\textbf{(C)} \: 516\qquad\textbf{(D)} \: 625\qquad\textbf{(E)} \: 687$\par \vspace{0.5em}\item A bug starts at a vertex of a grid made of equilateral triangles of side length $1$. At each step the bug moves in one of the $6$ possible directions along the grid lines randomly and independently with equal probability. What is the probability that after $5$ moves the bug never will have been more than $1$ unit away from the starting position?

\$\textbf\{(A)\}\ \frac\{13\}\{108\} \qquad\textbf\{(B)\}\  \frac\{7\}\{54\} \qquad\textbf\{(C)\}\  \frac\{29\}\{216\} \qquad\textbf\{(D)\}\
\frac\{4\}\{27\} \qquad\textbf\{(E)\}\ \frac\{1\}\{16\}\$\par \vspace{0.5em}\item Set $u_0 = \frac{1}{4}$, and for $k \ge 0$ let $u_{k+1}$ be determined by the recurrence 
\begin\{equation*\}
u\_\{k+1\} = 2u\_k - 2u\_k\^2.
\end\{equation*\}


This sequence tends to a limit; call it $L$. What is the least value of $k$ such that 
\begin\{equation*\}
|u\_k-L| \le \frac\{1\}\{2\^\{1000\}\}?
\end\{equation*\}


$\textbf{(A)}\: 10\qquad\textbf{(B)}\: 87\qquad\textbf{(C)}\: 123\qquad\textbf{(D)}\: 329\qquad\textbf{(E)}\: 401$\par \vspace{0.5em}\item Regular polygons with $5,6,7,$ and $8$ sides are inscribed in the same circle. No two of the polygons share a vertex, and no three of their sides intersect at a common point. At how many points inside the circle do two of their sides intersect?

$(\textbf{A})\: 52\qquad(\textbf{B}) \: 56\qquad(\textbf{C}) \: 60\qquad(\textbf{D}) \: 64\qquad(\textbf{E}) \: 68$\par \vspace{0.5em}\item A cube is constructed from $4$ white unit cubes and $4$ blue unit cubes. How many different ways are there to construct the $2 \times 2 \times 2$ cube using these smaller cubes? (Two constructions are considered the same if one can be rotated to match the other.)

$(\textbf{A})\: 7\qquad(\textbf{B}) \: 8\qquad(\textbf{C}) \: 9\qquad(\textbf{D}) \: 10\qquad(\textbf{E}) \: 11$\par \vspace{0.5em}\item For real numbers $x$, let 

\begin\{equation*\}
P(x)=1+\cos(x)+i\sin(x)-\cos(2x)-i\sin(2x)+\cos(3x)+i\sin(3x)
\end\{equation*\}

where $i = \sqrt{-1}$. For how many values of $x$ with $0\leq x<2\pi$ does 

\begin\{equation*\}
P(x)=0?
\end\{equation*\}


\$\textbf\{(A)\}\ 0 \qquad\textbf\{(B)\}\  1 \qquad\textbf\{(C)\}\  2 \qquad\textbf\{(D)\}\
3 \qquad\textbf\{(E)\}\ 4\$\par \vspace{0.5em}\item Right triangle $ABC$ has side lengths $BC=6$, $AC=8$, and $AB=10$. A circle centered at $O$ is tangent to line $BC$ at $B$ and passes through $A$. A circle centered at $P$ is tangent to line $AC$ at $A$ and passes through $B$. What is $OP$?

\$\textbf\{(A)\}\ \frac\{23\}\{8\} \qquad\textbf\{(B)\}\  \frac\{29\}\{10\} \qquad\textbf\{(C)\}\  \frac\{35\}\{12\} \qquad\textbf\{(D)\}\
\frac\{73\}\{25\} \qquad\textbf\{(E)\}\ 3\$\par \vspace{0.5em}\item What is the average number of pairs of consecutive integers in a randomly selected subset of $5$ distinct integers chosen from the set $\{ 1, 2, 3, , 30\}$? (For example the set $\{1, 17, 18, 19, 30\}$ has $2$ pairs of consecutive integers.)

\$\textbf\{(A)\}\ \frac\{2\}\{3\} \qquad\textbf\{(B)\}\ \frac\{29\}\{36\} \qquad\textbf\{(C)\}\ \frac\{5\}\{6\} \qquad\textbf\{(D)\}\
\frac\{29\}\{30\} \qquad\textbf\{(E)\}\ 1\$\par \vspace{0.5em}\item Triangle $ABC$ has side lengths $AB = 11, BC=24$, and $CA = 20$. The bisector of $\angle{BAC}$ intersects $\overline{BC}$ in point $D$, and intersects the circumcircle of $\triangle{ABC}$ in point $E \ne A$. The circumcircle of $\triangle{BED}$ intersects the line $AB$ in points $B$ and $F \ne B$. What is $CF$?

$\textbf{(A) } 28 \qquad \textbf{(B) } 20\sqrt{2} \qquad \textbf{(C) } 30 \qquad \textbf{(D) } 32 \qquad \textbf{(E) } 20\sqrt{3}$\par \vspace{0.5em}\item For $n$ a positive integer, let $R(n)$ be the sum of the remainders when $n$ is divided by $2$, $3$, $4$, $5$, $6$, $7$, $8$, $9$, and $10$. For example, $R(15) = 1+0+3+0+3+1+7+6+5=26$. How many two-digit positive integers $n$ satisfy $R(n) = R(n+1)\,?$

$\textbf{(A) }0\qquad\textbf{(B) }1\qquad\textbf{(C) }2\qquad\textbf{(D) }3\qquad\textbf{(E) }4$\par \vspace{0.5em}\end{enumerate}\newpage\section*{2022 AMC1212A}\begin{enumerate}[label=\arabic*., itemsep=0.5em]\item What is the value of 
\begin\{equation*\}
3+\frac\{1\}\{3+\frac\{1\}\{3+\frac13\}\}?
\end\{equation*\}

$\textbf{(A)}\ \frac{31}{10}\qquad\textbf{(B)}\ \frac{49}{15}\qquad\textbf{(C)}\ \frac{33}{10}\qquad\textbf{(D)}\ \frac{109}{33}\qquad\textbf{(E)}\ \frac{15}{4}$\par \vspace{0.5em}\item The sum of three numbers is $96.$ The first number is $6$ times the third number, and the third number is $40$ less than the second number. What is the absolute value of the difference between the first and second numbers?

$\textbf{(A) } 1 \qquad \textbf{(B) } 2 \qquad \textbf{(C) } 3 \qquad \textbf{(D) } 4 \qquad \textbf{(E) } 5$\par \vspace{0.5em}\item Five rectangles, $A$, $B$, $C$, $D$, and $E$, are arranged in a square as shown below. These rectangles have dimensions $1\times6$, $2\times4$, $5\times6$, $2\times7$, and $2\times3$, respectively. (The figure is not drawn to scale.) Which of the five rectangles is the shaded one in the middle?

\begin{center}
\begin{asy}
import olympiad;
import cse5;
size(150);
currentpen = black+1.25bp;
fill((3,2.5)--(3,4.5)--(5.3,4.5)--(5.3,2.5)--cycle,gray);
draw((0,0)--(7,0)--(7,7)--(0,7)--(0,0));
draw((3,0)--(3,4.5));
draw((0,4.5)--(5.3,4.5));
draw((5.3,7)--(5.3,2.5));
draw((7,2.5)--(3,2.5));
\end{asy}
\end{center}

$\textbf{(A) }A\qquad\textbf{(B) }B \qquad\textbf{(C) }C \qquad\textbf{(D) }D\qquad\textbf{(E) }E$\par \vspace{0.5em}\item The least common multiple of a positive integer $n$ and $18$ is $180$, and the greatest common divisor of $n$ and $45$ is $15$. What is the sum of the digits of $n$?

$\textbf{(A) } 3 \qquad \textbf{(B) } 6 \qquad \textbf{(C) } 8 \qquad \textbf{(D) } 9 \qquad \textbf{(E) } 12$\par \vspace{0.5em}\item The <em>taxicab distance</em> between points $(x_1, y_1)$ and $(x_2, y_2)$ in the coordinate plane is given by 
\begin\{equation*\}
|x\_1 - x\_2| + |y\_1 - y\_2|.
\end\{equation*\}

For how many points $P$ with integer coordinates is the taxicab distance between $P$ and the origin less than or equal to $20$?

$\textbf{(A)} \, 441 \qquad\textbf{(B)} \, 761 \qquad\textbf{(C)} \, 841 \qquad\textbf{(D)} \, 921  \qquad\textbf{(E)} \, 924 $\par \vspace{0.5em}\item A data set consists of $6$ (not distinct) positive integers: $1$, $7$, $5$, $2$, $5$, and $X$. The
average (arithmetic mean) of the $6$ numbers equals a value in the data set. What is
the sum of all possible values of $X$?

$\textbf{(A) } 10 \qquad \textbf{(B) } 26 \qquad \textbf{(C) } 32 \qquad \textbf{(D) } 36 \qquad \textbf{(E) } 40$\par \vspace{0.5em}\item A rectangle is partitioned into $5$ regions as shown. Each region is to be painted a solid color - red, orange, yellow, blue, or green - so that regions that touch are painted different colors, and colors can be used more than once. How many different colorings are possible?


\begin{center}
\begin{asy}
import olympiad;
import cse5;
size(5.5cm); draw((0,0)--(0,2)--(2,2)--(2,0)--cycle); draw((2,0)--(8,0)--(8,2)--(2,2)--cycle); draw((8,0)--(12,0)--(12,2)--(8,2)--cycle); draw((0,2)--(6,2)--(6,4)--(0,4)--cycle); draw((6,2)--(12,2)--(12,4)--(6,4)--cycle);
\end{asy}
\end{center}


$\textbf{(A) }120\qquad\textbf{(B) }270\qquad\textbf{(C) }360\qquad\textbf{(D) }540\qquad\textbf{(E) }720$\par \vspace{0.5em}\item The infinite product

\begin\{equation*\}
\sqrt[3]\{10\} \cdot \sqrt[3]\{\sqrt[3]\{10\}\} \cdot \sqrt[3]\{\sqrt[3]\{\sqrt[3]\{10\}\}\} \cdots
\end\{equation*\}

evaluates to a real number. What is that number?

$\textbf{(A) }\sqrt{10}\qquad\textbf{(B) }\sqrt[3]{100}\qquad\textbf{(C) }\sqrt[4]{1000}\qquad\textbf{(D) }10\qquad\textbf{(E) }10\sqrt[3]{10}$\par \vspace{0.5em}\item On Halloween $31$ children walked into the principal's office asking for candy. They
can be classified into three types: Some always lie; some always tell the truth; and
some alternately lie and tell the truth. The alternaters arbitrarily choose their first
response, either a lie or the truth, but each subsequent statement has the opposite
truth value from its predecessor. The principal asked everyone the same three
questions in this order.

"Are you a truth-teller?" The principal gave a piece of candy to each of the $22$
children who answered yes.

"Are you an alternater?" The principal gave a piece of candy to each of the $15$
children who answered yes.

"Are you a liar?" The principal gave a piece of candy to each of the $9$ children who
answered yes.

How many pieces of candy in all did the principal give to the children who always
tell the truth?

$\textbf{(A) } 7 \qquad \textbf{(B) } 12 \qquad \textbf{(C) } 21 \qquad \textbf{(D) } 27 \qquad \textbf{(E) } 31$\par \vspace{0.5em}\item How many ways are there to split the integers $1$ through $14$ into $7$ pairs such that in each pair, the greater number is at least $2$ times the lesser number?

$\textbf{(A) } 108 \qquad \textbf{(B) } 120 \qquad \textbf{(C) } 126 \qquad \textbf{(D) } 132 \qquad \textbf{(E) } 144$\par \vspace{0.5em}\item What is the product of all real numbers $x$ such that the distance on the number line between $\log_6x$ and $\log_69$ is twice the distance on the number line between $\log_610$ and $1$?

$\textbf{(A) } 10 \qquad \textbf{(B) } 18 \qquad \textbf{(C) } 25 \qquad \textbf{(D) } 36 \qquad \textbf{(E) } 81$\par \vspace{0.5em}\item Let $M$ be the midpoint of $\overline{AB}$ in regular tetrahedron $ABCD$. What is $\cos(\angle CMD)$?

$\textbf{(A) } \frac14 \qquad \textbf{(B) } \frac13 \qquad \textbf{(C) } \frac25 \qquad \textbf{(D) } \frac12 \qquad \textbf{(E) } \frac{\sqrt{3}}{2}$\par \vspace{0.5em}\item Let $\mathcal{R}$ be the region in the complex plane consisting of all complex numbers $z$ that can be written as the sum of complex numbers $z_1$ and $z_2$, where $z_1$ lies on the segment with endpoints $3$ and $4i$, and $z_2$ has magnitude at most $1$. What integer is closest to the area of $\mathcal{R}$?  

$\textbf{(A) } 13 \qquad \textbf{(B) } 14 \qquad \textbf{(C) } 15 \qquad \textbf{(D) } 16 \qquad \textbf{(E) } 17$\par \vspace{0.5em}\item What is the value of 
\begin\{equation*\}
(\log 5)\^3+(\log 20)\^3+(\log 8)(\log 0.25)
\end\{equation*\}
 where $\log$ denotes the base-ten logarithm?

$\textbf{(A) } \frac{3}{2} \qquad \textbf{(B) } \frac{7}{4} \qquad \textbf{(C) } 2 \qquad \textbf{(D) } \frac{9}{4} \qquad \textbf{(E) } 3$\par \vspace{0.5em}\item The roots of the polynomial $10x^3-39x^2+29x-6$ are the height, length, and width of a rectangular box (right rectangular prism). A new rectangular box is formed by lengthening each edge of the original box by $2$ units. What is the volume of the new box?

$\textbf{(A) } \frac{24}{5} \qquad \textbf{(B) } \frac{42}{5} \qquad \textbf{(C) } \frac{81}{5} \qquad \textbf{(D) } 30 \qquad \textbf{(E) } 48$\par \vspace{0.5em}\item A triangular number is a positive integer that can be expressed in the form $t_n=1+2+3+\cdots+n$, for some positive integer $n$. The three smallest triangular numbers that are also perfect squares are $t_1=1=1^2, t_8=36=6^2,$ and $t_{49}=1225=35^2$. What is the sum of the digits of the fourth smallest triangular number that is also a perfect square?

$\textbf{(A)} ~6 \qquad\textbf{(B)} ~9 \qquad\textbf{(C)} ~12 \qquad\textbf{(D)} ~18 \qquad\textbf{(E)} ~27 $\par \vspace{0.5em}\item Suppose $a$ is a real number such that the equation 
\begin\{equation*\}
a\cdot(\sin\{x\}+\sin\{(2x)\}) = \sin\{(3x)\}
\end\{equation*\}

has more than one solution in the interval $(0, \pi)$. The set of all such $a$ that can be written
in the form 
\begin\{equation*\}
(p,q) \cup (q,r),
\end\{equation*\}

where $p, q,$ and $r$ are real numbers with $p < q< r$. What is $p+q+r$?

$\textbf{(A) } {-}4 \qquad \textbf{(B) } {-}1 \qquad \textbf{(C) } 0 \qquad \textbf{(D) } 1 \qquad \textbf{(E) } 4$\par \vspace{0.5em}\item Let $T_k$ be the transformation of the coordinate plane that first rotates the plane $k$ degrees counterclockwise around the origin and then reflects the plane across the $y$-axis. What is the least positive integer $n$ such that performing the sequence of transformations $T_1, T_2, T_3, \dots, T_n$ returns the point $(1,0)$ back to itself?

$\textbf{(A) } 359 \qquad \textbf{(B) } 360\qquad \textbf{(C) } 719 \qquad \textbf{(D) } 720 \qquad \textbf{(E) } 721$\par \vspace{0.5em}\item Suppose that $13$ cards numbered $1, 2, 3, \ldots, 13$ are arranged in a row. The task is to pick them up in numerically increasing order, working repeatedly from left to right. In the example below, cards $1, 2, 3$ are picked up on the first pass, $4$ and $5$ on the second pass, $6$ on the third pass, $7, 8, 9, 10$ on the fourth pass, and $11, 12, 13$ on the fifth pass. For how many of the $13!$ possible orderings of the cards will the $13$ cards be picked up in exactly two passes?


\begin{center}
\begin{asy}
import olympiad;
import cse5;
size(11cm);
draw((0,0)--(2,0)--(2,3)--(0,3)--cycle);
label("7", (1,1.5));
draw((3,0)--(5,0)--(5,3)--(3,3)--cycle);
label("11", (4,1.5));
draw((6,0)--(8,0)--(8,3)--(6,3)--cycle);
label("8", (7,1.5));
draw((9,0)--(11,0)--(11,3)--(9,3)--cycle);
label("6", (10,1.5));
draw((12,0)--(14,0)--(14,3)--(12,3)--cycle);
label("4", (13,1.5));
draw((15,0)--(17,0)--(17,3)--(15,3)--cycle);
label("5", (16,1.5));
draw((18,0)--(20,0)--(20,3)--(18,3)--cycle);
label("9", (19,1.5));
draw((21,0)--(23,0)--(23,3)--(21,3)--cycle);
label("12", (22,1.5));
draw((24,0)--(26,0)--(26,3)--(24,3)--cycle);
label("1", (25,1.5));
draw((27,0)--(29,0)--(29,3)--(27,3)--cycle);
label("13", (28,1.5));
draw((30,0)--(32,0)--(32,3)--(30,3)--cycle);
label("10", (31,1.5));
draw((33,0)--(35,0)--(35,3)--(33,3)--cycle);
label("2", (34,1.5));
draw((36,0)--(38,0)--(38,3)--(36,3)--cycle);
label("3", (37,1.5));
\end{asy}
\end{center}

$\textbf{(A) } 4082 \qquad \textbf{(B) } 4095 \qquad \textbf{(C) } 4096 \qquad \textbf{(D) } 8178 \qquad \textbf{(E) } 8191$\par \vspace{0.5em}\item Isosceles trapezoid $ABCD$ has parallel sides $\overline{AD}$ and $\overline{BC},$ with $BC < AD$ and $AB = CD.$ There is a point $P$ in the plane such that $PA=1, PB=2, PC=3,$ and $PD=4.$ What is $\tfrac{BC}{AD}?$

$\textbf{(A) }\frac{1}{4}\qquad\textbf{(B) }\frac{1}{3}\qquad\textbf{(C) }\frac{1}{2}\qquad\textbf{(D) }\frac{2}{3}\qquad\textbf{(E) }\frac{3}{4}$\par \vspace{0.5em}\item Let 
\begin\{equation*\}
P(x) = x\^\{2022\} + x\^\{1011\} + 1.
\end\{equation*\}
 Which of the following polynomials is a factor of $P(x)$?

$\textbf{(A)} \, x^2 -x + 1 \qquad\textbf{(B)} \, x^2 + x + 1 \qquad\textbf{(C)} \, x^4 + 1 \qquad\textbf{(D)} \, x^6 - x^3 + 1  \qquad\textbf{(E)} \, x^6 + x^3 + 1 $\par \vspace{0.5em}\item Let $c$ be a real number, and let $z_1$ and $z_2$ be the two complex numbers satisfying the equation
$z^2 - cz + 10 = 0$. Points $z_1$, $z_2$, $\frac{1}{z_1}$, and $\frac{1}{z_2}$ are the vertices of (convex) quadrilateral $\mathcal{Q}$ in the complex plane. When the area of $\mathcal{Q}$ obtains its maximum possible value, $c$ is closest to which of the following?

$\textbf{(A) }4.5 \qquad\textbf{(B) }5 \qquad\textbf{(C) }5.5 \qquad\textbf{(D) }6\qquad\textbf{(E) }6.5$\par \vspace{0.5em}\item Let $h_n$ and $k_n$ be the unique relatively prime positive integers such that 
\begin\{equation*\}
\frac\{1\}\{1\}+\frac\{1\}\{2\}+\frac\{1\}\{3\}+\cdots+\frac\{1\}\{n\}=\frac\{h\_n\}\{k\_n\}.
\end\{equation*\}
 Let $L_n$ denote the least common multiple of the numbers $1, 2, 3, \ldots, n$. For how many integers with $1\le{n}\le{22}$ is $k_n<L_n$?

$\textbf{(A) }0 \qquad\textbf{(B) }3 \qquad\textbf{(C) }7 \qquad\textbf{(D) }8\qquad\textbf{(E) }10$\par \vspace{0.5em}\item How many strings of length $5$ formed from the digits $0$, $1$, $2$, $3$, $4$ are there such that for each $j \in \{1,2,3,4\}$, at least $j$ of the digits are less than $j$? (For example, $02214$ satisfies this condition
because it contains at least $1$ digit less than $1$, at least $2$ digits less than $2$, at least $3$ digits less
than $3$, and at least $4$ digits less than $4$. The string $23404$ does not satisfy the condition because it
does not contain at least $2$ digits less than $2$.)

$\textbf{(A) }500\qquad\textbf{(B) }625\qquad\textbf{(C) }1089\qquad\textbf{(D) }1199\qquad\textbf{(E) }1296$\par \vspace{0.5em}\item A circle with integer radius $r$ is centered at $(r, r)$. Distinct line segments of length $c_i$ connect points $(0, a_i)$ to $(b_i, 0)$ for $1 \le i \le 14$ and are tangent to the circle, where $a_i$, $b_i$, and $c_i$ are all positive integers and $c_1 \le c_2 \le \cdots \le c_{14}$. What is the ratio $\frac{c_{14}}{c_1}$ for the least possible value of $r$?

$\textbf{(A)} ~\frac{21}{5} \qquad\textbf{(B)} ~\frac{85}{13} \qquad\textbf{(C)} ~7 \qquad\textbf{(D)} ~\frac{39}{5} \qquad\textbf{(E)} ~17 $\par \vspace{0.5em}\end{enumerate}\newpage\section*{2022 AMC1212B}\begin{enumerate}[label=\arabic*., itemsep=0.5em]\item Define $x\diamond y$ to be $|x-y|$ for all real numbers $x$ and $y.$ What is the value of 
\begin\{equation*\}
(1\diamond(2\diamond3))-((1\diamond2)\diamond3)?
\end\{equation*\}


\$ \textbf\{(A)\}\ \{-\}2 \qquad
\textbf\{(B)\}\ \{-\}1 \qquad
\textbf\{(C)\}\ 0 \qquad
\textbf\{(D)\}\ 1 \qquad
\textbf\{(E)\}\ 2\$\par \vspace{0.5em}\item In rhombus $ABCD$, point $P$ lies on segment $\overline{AD}$ so that $\overline{BP}$ $\perp$ $\overline{AD}$, $AP = 3$, and $PD = 2$. What is the area of $ABCD$? (Note: The figure is not drawn to scale.)


\begin{center}
\begin{asy}
import olympiad;
import cse5;
import olympiad;
size(180);
real r = 3, s = 5, t = sqrt(r*r+s*s);
defaultpen(linewidth(0.6) + fontsize(10));
pair A = (0,0), B = (r,s), C = (r+t,s), D = (t,0), P = (r,0);
draw(A--B--C--D--A\^\^B--P\^\^rightanglemark(B,P,D));
label("$A$",A,SW);
label("$B$", B, NW);
label("$C$",C,NE);
label("$D$",D,SE);
label("$P$",P,S);
\end{asy}
\end{center}


\$\textbf\{(A) \}3\sqrt 5 \qquad
\textbf\{(B) \}10 \qquad
\textbf\{(C) \}6\sqrt 5 \qquad
\textbf\{(D) \}20\qquad
\textbf\{(E) \}25\$\par \vspace{0.5em}\item How many of the first ten numbers of the sequence $121, 11211, 1112111, \ldots$ are prime numbers?

$\textbf{(A) } 0 \qquad \textbf{(B) }1 \qquad \textbf{(C) }2 \qquad \textbf{(D) }3 \qquad \textbf{(E) }4$\par \vspace{0.5em}\item For how many values of the constant $k$ will the polynomial $x^{2}+kx+36$ have two distinct integer roots?

$\textbf{(A) }6 \qquad \textbf{(B) }8 \qquad \textbf{(C) }9 \qquad \textbf{(D) }14 \qquad \textbf{(E) }16$\par \vspace{0.5em}\item The point $(-1, -2)$ is rotated $270^{\circ}$ counterclockwise about the point $(3, 1)$. What are the coordinates of its new position?

$\textbf{(A) }\ (-3, -4) \qquad \textbf{(B) }\ (0,5) \qquad \textbf{(C) }\ (2,-1) \qquad \textbf{(D) }\ (4,3) \qquad \textbf{(E) }\ (6,-3)$\par \vspace{0.5em}\item Consider the following $100$ sets of $10$ elements each:

\begin\{align*\}
\&\\{1,2,3,\ldots,10\\}, \\
\&\\{11,12,13,\ldots,20\\},\\
\&\\{21,22,23,\ldots,30\\},\\
\&\vdots\\
\&\\{991,992,993,\ldots,1000\\}.
\end\{align*\}

How many of these sets contain exactly two multiples of $7$?

$\textbf{(A)}\ 40\qquad\textbf{(B)}\ 42\qquad\textbf{(C)}\ 43\qquad\textbf{(D)}\ 49\qquad\textbf{(E)}\ 50$\par \vspace{0.5em}\item Camila writes down five positive integers. The unique mode of these integers is $2$ greater than their median, and the median is $2$ greater than their arithmetic mean. What is the least possible value for the mode?

$\textbf{(A) }5\qquad\textbf{(B) }7\qquad\textbf{(C) }9\qquad\textbf{(D) }11\qquad\textbf{(E) }13$\par \vspace{0.5em}\item What is the graph of $y^4+1=x^4+2y^2$ in the coordinate plane?

$\textbf{(A) }\ \text{two intersecting parabolas} \qquad \textbf{(B) }\ \text{two nonintersecting parabolas} \qquad \textbf{(C) }\ \text{two intersecting circles} \qquad$

$\textbf{(D) }\ \text{a circle and a hyperbola} \qquad \textbf{(E) }\ \text{a circle and two parabolas}$\par \vspace{0.5em}\item The sequence $a_0,a_1,a_2,\cdots$ is a strictly increasing arithmetic sequence of positive integers such that 
\begin\{equation*\}
2\^\{a\_7\}=2\^\{27\} \cdot a\_7.
\end\{equation*\}
 What is the minimum possible value of $a_2$?

$\textbf{(A) }\ 8 \qquad \textbf{(B) }\ 12 \qquad \textbf{(C) }\ 16 \qquad \textbf{(D) }\ 17 \qquad \textbf{(E) }\ 22$\par \vspace{0.5em}\item Regular hexagon $ABCDEF$ has side length $2$. Let $G$ be the midpoint of $\overline{AB}$, and let $H$ be the midpoint of $\overline{DE}$. What is the perimeter of $GCHF$?

\$ \textbf\{(A) \}\ 4\sqrt3 \qquad
\textbf\{(B) \}\ 8 \qquad
\textbf\{(C) \}\ 4\sqrt5 \qquad
\textbf\{(D) \}\ 4\sqrt7 \qquad
\textbf\{(E) \}\ 12\$\par \vspace{0.5em}\item Let $ f(n) = \left( \frac{-1+i\sqrt{3}}{2} \right)^n + \left( \frac{-1-i\sqrt{3}}{2} \right)^n $, where $i = \sqrt{-1}$. What is $f(2022)$?

\$ \textbf\{(A) \}\ -2 \qquad
\textbf\{(B) \}\ -1 \qquad
\textbf\{(C) \}\ 0 \qquad
\textbf\{(D) \}\ \sqrt\{3\} \qquad
\textbf\{(E) \}\ 2\$\par \vspace{0.5em}\item Kayla rolls four fair $6$-sided dice. What is the probability that at least one of the numbers Kayla rolls is greater than $4$ and at least two of the numbers she rolls are greater than $2$?

$\textbf{(A) }\frac{2}{3} \qquad \textbf{(B) }\frac{19}{27} \qquad \textbf{(C) }\frac{59}{81} \qquad \textbf{(D) }\frac{61}{81} \qquad \textbf{(E) }\frac{7}{9}$\par \vspace{0.5em}\item The diagram below shows a rectangle with side lengths $4$ and $8$ and a square with side length $5$. Three vertices of the square lie on three different sides of the rectangle, as shown. What is the area of the region inside both the square and the rectangle?


\begin{center}
\begin{asy}
import olympiad;
import cse5;
size(5cm);
filldraw((4,0)--(8,3)--(8-3/4,4)--(1,4)--cycle,mediumgray);
draw((0,0)--(8,0)--(8,4)--(0,4)--cycle,linewidth(1.1));
draw((1,0)--(1,4)--(4,0)--(8,3)--(5,7)--(1,4),linewidth(1.1));
label("$4$", (8,2), E);
label("$8$", (4,0), S);
label("$5$", (3,11/2), NW);
draw((1,.35)--(1.35,.35)--(1.35,0),linewidth(1.1));
\end{asy}
\end{center}


\$\textbf\{(A) \}15\dfrac\{1\}\{8\}  \qquad
\textbf\{(B) \}15\dfrac\{3\}\{8\}  \qquad
\textbf\{(C) \}15\dfrac\{1\}\{2\}  \qquad
\textbf\{(D) \}15\dfrac\{5\}\{8\}  \qquad
\textbf\{(E) \}15\dfrac\{7\}\{8\} \$\par \vspace{0.5em}\item The graph of $y=x^2+2x-15$ intersects the $x$-axis at points $A$ and $C$ and the $y$-axis at point $B$. What is $\tan(\angle ABC)$?

$\textbf{(A) }\frac{1}{7} \qquad \textbf{(B) }\frac{1}{4} \qquad \textbf{(C) }\frac{3}{7} \qquad \textbf{(D) }\frac{1}{2} \qquad \textbf{(E) }\frac{4}{7}$\par \vspace{0.5em}\item One of the following numbers is not divisible by any prime number less than $10.$ Which is it?

$\textbf{(A) } 2^{606}-1 \qquad\textbf{(B) } 2^{606}+1 \qquad\textbf{(C) } 2^{607}-1 \qquad\textbf{(D) } 2^{607}+1\qquad\textbf{(E) } 2^{607}+3^{607}$\par \vspace{0.5em}\item Suppose $x$ and $y$ are positive real numbers such that

\begin\{equation*\}
x\^y=2\^\{64\}\text\{ and \}(\log\_2\{x\})\^\{\log\_2\{y\}\}=2\^\{7\}.
\end\{equation*\}

What is the greatest possible value of $\log_2{y}$?

$\textbf{(A) }3 \qquad \textbf{(B) }4 \qquad \textbf{(C) }3+\sqrt{2} \qquad \textbf{(D) }4+\sqrt{3} \qquad \textbf{(E) }7$\par \vspace{0.5em}\item How many $4 \times 4$ arrays whose entries are $0$s and $1$s are there such that the row sums (the sum of the entries in each row) are $1, 2, 3,$ and $4,$ in some order, and the column sums (the sum of the entries in each column) are also $1, 2, 3,$ and $4,$ in some order? For example, the array

\begin\{equation*\}
\left[
  \begin\{array\}\{cccc\}
    1 \& 1 \& 1 \& 0 \\
    0 \& 1 \& 1 \& 0 \\
    1 \& 1 \& 1 \& 1 \\
    0 \& 1 \& 0 \& 0 \\
  \end\{array\}
\right]
\end\{equation*\}

satisfies the condition.

$\textbf{(A) }144 \qquad \textbf{(B) }240 \qquad \textbf{(C) }336 \qquad \textbf{(D) }576 \qquad \textbf{(E) }624$\par \vspace{0.5em}\item Each square in a $5 \times 5$ grid is either filled or empty, and has up to eight adjacent neighboring squares, where neighboring squares share either a side or a corner. The grid is transformed by the following rules:

* Any filled square with two or three filled neighbors remains filled.

* Any empty square with exactly three filled neighbors becomes a filled square.

* All other squares remain empty or become empty.

A sample transformation is shown in the figure below.

\begin{center}
\begin{asy}
import olympiad;
import cse5;
import geometry;
        unitsize(0.6cm);

        void ds(pair x) \{
            filldraw(x -- (1,0) + x -- (1,1) + x -- (0,1)+x -- cycle,mediumgray,invisible);
        \}

        ds((1,1));
        ds((2,1));
        ds((3,1));
        ds((1,3));

        for (int i = 0; i <= 5; ++i) \{
            draw((0,i)--(5,i));
            draw((i,0)--(i,5));
        \}

        label("Initial", (2.5,-1));
        draw((6,2.5)--(8,2.5),Arrow);

        ds((10,2));
        ds((11,1));
        ds((11,0));

        for (int i = 0; i <= 5; ++i) \{
            draw((9,i)--(14,i));
            draw((i+9,0)--(i+9,5));
        \}

        label("Transformed", (11.5,-1));
\end{asy}
\end{center}

Suppose the $5 \times 5$ grid has a border of empty squares surrounding a $3 \times 3$ subgrid. How many initial configurations will lead to a transformed grid consisting of a single filled square in the center after a single transformation? (Rotations and reflections of the same configuration are considered different.)

\begin{center}
\begin{asy}
import olympiad;
import cse5;
import geometry;
        unitsize(0.6cm);

        void ds(pair x) \{
            filldraw(x -- (1,0) + x -- (1,1) + x -- (0,1)+x -- cycle,mediumgray,invisible);
        \}

        for (int i = 1; i < 4; ++ i) \{
            for (int j = 1; j < 4; ++j) \{
                label("?",(i + 0.5, j + 0.5));
            \}
        \}

        for (int i = 0; i <= 5; ++i) \{
            draw((0,i)--(5,i));
            draw((i,0)--(i,5));
        \}

        label("Initial", (2.5,-1));
        draw((6,2.5)--(8,2.5),Arrow);

        ds((11,2));

        for (int i = 0; i <= 5; ++i) \{
            draw((9,i)--(14,i));
            draw((i+9,0)--(i+9,5));
        \}

        label("Transformed", (11.5,-1));
\end{asy}
\end{center}

$\textbf{(A)}\ 14 \qquad\textbf{(B)}\ 18 \qquad\textbf{(C)}\ 22 \qquad\textbf{(D)}\ 26 \qquad\textbf{(E)}\ 30$\par \vspace{0.5em}\item In $\triangle{ABC}$ medians $\overline{AD}$ and $\overline{BE}$ intersect at $G$ and $\triangle{AGE}$ is equilateral. Then $\cos(C)$ can be written as $\frac{m\sqrt p}n$, where $m$ and $n$ are relatively prime positive integers and $p$ is a positive integer not divisible by the square of any prime. What is $m+n+p?$

$\textbf{(A) }44 \qquad \textbf{(B) }48 \qquad \textbf{(C) }52 \qquad \textbf{(D) }56 \qquad \textbf{(E) }60$\par \vspace{0.5em}\item Let $P(x)$ be a polynomial with rational coefficients such that when $P(x)$ is divided by the polynomial $x^2 + x + 1$, the remainder is $x + 2$, and when $P(x)$ is divided by the polynomial $x^2 + 1$, the remainder is $2x + 1$. There is a unique polynomial of least degree with these two properties. What is the sum of the squares of the coefficients of that polynomial?

$\textbf{(A) } 10 \qquad \textbf{(B) } 13 \qquad \textbf{(C) } 19 \qquad \textbf{(D) } 20 \qquad \textbf{(E) } 23$\par \vspace{0.5em}\item Let $S$ be the set of circles in the coordinate plane that are tangent to each of the three circles with equations $x^{2}+y^{2}=4$, $x^{2}+y^{2}=64$, and $(x-5)^{2}+y^{2}=3$. What is the sum of the areas of all circles in $S$?

\$\textbf\{(A) \} 48 \pi \qquad
\textbf\{(B) \} 68 \pi \qquad
\textbf\{(C) \} 96 \pi \qquad
\textbf\{(D) \} 102 \pi \qquad
\textbf\{(E) \} 136 \pi \qquad\$\par \vspace{0.5em}\item Ant Amelia starts on the number line at $0$ and crawls in the following manner. For $n=1,2,3,$ Amelia chooses a time duration $t_n$ and an increment $x_n$ independently and uniformly at random from the interval $(0,1).$ During the $n$th step of the process, Amelia moves $x_n$ units in the positive direction, using up $t_n$ minutes. If the total elapsed time has exceeded $1$ minute during the $n$th step, she stops at the end of that step; otherwise, she continues with the next step, taking at most $3$ steps in all. What is the probability that Amelias position when she stops will be greater than $1$?

$\textbf{(A) }\frac{1}{3} \qquad \textbf{(B) }\frac{1}{2} \qquad \textbf{(C) }\frac{2}{3} \qquad \textbf{(D) }\frac{3}{4} \qquad \textbf{(E) }\frac{5}{6}$\par \vspace{0.5em}\item Let $x_0,x_1,x_2,\dotsc$ be a sequence of numbers, where each $x_k$ is either $0$ or $1$. For each positive integer $n$, define 

\begin\{equation*\}
S\_n = \sum\_\{k=0\}\^\{n-1\} x\_k 2\^k
\end\{equation*\}

Suppose $7S_n \equiv 1 \pmod{2^n}$ for all $n \geq 1$. What is the value of the sum

\begin\{equation*\}
x\_\{2019\} + 2x\_\{2020\} + 4x\_\{2021\} + 8x\_\{2022\}?
\end\{equation*\}

$\textbf{(A) } 6 \qquad \textbf{(B) } 7 \qquad \textbf{(C) }12\qquad \textbf{(D) } 14\qquad \textbf{(E) }15$\par \vspace{0.5em}\item The figure below depicts a regular $7$-gon inscribed in a unit circle.

\begin{center}
\begin{asy}
import olympiad;
import cse5;
import geometry;
unitsize(3cm);
draw(circle((0,0),1),linewidth(1.5));
for (int i = 0; i < 7; ++i) \{
  for (int j = 0; j < i; ++j) \{
    draw(dir(i * 360/7) -- dir(j * 360/7),linewidth(1.5));
  \}
\}
for(int i = 0; i < 7; ++i) \{ 
  dot(dir(i * 360/7),5+black);
\}
\end{asy}
\end{center}

What is the sum of the $4$th powers of the lengths of all $21$ of its edges and diagonals?

$\textbf{(A) }49 \qquad \textbf{(B) }98 \qquad \textbf{(C) }147 \qquad \textbf{(D) }168 \qquad \textbf{(E) }196$\par \vspace{0.5em}\item Four regular hexagons surround a square with a side length $1$, each one sharing an edge with the square, as shown in the figure below. The area of the resulting 12-sided outer nonconvex polygon can be written as $m\sqrt{n} + p$, where $m$, $n$, and $p$ are integers and $n$ is not divisible by the square of any prime. What is $m + n + p$?


\begin{center}
\begin{asy}
import olympiad;
import cse5;
import geometry;
        unitsize(3cm);
        draw((0,0) -- (1,0) -- (1,1) -- (0,1) -- cycle);
        draw(shift((1/2,1-sqrt(3)/2))*polygon(6));
        draw(shift((1/2,sqrt(3)/2))*polygon(6));
        draw(shift((sqrt(3)/2,1/2))*rotate(90)*polygon(6));
        draw(shift((1-sqrt(3)/2,1/2))*rotate(90)*polygon(6));
		draw((0,1-sqrt(3))--(1,1-sqrt(3))--(3-sqrt(3),sqrt(3)-2)--(sqrt(3),0)--(sqrt(3),1)--(3-sqrt(3),3-sqrt(3))--(1,sqrt(3))--(0,sqrt(3))--(sqrt(3)-2,3-sqrt(3))--(1-sqrt(3),1)--(1-sqrt(3),0)--(sqrt(3)-2,sqrt(3)-2)--cycle,linewidth(2));
\end{asy}
\end{center}


\$\textbf\{(A) \} -12 \qquad
\textbf\{(B) \}-4 \qquad 
\textbf\{(C) \} 4 \qquad
\textbf\{(D) \}24 \qquad
\textbf\{(E) \}32\$\par \vspace{0.5em}\end{enumerate}\newpage\section*{2023 AMC1212A}\begin{enumerate}[label=\arabic*., itemsep=0.5em]\item Cities $A$ and $B$ are $45$ miles apart. Alicia lives in $A$ and Beth lives in $B$. Alicia bikes towards $B$ at 18 miles per hour. Leaving at the same time, Beth bikes toward $A$ at 12 miles per hour. How many miles from City $A$ will they be when they meet?

$\textbf{(A) }20\qquad\textbf{(B) }24\qquad\textbf{(C) }25\qquad\textbf{(D) }26\qquad\textbf{(E) }27$\par \vspace{0.5em}\item The weight of $\frac{1}{3}$ of a large pizza together with $3 \frac{1}{2}$ cups of orange slices is the same weight of $\frac{3}{4}$ of a large pizza together with $\frac{1}{2}$ cups of orange slices. A cup of orange slices weigh $\frac{1}{4}$ of a pound. What is the weight, in pounds, of a large pizza?

$\textbf{(A) }1\frac{4}{5}\qquad\textbf{(B) }2\qquad\textbf{(C) }2\frac{2}{5}\qquad\textbf{(D) }3\qquad\textbf{(E) }3\frac{3}{5}$\par \vspace{0.5em}\item How many positive perfect squares less than $2023$ are divisible by $5$?

$\textbf{(A) }8\qquad\textbf{(B) }9\qquad\textbf{(C) }10\qquad\textbf{(D) }11\qquad\textbf{(E) }12$\par \vspace{0.5em}\item How many digits are in the base-ten representation of $8^5 \cdot 5^{10} \cdot 15^5$?

$\textbf{(A)}~14\qquad\textbf{(B)}~15\qquad\textbf{(C)}~16\qquad\textbf{(D)}~17\qquad\textbf{(E)}~18\qquad$\par \vspace{0.5em}\item Janet rolls a standard $6$-sided die $4$ times and keeps a running total of the numbers she rolls. What is the probability that at some point, her running total will equal $3?$

$\textbf{(A) }\frac{2}{9}\qquad\textbf{(B) }\frac{49}{216}\qquad\textbf{(C) }\frac{25}{108}\qquad\textbf{(D) }\frac{17}{72}\qquad\textbf{(E) }\frac{13}{54}$\par \vspace{0.5em}\item Points $A$ and $B$ lie on the graph of $y=\log_{2}x$. The midpoint of $\overline{AB}$ is $(6, 2)$. What is the positive difference between the $x$-coordinates of $A$ and $B$?

$\textbf{(A)}~2\sqrt{11}\qquad\textbf{(B)}~4\sqrt{3}\qquad\textbf{(C)}~8\qquad\textbf{(D)}~4\sqrt{5}\qquad\textbf{(E)}~9$\par \vspace{0.5em}\item A digital display shows the current date as an $8$-digit integer consisting of a $4$-digit year, followed by a $2$-digit month, followed by a $2$-digit date within the month. For example, Arbor Day this year is displayed as $20230428$. For how many dates in $2023$ will each digit appear an even number of times in the 8-digital display for that date?

$\textbf{(A)}~5\qquad\textbf{(B)}~6\qquad\textbf{(C)}~7\qquad\textbf{(D)}~8\qquad\textbf{(E)}~9$\par \vspace{0.5em}\item Maureen is keeping track of the mean of her quiz scores this semester. If Maureen scores an $11$ on the next quiz, her mean will increase by $1$. If she scores an $11$ on each of the next three quizzes, her mean will increase by $2$. What is the mean of her quiz scores currently?

$\textbf{(A) }4\qquad\textbf{(B) }5\qquad\textbf{(C) }6\qquad\textbf{(D) }7\qquad\textbf{(E) }8$\par \vspace{0.5em}\item A square of area $2$ is inscribed in a square of area $3$, creating four congruent triangles, as shown below. What is the ratio of the shorter leg to the longer leg in the shaded right triangle?

\begin{center}
\begin{asy}
import olympiad;
import cse5;
size(200);
defaultpen(linewidth(0.6pt)+fontsize(10pt));
real y = sqrt(3);
pair A,B,C,D,E,F,G,H;
A = (0,0);
B = (0,y);
C = (y,y);
D = (y,0);
E = ((y + 1)/2,y);
F = (y, (y - 1)/2);
G = ((y - 1)/2, 0);
H = (0,(y + 1)/2);
fill(H--B--E--cycle, gray);
draw(A--B--C--D--cycle);
draw(E--F--G--H--cycle);
\end{asy}
\end{center}


$\textbf{(A) }\frac15\qquad\textbf{(B) }\frac14\qquad\textbf{(C) }2-\sqrt3\qquad\textbf{(D) }\sqrt3-\sqrt2\qquad\textbf{(E) }\sqrt2-1$\par \vspace{0.5em}\item Positive real numbers $x$ and $y$ satisfy $y^3 = x^2$ and $(y-x)^2 = 4y^2$. What is $x+y$?

$\textbf{(A)}\ 12 \qquad \textbf{(B)}\ 18 \qquad \textbf{(C)}\ 24 \qquad \textbf{(D)}\ 36 \qquad \textbf{(E)}\ 42$\par \vspace{0.5em}\item What is the degree measure of the acute angle formed by lines with slopes $2$ and $\tfrac{1}{3}$?

$\textbf{(A)}~30\qquad\textbf{(B)}~37.5\qquad\textbf{(C)}~45\qquad\textbf{(D)}~52.5\qquad\textbf{(E)}~60$\par \vspace{0.5em}\item What is the value of

\begin\{equation*\}
2\^3 - 1\^3 + 4\^3 - 3\^3 + 6\^3 - 5\^3 + \dots + 18\^3 - 17\^3?
\end\{equation*\}


$\textbf{(A) } 2023 \qquad\textbf{(B) } 2679 \qquad\textbf{(C) } 2941 \qquad\textbf{(D) } 3159 \qquad\textbf{(E) } 3235$\par \vspace{0.5em}\item In a table tennis tournament every participant played every other participant exactly once. Although there were twice as many right-handed players as left-handed players, the number of games won by left-handed players was $40\%$ more than the number of games won by right-handed players. (There were no ties and no ambidextrous players.) What is the total number of games played?

$\textbf{(A) }15\qquad\textbf{(B) }36\qquad\textbf{(C) }45\qquad\textbf{(D) }48\qquad\textbf{(E) }66$\par \vspace{0.5em}\item How many complex numbers satisfy the equation $z^{5}=\overline{z}$, where $\overline{z}$ is the conjugate of the complex number $z$?

$\textbf{(A)}~2\qquad\textbf{(B)}~3\qquad\textbf{(C)}~5\qquad\textbf{(D)}~6\qquad\textbf{(E)}~7$\par \vspace{0.5em}\item Usain is walking for exercise by zigzagging across a $100$-meter by $30$-meter rectangular field, beginning at point $A$ and ending on the segment $\overline{BC}$. He wants to increase the distance walked by zigzagging as shown in the figure below $(APQRS)$. What angle $\theta$$\angle PAB=\angle QPC=\angle RQB=\cdots$ will produce in a length that is $120$ meters? (This figure is not drawn to scale. Do not assume that the zigzag path has exactly four segments as shown; there could be more or fewer.)


\begin{center}
\begin{asy}
import olympiad;
import cse5;
import olympiad;
draw((-50,15)--(50,15));
draw((50,15)--(50,-15));
draw((50,-15)--(-50,-15));
draw((-50,-15)--(-50,15));
draw((-50,-15)--(-22.5,15));
draw((-22.5,15)--(5,-15));
draw((5,-15)--(32.5,15));
draw((32.5,15)--(50,-4.090909090909));
label("$\theta$", (-41.5,-10.5));
label("$\theta$", (-13,10.5));
label("$\theta$", (15.5,-10.5));
label("$\theta$", (43,10.5));
dot((-50,15));
dot((-50,-15));
dot((50,15));
dot((50,-15));
dot((50,-4.09090909090909));
label("$D$",(-58,15));
label("$A$",(-58,-15));
label("$C$",(58,15));
label("$B$",(58,-15));
label("$S$",(58,-4.0909090909));
dot((-22.5,15));
dot((5,-15));
dot((32.5,15));
label("$P$",(-22.5,23));
label("$Q$",(5,-23));
label("$R$",(32.5,23));
\end{asy}
\end{center}


$\textbf{(A)}~\arccos\frac{5}{6}\qquad\textbf{(B)}~\arccos\frac{4}{5}\qquad\textbf{(C)}~\arccos\frac{3}{10}\qquad\textbf{(D)}~\arcsin\frac{4}{5}\qquad\textbf{(E)}~\arcsin\frac{5}{6}$\par \vspace{0.5em}\item Consider the set of complex numbers $z$ satisfying $|1+z+z^{2}|=4$. The maximum value of the imaginary part of $z$ can be written in the form $\tfrac{\sqrt{m}}{n}$, where $m$ and $n$ are relatively prime positive integers. What is $m+n$?

$\textbf{(A)}~20\qquad\textbf{(B)}~21\qquad\textbf{(C)}~22\qquad\textbf{(D)}~23\qquad\textbf{(E)}~24$\par \vspace{0.5em}\item Flora the frog starts at $0$ on the number line and makes a sequence of jumps to the right. In any one jump, independent of previous jumps, Flora leaps a positive integer distance $m$ with probability $\frac{1}{2^m}$. What is the probability that Flora will eventually land at $10$?

$\textbf{(A) } \frac{5}{512} \qquad \textbf{(B) } \frac{45}{1024} \qquad \textbf{(C) } \frac{127}{1024} \qquad \textbf{(D) } \frac{511}{1024} \qquad \textbf{(E) } \frac{1}{2}$\par \vspace{0.5em}\item Circle $C_1$ and $C_2$ each have radius $1$, and the distance between their centers is $\frac{1}{2}$. Circle $C_3$ is the largest circle internally tangent to both $C_1$ and $C_2$. Circle $C_4$ is internally tangent to both $C_1$ and $C_2$ and externally tangent to $C_3$. What is the radius of $C_4$?


\begin{center}
\begin{asy}
import olympiad;
import cse5;
import olympiad; 
size(10cm); 
draw(circle((0,0),0.75)); 
draw(circle((-0.25,0),1)); 
draw(circle((0.25,0),1)); 
draw(circle((0,6/7),3/28)); 
pair A = (0,0), B = (-0.25,0), C = (0.25,0), D = (0,6/7), E = (-0.95710678118, 0.70710678118), F = (0.95710678118, -0.70710678118);
dot(B\^\^C); 
draw(B--E, dashed);
draw(C--F, dashed);
draw(B--C); 
label("$C_4$", D); 
label("$C_1$", (-1.375, 0)); 
label("$C_2$", (1.375,0));
label("$\frac{1}{2}$", (0, -.125));
label("$C_3$", (-0.4, -0.4));
label("$1$", (-.85, 0.70));
label("$1$", (.85, -.7));
import olympiad; 
markscalefactor=0.005;
\end{asy}
\end{center}


$\textbf{(A) } \frac{1}{14} \qquad \textbf{(B) } \frac{1}{12} \qquad \textbf{(C) } \frac{1}{10} \qquad \textbf{(D) } \frac{3}{28} \qquad \textbf{(E) } \frac{1}{9}$\par \vspace{0.5em}\item What is the product of all the solutions to the equation 
\begin\{equation*\}
\log\_\{7x\}2023 \cdot \log\_\{289x\} 2023 = \log\_\{2023x\} 2023?
\end\{equation*\}


\$\textbf\{(A) \}(\log\_\{2023\}7 \cdot \log\_\{2023\}289)\^2 \qquad\textbf\{(B) \}\log\_\{2023\}7 \cdot \log\_\{2023\}289\qquad\textbf\{(C) \} 1
\\
\\
\textbf\{(D) \}\log\_\{7\}2023 \cdot \log\_\{289\}2023\qquad\textbf\{(E) \}(\log\_\{7\}2023 \cdot \log\_\{289\}2023)\^2\$\par \vspace{0.5em}\item Rows 1, 2, 3, 4, and 5 of a triangular array of integers are shown below:


\begin{center}
\begin{asy}
import olympiad;
import cse5;
size(4.5cm);
label("$1$", (0,0));
label("$1$", (-0.5,-2/3));
label("$1$", (0.5,-2/3));
label("$1$", (-1,-4/3));
label("$3$", (0,-4/3));
label("$1$", (1,-4/3));
label("$1$", (-1.5,-2));
label("$5$", (-0.5,-2));
label("$5$", (0.5,-2));
label("$1$", (1.5,-2));
label("$1$", (-2,-8/3));
label("$7$", (-1,-8/3));
label("$11$", (0,-8/3));
label("$7$", (1,-8/3));
label("$1$", (2,-8/3));
\end{asy}
\end{center}


Each row after the first row is formed by placing a 1 at each end of the row, and each interior entry is 1 greater than the sum of the two numbers diagonally above it in the previous row. What is the units digit of the sum of the 2023 numbers in the 2023rd row?

$\textbf{(A) }1\qquad\textbf{(B) }3\qquad\textbf{(C) }5\qquad\textbf{(D) }7\qquad\textbf{(E) }9$\par \vspace{0.5em}\item If $A$ and $B$ are vertices of a polyhedron, define the distance $d(A, B)$ to be the minimum number of edges of the polyhedron one must traverse in order to connect $A$ and $B$. For example, if $\overline{AB}$ is an edge of the polyhedron, then $d(A, B) = 1$, but if $\overline{AC}$ and $\overline{CB}$ are edges and $\overline{AB}$ is not an edge, then $d(A, B) = 2$. Let $Q$, $R$, and $S$ be randomly chosen distinct vertices of a regular icosahedron (regular polyhedron made up of 20 equilateral triangles). What is the probability that $d(Q, R) > d(R, S)$?

$\textbf{(A)}~\frac{7}{22}\qquad\textbf{(B)}~\frac13\qquad\textbf{(C)}~\frac38\qquad\textbf{(D)}~\frac5{12}\qquad\textbf{(E)}~\frac12$\par \vspace{0.5em}\item Let $f$ be the unique function defined on the positive integers such that
\begin\{equation*\}
\sum\_\{d\mid n\}d\cdot f\left(\frac\{n\}\{d\}\right)=1
\end\{equation*\}
for all positive integers $n$, where the sum is taken over all positive divisors of $n$. What is $f(2023)$?

$\textbf{(A)}~-1536\qquad\textbf{(B)}~96\qquad\textbf{(C)}~108\qquad\textbf{(D)}~116\qquad\textbf{(E)}~144$\par \vspace{0.5em}\item How many ordered pairs of positive real numbers $(a,b)$ satisfy the equation

\begin\{equation*\}
(1+2a)(2+2b)(2a+b) = 32ab?
\end\{equation*\}


$\textbf{(A) }0\qquad\textbf{(B) }1\qquad\textbf{(C) }2\qquad\textbf{(D) }3\qquad\textbf{(E) }\text{an infinite number}$\par \vspace{0.5em}\item Let $K$ be the number of sequences $A_1$, $A_2$, $\dots$, $A_n$ such that $n$ is a positive integer less than or equal to $10$, each $A_i$ is a subset of $\{1, 2, 3, \dots, 10\}$, and $A_{i-1}$ is a subset of $A_i$ for each $i$ between $2$ and $n$, inclusive. For example, $\{\}$, $\{5, 7\}$, $\{2, 5, 7\}$, $\{2, 5, 7\}$, $\{2, 5, 6, 7, 9\}$ is one such sequence, with $n = 5$.What is the remainder when $K$ is divided by $10$?

$\textbf{(A) } 1 \qquad \textbf{(B) } 3 \qquad \textbf{(C) } 5 \qquad \textbf{(D) } 7 \qquad \textbf{(E) } 9$\par \vspace{0.5em}\item There is a unique sequence of integers $a_1, a_2, \cdots a_{2023}$ such that

\begin\{equation*\}
\tan2023x = \frac\{a\_1 \tan x + a\_3 \tan\^3 x + a\_5 \tan\^5 x + \cdots + a\_\{2023\} \tan\^\{2023\} x\}\{1 + a\_2 \tan\^2 x + a\_4 \tan\^4 x \cdots + a\_\{2022\} \tan\^\{2022\} x\}
\end\{equation*\}
whenever $\tan 2023x$ is defined. What is $a_{2023}?$

$\textbf{(A) } -2023 \qquad\textbf{(B) } -2022 \qquad\textbf{(C) } -1 \qquad\textbf{(D) } 1 \qquad\textbf{(E) } 2023$\par \vspace{0.5em}\end{enumerate}\newpage\section*{2023 AMC1212B}\begin{enumerate}[label=\arabic*., itemsep=0.5em]\item Mrs. Jones is pouring orange juice into four identical glasses for her four sons. She fills the first three glasses completely but runs out of juice when the fourth glass is only $\frac{1}{3}$ full. What fraction of a glass must Mrs. Jones pour from each of the first three glasses into the fourth glass so that all four glasses will have the same amount of juice?

$\textbf{(A) }\frac{1}{12}\qquad\textbf{(B) }\frac{1}{4}\qquad\textbf{(C) }\frac{1}{6}\qquad\textbf{(D) }\frac{1}{8}\qquad\textbf{(E) }\frac{2}{9}$\par \vspace{0.5em}\item Carlos went to a sports store to buy running shoes. Running shoes were on sale, with prices reduced by $20\%$on every pair of shoes. Carlos also knew that he had to pay a $7.5\%$ sales tax on the discounted price. He had $43$ dollars. What is the original (before discount) price of the most expensive shoes he could afford to buy?

$\textbf{(A) }46\qquad\textbf{(B) }50\qquad\textbf{(C) }48\qquad\textbf{(D) }47\qquad\textbf{(E) }49$\par \vspace{0.5em}\item A $3-4-5$ right triangle is inscribed in circle $A$, and a $5-12-13$ right triangle is inscribed in circle $B$. What is the ratio of the area of circle $A$ to the area of circle $B$?

$\textbf{(A)}~\frac{9}{25}\qquad\textbf{(B)}~\frac{1}{9}\qquad\textbf{(C)}~\frac{1}{5}\qquad\textbf{(D)}~\frac{25}{169}\qquad\textbf{(E)}~\frac{4}{25}$\par \vspace{0.5em}\item Jackson's paintbrush makes a narrow strip with a width of $6.5$ millimeters. Jackson has enough paint to make a strip $25$ meters long. How many square centimeters of paper could Jackson cover with paint?

$\textbf{(A) }162,500\qquad\textbf{(B) }162.5\qquad\textbf{(C) }1,625\qquad\textbf{(D) }1,625,000\qquad\textbf{(E) }16,250$\par \vspace{0.5em}\item You are playing a game. A $2 \times 1$ rectangle covers two adjacent squares (oriented either horizontally or vertically) of a $3 \times 3$ grid of squares, but you are not told which two squares are covered. Your goal is to find at least one square that is covered by the rectangle. A "turn" consists of you guessing a square, after which you are told whether that square is covered by the hidden rectangle. What is the minimum number of turns you need to ensure that at least one of your guessed squares is covered by the rectangle?

$\textbf{(A)}~3\qquad\textbf{(B)}~5\qquad\textbf{(C)}~4\qquad\textbf{(D)}~8\qquad\textbf{(E)}~6$\par \vspace{0.5em}\item When the roots of the polynomial


\begin\{equation*\}
P(x)  = (x-1)\^1 (x-2)\^2 (x-3)\^3 \cdot \cdot \cdot (x-10)\^\{10\}
\end\{equation*\}


are removed from the number line, what remains is the union of 11 disjoint open intervals. On how many of these intervals is $P(x)$ positive?

$\textbf{(A)}~3\qquad\textbf{(B)}~7\qquad\textbf{(C)}~6\qquad\textbf{(D)}~4\qquad\textbf{(E)}~5$\par \vspace{0.5em}\item For how many integers $n$ does the expression
\begin\{equation*\}
\sqrt\{\frac\{\log (n\^2) - (\log n)\^2\}\{\log n - 3\}\}
\end\{equation*\}
represent a real number, where log denotes the base $10$ logarithm?

$\textbf{(A) }900 \qquad \textbf{(B) }3\qquad \textbf{(C) }902 \qquad \textbf{(D) } 2  \qquad \textbf{(E) }901$\par \vspace{0.5em}\item How many nonempty subsets $B$ of $\{0, 1, 2, 3, \dots, 12\}$ have the property that the number of elements in $B$ is equal to the least element of $B$? For example, $B = \{4, 6, 8, 11\}$ satisfies the condition.

$\textbf{(A)}\ 256 \qquad\textbf{(B)}\ 136 \qquad\textbf{(C)}\ 108 \qquad\textbf{(D)}\ 144 \qquad\textbf{(E)}\ 156$\par \vspace{0.5em}\item What is the area of the region in the coordinate plane defined by

$\left||x|-1\right|+\left||y|-1\right|\leq 1?$

$\textbf{(A)}~2\qquad\textbf{(B)}~8\qquad\textbf{(C)}~4\qquad\textbf{(D)}~15\qquad\textbf{(E)}~12$\par \vspace{0.5em}\item In the $xy$-plane, a circle of radius $4$ with center on the positive $x$-axis is tangent to the $y$-axis at the origin, and a circle with radius $10$ with center on the positive $y$-axis is tangent to the $x$-axis at the origin. What is the slope of the line passing through the two points at which these circles intersect?

$\textbf{(A)}\ \dfrac{2}{7} \qquad\textbf{(B)}\ \dfrac{3}{7}  \qquad\textbf{(C)}\ \dfrac{2}{\sqrt{29}}  \qquad\textbf{(D)}\ \dfrac{1}{\sqrt{29}}  \qquad\textbf{(E)}\ \dfrac{2}{5}$\par \vspace{0.5em}\item What is the maximum area of an isosceles trapezoid that has legs of length $1$ and one base twice as long as the other?

$\textbf{(A) }\frac 54 \qquad \textbf{(B) } \frac 87 \qquad \textbf{(C)} \frac{5\sqrt2}4 \qquad \textbf{(D) } \frac 32  \qquad \textbf{(E) } \frac{3\sqrt3}4$\par \vspace{0.5em}\item For complex numbers $u=a+bi$ and $v=c+di$, define the binary operation $\otimes$ by
\begin\{equation*\}
u\otimes v=ac+bdi.
\end\{equation*\}
Suppose $z$ is a complex number such that $z\otimes z=z^{2}+40$. What is $|z|$?

$\textbf{(A) }2\qquad\textbf{(B) }5\qquad\textbf{(C) }\sqrt{5}\qquad\textbf{(D) }\sqrt{10}\qquad\textbf{(E) }5\sqrt{2}$\par \vspace{0.5em}\item A rectangular box $P$ has distinct edge lengths $a$, $b$, and $c$. The sum of the lengths of all $12$ edges of $P$ is $13$, the sum of the areas of all $6$ faces of $P$ is $\frac{11}{2}$, and the volume of $P$ is $\frac{1}{2}$. What is the length of the longest interior diagonal connecting two vertices of $P$?

$\textbf{(A)}~2\qquad\textbf{(B)}~\frac{3}{8}\qquad\textbf{(C)}~\frac{9}{8}\qquad\textbf{(D)}~\frac{9}{4}\qquad\textbf{(E)}~\frac{3}{2}$\par \vspace{0.5em}\item For how many ordered pairs $(a,b)$ of integers does the polynomial $x^3+ax^2+bx+6$ have $3$ distinct integer roots?

$\textbf{(A)}\ 5 \qquad\textbf{(B)}\ 6 \qquad\textbf{(C)}\ 8 \qquad\textbf{(D)}\ 7 \qquad\textbf{(E)}\ 4$\par \vspace{0.5em}\item Suppose $a$, $b$, and $c$ are positive integers such that
\begin\{equation*\}
\frac\{a\}\{14\}+\frac\{b\}\{15\}=\frac\{c\}\{210\}.
\end\{equation*\}
Which of the following statements are necessarily true?

I. If $\gcd(a,14)=1$ or $\gcd(b,15)=1$ or both, then $\gcd(c,210)=1$.

II. If $\gcd(c,210)=1$, then $\gcd(a,14)=1$ or $\gcd(b,15)=1$ or both.

III. $\gcd(c,210)=1$ if and only if $\gcd(a,14)=\gcd(b,15)=1$.

$\textbf{(A)}~\text{I, II, and III}\qquad\textbf{(B)}~\text{I only}\qquad\textbf{(C)}~\text{I and II only}\qquad\textbf{(D)}~\text{III only}\qquad\textbf{(E)}~\text{II and III only}$\par \vspace{0.5em}\item In Coinland, there are three types of coins, each worth $6, 10,$ and $15.$ What is the sum of the digits of the maximum amount of money that is impossible to have?

$\textbf{(A) }8\qquad\textbf{(B) }10\qquad\textbf{(C) }7\qquad\textbf{(D) }11\qquad\textbf{(E) }9$\par \vspace{0.5em}\item Triangle $ABC$ has side lengths in arithmetic progression, and the smallest side has length $6$. If the triangle has an angle of $120^\circ,$ what is the area of $ABC$?

$\textbf{(A) }12\sqrt{3}\qquad\textbf{(B) }8\sqrt{6}\qquad\textbf{(C) }14\sqrt{2}\qquad\textbf{(D) }20\sqrt{2}\qquad\textbf{(E) }15\sqrt{3}$\par \vspace{0.5em}\item Last academic year Yolanda and Zelda took different courses that did not necessarily administer the same number of quizzes during each of the two semesters. Yolanda's average on all the quizzes she took during the first semester was $3$ points higher than Zelda's average on all the quizzes she took during the first semester. Yolanda's average on all the quizzes she took during the second semester was $18$ points higher than her average for the first semester and was again $3$ points higher than Zelda's average on all the quizzes Zelda took during her second semester. Which one of the following statements cannot possibly be true?

$\textbf{(A)}$ Yolanda's quiz average for the academic year was $22$ points higher than Zelda's.

$\textbf{(B)}$ Zelda's quiz average for the academic year was higher than Yolanda's.

$\textbf{(C)}$ Yolanda's quiz average for the academic year was $3$ points higher than Zelda's.

$\textbf{(D)}$ Zelda's quiz average for the academic year equaled Yolanda's.

$\textbf{(E)}$ If Zelda had scored $3$ points higher on each quiz she took, then she would have had the same average for the academic year as Yolanda.\par \vspace{0.5em}\item Each of $2023$ balls is placed in one of $3$ bins. Which of the following is closest to the probability that each of the bins will contain an odd number of balls?

$\textbf{(A) } \frac{2}{3} \qquad \textbf{(B) } \frac{3}{10} \qquad \textbf{(C) } \frac{1}{2} \qquad \textbf{(D) } \frac{1}{3} \qquad \textbf{(E) } \frac{1}{4}$\par \vspace{0.5em}\item Cyrus the frog jumps $2$ units in a direction, then $2$ more in another direction. What is the probability that he lands less than $1$ unit away from his starting position?

$\textbf{(A)}~\frac{1}{6}\qquad\textbf{(B)}~\frac{1}{5}\qquad\textbf{(C)}~\frac{\sqrt{3}}{8}\qquad\textbf{(D)}~\frac{\arctan \frac{1}{2}}{\pi}\qquad\textbf{(E)}~\frac{2\arcsin \frac{1}{4}}{\pi}$\par \vspace{0.5em}\item A lampshade is made in the form of the lateral surface of the frustum of a right circular cone. The height of the frustum is $3\sqrt3$ inches, its top diameter is $6$ inches, and its bottom diameter is $12$ inches. A bug is at the bottom of the lampshade and there is a glob of honey on the top edge of the lampshade at the spot farthest from the bug. The bug wants to crawl to the honey, but it must stay on the surface of the lampshade. What is the length in inches of its shortest path to the honey?

$\textbf{(A) } 6 + 3\pi\qquad \textbf{(B) }6 + 6\pi\qquad \textbf{(C) } 6\sqrt3 \qquad \textbf{(D) } 6\sqrt5 \qquad \textbf{(E) } 6\sqrt3 + \pi$\par \vspace{0.5em}\item A real-valued function $f$ has the property that for all real numbers $a$ and $b,$
\begin\{equation*\}
f(a + b)  + f(a - b) = 2f(a) f(b).
\end\{equation*\}
Which one of the following cannot be the value of $f(1)?$

$\textbf{(A) } 0 \qquad \textbf{(B) } 1 \qquad \textbf{(C) } -1 \qquad \textbf{(D) } 2 \qquad \textbf{(E) } -2$\par \vspace{0.5em}\item When $n$ standard six-sided dice are rolled, the product of the numbers rolled can be any of $936$ possible values. What is $n$?

$\textbf{(A)}~11\qquad\textbf{(B)}~6\qquad\textbf{(C)}~8\qquad\textbf{(D)}~10\qquad\textbf{(E)}~9$\par \vspace{0.5em}\item Suppose that $a$, $b$, $c$ and $d$ are positive integers satisfying all of the following relations.


\begin\{equation*\}
abcd=2\^6\cdot 3\^9\cdot 5\^7
\end\{equation*\}


\begin\{equation*\}
\text\{lcm\}(a,b)=2\^3\cdot 3\^2\cdot 5\^3
\end\{equation*\}


\begin\{equation*\}
\text\{lcm\}(a,c)=2\^3\cdot 3\^3\cdot 5\^3
\end\{equation*\}


\begin\{equation*\}
\text\{lcm\}(a,d)=2\^3\cdot 3\^3\cdot 5\^3
\end\{equation*\}


\begin\{equation*\}
\text\{lcm\}(b,c)=2\^1\cdot 3\^3\cdot 5\^2
\end\{equation*\}


\begin\{equation*\}
\text\{lcm\}(b,d)=2\^2\cdot 3\^3\cdot 5\^2
\end\{equation*\}


\begin\{equation*\}
\text\{lcm\}(c,d)=2\^2\cdot 3\^3\cdot 5\^2
\end\{equation*\}


What is $\text{gcd}(a,b,c,d)$?

$\textbf{(A)}~30\qquad\textbf{(B)}~45\qquad\textbf{(C)}~3\qquad\textbf{(D)}~15\qquad\textbf{(E)}~6$\par \vspace{0.5em}\item A regular pentagon with area $\sqrt{5}+1$ is printed on paper and cut out. The five vertices of the pentagon are folded into the center of the pentagon, creating a smaller pentagon. What is the area of the new pentagon?

$\textbf{(A)}~4-\sqrt{5}\qquad\textbf{(B)}~\sqrt{5}-1\qquad\textbf{(C)}~8-3\sqrt{5}\qquad\textbf{(D)}~\frac{\sqrt{5}+1}{2}\qquad\textbf{(E)}~\frac{2+\sqrt{5}}{3}$\par \vspace{0.5em}\end{enumerate}\newpage\section*{2024 AMC1212A}\begin{enumerate}[label=\arabic*., itemsep=0.5em]\item What is the value of $9901\cdot101-99\cdot10101?$

$\textbf{(A)}~2\qquad\textbf{(B)}~20\qquad\textbf{(C)}~200\qquad\textbf{(D)}~202\qquad\textbf{(E)}~2020$\par \vspace{0.5em}\item A model used to estimate the time it will take to hike to the top of the mountain on a trail is of the form $T=aL+bG,$ where $a$ and $b$ are constants, $T$ is the time in minutes, $L$ is the length of the trail in miles, and $G$ is the altitude gain in feet. The model estimates that it will take $69$ minutes to hike to the top if a trail is $1.5$ miles long and ascends $800$ feet, as well as if a trail is $1.2$ miles long and ascends $1100$ feet. How many minutes does the model estimates it will take to hike to the top if the trail is $4.2$ miles long and ascends $4000$ feet?

$\textbf{(A) }240\qquad\textbf{(B) }246\qquad\textbf{(C) }252\qquad\textbf{(D) }258\qquad\textbf{(E) }264$\par \vspace{0.5em}\item The number $2024$ is written as the sum of not necessarily distinct two-digit numbers. What is the least number of two-digit numbers needed to write this sum?

$\textbf{(A) }20\qquad\textbf{(B) }21\qquad\textbf{(C) }22\qquad\textbf{(D) }23\qquad\textbf{(E) }24$\par \vspace{0.5em}\item What is the least value of $n$ such that $n!$ is a multiple of $2024$?

\$
\textbf\{(A) \}11 \qquad
\textbf\{(B) \}21 \qquad
\textbf\{(C) \}22 \qquad
\textbf\{(D) \}23 \qquad
\textbf\{(E) \}253 \qquad
\$\par \vspace{0.5em}\item A data set containing $20$ numbers, some of which are $6$, has mean $45$. When all the 6s are removed, the data set has mean $66$. How many 6s were in the original data set?

$\textbf{(A) }4\qquad\textbf{(B) }5\qquad\textbf{(C) }6\qquad\textbf{(D) }7\qquad\textbf{(E) }8$\par \vspace{0.5em}\item The product of three integers is $60$. What is the least possible positive sum of the three integers?

$\textbf{(A) } 2 \qquad \textbf{(B) } 3 \qquad \textbf{(C) } 5 \qquad \textbf{(D) } 6 \qquad \textbf{(E) } 13$\par \vspace{0.5em}\item In $\Delta ABC$, $\angle ABC = 90^\circ$ and $BA = BC = \sqrt{2}$. Points $P_1, P_2, \dots, P_{2024}$ lie on hypotenuse $\overline{AC}$ so that $AP_1= P_1P_2 = P_2P_3 = \dots = P_{2023}P_{2024} = P_{2024}C$. What is the length of the vector sum

\begin\{equation*\}
\overrightarrow\{BP\_1\} + \overrightarrow\{BP\_2\} + \overrightarrow\{BP\_3\} + \dots + \overrightarrow\{BP\_\{2024\}\}?
\end\{equation*\}

\$
\textbf\{(A) \}1011 \qquad
\textbf\{(B) \}1012 \qquad
\textbf\{(C) \}2023 \qquad
\textbf\{(D) \}2024 \qquad
\textbf\{(E) \}2025 \qquad
\$\par \vspace{0.5em}\item How many angles $\theta$ with $0\le\theta\le2\pi$ satisfy $\log(\sin(3\theta))+\log(\cos(2\theta))=0$?  

$ \textbf{(A) }0 \qquad \textbf{(B) }1 \qquad \textbf{(C) }2 \qquad \textbf{(D) }3 \qquad \textbf{(E) }4 \qquad $\par \vspace{0.5em}\item Let $M$ be the greatest integer such that both $M + 1213$ and $M + 3773$ are perfect squares. What is the units digit of $M$?

\$
\textbf\{(A) \}1 \qquad
\textbf\{(B) \}2 \qquad
\textbf\{(C) \}3 \qquad
\textbf\{(D) \}6 \qquad
\textbf\{(E) \}8 \qquad
\$\par \vspace{0.5em}\item Let $\alpha$ be the radian measure of the smallest angle in a $3{-}4{-}5$ right triangle. Let $\beta$ be the radian measure of the smallest angle in a $7{-}24{-}25$ right triangle. In terms of $\alpha$, what is $\beta$?

\$
\textbf\{(A) \}\frac\{\alpha\}\{3\}\qquad
\textbf\{(B) \}\alpha - \frac\{\pi\}\{8\}\qquad
\textbf\{(C) \}\frac\{\pi\}\{2\} - 2\alpha \qquad
\textbf\{(D) \}\frac\{\alpha\}\{2\}\qquad
\textbf\{(E) \}\pi - 4\alpha\qquad
\$\par \vspace{0.5em}\item There are exactly $K$ positive integers $b$ with $5 \leq b \leq 2024$ such that the base-$b$ integer $2024_b$ is divisible by $16$ (where $16$ is in base ten). What is the sum of the digits of $K$?

$\textbf{(A) }16\qquad\textbf{(B) }17\qquad\textbf{(C) }18\qquad\textbf{(D) }20\qquad\textbf{(E) }21$\par \vspace{0.5em}\item The first three terms of a geometric sequence are the integers $a,\,720,$ and $b,$ where $a<720<b.$ What is the sum of the digits of the least possible value of $b?$

$\textbf{(A) } 9 \qquad \textbf{(B) } 12 \qquad \textbf{(C) } 16 \qquad \textbf{(D) } 18 \qquad \textbf{(E) } 21$\par \vspace{0.5em}\item The graph of $y=e^{x+1}+e^{-x}-2$ has an axis of symmetry. What is the reflection of the point $(-1,\tfrac{1}{2})$ over this axis?

$\textbf{(A) }\left(-1,-\frac{3}{2}\right)\qquad\textbf{(B) }(-1,0)\qquad\textbf{(C) }\left(-1,\tfrac{1}{2}\right)\qquad\textbf{(D) }\left(0,\frac{1}{2}\right)\qquad\textbf{(E) }\left(3,\frac{1}{2}\right)$\par \vspace{0.5em}\item The numbers, in order, of each row and the numbers, in order, of each column of a $5 \times 5$ array of integers form an arithmetic progression of length $5{.}$ The numbers in positions $(5, 5), \,(2,4),\,(4,3),$ and $(3, 1)$ are $0, 48, 16,$ and $12{,}$ respectively. What number is in position $(1, 2)?$

\begin\{equation*\}
\begin\{bmatrix\} . \& ? \&.\&.\&. \\ .\&.\&.\&48\&.\\ 12\&.\&.\&.\&.\\ .\&.\&16\&.\&.\\ .\&.\&.\&.\&0\end\{bmatrix\}
\end\{equation*\}

$\textbf{(A) } 19 \qquad \textbf{(B) } 24 \qquad \textbf{(C) } 29 \qquad \textbf{(D) } 34 \qquad \textbf{(E) } 39$\par \vspace{0.5em}\item The roots of $x^3 + 2x^2 - x + 3$ are $p, q,$ and $r.$ What is the value of 
\begin\{equation*\}
(p\^2 + 4)(q\^2 + 4)(r\^2 + 4)?
\end\{equation*\}

$\textbf{(A) } 64 \qquad \textbf{(B) } 75 \qquad \textbf{(C) } 100 \qquad \textbf{(D) } 125 \qquad \textbf{(E) } 144$\par \vspace{0.5em}\item A set of $12$ tokens ---- $3$ red, $2$ white, $1$ blue, and $6$ black ---- is to be distributed at random to $3$ game players, $4$ tokens per player. The probability that some player gets all the red tokens, another gets all the white tokens, and the remaining player gets the blue token can be written as $\frac{m}{n}$, where $m$ and $n$ are relatively prime positive integers. What is $m+n$?

\$
\textbf\{(A) \}387 \qquad
\textbf\{(B) \}388 \qquad
\textbf\{(C) \}389 \qquad
\textbf\{(D) \}390 \qquad
\textbf\{(E) \}391 \qquad
\$\par \vspace{0.5em}\item Integers $a$, $b$, and $c$ satisfy $ab + c = 100$, $bc + a = 87$, and $ca + b = 60$. What is $ab + bc + ca$?

\$
\textbf\{(A) \}212 \qquad
\textbf\{(B) \}247 \qquad
\textbf\{(C) \}258 \qquad
\textbf\{(D) \}276 \qquad
\textbf\{(E) \}284 \qquad
\$\par \vspace{0.5em}\item On top of a rectangular card with sides of length $1$ and $2+\sqrt{3}$, an identical card is placed so that two of their diagonals line up, as shown ($\overline{AC}$, in this case).


\begin{center}
\begin{asy}
import olympiad;
import cse5;
defaultpen(fontsize(12)+0.85); size(150);
real h=2.25;
pair C=origin,B=(0,h),A=(1,h),D=(1,0),Dp=reflect(A,C)*D,Bp=reflect(A,C)*B;
pair L=extension(A,Dp,B,C),R=extension(Bp,C,A,D);
draw(L--B--A--Dp--C--Bp--A);
draw(C--D--R);
draw(L--C\^\^R--A,dashed+0.6);
draw(A--C,black+0.6);
dot("$C$",C,2*dir(C-R)); dot("$A$",A,1.5*dir(A-L)); dot("$B$",B,dir(B-R));
\end{asy}
\end{center}


Continue the process, adding a third card to the second, and so on, lining up successive diagonals after rotating clockwise. In total, how many cards must be used until a vertex of a new card lands exactly on the vertex labeled $B$ in the figure?

$\textbf{(A) }6\qquad\textbf{(B) }8\qquad\textbf{(C) }10\qquad\textbf{(D) }12\qquad\textbf{(E) }\text{No new vertex will land on }B.$\par \vspace{0.5em}\item Cyclic quadrilateral $ABCD$ has lengths $BC=CD=3$ and $DA=5$ with $\angle CDA=120^\circ$. What is the length of the shorter diagonal of $ABCD$?

\$
\textbf\{(A) \}\frac\{31\}7 \qquad
\textbf\{(B) \}\frac\{33\}7 \qquad
\textbf\{(C) \}5 \qquad
\textbf\{(D) \}\frac\{39\}7 \qquad
\textbf\{(E) \}\frac\{41\}7 \qquad
\$\par \vspace{0.5em}\item Points $P$ and $Q$ are chosen uniformly and independently at random on sides $\overline {AB}$ and $\overline{AC},$ respectively, of equilateral triangle $\Delta ABC.$ Which of the following intervals contains the probability that the area of $\triangle APQ$ is less than half the area of $\triangle ABC?$

$\textbf{(A) } \left[\frac 38, \frac 12\right] \qquad \textbf{(B) } \left(\frac 12, \frac 23\right] \qquad \textbf{(C) } \left(\frac 23, \frac 34\right] \qquad \textbf{(D) } \left(\frac 34, \frac 78\right] \qquad \textbf{(E) } \left(\frac 78, 1\right]$\par \vspace{0.5em}\item Suppose that $a_1 = 2$ and the sequence $(a_n)$ satisfies the recurrence relation 
\begin\{equation*\}
\frac\{a\_n -1\}\{n-1\}=\frac\{a\_\{n-1\}+1\}\{(n-1)+1\}
\end\{equation*\}
for all $n \ge 2.$ What is the greatest integer less than or equal to 
\begin\{equation*\}
\sum\^\{100\}\_\{n=1\} a\_n\^2?
\end\{equation*\}

$\textbf{(A) } 338{,}550 \qquad \textbf{(B) } 338{,}551 \qquad \textbf{(C) } 338{,}552 \qquad \textbf{(D) } 338{,}553 \qquad \textbf{(E) } 338{,}554$\par \vspace{0.5em}\item The figure below shows a dotted grid $8$ cells wide and $3$ cells tall consisting of $1''\times1''$ squares. Carl places $1$-inch toothpicks along some of the sides of the squares to create a closed loop that does not intersect itself. The numbers in the cells indicate the number of sides of that square that are to be covered by toothpicks, and any number of toothpicks are allowed if no number is written. In how many ways can Carl place the toothpicks?


\begin{center}
\begin{asy}
import olympiad;
import cse5;
size(6cm);
for (int i=0; i<9; ++i) \{
  draw((i,0)--(i,3),dotted);
\}
for (int i=0; i<4; ++i)\{
  draw((0,i)--(8,i),dotted);
\}
for (int i=0; i<8; ++i) \{
  for (int j=0; j<3; ++j) \{
    if (j==1) \{
      label("1",(i+0.5,1.5));
\}\}\}
\end{asy}
\end{center}


$\textbf{(A) }130\qquad\textbf{(B) }144\qquad\textbf{(C) }146\qquad\textbf{(D) }162\qquad\textbf{(E) }196$\par \vspace{0.5em}\item What is the value of 


\begin\{equation*\}
\tan\^2 \frac \{\pi\}\{16\} \cdot \tan\^2 \frac \{3\pi\}\{16\}\~ + \~ \tan\^2 \frac \{\pi\}\{16\} \cdot \tan\^2 \frac \{5\pi\}\{16\} \~+\~\tan\^2 \frac \{3\pi\}\{16\} \cdot \tan\^2 \frac \{7\pi\}\{16\} \~+\~ \tan\^2 \frac \{5\pi\}\{16\} \cdot \tan\^2 \frac \{7\pi\}\{16\}?
\end\{equation*\}


$\textbf{(A) } 28 \qquad \textbf{(B) } 68 \qquad \textbf{(C) } 70 \qquad \textbf{(D) } 72 \qquad \textbf{(E) } 84$\par \vspace{0.5em}\item A $\textit{disphenoid}$ is a tetrahedron whose triangular faces are congruent to one another. What is the least total surface area of a disphenoid whose faces are scalene triangles with integer side lengths?

$\textbf{(A) }\sqrt{3}\qquad\textbf{(B) }3\sqrt{15}\qquad\textbf{(C) }15\qquad\textbf{(D) }15\sqrt{7}\qquad\textbf{(E) }24\sqrt{6}$\par \vspace{0.5em}\item A graph is $\textit{symmetric}$ about a line if the graph remains unchanged after reflection in that line. For how many quadruples of integers $(a,b,c,d)$, where $|a|,|b|,|c|,|d|\le5$ and $c$ and $d$ are not both $0$, is the graph of 
\begin\{equation*\}
y=\frac\{ax+b\}\{cx+d\}
\end\{equation*\}
symmetric about the line $y=x$?

$\textbf{(A) }1282\qquad\textbf{(B) }1292\qquad\textbf{(C) }1310\qquad\textbf{(D) }1320\qquad\textbf{(E) }1330$\par \vspace{0.5em}\end{enumerate}\newpage\section*{2024 AMC1212B}\begin{enumerate}[label=\arabic*., itemsep=0.5em]\item In a long line of people arranged left to right, the 1013th person from the left is also the 1010th person from the right. How many people are in the line?

$\textbf{(A) } 2021 \qquad\textbf{(B) } 2022 \qquad\textbf{(C) } 2023 \qquad\textbf{(D) } 2024 \qquad\textbf{(E) } 2025$\par \vspace{0.5em}\item What is $10! - 7! \cdot 6!$?

$\textbf{(A) }-120 \qquad\textbf{(B) }0 \qquad\textbf{(C) }120 \qquad\textbf{(D) }600 \qquad\textbf{(E) }720 \qquad$\par \vspace{0.5em}\item For how many integer values of $x$ is $|2x|\leq 7\pi?$

$\textbf{(A) }16 \qquad\textbf{(B) }17\qquad\textbf{(C) }19\qquad\textbf{(D) }20\qquad\textbf{(E) }21$\par \vspace{0.5em}\item Balls numbered $1,2,3,\ldots$ are deposited in $5$ bins, labeled $A,B,C,D,$ and $E$, using the following procedure. Ball $1$ is deposited in bin $A$, and balls $2$ and $3$ are deposited in $B$. The next three balls are deposited in bin $C$, the next $4$ in bin $D$, and so on, cycling back to bin $A$ after balls are deposited in bin $E$. (For example, $22,23,\ldots,28$ are deposited in bin $B$ at step 7 of this process.) In which bin is ball $2024$ deposited?

$\textbf{(A) }A\qquad\textbf{(B) }B\qquad\textbf{(C) }C\qquad\textbf{(D) }D\qquad\textbf{(E) }E$\par \vspace{0.5em}\item In the following expression, Melanie changed some of the plus signs to minus signs:
\begin\{equation*\}
1 + 3+5+7+\cdots+97+99
\end\{equation*\}
When the new expression was evaluated, it was negative. What is the least number of plus signs that Melanie could have changed to minus signs?

\$
\textbf\{(A) \}14 \qquad
\textbf\{(B) \}15 \qquad
\textbf\{(C) \}16 \qquad
\textbf\{(D) \}17 \qquad
\textbf\{(E) \}18 \qquad
\$\par \vspace{0.5em}\item The national debt of the United States is on track to reach $5 \cdot 10^{13}$ dollars by $2033$. How many digits does this number of dollars have when written as a numeral in base $5$? (The approximation of $\log_{10} 5$ as $0.7$ is sufficient for this problem.)

\$
\textbf\{(A) \}18 \qquad
\textbf\{(B) \}20 \qquad
\textbf\{(C) \}22 \qquad
\textbf\{(D) \}24 \qquad
\textbf\{(E) \}26 \qquad
\$\par \vspace{0.5em}\item In the figure below $WXYZ$ is a rectangle with $WX=4$ and $WZ=8$. Point $M$ lies $\overline{XY}$, point $A$ lies on $\overline{YZ}$, and $\angle WMA$ is a right angle. The areas of $\triangle WXM$ and $\triangle WAZ$ are equal. What is the area of $\triangle WMA$?


\begin{center}
\begin{asy}
import olympiad;
import cse5;
pair X = (0, 0);
pair W = (0, 4);
pair Y = (8, 0);
pair Z = (8, 4);
label("$X$", X, dir(180));
label("$W$", W, dir(180));
label("$Y$", Y, dir(0));
label("$Z$", Z, dir(0));

draw(W--X--Y--Z--cycle);
dot(X);
dot(Y);
dot(W);
dot(Z);
pair M = (2, 0);
pair A = (8, 3);
label("$A$", A, dir(0));
dot(M);
dot(A);
draw(W--M--A--cycle);
markscalefactor = 0.05;
draw(rightanglemark(W, M, A));
label("$M$", M, dir(-90));
\end{asy}
\end{center}


\$
\textbf\{(A) \}13 \qquad
\textbf\{(B) \}14 \qquad
\textbf\{(C) \}15 \qquad
\textbf\{(D) \}16 \qquad
\textbf\{(E) \}17 \qquad
\$\par \vspace{0.5em}\item What value of $x$ satisfies
\begin\{equation*\}
\frac\{\log\_2x\cdot\log\_3x\}\{\log\_2x+\log\_3x\}=2?
\end\{equation*\}

\$
\textbf\{(A) \}25\qquad
\textbf\{(B) \}32\qquad
\textbf\{(C) \}36\qquad
\textbf\{(D) \}42\qquad
\textbf\{(E) \}48\qquad
\$\par \vspace{0.5em}\item A dartboard is the region $B$ in the coordinate plane consisting of points $(x,y)$ such that $|x| + |y| \le 8$ . A target $T$ is the region where $(x^2 + y^2 - 25)^2 \le 49.$ A dart is thrown and lands at a random point in $B$. The probability that the dart lands in $T$ can be expressed as $\frac{m}{n} \cdot \pi,$ where $m$ and $n$ are relatively prime positive integers. What is $m + n?$

\$
\textbf\{(A) \}39 \qquad
\textbf\{(B) \}71 \qquad
\textbf\{(C) \}73 \qquad
\textbf\{(D) \}75 \qquad
\textbf\{(E) \}135 \qquad
\$\par \vspace{0.5em}\item A list of 9 real numbers consists of $1$, $2.2 $, $3.2 $, $5.2 $, $6.2 $, and $7$, as well as $x, y,z$ with $x\leq y\leq z$. The range of the list is $7$, and the mean and median are both positive integers. How many ordered triples $(x,y,z)$ are possible?

$\textbf{(A) }1 \qquad\textbf{(B) }2 \qquad\textbf{(C) }3 \qquad\textbf{(D) }4 \qquad\textbf{(E) }\text{infinitely many}\qquad$\par \vspace{0.5em}\item Let $x_{n} = \sin^2(n^\circ)$. What is the mean of $x_{1}, x_{2}, x_{3}, \cdots, x_{90}$?

\$
\textbf\{(A) \}\frac\{11\}\{45\} \qquad
\textbf\{(B) \}\frac\{22\}\{45\} \qquad
\textbf\{(C) \}\frac\{89\}\{180\} \qquad
\textbf\{(D) \}\frac\{1\}\{2\} \qquad
\textbf\{(E) \}\frac\{91\}\{180\} \qquad
\$\par \vspace{0.5em}\item Suppose $z$ is a complex number with positive imaginary part, with real part greater than $1$, and with $|z| = 2$. In the complex plane, the four points $0$, $z$, $z^{2}$, and $z^{3}$ are the vertices of a quadrilateral with area $15$. What is the imaginary part of $z$?

$\textbf{(A)}~\frac{3}{4}\qquad\textbf{(B)}~1\qquad\textbf{(C)}~\frac{4}{3}\qquad\textbf{(D)}~\frac{3}{2}\qquad\textbf{(E)}~\frac{5}{3}$\par \vspace{0.5em}\item There are real numbers $x,y,h$ and $k$ that satisfy the system of equations
\begin\{equation*\}
x\^2 + y\^2 - 6x - 8y = h
\end\{equation*\}

\begin\{equation*\}
x\^2 + y\^2 - 10x + 4y = k
\end\{equation*\}
What is the minimum possible value of $h+k$?

\$
\textbf\{(A) \}-54 \qquad
\textbf\{(B) \}-46 \qquad
\textbf\{(C) \}-34 \qquad
\textbf\{(D) \}-16 \qquad
\textbf\{(E) \}16 \qquad
\$\par \vspace{0.5em}\item How many different remainders can result when the $100$th power of an integer is divided by $125$?

$\textbf{(A) }1 \qquad\textbf{(B) }2 \qquad\textbf{(C) }5 \qquad\textbf{(D) }25 \qquad\textbf{(E) }125 \qquad$\par \vspace{0.5em}\item A triangle in the coordinate plane has vertices $A(\log_21,\log_22)$, $B(\log_23,\log_24)$, and $C(\log_27,\log_28)$. What is the area of $\triangle ABC$?

\$
\textbf\{(A) \}\log\_2\frac\{\sqrt3\}7\qquad
\textbf\{(B) \}\log\_2\frac3\{\sqrt7\}\qquad
\textbf\{(C) \}\log\_2\frac7\{\sqrt3\}\qquad
\textbf\{(D) \}\log\_2\frac\{11\}\{\sqrt7\}\qquad
\textbf\{(E) \}\log\_2\frac\{11\}\{\sqrt3\}\qquad
\$\par \vspace{0.5em}\item A group of $16$ people will be partitioned into $4$ indistinguishable $4$-person committees. Each committee will have one chairperson and one secretary. The number of different ways to make these assignments can be written as $3^{r}M$, where $r$ and $M$ are positive integers and $M$ is not divisible by $3$. What is $r$?

\$
\textbf\{(A) \}5 \qquad
\textbf\{(B) \}6 \qquad
\textbf\{(C) \}7 \qquad
\textbf\{(D) \}8 \qquad
\textbf\{(E) \}9 \qquad\$\par \vspace{0.5em}\item Integers $a$ and $b$ are randomly chosen without replacement from the set of integers with absolute value not exceeding $10$. What is the probability that the polynomial $x^3 + ax^2 + bx + 6$ has $3$ distinct integer roots?

$\textbf{(A) }\frac{1}{240} \qquad \textbf{(B) }\frac{1}{221} \qquad \textbf{(C) }\frac{1}{105} \qquad \textbf{(D) }\frac{1}{84} \qquad \textbf{(E) }\frac{1}{63}$\par \vspace{0.5em}\item The Fibonacci numbers are defined by $F_1=1,$ $F_2=1,$ and $F_n=F_{n-1}+F_{n-2}$ for $n\geq 3.$ What is
\begin\{equation*\}
\dfrac\{F\_2\}\{F\_1\}+\dfrac\{F\_4\}\{F\_2\}+\dfrac\{F\_6\}\{F\_3\}+\cdots+\dfrac\{F\_\{20\}\}\{F\_\{10\}\}?
\end\{equation*\}

$\textbf{(A) }318 \qquad\textbf{(B) }319\qquad\textbf{(C) }320\qquad\textbf{(D) }321\qquad\textbf{(E) }322$\par \vspace{0.5em}\item Equilateral $\triangle ABC$ with side length $14$ is rotated about its center by angle $\theta$, where $0 < \theta < 60^{\circ}$, to form $\triangle DEF$. See the figure. The area of hexagon $ADBECF$ is $91\sqrt{3}$. What is $\tan\theta$?

\begin{center}
\begin{asy}
import olympiad;
import cse5;
// Credit to shihan for this diagram.

defaultpen(fontsize(13)); size(200);
pair O=(0,0),A=dir(225),B=dir(-15),C=dir(105),D=rotate(38.21,O)*A,E=rotate(38.21,O)*B,F=rotate(38.21,O)*C;
draw(A--B--C--A,gray+0.4);draw(D--E--F--D,gray+0.4); draw(A--D--B--E--C--F--A,black+0.9); dot(O); dot("$A$",A,dir(A)); dot("$B$",B,dir(B)); dot("$C$",C,dir(C)); dot("$D$",D,dir(D)); dot("$E$",E,dir(E)); dot("$F$",F,dir(F));
\end{asy}
\end{center}


$\textbf{(A)}~\frac{3}{4}\qquad\textbf{(B)}~\frac{5\sqrt{3}}{11}\qquad\textbf{(C)}~\frac{4}{5}\qquad\textbf{(D)}~\frac{11}{13}\qquad\textbf{(E)}~\frac{7\sqrt{3}}{13}$\par \vspace{0.5em}\item Suppose $A$, $B$, and $C$ are points in the plane with $AB=40$ and $AC=42$, and let $x$ be the length of the line segment from $A$ to the midpoint of $\overline{BC}$. Define a function $f$ by letting $f(x)$ be the area of $\triangle ABC$. Then the domain of $f$ is an open interval $(p,q)$, and the maximum value $r$ of $f(x)$ occurs at $x=s$. What is $p+q+r+s$?

\$
\textbf\{(A) \}909\qquad
\textbf\{(B) \}910\qquad
\textbf\{(C) \}911\qquad
\textbf\{(D) \}912\qquad
\textbf\{(E) \}913\qquad
\$\par \vspace{0.5em}\item The measures of the smallest angles of three different right triangles sum to $90^\circ$. All three triangles have side lengths that are primitive Pythagorean triples. Two of them are $3-4-5$ and $5-12-13$. What is the perimeter of the third triangle?

\$
\textbf\{(A) \}40 \qquad
\textbf\{(B) \}126 \qquad
\textbf\{(C) \}154 \qquad
\textbf\{(D) \}176 \qquad
\textbf\{(E) \}208 \qquad
\$\par \vspace{0.5em}\item Let $\triangle{ABC}$ be a triangle with integer side lengths and the property that $\angle{B} = 2\angle{A}$. What is the least possible perimeter of such a triangle?

\$
\textbf\{(A) \}13 \qquad
\textbf\{(B) \}14 \qquad
\textbf\{(C) \}15 \qquad
\textbf\{(D) \}16 \qquad
\textbf\{(E) \}17 \qquad
\$\par \vspace{0.5em}\item A right pyramid has regular octagon $ABCDEFGH$ with side length $1$ as its base and apex $V.$ Segments $\overline{AV}$ and $\overline{DV}$ are perpendicular. What is the square of the height of the pyramid?

\$
\textbf\{(A) \}1 \qquad
\textbf\{(B) \}\frac\{1+\sqrt2\}\{2\} \qquad
\textbf\{(C) \}\sqrt2 \qquad
\textbf\{(D) \}\frac32 \qquad
\textbf\{(E) \}\frac\{2+\sqrt2\}\{3\} \qquad
\$\par \vspace{0.5em}\item What is the number of ordered triples $(a,b,c)$ of positive integers, with $a\le b\le c\le 9$, such that there exists a (non-degenerate) triangle $\triangle ABC$ with an integer inradius for which $a$, $b$, and $c$ are the lengths of the altitudes from $A$ to $\overline{BC}$, $B$ to $\overline{AC}$, and $C$ to $\overline{AB}$, respectively? (Recall that the inradius of a triangle is the radius of the largest possible circle that can be inscribed in the triangle.)

\$
\textbf\{(A) \}2\qquad
\textbf\{(B) \}3\qquad
\textbf\{(C) \}4\qquad
\textbf\{(D) \}5\qquad
\textbf\{(E) \}6\qquad
\$\par \vspace{0.5em}\item Pablo will decorate each of $6$ identical white balls with either a striped or a dotted pattern, using either red or blue paint. He will decide on the color and pattern for each ball by flipping a fair coin for each of the $12$ decisions he must make. After the paint dries, he will place the $6$ balls in an urn. Frida will randomly select one ball from the urn and note its color and pattern. The events "the ball Frida selects is red" and "the ball Frida selects is striped" may or may not be independent, depending on the outcome of Pablo's coin flips. The probability that these two events are independent can be written as $\frac mn,$ where $m$ and $n$ are relatively prime positive integers. What is $m?$ (Recall that two events $A$ and $B$ are independent if $P(A \text{ and }B) = P(A) \cdot P(B).$)

$\textbf{(A) } 243 \qquad \textbf{(B) } 245 \qquad \textbf{(C) } 247 \qquad \textbf{(D) } 249\qquad \textbf{(E) } 251$\par \vspace{0.5em}\end{enumerate}\newpage
\end{document}
