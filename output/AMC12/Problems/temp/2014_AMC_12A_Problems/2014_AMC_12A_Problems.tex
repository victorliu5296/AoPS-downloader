
\documentclass{article}
\usepackage{amsmath, amssymb}
\usepackage{geometry}
\geometry{a4paper, margin=0.75in}
\usepackage{enumitem}
\usepackage{hyperref}
\usepackage{fancyhdr}
\usepackage{tikz}
\usepackage{graphicx}
\usepackage{asymptote}
\usepackage{arcs}
\usepackage{xwatermark}
\begin{asydef}
  // Global Asymptote settings
  settings.outformat = "pdf";
  settings.render = 0;
  settings.prc = false;
  import olympiad;
  import cse5;
  size(8cm);
\end{asydef}
\pagestyle{fancy}
\fancyhead[L]{\textbf{AMC12 Problems}}
\fancyhead[R]{\textbf{2014}}
\fancyfoot[C]{\thepage}
\renewcommand{\headrulewidth}{0.4pt}
\renewcommand{\footrulewidth}{0.4pt}

\title{AMC12 Problems \\ 2014}
\date{}
\begin{document}\maketitle\thispagestyle{fancy}\newpage\section*{Problems}\begin{enumerate}[label=\arabic*., itemsep=0.5em]\item What is \(10 \cdot \left(\tfrac{1}{2} + \tfrac{1}{5} + \tfrac{1}{10}\right)^{-1}?\)

\( \textbf{(A)}\ 3\qquad\textbf{(B)}\ 8\qquad\textbf{(C)}\ \frac{25}{2}\qquad\textbf{(D)}\ \frac{170}{3}\qquad\textbf{(E)}\ 170\)\par \vspace{0.5em}\item At the theater children get in for half price.  The price for \(5\) adult tickets and \(4\) child tickets is \(\$24.50\).  How much would \(8\) adult tickets and \(6\) child tickets cost?

\(\textbf{(A) }\$35\qquad
\textbf{(B) }\$38.50\qquad
\textbf{(C) }\$40\qquad
\textbf{(D) }\$42\qquad
\textbf{(E) }\$42.50\)\par \vspace{0.5em}\item Walking down Jane Street, Ralph passed four houses in a row, each painted a different color. He passed the orange house before the red house, and he passed the blue house before the yellow house. The blue house was not next to the yellow house. How many orderings of the colored houses are possible?

\( \textbf{(A)}\ 2\qquad\textbf{(B)}\ 3\qquad\textbf{(C)}\ 4\qquad\textbf{(D)}\ 5\qquad\textbf{(E)}\ 6\)\par \vspace{0.5em}\item Suppose that \(a\) cows give \(b\) gallons of milk in \(c\) days. At this rate, how many gallons of milk will \(d\) cows give in \(e\) days?

\( \textbf{(A)}\ \frac{bde}{ac}\qquad\textbf{(B)}\ \frac{ac}{bde}\qquad\textbf{(C)}\ \frac{abde}{c}\qquad\textbf{(D)}\ \frac{bcde}{a}\qquad\textbf{(E)}\ \frac{abc}{de}\)\par \vspace{0.5em}\item On an algebra quiz, \(10\%\) of the students scored \(70\) points, \(35\%\) scored \(80\) points, \(30\%\) scored \(90\) points, and the rest scored \(100\) points. What is the difference between the mean and median score of the students' scores on this quiz?

\( \textbf{(A)}\ 1\qquad\textbf{(B)}\ 2\qquad\textbf{(C)}\ 3\qquad\textbf{(D)}\ 4\qquad\textbf{(E)}\ 5\)\par \vspace{0.5em}\item The difference between a two-digit number and the number obtained by reversing its digits is \(5\) times the sum of the digits of either number.  What is the sum of the two digit number and its reverse?

\(\textbf{(A) }44\qquad
\textbf{(B) }55\qquad
\textbf{(C) }77\qquad
\textbf{(D) }99\qquad
\textbf{(E) }110\)\par \vspace{0.5em}\item The first three terms of a geometric progression are \(\sqrt 3\), \(\sqrt[3]3\), and \(\sqrt[6]3\).  What is the fourth term?

\(\textbf{(A) }1\qquad
\textbf{(B) }\sqrt[7]3\qquad
\textbf{(C) }\sqrt[8]3\qquad
\textbf{(D) }\sqrt[9]3\qquad
\textbf{(E) }\sqrt[10]3\qquad\)\par \vspace{0.5em}\item A customer who intends to purchase an appliance has three coupons, only one of which may be used:

Coupon 1: \(10\%\) off the listed price if the listed price is at least \(\$50\)

Coupon 2: \(\$20\) off the listed price if the listed price is at least \(\$100\)

Coupon 3: \(18\%\) off the amount by which the listed price exceeds \(\$100\)

For which of the following listed prices will coupon \(1\) offer a greater price reduction than either coupon \(2\) or coupon \(3\)?

\(\textbf{(A) }\$179.95\qquad
\textbf{(B) }\$199.95\qquad
\textbf{(C) }\$219.95\qquad
\textbf{(D) }\$239.95\qquad
\textbf{(E) }\$259.95\qquad\)\par \vspace{0.5em}\item Five positive consecutive integers starting with \(a\) have average \(b\). What is the average of \(5\) consecutive integers that start with \(b\)?

\( \textbf{(A)}\ a+3\qquad\textbf{(B)}\ a+4\qquad\textbf{(C)}\ a+5\qquad\textbf{(D)}\ a+6\qquad\textbf{(E)}\ a+7\)\par \vspace{0.5em}\item Three congruent isosceles triangles are constructed with their bases on the sides of an equilateral triangle of side length \(1\).  The sum of the areas of the three isosceles triangles is the same as the area of the equilateral triangle.  What is the length of one of the two congruent sides of one of the isosceles triangles?

\(\textbf{(A) }\dfrac{\sqrt3}4\qquad
\textbf{(B) }\dfrac{\sqrt3}3\qquad
\textbf{(C) }\dfrac23\qquad
\textbf{(D) }\dfrac{\sqrt2}2\qquad
\textbf{(E) }\dfrac{\sqrt3}2\)\par \vspace{0.5em}\item David drives from his home to the airport to catch a flight.  He drives \(35\) miles in the first hour, but realizes that he will be \(1\) hour late if he continues at this speed.  He increases his speed by \(15\) miles per hour for the rest of the way to the airport and arrives \(30\) minutes early.  How many miles is the airport from his home?

\(\textbf{(A) }140\qquad
\textbf{(B) }175\qquad
\textbf{(C) }210\qquad
\textbf{(D) }245\qquad
\textbf{(E) }280\qquad\)\par \vspace{0.5em}\item Two circles intersect at points \(A\) and \(B\).  The minor arcs \(AB\) measure \(30^\circ\) on one circle and \(60^\circ\) on the other circle.  What is the ratio of the area of the larger circle to the area of the smaller circle?

\(\textbf{(A) }2\qquad
\textbf{(B) }1+\sqrt3\qquad
\textbf{(C) }3\qquad
\textbf{(D) }2+\sqrt3\qquad
\textbf{(E) }4\qquad\)\par \vspace{0.5em}\item A fancy bed and breakfast inn has \(5\) rooms, each with a distinctive color-coded decor.  One day \(5\) friends arrive to spend the night.  There are no other guests that night.  The friends can room in any combination they wish, but with no more than \(2\) friends per room.  In how many ways can the innkeeper assign the guests to the rooms?

\(\textbf{(A) }2100\qquad
\textbf{(B) }2220\qquad
\textbf{(C) }3000\qquad
\textbf{(D) }3120\qquad
\textbf{(E) }3125\qquad\)\par \vspace{0.5em}\item Let \(a<b<c\) be three integers such that \(a,b,c\) is an arithmetic progression and \(a,c,b\) is a geometric progression.  What is the smallest possible value of \(c\)?

\(\textbf{(A) }-2\qquad
\textbf{(B) }1\qquad
\textbf{(C) }2\qquad
\textbf{(D) }4\qquad
\textbf{(E) }6\qquad\)\par \vspace{0.5em}\item A five-digit palindrome is a positive integer with respective digits \(abcba\), where \(a\) is non-zero.  Let \(S\) be the sum of all five-digit palindromes.  What is the sum of the digits of \(S\)?

\(\textbf{(A) }9\qquad
\textbf{(B) }18\qquad
\textbf{(C) }27\qquad
\textbf{(D) }36\qquad
\textbf{(E) }45\qquad\)\par \vspace{0.5em}\item The product \((8)(888\ldots 8)\), where the second factor has \(k\) digits, is an integer whose digits have a sum of \(1000\).  What is \(k\)?

\(\textbf{(A) }901\qquad
\textbf{(B) }911\qquad
\textbf{(C) }919\qquad
\textbf{(D) }991\qquad
\textbf{(E) }999\qquad\)\par \vspace{0.5em}\item A \(4\times 4\times h\) rectangular box contains a sphere of radius \(2\) and eight smaller spheres of radius \(1\).  The smaller spheres are each tangent to three sides of the box, and the larger sphere is tangent to each of the smaller spheres.  What is \(h\)?

<center>
\begin{center}
\begin{asy}
import olympiad;
import cse5;
import graph3;
import solids;
real h=2+2*sqrt(7);
currentprojection=orthographic((0.75,-5,h/2+1),target=(2,2,h/2));
currentlight=light(4,-4,4);
draw((0,0,0)--(4,0,0)--(4,4,0)--(0,4,0)--(0,0,0)^^(4,0,0)--(4,0,h)--(4,4,h)--(0,4,h)--(0,4,0));
draw(shift((1,3,1))*unitsphere,gray(0.85));
draw(shift((3,3,1))*unitsphere,gray(0.85));
draw(shift((3,1,1))*unitsphere,gray(0.85));
draw(shift((1,1,1))*unitsphere,gray(0.85));
draw(shift((2,2,h/2))*scale(2,2,2)*unitsphere,gray(0.85));
draw(shift((1,3,h-1))*unitsphere,gray(0.85));
draw(shift((3,3,h-1))*unitsphere,gray(0.85));
draw(shift((3,1,h-1))*unitsphere,gray(0.85));
draw(shift((1,1,h-1))*unitsphere,gray(0.85));
draw((0,0,0)--(0,0,h)--(4,0,h)^^(0,0,h)--(0,4,h));
\end{asy}
\end{center}
</center>

\(\textbf{(A) }2+2\sqrt 7\qquad
\textbf{(B) }3+2\sqrt 5\qquad
\textbf{(C) }4+2\sqrt 7\qquad
\textbf{(D) }4\sqrt 5\qquad
\textbf{(E) }4\sqrt 7\qquad\)\par \vspace{0.5em}\item The domain of the function \(f(x)=\log_{\frac12}(\log_4(\log_{\frac14}(\log_{16}(\log_{\frac1{16}}x))))\) is an interval of length \(\tfrac mn\), where \(m\) and \(n\) are relatively prime positive integers.  What is \(m+n\)?

\(\textbf{(A) }19\qquad
\textbf{(B) }31\qquad
\textbf{(C) }271\qquad
\textbf{(D) }319\qquad
\textbf{(E) }511\qquad\)\par \vspace{0.5em}\item There are exactly \(N\) distinct rational numbers \(k\) such that \(|k|<200\) and 
\begin{equation*}
5x^2+kx+12=0
\end{equation*}
 has at least one integer solution for \(x\).  What is \(N\)?

\(\textbf{(A) }6\qquad
\textbf{(B) }12\qquad
\textbf{(C) }24\qquad
\textbf{(D) }48\qquad
\textbf{(E) }78\qquad\)\par \vspace{0.5em}\item In \(\triangle BAC\), \(\angle BAC=40^\circ\), \(AB=10\), and \(AC=6\).  Points \(D\) and \(E\) lie on \(\overline{AB}\) and \(\overline{AC}\) respectively.  What is the minimum possible value of \(BE+DE+CD\)?

\(\textbf{(A) }6\sqrt 3+3\qquad
\textbf{(B) }\dfrac{27}2\qquad
\textbf{(C) }8\sqrt 3\qquad
\textbf{(D) }14\qquad
\textbf{(E) }3\sqrt 3+9\qquad\)\par \vspace{0.5em}\item For every real number \(x\), let \(\lfloor x\rfloor\) denote the greatest integer not exceeding \(x\), and let 
\begin{equation*}
f(x)=\lfloor x\rfloor(2014^{x-\lfloor x\rfloor}-1).
\end{equation*}
  The set of all numbers \(x\) such that \(1\leq x<2014\) and \(f(x)\leq 1\) is a union of disjoint intervals.  What is the sum of the lengths of those intervals?

\(\textbf{(A) }1\qquad
\textbf{(B) }\dfrac{\log 2015}{\log 2014}\qquad
\textbf{(C) }\dfrac{\log 2014}{\log 2013}\qquad
\textbf{(D) }\dfrac{2014}{2013}\qquad
\textbf{(E) }2014^{\frac1{2014}}\qquad\)\par \vspace{0.5em}\item The number \(5^{867}\) is between \(2^{2013}\) and \(2^{2014}\).  How many pairs of integers \((m,n)\) are there such that \(1\leq m\leq 2012\) and 
\begin{equation*}
5^n<2^m<2^{m+2}<5^{n+1}?
\end{equation*}

\(\textbf{(A) }278\qquad
\textbf{(B) }279\qquad
\textbf{(C) }280\qquad
\textbf{(D) }281\qquad
\textbf{(E) }282\qquad\)\par \vspace{0.5em}\item The fraction 
\begin{equation*}
\dfrac1{99^2}=0.\overline{b_{n-1}b_{n-2}\ldots b_2b_1b_0},
\end{equation*}
 where \(n\) is the length of the period of the repeating decimal expansion.  What is the sum \(b_0+b_1+\cdots+b_{n-1}\)?

\(\textbf{(A) }874\qquad
\textbf{(B) }883\qquad
\textbf{(C) }887\qquad
\textbf{(D) }891\qquad
\textbf{(E) }892\qquad\)\par \vspace{0.5em}\item Let \(f_0(x)=x+|x-100|-|x+100|\), and for \(n\geq 1\), let \(f_n(x)=|f_{n-1}(x)|-1\).  For how many values of \(x\) is \(f_{100}(x)=0\)?

\(\textbf{(A) }299\qquad
\textbf{(B) }300\qquad
\textbf{(C) }301\qquad
\textbf{(D) }302\qquad
\textbf{(E) }303\qquad\)\par \vspace{0.5em}\item The parabola \(P\) has focus \((0,0)\) and goes through the points \((4,3)\) and \((-4,-3)\).  For how many points \((x,y)\in P\) with integer coordinates is it true that \(|4x+3y|\leq 1000\)?

\(\textbf{(A) }38\qquad
\textbf{(B) }40\qquad
\textbf{(C) }42\qquad
\textbf{(D) }44\qquad
\textbf{(E) }46\qquad\)\par \vspace{0.5em}\end{enumerate}
\end{document}
