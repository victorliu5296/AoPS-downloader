
\documentclass{article}
\usepackage{amsmath, amssymb}
\usepackage{geometry}
\geometry{a4paper, margin=0.75in}
\usepackage{enumitem}
\usepackage{hyperref}
\usepackage{fancyhdr}
\usepackage{tikz}
\usepackage{graphicx}
\usepackage{asymptote}
\usepackage{arcs}
\usepackage{xwatermark}
\begin{asydef}
  // Global Asymptote settings
  settings.outformat = "pdf";
  settings.render = 0;
  settings.prc = false;
  import olympiad;
  import cse5;
  size(8cm);
\end{asydef}
\pagestyle{fancy}
\fancyhead[L]{\textbf{AMC12 Problems}}
\fancyhead[R]{\textbf{2024}}
\fancyfoot[C]{\thepage}
\renewcommand{\headrulewidth}{0.4pt}
\renewcommand{\footrulewidth}{0.4pt}

\title{AMC12 Problems \\ 2024}
\date{}
\begin{document}\maketitle\thispagestyle{fancy}\newpage\section*{2012 AMC 12B}\begin{enumerate}[label=\arabic*., itemsep=0.5em]\item Each third-grade classroom at Pearl Creek Elementary has 18 students and 2 pet rabbits. How many more students than rabbits are there in all 4 of the third-grade classrooms?

\( \textbf{(A)}\ 48\qquad\textbf{(B)}\ 56\qquad\textbf{(C)}\ 64\qquad\textbf{(D)}\ 72\qquad\textbf{(E)}\ 80 \)\par \vspace{0.5em}\item A circle of radius 5 is inscribed in a rectangle as shown. The ratio of the length of the rectangle to its width is 2:1. What is the area of the rectangle?

\begin{center}
\begin{asy}
import olympiad;
import cse5;
draw((0,0)--(0,10)--(20,10)--(20,0)--cycle); 
draw(circle((10,5),5));
\end{asy}
\end{center}

\(\textbf{(A)}\ 50\qquad\textbf{(B)}\ 100\qquad\textbf{(C)}\ 125\qquad\textbf{(D)}\ 150\qquad\textbf{(E)}\ 200\)\par \vspace{0.5em}\item For a science project, Sammy observed a chipmunk and squirrel stashing acorns in holes. The chipmunk hid 3 acorns in each of the holes it dug. The squirrel hid 4 acorns in each of the holes it dug. They each hid the same number of acorns, although the squirrel needed 4 fewer holes. How many acorns did the chipmunk hide? 

\(\textbf{(A)}\ 30\qquad\textbf{(B)}\ 36\qquad\textbf{(C)}\ 42\qquad\textbf{(D)}\ 48\qquad\textbf{(E)}\ 54\)\par \vspace{0.5em}\item Suppose that the euro is worth \(1.30\) dollars. If Diana has \(500\) dollars and Etienne has \(400\) euros, by what percent is the value of Etienne's money greater than the value of Diana's money?

\(\textbf{(A)}\ 2\qquad\textbf{(B)}\ 4\qquad\textbf{(C)}\ 6.5\qquad\textbf{(D)}\ 8\qquad\textbf{(E)}\ 13\)\par \vspace{0.5em}\item Two integers have a sum of 26. When two more integers are added to the first two, the sum is 41. Finally, when two more integers are added to the sum of the previous 4 integers, the sum is 57. What is the minimum number of even integers among the 6 integers? 

\(\textbf{(A)}\ 1\qquad\textbf{(B)}\ 2\qquad\textbf{(C)}\ 3\qquad\textbf{(D)}\ 4\qquad\textbf{(E)}\ 5\)\par \vspace{0.5em}\item In order to estimate the value of \(x-y\) where \(x\) and \(y\) are real numbers with \(x > y > 0\), Xiaoli rounded \(x\) up by a small amount, rounded \(y\) down by the same amount, and then subtracted her rounded values. Which of the following statements is necessarily correct?

\(\textbf{(A)}\ \text{Her estimate is larger than }x-y\)

\(\textbf{(B)}\ \text{Her estimate is smaller than }x-y\)

\(\textbf{(C)}\ \text{Her estimate equals }x-y\)

\(\textbf{(D)}\ \text{Her estimate equals }y - x\)

\(\textbf{(E)}\ \text{Her estimate is 0}\)\par \vspace{0.5em}\item Small lights are hung on a string 6 inches apart in the order red, red, green, green, green, red, red, green, green, green, and so on continuing this pattern of 2 red lights followed by 3 green lights. How many feet separate the 3rd red light and the 21st red light?

'''Note:''' 1 foot is equal to 12 inches.

\(\textbf{(A)}\ 18\qquad\textbf{(B)}\ 18.5\qquad\textbf{(C)}\ 20\qquad\textbf{(D)}\ 20.5\qquad\textbf{(E)}\ 22.5 \)\par \vspace{0.5em}\item A dessert chef prepares the dessert for every day of a week starting with Sunday. The dessert each day is either cake, pie, ice cream, or pudding. The same dessert may not be served two days in a row. There must be cake on Friday because of a birthday. How many different dessert menus for the week are possible?

\(\textbf{(A)}\ 729\qquad\textbf{(B)}\ 972\qquad\textbf{(C)}\ 1024\qquad\textbf{(D)}\ 2187\qquad\textbf{(E)}\ 2304 \)\par \vspace{0.5em}\item It takes Clea 60 seconds to walk down an escalator when it is not moving, and 24 seconds when it is moving. How many seconds would it take Clea to ride the escalator down when she is not walking?

\(\textbf{(A)}\ 36\qquad\textbf{(B)}\ 40\qquad\textbf{(C)}\ 42\qquad\textbf{(D)}\ 48\qquad\textbf{(E)}\ 52 \)\par \vspace{0.5em}\item What is the area of the polygon whose vertices are the points of intersection of the curves \(x^2 + y^2 =25\) and \((x-4)^2 + 9y^2 = 81\)?

\(\textbf{(A)}\ 24\qquad\textbf{(B)}\ 27\qquad\textbf{(C)}\ 36\qquad\textbf{(D)}\ 37.5\qquad\textbf{(E)}\ 42\)\par \vspace{0.5em}\item In the equation below, \(A\) and \(B\) are consecutive positive integers, and \(A\), \(B\), and \(A+B\) represent number bases: 
\begin{equation*}
132_A+43_B=69_{A+B}.
\end{equation*}

What is \(A+B\)?

\(\textbf{(A)}\ 9\qquad\textbf{(B)}\ 11\qquad\textbf{(C)}\ 13\qquad\textbf{(D)}\ 15\qquad\textbf{(E)}\ 17 \)\par \vspace{0.5em}\item How many sequences of zeros and ones of length 20 have all the zeros consecutive, or all the ones consecutive, or both?

\(\textbf{(A)}\ 190\qquad\textbf{(B)}\ 192\qquad\textbf{(C)}\ 211\qquad\textbf{(D)}\ 380\qquad\textbf{(E)}\ 382 \)\par \vspace{0.5em}\item Two parabolas have equations \(y= x^2 + ax +b\) and \(y= x^2 + cx +d\), where \(a\), \(b\), \(c\), and \(d\) are integers, each chosen independently by rolling a fair six-sided die. What is the probability that the parabolas will have at least one point in common?

\(\textbf{(A)}\ \frac{1}{2}\qquad\textbf{(B)}\ \frac{25}{36}\qquad\textbf{(C)}\ \frac{5}{6}\qquad\textbf{(D)}\ \frac{31}{36}\qquad\textbf{(E)}\ 1 \)\par \vspace{0.5em}\item Bernardo and Silvia play the following game. An integer between \(0\) and \(999\) inclusive is selected and given to Bernardo. Whenever Bernardo receives a number, he doubles it and passes the result to Silvia. Whenever Silvia receives a number, she adds \(50\) to it and passes the result to Bernardo. The winner is the last person who produces a number less than \(1000\). Let \(N\) be the smallest initial number that results in a win for Bernardo. What is the sum of the digits of \(N\)?

\( \textbf{(A)}\ 7\qquad\textbf{(B)}\ 8\qquad\textbf{(C)}\ 9\qquad\textbf{(D)}\ 10\qquad\textbf{(E)}\ 11 \)\par \vspace{0.5em}\item Jesse cuts a circular paper disk of radius \(12\) along two radii to form two sectors, the smaller having a central angle of \(120\) degrees. He makes two circular cones, using each sector to form the lateral surface of a cone. What is the ratio of the volume of the smaller cone to that of the larger one?

\(\textbf{(A)}\ \frac{1}{8}\qquad\textbf{(B)}\ \frac{1}{4}\qquad\textbf{(C)}\ \frac{\sqrt{10}}{10}\qquad\textbf{(D)}\ \frac{\sqrt{5}}{6}\qquad\textbf{(E)}\ \frac{\sqrt{5}}{5}\)\par \vspace{0.5em}\item Amy, Beth, and Jo listen to four different songs and discuss which ones they like. No song is liked by all three. Furthermore, for each of the three pairs of the girls, there is at least one song liked by those girls but disliked by the third. In how many different ways is this possible?

\(\textbf{(A)}\ 108\qquad\textbf{(B)}\ 132\qquad\textbf{(C)}\ 671\qquad\textbf{(D)}\ 846\qquad\textbf{(E)}\ 1105 \)\par \vspace{0.5em}\item Square \(PQRS\) lies in the first quadrant. Points \((3,0), (5,0), (7,0),\) and \((13,0)\) lie on lines \(SP, RQ, PQ,\) and \(SR\), respectively. What is the sum of the coordinates of the center of the square \(PQRS\)?

\(\textbf{(A)}\ 6\qquad\textbf{(B)}\ \frac{31}{5}\qquad\textbf{(C)}\ \frac{32}{5}\qquad\textbf{(D)}\ \frac{33}{5}\qquad\textbf{(E)}\ \frac{34}{5} \)\par \vspace{0.5em}\item Let \((a_1,a_2, \dots ,a_{10})\) be a list of the first \(10\) positive integers such that for each \(2 \le i \le 10\) either \(a_i+1\) or \(a_i-1\) or both appear somewhere before \(a_i\) in the list. How many such lists are there?

\(\textbf{(A)}\ 120\qquad\textbf{(B)}\ 512\qquad\textbf{(C)}\ 1024\qquad\textbf{(D)}\ 181,440\qquad\textbf{(E)}\ 362,880\)


label("$P_1$", P[1], dir(P[1]));
label("$P_2$", P[2], dir(P[2]));
label("$P_3$", P[3], dir(-45));
label("$P_4$", P[4], dir(P[4]));
label("$P'_1$", Pp[1], dir(Pp[1]));
label("$P'_2$", Pp[2], dir(Pp[2]));
label("$P'_3$", Pp[3], dir(-100));
label("$P'_4$", Pp[4], dir(Pp[4]));\par \vspace{0.5em}\item A trapezoid has side lengths \(3\), \(5\), \(7\), and \(11\). The sums of all the possible areas of the trapezoid can be written in the form of \(r_1\sqrt{n_1}+r_2\sqrt{n_2}+r_3\), where \(r_1\), \(r_2\), and \(r_3\) are rational numbers and \(n_1\) and \(n_2\) are positive integers not divisible by the square of any prime. What is the greatest integer less than or equal to \(r_1+r_2+r_3+n_1+n_2\)?

\(\textbf{(A)}\ 57\qquad\textbf{(B)}\ 59\qquad\textbf{(C)}\ 61\qquad\textbf{(D)}\ 63\qquad\textbf{(E)}\ 65\)\par \vspace{0.5em}\item Square \(AXYZ\) is inscribed in equiangular hexagon \(ABCDEF\) with \(X\) on \(\overline{BC}\), \(Y\) on \(\overline{DE}\), and \(Z\) on \(\overline{EF}\). Suppose that \(AB=40\), and \(EF=41(\sqrt{3}-1)\). What is the side-length of the square?


\begin{center}
\begin{asy}
import olympiad;
import cse5;
size(200);
defaultpen(linewidth(1));
pair A=origin,B=(2.5,0),C=B+2.5*dir(60), D=C+1.75*dir(120),E=D-(3.19,0),F=E-1.8*dir(60);
pair X=waypoint(B--C,0.345),Z=rotate(90,A)*X,Y=rotate(90,Z)*A;
draw(A--B--C--D--E--F--cycle);
draw(A--X--Y--Z--cycle,linewidth(0.9)+linetype("2 2"));
dot("$A$",A,W,linewidth(4));
dot("$B$",B,dir(0),linewidth(4));
dot("$C$",C,dir(0),linewidth(4));
dot("$D$",D,dir(20),linewidth(4));
dot("$E$",E,dir(100),linewidth(4));
dot("$F$",F,W,linewidth(4));
dot("$X$",X,dir(0),linewidth(4));
dot("$Y$",Y,N,linewidth(4));
dot("$Z$",Z,W,linewidth(4));
\end{asy}
\end{center}


\(\textbf{(A)}\ 29\sqrt{3} \qquad\textbf{(B)}\ \frac{21}{2}\sqrt{2}+\frac{41}{2}\sqrt{3}\qquad\textbf{(C)}\ 20\sqrt{3}+16\)

\(\textbf{(D)}\ 20\sqrt{2}+13\sqrt{3} \qquad\textbf{(E)}\ 21\sqrt{6} \)\par \vspace{0.5em}\item A bug travels from \(A\) to \(B\) along the segments in the hexagonal lattice pictured below. The segments marked with an arrow can be traveled only in the direction of the arrow, and the bug never travels the same segment more than once. How many different paths are there?


\begin{center}
\begin{asy}
import olympiad;
import cse5;
size(10cm);
draw((0.0,0.0)--(1.0,1.7320508075688772)--(3.0,1.7320508075688772)--(4.0,3.4641016151377544)--(6.0,3.4641016151377544)--(7.0,5.196152422706632)--(9.0,5.196152422706632)--(10.0,6.928203230275509)--(12.0,6.928203230275509));
draw((0.0,0.0)--(1.0,1.7320508075688772)--(3.0,1.7320508075688772)--(4.0,3.4641016151377544)--(6.0,3.4641016151377544)--(7.0,5.196152422706632)--(9.0,5.196152422706632)--(10.0,6.928203230275509)--(12.0,6.928203230275509));
draw((3.0,-1.7320508075688772)--(4.0,0.0)--(6.0,0.0)--(7.0,1.7320508075688772)--(9.0,1.7320508075688772)--(10.0,3.4641016151377544)--(12.0,3.464101615137755)--(13.0,5.196152422706632)--(15.0,5.196152422706632));
draw((6.0,-3.4641016151377544)--(7.0,-1.7320508075688772)--(9.0,-1.7320508075688772)--(10.0,0.0)--(12.0,0.0)--(13.0,1.7320508075688772)--(15.0,1.7320508075688776)--(16.0,3.464101615137755)--(18.0,3.4641016151377544));
draw((9.0,-5.196152422706632)--(10.0,-3.464101615137755)--(12.0,-3.464101615137755)--(13.0,-1.7320508075688776)--(15.0,-1.7320508075688776)--(16.0,0)--(18.0,0.0)--(19.0,1.7320508075688772)--(21.0,1.7320508075688767));
draw((12.0,-6.928203230275509)--(13.0,-5.196152422706632)--(15.0,-5.196152422706632)--(16.0,-3.464101615137755)--(18.0,-3.4641016151377544)--(19.0,-1.7320508075688772)--(21.0,-1.7320508075688767)--(22.0,0));
draw((0.0,-0.0)--(1.0,-1.7320508075688772)--(3.0,-1.7320508075688772)--(4.0,-3.4641016151377544)--(6.0,-3.4641016151377544)--(7.0,-5.196152422706632)--(9.0,-5.196152422706632)--(10.0,-6.928203230275509)--(12.0,-6.928203230275509));
draw((3.0,1.7320508075688772)--(4.0,-0.0)--(6.0,-0.0)--(7.0,-1.7320508075688772)--(9.0,-1.7320508075688772)--(10.0,-3.4641016151377544)--(12.0,-3.464101615137755)--(13.0,-5.196152422706632)--(15.0,-5.196152422706632));
draw((6.0,3.4641016151377544)--(7.0,1.7320508075688772)--(9.0,1.7320508075688772)--(10.0,-0.0)--(12.0,-0.0)--(13.0,-1.7320508075688772)--(15.0,-1.7320508075688776)--(16.0,-3.464101615137755)--(18.0,-3.4641016151377544));
draw((9.0,5.1961524)--(10.0,3.464101)--(12.0,3.46410)--(13.0,1.73205)--(15.0,1.732050)--(16.0,0)--(18.0,-0.0)--(19.0,-1.7320)--(21.0,-1.73205080));
draw((12.0,6.928203)--(13.0,5.1961524)--(15.0,5.1961524)--(16.0,3.464101615)--(18.0,3.4641016)--(19.0,1.7320508)--(21.0,1.732050)--(22.0,0));
dot((0,0));
dot((22,0));
label("$A$",(0,0),WNW);
label("$B$",(22,0),E);
filldraw((2.0,1.7320508075688772)--(1.6,1.2320508075688772)--(1.75,1.7320508075688772)--(1.6,2.232050807568877)--cycle,black);
filldraw((5.0,3.4641016151377544)--(4.6,2.9641016151377544)--(4.75,3.4641016151377544)--(4.6,3.9641016151377544)--cycle,black);
filldraw((8.0,5.196152422706632)--(7.6,4.696152422706632)--(7.75,5.196152422706632)--(7.6,5.696152422706632)--cycle,black);
filldraw((11.0,6.928203230275509)--(10.6,6.428203230275509)--(10.75,6.928203230275509)--(10.6,7.428203230275509)--cycle,black);
filldraw((4.6,0.0)--(5.0,-0.5)--(4.85,0.0)--(5.0,0.5)--cycle,white);
filldraw((8.0,1.732050)--(7.6,1.2320)--(7.75,1.73205)--(7.6,2.2320)--cycle,black);
filldraw((11.0,3.4641016)--(10.6,2.9641016)--(10.75,3.46410161)--(10.6,3.964101)--cycle,black);
filldraw((14.0,5.196152422706632)--(13.6,4.696152422706632)--(13.75,5.196152422706632)--(13.6,5.696152422706632)--cycle,black);
filldraw((8.0,-1.732050)--(7.6,-2.232050)--(7.75,-1.7320508)--(7.6,-1.2320)--cycle,black);
filldraw((10.6,0.0)--(11,-0.5)--(10.85,0.0)--(11,0.5)--cycle,white);
filldraw((14.0,1.7320508075688772)--(13.6,1.2320508075688772)--(13.75,1.7320508075688772)--(13.6,2.232050807568877)--cycle,black);
filldraw((17.0,3.464101615137755)--(16.6,2.964101615137755)--(16.75,3.464101615137755)--(16.6,3.964101615137755)--cycle,black);
filldraw((11.0,-3.464101615137755)--(10.6,-3.964101615137755)--(10.75,-3.464101615137755)--(10.6,-2.964101615137755)--cycle,black);
filldraw((14.0,-1.7320508075688776)--(13.6,-2.2320508075688776)--(13.75,-1.7320508075688776)--(13.6,-1.2320508075688776)--cycle,black);
filldraw((16.6,0)--(17,-0.5)--(16.85,0)--(17,0.5)--cycle,white);
filldraw((20.0,1.7320508075688772)--(19.6,1.2320508075688772)--(19.75,1.7320508075688772)--(19.6,2.232050807568877)--cycle,black);
filldraw((14.0,-5.196152422706632)--(13.6,-5.696152422706632)--(13.75,-5.196152422706632)--(13.6,-4.696152422706632)--cycle,black);
filldraw((17.0,-3.464101615137755)--(16.6,-3.964101615137755)--(16.75,-3.464101615137755)--(16.6,-2.964101615137755)--cycle,black);
filldraw((20.0,-1.7320508075688772)--(19.6,-2.232050807568877)--(19.75,-1.7320508075688772)--(19.6,-1.2320508075688772)--cycle,black);
filldraw((2.0,-1.7320508075688772)--(1.6,-1.2320508075688772)--(1.75,-1.7320508075688772)--(1.6,-2.232050807568877)--cycle,black);
filldraw((5.0,-3.4641016)--(4.6,-2.964101)--(4.75,-3.4641)--(4.6,-3.9641016)--cycle,black);
filldraw((8.0,-5.1961524)--(7.6,-4.6961524)--(7.75,-5.19615242)--(7.6,-5.696152422)--cycle,black);
filldraw((11.0,-6.9282032)--(10.6,-6.4282032)--(10.75,-6.928203)--(10.6,-7.428203)--cycle,black);
\end{asy}
\end{center}


\(\textbf{(A)}\ 2112\qquad\textbf{(B)}\ 2304\qquad\textbf{(C)}\ 2368\qquad\textbf{(D)}\ 2384\qquad\textbf{(E)}\ 2400\)\par \vspace{0.5em}\item Consider all polynomials of a complex variable, \(P(z)=4z^4+az^3+bz^2+cz+d\), where \(a,b,c,\) and \(d\) are integers, \(0\le d\le c\le b\le a\le 4\), and the polynomial has a zero \(z_0\) with \(|z_0|=1.\) What is the sum of all values \(P(1)\) over all the polynomials with these properties?

\(\textbf{(A)}\ 84\qquad\textbf{(B)}\ 92\qquad\textbf{(C)}\ 100\qquad\textbf{(D)}\ 108\qquad\textbf{(E)}\ 120 \)\par \vspace{0.5em}\item Define the function \(f_1\) on the positive integers by setting \(f_1(1)=1\) and if \(n=p_1^{e_1}p_2^{e_2}\cdots p_k^{e_k}\) is the prime factorization of \(n>1\), then 
\begin{equation*}
f_1(n)=(p_1+1)^{e_1-1}(p_2+1)^{e_2-1}\cdots (p_k+1)^{e_k-1}.
\end{equation*}

For every \(m\ge 2\), let \(f_m(n)=f_1(f_{m-1}(n))\). For how many \(N\)s in the range \(1\le N\le 400\) is the sequence \((f_1(N),f_2(N),f_3(N),\dots )\) unbounded?

'''Note:''' A sequence of positive numbers is unbounded if for every integer \(B\), there is a member of the sequence greater than \(B\).

\(\textbf{(A)}\ 15\qquad\textbf{(B)}\ 16\qquad\textbf{(C)}\ 17\qquad\textbf{(D)}\ 18\qquad\textbf{(E)}\ 19 \)\par \vspace{0.5em}\item Let \(S=\{(x,y) : x\in \{0,1,2,3,4\}, y\in \{0,1,2,3,4,5\},\text{ and } (x,y)\ne (0,0)\}\). 
Let \(T\) be the set of all right triangles whose vertices are in \(S\). For every right triangle \(t=\triangle{ABC}\) with vertices \(A\), \(B\), and \(C\) in counter-clockwise order and right angle at \(A\), let \(f(t)=\tan(\angle{CBA})\). What is 
\begin{equation*}
\prod_{t\in T} f(t)?
\end{equation*}


\(\textbf{(A)}\ 1\qquad\textbf{(B)}\ \frac{625}{144}\qquad\textbf{(C)}\ \frac{125}{24}\qquad\textbf{(D)}\ 6\qquad\textbf{(E)}\ \frac{625}{24} \)\par \vspace{0.5em}\end{enumerate}
\end{document}
