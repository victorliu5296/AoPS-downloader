
\documentclass{article}
\usepackage{amsmath, amssymb}
\usepackage{geometry}
\geometry{a4paper, margin=0.75in}
\usepackage{enumitem}
\usepackage{hyperref}
\usepackage{fancyhdr}
\usepackage{tikz}
\usepackage{graphicx}
\usepackage{asymptote}
\usepackage{arcs}
\usepackage{xwatermark}
\begin{asydef}
  // Global Asymptote settings
  settings.outformat = "pdf";
  settings.render = 0;
  settings.prc = false;
  import olympiad;
  import cse5;
  size(8cm);
\end{asydef}
\pagestyle{fancy}
\fancyhead[L]{\textbf{AMC12 Problems}}
\fancyhead[R]{\textbf{2022}}
\fancyfoot[C]{\thepage}
\renewcommand{\headrulewidth}{0.4pt}
\renewcommand{\footrulewidth}{0.4pt}

\title{AMC12 Problems \\ 2022}
\date{}
\begin{document}\maketitle\thispagestyle{fancy}\newpage\section*{Problems}\begin{enumerate}[label=\arabic*., itemsep=0.5em]\item What is the value of 
\begin{equation*}
3+\frac{1}{3+\frac{1}{3+\frac13}}?
\end{equation*}

\(\textbf{(A)}\ \frac{31}{10}\qquad\textbf{(B)}\ \frac{49}{15}\qquad\textbf{(C)}\ \frac{33}{10}\qquad\textbf{(D)}\ \frac{109}{33}\qquad\textbf{(E)}\ \frac{15}{4}\)\par \vspace{0.5em}\item The sum of three numbers is \(96.\) The first number is \(6\) times the third number, and the third number is \(40\) less than the second number. What is the absolute value of the difference between the first and second numbers?

\(\textbf{(A) } 1 \qquad \textbf{(B) } 2 \qquad \textbf{(C) } 3 \qquad \textbf{(D) } 4 \qquad \textbf{(E) } 5\)\par \vspace{0.5em}\item Five rectangles, \(A\), \(B\), \(C\), \(D\), and \(E\), are arranged in a square as shown below. These rectangles have dimensions \(1\times6\), \(2\times4\), \(5\times6\), \(2\times7\), and \(2\times3\), respectively. (The figure is not drawn to scale.) Which of the five rectangles is the shaded one in the middle?

\begin{center}
\begin{asy}
import olympiad;
import cse5;
size(150);
currentpen = black+1.25bp;
fill((3,2.5)--(3,4.5)--(5.3,4.5)--(5.3,2.5)--cycle,gray);
draw((0,0)--(7,0)--(7,7)--(0,7)--(0,0));
draw((3,0)--(3,4.5));
draw((0,4.5)--(5.3,4.5));
draw((5.3,7)--(5.3,2.5));
draw((7,2.5)--(3,2.5));
\end{asy}
\end{center}

\(\textbf{(A) }A\qquad\textbf{(B) }B \qquad\textbf{(C) }C \qquad\textbf{(D) }D\qquad\textbf{(E) }E\)\par \vspace{0.5em}\item The least common multiple of a positive integer \(n\) and \(18\) is \(180\), and the greatest common divisor of \(n\) and \(45\) is \(15\). What is the sum of the digits of \(n\)?

\(\textbf{(A) } 3 \qquad \textbf{(B) } 6 \qquad \textbf{(C) } 8 \qquad \textbf{(D) } 9 \qquad \textbf{(E) } 12\)\par \vspace{0.5em}\item The <em>taxicab distance</em> between points \((x_1, y_1)\) and \((x_2, y_2)\) in the coordinate plane is given by 
\begin{equation*}
|x_1 - x_2| + |y_1 - y_2|.
\end{equation*}

For how many points \(P\) with integer coordinates is the taxicab distance between \(P\) and the origin less than or equal to \(20\)?

\(\textbf{(A)} \, 441 \qquad\textbf{(B)} \, 761 \qquad\textbf{(C)} \, 841 \qquad\textbf{(D)} \, 921  \qquad\textbf{(E)} \, 924 \)\par \vspace{0.5em}\item A data set consists of \(6\) (not distinct) positive integers: \(1\), \(7\), \(5\), \(2\), \(5\), and \(X\). The
average (arithmetic mean) of the \(6\) numbers equals a value in the data set. What is
the sum of all possible values of \(X\)?

\(\textbf{(A) } 10 \qquad \textbf{(B) } 26 \qquad \textbf{(C) } 32 \qquad \textbf{(D) } 36 \qquad \textbf{(E) } 40\)\par \vspace{0.5em}\item A rectangle is partitioned into \(5\) regions as shown. Each region is to be painted a solid color - red, orange, yellow, blue, or green - so that regions that touch are painted different colors, and colors can be used more than once. How many different colorings are possible?


\begin{center}
\begin{asy}
import olympiad;
import cse5;
size(5.5cm); draw((0,0)--(0,2)--(2,2)--(2,0)--cycle); draw((2,0)--(8,0)--(8,2)--(2,2)--cycle); draw((8,0)--(12,0)--(12,2)--(8,2)--cycle); draw((0,2)--(6,2)--(6,4)--(0,4)--cycle); draw((6,2)--(12,2)--(12,4)--(6,4)--cycle);
\end{asy}
\end{center}


\(\textbf{(A) }120\qquad\textbf{(B) }270\qquad\textbf{(C) }360\qquad\textbf{(D) }540\qquad\textbf{(E) }720\)\par \vspace{0.5em}\item The infinite product

\begin{equation*}
\sqrt[3]{10} \cdot \sqrt[3]{\sqrt[3]{10}} \cdot \sqrt[3]{\sqrt[3]{\sqrt[3]{10}}} \cdots
\end{equation*}

evaluates to a real number. What is that number?

\(\textbf{(A) }\sqrt{10}\qquad\textbf{(B) }\sqrt[3]{100}\qquad\textbf{(C) }\sqrt[4]{1000}\qquad\textbf{(D) }10\qquad\textbf{(E) }10\sqrt[3]{10}\)\par \vspace{0.5em}\item On Halloween \(31\) children walked into the principal's office asking for candy. They
can be classified into three types: Some always lie; some always tell the truth; and
some alternately lie and tell the truth. The alternaters arbitrarily choose their first
response, either a lie or the truth, but each subsequent statement has the opposite
truth value from its predecessor. The principal asked everyone the same three
questions in this order.

"Are you a truth-teller?" The principal gave a piece of candy to each of the \(22\)
children who answered yes.

"Are you an alternater?" The principal gave a piece of candy to each of the \(15\)
children who answered yes.

"Are you a liar?" The principal gave a piece of candy to each of the \(9\) children who
answered yes.

How many pieces of candy in all did the principal give to the children who always
tell the truth?

\(\textbf{(A) } 7 \qquad \textbf{(B) } 12 \qquad \textbf{(C) } 21 \qquad \textbf{(D) } 27 \qquad \textbf{(E) } 31\)\par \vspace{0.5em}\item How many ways are there to split the integers \(1\) through \(14\) into \(7\) pairs such that in each pair, the greater number is at least \(2\) times the lesser number?

\(\textbf{(A) } 108 \qquad \textbf{(B) } 120 \qquad \textbf{(C) } 126 \qquad \textbf{(D) } 132 \qquad \textbf{(E) } 144\)\par \vspace{0.5em}\item What is the product of all real numbers \(x\) such that the distance on the number line between \(\log_6x\) and \(\log_69\) is twice the distance on the number line between \(\log_610\) and \(1\)?

\(\textbf{(A) } 10 \qquad \textbf{(B) } 18 \qquad \textbf{(C) } 25 \qquad \textbf{(D) } 36 \qquad \textbf{(E) } 81\)\par \vspace{0.5em}\item Let \(M\) be the midpoint of \(\overline{AB}\) in regular tetrahedron \(ABCD\). What is \(\cos(\angle CMD)\)?

\(\textbf{(A) } \frac14 \qquad \textbf{(B) } \frac13 \qquad \textbf{(C) } \frac25 \qquad \textbf{(D) } \frac12 \qquad \textbf{(E) } \frac{\sqrt{3}}{2}\)\par \vspace{0.5em}\item Let \(\mathcal{R}\) be the region in the complex plane consisting of all complex numbers \(z\) that can be written as the sum of complex numbers \(z_1\) and \(z_2\), where \(z_1\) lies on the segment with endpoints \(3\) and \(4i\), and \(z_2\) has magnitude at most \(1\). What integer is closest to the area of \(\mathcal{R}\)?  

\(\textbf{(A) } 13 \qquad \textbf{(B) } 14 \qquad \textbf{(C) } 15 \qquad \textbf{(D) } 16 \qquad \textbf{(E) } 17\)\par \vspace{0.5em}\item What is the value of 
\begin{equation*}
(\log 5)^3+(\log 20)^3+(\log 8)(\log 0.25)
\end{equation*}
 where \(\log\) denotes the base-ten logarithm?

\(\textbf{(A) } \frac{3}{2} \qquad \textbf{(B) } \frac{7}{4} \qquad \textbf{(C) } 2 \qquad \textbf{(D) } \frac{9}{4} \qquad \textbf{(E) } 3\)\par \vspace{0.5em}\item The roots of the polynomial \(10x^3-39x^2+29x-6\) are the height, length, and width of a rectangular box (right rectangular prism). A new rectangular box is formed by lengthening each edge of the original box by \(2\) units. What is the volume of the new box?

\(\textbf{(A) } \frac{24}{5} \qquad \textbf{(B) } \frac{42}{5} \qquad \textbf{(C) } \frac{81}{5} \qquad \textbf{(D) } 30 \qquad \textbf{(E) } 48\)\par \vspace{0.5em}\item A triangular number is a positive integer that can be expressed in the form \(t_n=1+2+3+\cdots+n\), for some positive integer \(n\). The three smallest triangular numbers that are also perfect squares are \(t_1=1=1^2, t_8=36=6^2,\) and \(t_{49}=1225=35^2\). What is the sum of the digits of the fourth smallest triangular number that is also a perfect square?

\(\textbf{(A)} ~6 \qquad\textbf{(B)} ~9 \qquad\textbf{(C)} ~12 \qquad\textbf{(D)} ~18 \qquad\textbf{(E)} ~27 \)\par \vspace{0.5em}\item Suppose \(a\) is a real number such that the equation 
\begin{equation*}
a\cdot(\sin{x}+\sin{(2x)}) = \sin{(3x)}
\end{equation*}

has more than one solution in the interval \((0, \pi)\). The set of all such \(a\) that can be written
in the form 
\begin{equation*}
(p,q) \cup (q,r),
\end{equation*}

where \(p, q,\) and \(r\) are real numbers with \(p < q< r\). What is \(p+q+r\)?

\(\textbf{(A) } {-}4 \qquad \textbf{(B) } {-}1 \qquad \textbf{(C) } 0 \qquad \textbf{(D) } 1 \qquad \textbf{(E) } 4\)\par \vspace{0.5em}\item Let \(T_k\) be the transformation of the coordinate plane that first rotates the plane \(k\) degrees counterclockwise around the origin and then reflects the plane across the \(y\)-axis. What is the least positive integer \(n\) such that performing the sequence of transformations \(T_1, T_2, T_3, \dots, T_n\) returns the point \((1,0)\) back to itself?

\(\textbf{(A) } 359 \qquad \textbf{(B) } 360\qquad \textbf{(C) } 719 \qquad \textbf{(D) } 720 \qquad \textbf{(E) } 721\)\par \vspace{0.5em}\item Suppose that \(13\) cards numbered \(1, 2, 3, \ldots, 13\) are arranged in a row. The task is to pick them up in numerically increasing order, working repeatedly from left to right. In the example below, cards \(1, 2, 3\) are picked up on the first pass, \(4\) and \(5\) on the second pass, \(6\) on the third pass, \(7, 8, 9, 10\) on the fourth pass, and \(11, 12, 13\) on the fifth pass. For how many of the \(13!\) possible orderings of the cards will the \(13\) cards be picked up in exactly two passes?


\begin{center}
\begin{asy}
import olympiad;
import cse5;
size(11cm);
draw((0,0)--(2,0)--(2,3)--(0,3)--cycle);
label("7", (1,1.5));
draw((3,0)--(5,0)--(5,3)--(3,3)--cycle);
label("11", (4,1.5));
draw((6,0)--(8,0)--(8,3)--(6,3)--cycle);
label("8", (7,1.5));
draw((9,0)--(11,0)--(11,3)--(9,3)--cycle);
label("6", (10,1.5));
draw((12,0)--(14,0)--(14,3)--(12,3)--cycle);
label("4", (13,1.5));
draw((15,0)--(17,0)--(17,3)--(15,3)--cycle);
label("5", (16,1.5));
draw((18,0)--(20,0)--(20,3)--(18,3)--cycle);
label("9", (19,1.5));
draw((21,0)--(23,0)--(23,3)--(21,3)--cycle);
label("12", (22,1.5));
draw((24,0)--(26,0)--(26,3)--(24,3)--cycle);
label("1", (25,1.5));
draw((27,0)--(29,0)--(29,3)--(27,3)--cycle);
label("13", (28,1.5));
draw((30,0)--(32,0)--(32,3)--(30,3)--cycle);
label("10", (31,1.5));
draw((33,0)--(35,0)--(35,3)--(33,3)--cycle);
label("2", (34,1.5));
draw((36,0)--(38,0)--(38,3)--(36,3)--cycle);
label("3", (37,1.5));
\end{asy}
\end{center}

\(\textbf{(A) } 4082 \qquad \textbf{(B) } 4095 \qquad \textbf{(C) } 4096 \qquad \textbf{(D) } 8178 \qquad \textbf{(E) } 8191\)\par \vspace{0.5em}\item Isosceles trapezoid \(ABCD\) has parallel sides \(\overline{AD}\) and \(\overline{BC},\) with \(BC < AD\) and \(AB = CD.\) There is a point \(P\) in the plane such that \(PA=1, PB=2, PC=3,\) and \(PD=4.\) What is \(\tfrac{BC}{AD}?\)

\(\textbf{(A) }\frac{1}{4}\qquad\textbf{(B) }\frac{1}{3}\qquad\textbf{(C) }\frac{1}{2}\qquad\textbf{(D) }\frac{2}{3}\qquad\textbf{(E) }\frac{3}{4}\)\par \vspace{0.5em}\item Let 
\begin{equation*}
P(x) = x^{2022} + x^{1011} + 1.
\end{equation*}
 Which of the following polynomials is a factor of \(P(x)\)?

\(\textbf{(A)} \, x^2 -x + 1 \qquad\textbf{(B)} \, x^2 + x + 1 \qquad\textbf{(C)} \, x^4 + 1 \qquad\textbf{(D)} \, x^6 - x^3 + 1  \qquad\textbf{(E)} \, x^6 + x^3 + 1 \)\par \vspace{0.5em}\item Let \(c\) be a real number, and let \(z_1\) and \(z_2\) be the two complex numbers satisfying the equation
\(z^2 - cz + 10 = 0\). Points \(z_1\), \(z_2\), \(\frac{1}{z_1}\), and \(\frac{1}{z_2}\) are the vertices of (convex) quadrilateral \(\mathcal{Q}\) in the complex plane. When the area of \(\mathcal{Q}\) obtains its maximum possible value, \(c\) is closest to which of the following?

\(\textbf{(A) }4.5 \qquad\textbf{(B) }5 \qquad\textbf{(C) }5.5 \qquad\textbf{(D) }6\qquad\textbf{(E) }6.5\)\par \vspace{0.5em}\item Let \(h_n\) and \(k_n\) be the unique relatively prime positive integers such that 
\begin{equation*}
\frac{1}{1}+\frac{1}{2}+\frac{1}{3}+\cdots+\frac{1}{n}=\frac{h_n}{k_n}.
\end{equation*}
 Let \(L_n\) denote the least common multiple of the numbers \(1, 2, 3, \ldots, n\). For how many integers with \(1\le{n}\le{22}\) is \(k_n<L_n\)?

\(\textbf{(A) }0 \qquad\textbf{(B) }3 \qquad\textbf{(C) }7 \qquad\textbf{(D) }8\qquad\textbf{(E) }10\)\par \vspace{0.5em}\item How many strings of length \(5\) formed from the digits \(0\), \(1\), \(2\), \(3\), \(4\) are there such that for each \(j \in \{1,2,3,4\}\), at least \(j\) of the digits are less than \(j\)? (For example, \(02214\) satisfies this condition
because it contains at least \(1\) digit less than \(1\), at least \(2\) digits less than \(2\), at least \(3\) digits less
than \(3\), and at least \(4\) digits less than \(4\). The string \(23404\) does not satisfy the condition because it
does not contain at least \(2\) digits less than \(2\).)

\(\textbf{(A) }500\qquad\textbf{(B) }625\qquad\textbf{(C) }1089\qquad\textbf{(D) }1199\qquad\textbf{(E) }1296\)\par \vspace{0.5em}\item A circle with integer radius \(r\) is centered at \((r, r)\). Distinct line segments of length \(c_i\) connect points \((0, a_i)\) to \((b_i, 0)\) for \(1 \le i \le 14\) and are tangent to the circle, where \(a_i\), \(b_i\), and \(c_i\) are all positive integers and \(c_1 \le c_2 \le \cdots \le c_{14}\). What is the ratio \(\frac{c_{14}}{c_1}\) for the least possible value of \(r\)?

\(\textbf{(A)} ~\frac{21}{5} \qquad\textbf{(B)} ~\frac{85}{13} \qquad\textbf{(C)} ~7 \qquad\textbf{(D)} ~\frac{39}{5} \qquad\textbf{(E)} ~17 \)\par \vspace{0.5em}\end{enumerate}
\end{document}
