
\documentclass{article}
\usepackage{amsmath, amssymb}
\usepackage{geometry}
\geometry{a4paper, margin=0.75in}
\usepackage{enumitem}
\usepackage{hyperref}
\usepackage{fancyhdr}
\usepackage{tikz}
\usepackage{graphicx}
\usepackage{asymptote}
\usepackage{arcs}
\usepackage{xwatermark}
\begin{asydef}
  // Global Asymptote settings
  settings.outformat = "pdf";
  settings.render = 0;
  settings.prc = false;
  import olympiad;
  import cse5;
  size(8cm);
\end{asydef}
\pagestyle{fancy}
\fancyhead[L]{\textbf{AMC12 Problems}}
\fancyhead[R]{\textbf{2024}}
\fancyfoot[C]{\thepage}
\renewcommand{\headrulewidth}{0.4pt}
\renewcommand{\footrulewidth}{0.4pt}

\title{AMC12 Problems \\ 2024}
\date{}
\begin{document}\maketitle\thispagestyle{fancy}\newpage\section*{2020 AMC 12A}\begin{enumerate}[label=\arabic*., itemsep=0.5em]\item Carlos took \(70\%\) of a whole pie. Maria took one third of the remainder. What portion of the whole pie was left?

\(\textbf{(A)}\ 10\%\qquad\textbf{(B)}\ 15\%\qquad\textbf{(C)}\ 20\%\qquad\textbf{(D)}\ 30\%\qquad\textbf{(E)}\ 35\%\)\par \vspace{0.5em}\item The acronym AMC is shown in the rectangular grid below with grid lines spaced \(1\) unit apart. In units, what is the sum of the lengths of the line segments that form the acronym AMC\(?\)


\begin{center}
\begin{asy}
import olympiad;
import cse5;
import olympiad;
unitsize(25);
for (int i = 0; i < 3; ++i) {
for (int j = 0; j < 9; ++j) {
pair A = (j,i);

}
}
for (int i = 0; i < 3; ++i) {
for (int j = 0; j < 9; ++j) {
if (j != 8) {
draw((j,i)--(j+1,i), dashed);
}
if (i != 2) {
draw((j,i)--(j,i+1), dashed);
}
}
}
draw((0,0)--(2,2),linewidth(2));
draw((2,0)--(2,2),linewidth(2));
draw((1,1)--(2,1),linewidth(2));
draw((3,0)--(3,2),linewidth(2));
draw((5,0)--(5,2),linewidth(2));
draw((4,1)--(3,2),linewidth(2));
draw((4,1)--(5,2),linewidth(2));
draw((6,0)--(8,0),linewidth(2));
draw((6,2)--(8,2),linewidth(2));
draw((6,0)--(6,2),linewidth(2));
\end{asy}
\end{center}


\(\textbf{(A) } 17 \qquad \textbf{(B) } 15 + 2\sqrt{2} \qquad \textbf{(C) } 13 + 4\sqrt{2} \qquad \textbf{(D) } 11 + 6\sqrt{2} \qquad \textbf{(E) } 21\)\par \vspace{0.5em}\item A driver travels for \(2\) hours at \(60\) miles per hour, during which her car gets \(30\) miles per gallon of gasoline. She is paid \(\$0.50\) per mile, and her only expense is gasoline at \(\$2.00\) per gallon. What is her net rate of pay, in dollars per hour, after this expense?

\(\textbf{(A) }20 \qquad\textbf{(B) }22 \qquad\textbf{(C) }24 \qquad\textbf{(D) } 25\qquad\textbf{(E) } 26\)\par \vspace{0.5em}\item How many \(4\)-digit positive integers (that is, integers between \(1000\) and \(9999\), inclusive) having only even digits are divisible by \(5?\)

\(\textbf{(A) } 80 \qquad \textbf{(B) } 100 \qquad \textbf{(C) } 125 \qquad \textbf{(D) } 200 \qquad \textbf{(E) } 500\)\par \vspace{0.5em}\item The \(25\) integers from \(-10\) to \(14,\) inclusive, can be arranged to form a \(5\)-by-\(5\) square in which the sum of the numbers in each row, the sum of the numbers in each column, and the sum of the numbers along each of the main diagonals are all the same. What is the value of this common sum?

\(\textbf{(A) }2 \qquad\textbf{(B) } 5\qquad\textbf{(C) } 10\qquad\textbf{(D) } 25\qquad\textbf{(E) } 50\)\par \vspace{0.5em}\item In the plane figure shown below, \(3\) of the unit squares have been shaded. What is the least number of additional unit squares that must be shaded so that the resulting figure has two lines of symmetry\(?\)


\begin{center}
\begin{asy}
import olympiad;
import cse5;
import olympiad;
unitsize(25);
filldraw((1,3)--(1,4)--(2,4)--(2,3)--cycle, gray(0.7));
filldraw((2,1)--(2,2)--(3,2)--(3,1)--cycle, gray(0.7));
filldraw((4,0)--(5,0)--(5,1)--(4,1)--cycle, gray(0.7));
for (int i = 0; i < 5; ++i) {
for (int j = 0; j < 6; ++j) {
pair A = (j,i);

}
}
for (int i = 0; i < 5; ++i) {
for (int j = 0; j < 6; ++j) {
if (j != 5) {
draw((j,i)--(j+1,i));
}
if (i != 4) {
draw((j,i)--(j,i+1));
}
}
}
\end{asy}
\end{center}


\(\textbf{(A) } 4 \qquad \textbf{(B) } 5 \qquad \textbf{(C) } 6 \qquad \textbf{(D) } 7 \qquad \textbf{(E) } 8\)\par \vspace{0.5em}\item Seven cubes, whose volumes are \(1\), \(8\), \(27\), \(64\), \(125\), \(216\), and \(343\) cubic units, are stacked vertically to form a tower in which the volumes of the cubes decrease from bottom to top. Except for the bottom cube, the bottom face of each cube lies completely on top of the cube below it. What is the total surface area of the tower (including the bottom) in square units?

\(\textbf{(A) } 644    \qquad \textbf{(B) } 658   \qquad \textbf{(C) } 664   \qquad \textbf{(D) } 720   \qquad \textbf{(E) } 749\)\par \vspace{0.5em}\item What is the median of the following list of \(4040\) numbers\(?\)

\begin{equation*}
1, 2, 3, \ldots, 2020, 1^2, 2^2, 3^2, \ldots, 2020^2
\end{equation*}

\( \textbf{(A)}\ 1974.5\qquad\textbf{(B)}\ 1975.5\qquad\textbf{(C)}\ 1976.5\qquad\textbf{(D)}\ 1977.5\qquad\textbf{(E)}\ 1978.5 \)\par \vspace{0.5em}\item How many solutions does the equation \(\tan{(2x)} = \cos{(\tfrac{x}{2})}\) have on the interval \([0, 2\pi]?\)

\(\textbf{(A) } 1 \qquad \textbf{(B) } 2 \qquad \textbf{(C) } 3 \qquad \textbf{(D) } 4 \qquad \textbf{(E) } 5\)\par \vspace{0.5em}\item There is a unique positive integer \(n\) such that
\begin{equation*}
\log_2{(\log_{16}{n})} = \log_4{(\log_4{n})}.
\end{equation*}
What is the sum of the digits of \(n?\)

\(\textbf{(A) } 4 \qquad \textbf{(B) } 7 \qquad \textbf{(C) } 8 \qquad \textbf{(D) } 11 \qquad \textbf{(E) } 13\)\par \vspace{0.5em}\item A frog sitting at the point \((1, 2)\) begins a sequence of jumps, where each jump is parallel to one of the coordinate axes and has length \(1\), and the direction of each jump (up, down, right, or left) is chosen independently at random. The sequence ends when the frog reaches a side of the square with vertices \((0,0), (0,4), (4,4),\) and \((4,0)\). What is the probability that the sequence of jumps ends on a vertical side of the square\(?\)

\(\textbf{(A) } \frac{1}{2} \qquad \textbf{(B) } \frac{5}{8} \qquad \textbf{(C) } \frac{2}{3} \qquad \textbf{(D) } \frac{3}{4} \qquad \textbf{(E) } \frac{7}{8}\)\par \vspace{0.5em}\item Line \(\ell\) in the coordinate plane has the equation \(3x - 5y + 40 = 0\). This line is rotated \(45^{\circ}\) counterclockwise about the point \((20, 20)\) to obtain line \(k\). What is the \(x\)-coordinate of the \(x\)-intercept of line \(k?\)

\(\textbf{(A) } 10 \qquad \textbf{(B) } 15 \qquad \textbf{(C) } 20 \qquad \textbf{(D) } 25 \qquad \textbf{(E) } 30\)\par \vspace{0.5em}\item There are integers \(a\), \(b\), and \(c\), each greater than 1, such that
\begin{equation*}
\sqrt[a]{N \sqrt[b]{N \sqrt[c]{N}}} = \sqrt[36]{N^{25}}
\end{equation*}
for all \(N > 1\). What is \(b\)?

\(\textbf{(A)}\ 2\qquad\textbf{(B)}\ 3\qquad\textbf{(C)}\ 4\qquad\textbf{(D)}\ 5\qquad\textbf{(E)}\ 6\)\par \vspace{0.5em}\item Regular octagon \(ABCDEFGH\) has area \(n\). Let \(m\) be the area of quadrilateral \(ACEG\). What is \(\tfrac{m}{n}?\)

\(\textbf{(A) } \frac{\sqrt{2}}{4} \qquad \textbf{(B) } \frac{\sqrt{2}}{2} \qquad \textbf{(C) } \frac{3}{4} \qquad \textbf{(D) } \frac{3\sqrt{2}}{5} \qquad \textbf{(E) } \frac{2\sqrt{2}}{3}\)\par \vspace{0.5em}\item In the complex plane, let \(A\) be the set of solutions to \(z^3 - 8 = 0\) and let \(B\) be the set of solutions to \(z^3 - 8z^2 - 8z + 64 = 0\). What is the greatest distance between a point of \(A\) and a point of \(B?\)

\(\textbf{(A) } 2\sqrt{3} \qquad \textbf{(B) } 6 \qquad \textbf{(C) } 9 \qquad \textbf{(D) } 2\sqrt{21} \qquad \textbf{(E) } 9 + \sqrt{3}\)\par \vspace{0.5em}\item A point is chosen at random within the square in the coordinate plane whose vertices are \((0, 0), (2020, 0), (2020, 2020),\) and \((0, 2020)\). The probability that the point is within \(d\) units of a lattice point is \(\tfrac{1}{2}\). (A point \((x, y)\) is a lattice point if \(x\) and \(y\) are both integers.) What is \(d\) to the nearest tenth\(?\)

\(\textbf{(A) } 0.3 \qquad \textbf{(B) } 0.4 \qquad \textbf{(C) } 0.5 \qquad \textbf{(D) } 0.6 \qquad \textbf{(E) } 0.7\)\par \vspace{0.5em}\item The vertices of a quadrilateral lie on the graph of \(y = \ln x\), and the \(x\)-coordinates of these vertices are consecutive positive integers. The area of the quadrilateral is \(\ln \frac{91}{90}\). What is the \(x\)-coordinate of the leftmost vertex?

\(\textbf{(A)}\ 6\qquad\textbf{(B)}\ 7\qquad\textbf{(C)}\ 10\qquad\textbf{(D)}\ 12\qquad\textbf{(E)}\ 13\)\par \vspace{0.5em}\item Quadrilateral \(ABCD\) satisfies \(\angle ABC = \angle ACD = 90^{\circ}, AC = 20\), and \(CD = 30\). Diagonals \(\overline{AC}\) and \(\overline{BD}\) intersect at point \(E\), and \(AE = 5\). What is the area of quadrilateral \(ABCD\)?

\(\textbf{(A) } 330 \qquad\textbf{(B) } 340 \qquad\textbf{(C) } 350 \qquad\textbf{(D) } 360 \qquad\textbf{(E) } 370\)\par \vspace{0.5em}\item There exists a unique strictly increasing sequence of nonnegative integers \(a_1 < a_2 <  < a_k\) such that
\begin{equation*}
\frac{2^{289}+1}{2^{17}+1} = 2^{a_1} + 2^{a_2} +  + 2^{a_k}.
\end{equation*}
What is \(k?\)

\(\textbf{(A) } 117 \qquad \textbf{(B) } 136 \qquad \textbf{(C) } 137 \qquad \textbf{(D) } 273 \qquad \textbf{(E) } 306\)\par \vspace{0.5em}\item Let \(T\) be the triangle in the coordinate plane with vertices \(\left(0,0\right)\), \(\left(4,0\right)\), and \(\left(0,3\right)\). Consider the following five isometries (rigid transformations) of the plane: rotations of \(90^{\circ}\), \(180^{\circ}\), and \(270^{\circ}\) counterclockwise around the origin, reflection across the \(x\)-axis, and reflection across the \(y\)-axis. How many of the \(125\) sequences of three of these transformations (not necessarily distinct) will return \(T\) to its original position? (For example, a \(180^{\circ}\) rotation, followed by a reflection across the \(x\)-axis, followed by a reflection across the \(y\)-axis will return \(T\) to its original position, but a \(90^{\circ}\) rotation, followed by a reflection across the \(x\)-axis, followed by another reflection across the \(x\)-axis will not return \(T\) to its original position.)

\(\textbf{(A) } 12\qquad\textbf{(B) } 15\qquad\textbf{(C) }17 \qquad\textbf{(D) }20 \qquad\textbf{(E) }25\)\par \vspace{0.5em}\item How many positive integers \(n\) are there such that \(n\) is a multiple of \(5\), and the least common multiple of \(5!\) and \(n\) equals \(5\) times the greatest common divisor of \(10!\) and \(n?\)

\(\textbf{(A) } 12 \qquad \textbf{(B) } 24 \qquad \textbf{(C) } 36 \qquad \textbf{(D) } 48 \qquad \textbf{(E) } 72\)\par \vspace{0.5em}\item Let \((a_n)\) and \((b_n)\) be the sequences of real numbers such that

\begin{equation*}
(2 + i)^n = a_n + b_ni
\end{equation*}
for all integers \(n\geq 0\), where \(i = \sqrt{-1}\). What is
\begin{equation*}
\sum_{n=0}^\infty\frac{a_nb_n}{7^n}\,?
\end{equation*}

\(\textbf{(A) }\frac 38\qquad\textbf{(B) }\frac7{16}\qquad\textbf{(C) }\frac12\qquad\textbf{(D) }\frac9{16}\qquad\textbf{(E) }\frac47\)\par \vspace{0.5em}\item Jason rolls three fair standard six-sided dice. Then he looks at the rolls and chooses a subset of the dice (possibly empty, possibly all three dice) to reroll. After rerolling, he wins if and only if the sum of the numbers face up on the three dice is exactly \(7\). Jason always plays to optimize his chances of winning. What is the probability that he chooses to reroll exactly two of the dice?

\(\textbf{(A) } \frac{7}{36} \qquad\textbf{(B) } \frac{5}{24} \qquad\textbf{(C) } \frac{2}{9} \qquad\textbf{(D) } \frac{17}{72} \qquad\textbf{(E) } \frac{1}{4}\)\par \vspace{0.5em}\item Suppose that \(\triangle ABC\) is an equilateral triangle of side length \(s\), with the property that there is a unique point \(P\) inside the triangle such that \(AP = 1\), \(BP = \sqrt{3}\), and \(CP = 2\). What is \(s?\)

\(\textbf{(A) } 1 + \sqrt{2} \qquad \textbf{(B) } \sqrt{7} \qquad \textbf{(C) } \frac{8}{3} \qquad \textbf{(D) } \sqrt{5 + \sqrt{5}} \qquad \textbf{(E) } 2\sqrt{2}\)\par \vspace{0.5em}\item The number \(a = \tfrac{p}{q}\), where \(p\) and \(q\) are relatively prime positive integers, has the property that the sum of all real numbers \(x\) satisfying
\begin{equation*}
\lfloor x \rfloor \cdot \{x\} = a \cdot x^2
\end{equation*}
is \(420\), where \(\lfloor x \rfloor\) denotes the greatest integer less than or equal to \(x\) and \(\{x\} = x - \lfloor x \rfloor\) denotes the fractional part of \(x\). What is \(p + q?\)

\(\textbf{(A) } 245 \qquad \textbf{(B) } 593 \qquad \textbf{(C) } 929 \qquad \textbf{(D) } 1331 \qquad \textbf{(E) } 1332\)\par \vspace{0.5em}\end{enumerate}
\end{document}
