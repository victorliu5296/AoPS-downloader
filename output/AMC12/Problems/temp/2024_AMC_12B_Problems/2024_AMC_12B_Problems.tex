
\documentclass{article}
\usepackage{amsmath, amssymb}
\usepackage{geometry}
\geometry{a4paper, margin=0.75in}
\usepackage{enumitem}
\usepackage{hyperref}
\usepackage{fancyhdr}
\usepackage{tikz}
\usepackage{graphicx}
\usepackage{asymptote}
\usepackage{arcs}
\usepackage{xwatermark}
\begin{asydef}
  // Global Asymptote settings
  settings.outformat = "pdf";
  settings.render = 0;
  settings.prc = false;
  import olympiad;
  import cse5;
  size(8cm);
\end{asydef}
\pagestyle{fancy}
\fancyhead[L]{\textbf{AMC12 Problems}}
\fancyhead[R]{\textbf{2024}}
\fancyfoot[C]{\thepage}
\renewcommand{\headrulewidth}{0.4pt}
\renewcommand{\footrulewidth}{0.4pt}

\title{AMC12 Problems \\ 2024}
\date{}
\begin{document}\maketitle\thispagestyle{fancy}\newpage\section*{Problems}\begin{enumerate}[label=\arabic*., itemsep=0.5em]\item In a long line of people arranged left to right, the 1013th person from the left is also the 1010th person from the right. How many people are in the line?

\(\textbf{(A) } 2021 \qquad\textbf{(B) } 2022 \qquad\textbf{(C) } 2023 \qquad\textbf{(D) } 2024 \qquad\textbf{(E) } 2025\)\par \vspace{0.5em}\item What is \(10! - 7! \cdot 6!\)?

\(\textbf{(A) }-120 \qquad\textbf{(B) }0 \qquad\textbf{(C) }120 \qquad\textbf{(D) }600 \qquad\textbf{(E) }720 \qquad\)\par \vspace{0.5em}\item For how many integer values of \(x\) is \(|2x|\leq 7\pi?\)

\(\textbf{(A) }16 \qquad\textbf{(B) }17\qquad\textbf{(C) }19\qquad\textbf{(D) }20\qquad\textbf{(E) }21\)\par \vspace{0.5em}\item Balls numbered \(1,2,3,\ldots\) are deposited in \(5\) bins, labeled \(A,B,C,D,\) and \(E\), using the following procedure. Ball \(1\) is deposited in bin \(A\), and balls \(2\) and \(3\) are deposited in \(B\). The next three balls are deposited in bin \(C\), the next \(4\) in bin \(D\), and so on, cycling back to bin \(A\) after balls are deposited in bin \(E\). (For example, \(22,23,\ldots,28\) are deposited in bin \(B\) at step 7 of this process.) In which bin is ball \(2024\) deposited?

\(\textbf{(A) }A\qquad\textbf{(B) }B\qquad\textbf{(C) }C\qquad\textbf{(D) }D\qquad\textbf{(E) }E\)\par \vspace{0.5em}\item In the following expression, Melanie changed some of the plus signs to minus signs:
\begin{equation*}
1 + 3+5+7+\cdots+97+99
\end{equation*}
When the new expression was evaluated, it was negative. What is the least number of plus signs that Melanie could have changed to minus signs?

\(
\textbf{(A) }14 \qquad
\textbf{(B) }15 \qquad
\textbf{(C) }16 \qquad
\textbf{(D) }17 \qquad
\textbf{(E) }18 \qquad
\)\par \vspace{0.5em}\item The national debt of the United States is on track to reach \(5 \cdot 10^{13}\) dollars by \(2033\). How many digits does this number of dollars have when written as a numeral in base \(5\)? (The approximation of \(\log_{10} 5\) as \(0.7\) is sufficient for this problem.)

\(
\textbf{(A) }18 \qquad
\textbf{(B) }20 \qquad
\textbf{(C) }22 \qquad
\textbf{(D) }24 \qquad
\textbf{(E) }26 \qquad
\)\par \vspace{0.5em}\item In the figure below \(WXYZ\) is a rectangle with \(WX=4\) and \(WZ=8\). Point \(M\) lies \(\overline{XY}\), point \(A\) lies on \(\overline{YZ}\), and \(\angle WMA\) is a right angle. The areas of \(\triangle WXM\) and \(\triangle WAZ\) are equal. What is the area of \(\triangle WMA\)?


\begin{center}
\begin{asy}
import olympiad;
import cse5;
pair X = (0, 0);
pair W = (0, 4);
pair Y = (8, 0);
pair Z = (8, 4);
label("$X$", X, dir(180));
label("$W$", W, dir(180));
label("$Y$", Y, dir(0));
label("$Z$", Z, dir(0));

draw(W--X--Y--Z--cycle);
dot(X);
dot(Y);
dot(W);
dot(Z);
pair M = (2, 0);
pair A = (8, 3);
label("$A$", A, dir(0));
dot(M);
dot(A);
draw(W--M--A--cycle);
markscalefactor = 0.05;
draw(rightanglemark(W, M, A));
label("$M$", M, dir(-90));
\end{asy}
\end{center}


\(
\textbf{(A) }13 \qquad
\textbf{(B) }14 \qquad
\textbf{(C) }15 \qquad
\textbf{(D) }16 \qquad
\textbf{(E) }17 \qquad
\)\par \vspace{0.5em}\item What value of \(x\) satisfies
\begin{equation*}
\frac{\log_2x\cdot\log_3x}{\log_2x+\log_3x}=2?
\end{equation*}

\(
\textbf{(A) }25\qquad
\textbf{(B) }32\qquad
\textbf{(C) }36\qquad
\textbf{(D) }42\qquad
\textbf{(E) }48\qquad
\)\par \vspace{0.5em}\item A dartboard is the region \(B\) in the coordinate plane consisting of points \((x,y)\) such that \(|x| + |y| \le 8\) . A target \(T\) is the region where \((x^2 + y^2 - 25)^2 \le 49.\) A dart is thrown and lands at a random point in \(B\). The probability that the dart lands in \(T\) can be expressed as \(\frac{m}{n} \cdot \pi,\) where \(m\) and \(n\) are relatively prime positive integers. What is \(m + n?\)

\(
\textbf{(A) }39 \qquad
\textbf{(B) }71 \qquad
\textbf{(C) }73 \qquad
\textbf{(D) }75 \qquad
\textbf{(E) }135 \qquad
\)\par \vspace{0.5em}\item A list of 9 real numbers consists of \(1\), \(2.2 \), \(3.2 \), \(5.2 \), \(6.2 \), and \(7\), as well as \(x, y,z\) with \(x\leq y\leq z\). The range of the list is \(7\), and the mean and median are both positive integers. How many ordered triples \((x,y,z)\) are possible?

\(\textbf{(A) }1 \qquad\textbf{(B) }2 \qquad\textbf{(C) }3 \qquad\textbf{(D) }4 \qquad\textbf{(E) }\text{infinitely many}\qquad\)\par \vspace{0.5em}\item Let \(x_{n} = \sin^2(n^\circ)\). What is the mean of \(x_{1}, x_{2}, x_{3}, \cdots, x_{90}\)?

\(
\textbf{(A) }\frac{11}{45} \qquad
\textbf{(B) }\frac{22}{45} \qquad
\textbf{(C) }\frac{89}{180} \qquad
\textbf{(D) }\frac{1}{2} \qquad
\textbf{(E) }\frac{91}{180} \qquad
\)\par \vspace{0.5em}\item Suppose \(z\) is a complex number with positive imaginary part, with real part greater than \(1\), and with \(|z| = 2\). In the complex plane, the four points \(0\), \(z\), \(z^{2}\), and \(z^{3}\) are the vertices of a quadrilateral with area \(15\). What is the imaginary part of \(z\)?

\(\textbf{(A)}~\frac{3}{4}\qquad\textbf{(B)}~1\qquad\textbf{(C)}~\frac{4}{3}\qquad\textbf{(D)}~\frac{3}{2}\qquad\textbf{(E)}~\frac{5}{3}\)\par \vspace{0.5em}\item There are real numbers \(x,y,h\) and \(k\) that satisfy the system of equations
\begin{equation*}
x^2 + y^2 - 6x - 8y = h
\end{equation*}

\begin{equation*}
x^2 + y^2 - 10x + 4y = k
\end{equation*}
What is the minimum possible value of \(h+k\)?

\(
\textbf{(A) }-54 \qquad
\textbf{(B) }-46 \qquad
\textbf{(C) }-34 \qquad
\textbf{(D) }-16 \qquad
\textbf{(E) }16 \qquad
\)\par \vspace{0.5em}\item How many different remainders can result when the \(100\)th power of an integer is divided by \(125\)?

\(\textbf{(A) }1 \qquad\textbf{(B) }2 \qquad\textbf{(C) }5 \qquad\textbf{(D) }25 \qquad\textbf{(E) }125 \qquad\)\par \vspace{0.5em}\item A triangle in the coordinate plane has vertices \(A(\log_21,\log_22)\), \(B(\log_23,\log_24)\), and \(C(\log_27,\log_28)\). What is the area of \(\triangle ABC\)?

\(
\textbf{(A) }\log_2\frac{\sqrt3}7\qquad
\textbf{(B) }\log_2\frac3{\sqrt7}\qquad
\textbf{(C) }\log_2\frac7{\sqrt3}\qquad
\textbf{(D) }\log_2\frac{11}{\sqrt7}\qquad
\textbf{(E) }\log_2\frac{11}{\sqrt3}\qquad
\)\par \vspace{0.5em}\item A group of \(16\) people will be partitioned into \(4\) indistinguishable \(4\)-person committees. Each committee will have one chairperson and one secretary. The number of different ways to make these assignments can be written as \(3^{r}M\), where \(r\) and \(M\) are positive integers and \(M\) is not divisible by \(3\). What is \(r\)?

\(
\textbf{(A) }5 \qquad
\textbf{(B) }6 \qquad
\textbf{(C) }7 \qquad
\textbf{(D) }8 \qquad
\textbf{(E) }9 \qquad\)\par \vspace{0.5em}\item Integers \(a\) and \(b\) are randomly chosen without replacement from the set of integers with absolute value not exceeding \(10\). What is the probability that the polynomial \(x^3 + ax^2 + bx + 6\) has \(3\) distinct integer roots?

\(\textbf{(A) }\frac{1}{240} \qquad \textbf{(B) }\frac{1}{221} \qquad \textbf{(C) }\frac{1}{105} \qquad \textbf{(D) }\frac{1}{84} \qquad \textbf{(E) }\frac{1}{63}\)\par \vspace{0.5em}\item The Fibonacci numbers are defined by \(F_1=1,\) \(F_2=1,\) and \(F_n=F_{n-1}+F_{n-2}\) for \(n\geq 3.\) What is
\begin{equation*}
\dfrac{F_2}{F_1}+\dfrac{F_4}{F_2}+\dfrac{F_6}{F_3}+\cdots+\dfrac{F_{20}}{F_{10}}?
\end{equation*}

\(\textbf{(A) }318 \qquad\textbf{(B) }319\qquad\textbf{(C) }320\qquad\textbf{(D) }321\qquad\textbf{(E) }322\)\par \vspace{0.5em}\item Equilateral \(\triangle ABC\) with side length \(14\) is rotated about its center by angle \(\theta\), where \(0 < \theta < 60^{\circ}\), to form \(\triangle DEF\). See the figure. The area of hexagon \(ADBECF\) is \(91\sqrt{3}\). What is \(\tan\theta\)?

\begin{center}
\begin{asy}
import olympiad;
import cse5;
// Credit to shihan for this diagram.

defaultpen(fontsize(13)); size(200);
pair O=(0,0),A=dir(225),B=dir(-15),C=dir(105),D=rotate(38.21,O)*A,E=rotate(38.21,O)*B,F=rotate(38.21,O)*C;
draw(A--B--C--A,gray+0.4);draw(D--E--F--D,gray+0.4); draw(A--D--B--E--C--F--A,black+0.9); dot(O); dot("$A$",A,dir(A)); dot("$B$",B,dir(B)); dot("$C$",C,dir(C)); dot("$D$",D,dir(D)); dot("$E$",E,dir(E)); dot("$F$",F,dir(F));
\end{asy}
\end{center}


\(\textbf{(A)}~\frac{3}{4}\qquad\textbf{(B)}~\frac{5\sqrt{3}}{11}\qquad\textbf{(C)}~\frac{4}{5}\qquad\textbf{(D)}~\frac{11}{13}\qquad\textbf{(E)}~\frac{7\sqrt{3}}{13}\)\par \vspace{0.5em}\item Suppose \(A\), \(B\), and \(C\) are points in the plane with \(AB=40\) and \(AC=42\), and let \(x\) be the length of the line segment from \(A\) to the midpoint of \(\overline{BC}\). Define a function \(f\) by letting \(f(x)\) be the area of \(\triangle ABC\). Then the domain of \(f\) is an open interval \((p,q)\), and the maximum value \(r\) of \(f(x)\) occurs at \(x=s\). What is \(p+q+r+s\)?

\(
\textbf{(A) }909\qquad
\textbf{(B) }910\qquad
\textbf{(C) }911\qquad
\textbf{(D) }912\qquad
\textbf{(E) }913\qquad
\)\par \vspace{0.5em}\item The measures of the smallest angles of three different right triangles sum to \(90^\circ\). All three triangles have side lengths that are primitive Pythagorean triples. Two of them are \(3-4-5\) and \(5-12-13\). What is the perimeter of the third triangle?

\(
\textbf{(A) }40 \qquad
\textbf{(B) }126 \qquad
\textbf{(C) }154 \qquad
\textbf{(D) }176 \qquad
\textbf{(E) }208 \qquad
\)\par \vspace{0.5em}\item Let \(\triangle{ABC}\) be a triangle with integer side lengths and the property that \(\angle{B} = 2\angle{A}\). What is the least possible perimeter of such a triangle?

\(
\textbf{(A) }13 \qquad
\textbf{(B) }14 \qquad
\textbf{(C) }15 \qquad
\textbf{(D) }16 \qquad
\textbf{(E) }17 \qquad
\)\par \vspace{0.5em}\item A right pyramid has regular octagon \(ABCDEFGH\) with side length \(1\) as its base and apex \(V.\) Segments \(\overline{AV}\) and \(\overline{DV}\) are perpendicular. What is the square of the height of the pyramid?

\(
\textbf{(A) }1 \qquad
\textbf{(B) }\frac{1+\sqrt2}{2} \qquad
\textbf{(C) }\sqrt2 \qquad
\textbf{(D) }\frac32 \qquad
\textbf{(E) }\frac{2+\sqrt2}{3} \qquad
\)\par \vspace{0.5em}\item What is the number of ordered triples \((a,b,c)\) of positive integers, with \(a\le b\le c\le 9\), such that there exists a (non-degenerate) triangle \(\triangle ABC\) with an integer inradius for which \(a\), \(b\), and \(c\) are the lengths of the altitudes from \(A\) to \(\overline{BC}\), \(B\) to \(\overline{AC}\), and \(C\) to \(\overline{AB}\), respectively? (Recall that the inradius of a triangle is the radius of the largest possible circle that can be inscribed in the triangle.)

\(
\textbf{(A) }2\qquad
\textbf{(B) }3\qquad
\textbf{(C) }4\qquad
\textbf{(D) }5\qquad
\textbf{(E) }6\qquad
\)\par \vspace{0.5em}\item Pablo will decorate each of \(6\) identical white balls with either a striped or a dotted pattern, using either red or blue paint. He will decide on the color and pattern for each ball by flipping a fair coin for each of the \(12\) decisions he must make. After the paint dries, he will place the \(6\) balls in an urn. Frida will randomly select one ball from the urn and note its color and pattern. The events "the ball Frida selects is red" and "the ball Frida selects is striped" may or may not be independent, depending on the outcome of Pablo's coin flips. The probability that these two events are independent can be written as \(\frac mn,\) where \(m\) and \(n\) are relatively prime positive integers. What is \(m?\) (Recall that two events \(A\) and \(B\) are independent if \(P(A \text{ and }B) = P(A) \cdot P(B).\))

\(\textbf{(A) } 243 \qquad \textbf{(B) } 245 \qquad \textbf{(C) } 247 \qquad \textbf{(D) } 249\qquad \textbf{(E) } 251\)\par \vspace{0.5em}\end{enumerate}
\end{document}
