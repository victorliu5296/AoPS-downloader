
\documentclass{article}
\usepackage{amsmath, amssymb}
\usepackage{geometry}
\geometry{a4paper, margin=0.75in}
\usepackage{enumitem}
\usepackage{hyperref}
\usepackage{fancyhdr}
\usepackage{tikz}
\usepackage{graphicx}
\usepackage{asymptote}
\usepackage{arcs}
\usepackage{xwatermark}
\begin{asydef}
  // Global Asymptote settings
  settings.outformat = "pdf";
  settings.render = 0;
  settings.prc = false;
  import olympiad;
  import cse5;
  size(8cm);
\end{asydef}
\pagestyle{fancy}
\fancyhead[L]{\textbf{AMC12 Problems}}
\fancyhead[R]{\textbf{2024}}
\fancyfoot[C]{\thepage}
\renewcommand{\headrulewidth}{0.4pt}
\renewcommand{\footrulewidth}{0.4pt}

\title{AMC12 Problems \\ 2024}
\date{}
\begin{document}\maketitle\thispagestyle{fancy}\newpage\section*{2021 AMC 12A}\begin{enumerate}[label=\arabic*., itemsep=0.5em]\item What is the value of
\begin{equation*}
2^{1+2+3}-(2^1+2^2+2^3)?
\end{equation*}

\(\textbf{(A) }0 \qquad \textbf{(B) }50 \qquad \textbf{(C) }52 \qquad \textbf{(D) }54 \qquad \textbf{(E) }57\)\par \vspace{0.5em}\item Under what conditions is \(\sqrt{a^2+b^2}=a+b\) true, where \(a\) and \(b\) are real numbers?

\(\textbf{(A) }\) It is never true.

\(\textbf{(B) }\) It is true if and only if \(ab=0\).

\(\textbf{(C) }\) It is true if and only if \(a+b\ge 0\).

\(\textbf{(D) }\) It is true if and only if \(ab=0\) and \(a+b\ge 0\).

\(\textbf{(E) }\) It is always true.\par \vspace{0.5em}\item The sum of two natural numbers is \(17{,}402\). One of the two numbers is divisible by \(10\). If the units digit of that number is erased, the other number is obtained. What is the difference of these two numbers?

\(\textbf{(A)} ~10{,}272\qquad\textbf{(B)} ~11{,}700\qquad\textbf{(C)} ~13{,}362\qquad\textbf{(D)} ~14{,}238\qquad\textbf{(E)} ~15{,}426\)\par \vspace{0.5em}\item Tom has a collection of \(13\) snakes, \(4\) of which are purple and \(5\) of which are happy. He observes that

* all of his happy snakes can add,

* none of his purple snakes can subtract, and

* all of his snakes that can't subtract also can't add.

Which of these conclusions can be drawn about Tom's snakes?

\(\textbf{(A) }\) Purple snakes can add.

\(\textbf{(B) }\) Purple snakes are happy.

\(\textbf{(C) }\) Snakes that can add are purple.

\(\textbf{(D) }\) Happy snakes are not purple.

\(\textbf{(E) }\) Happy snakes can't subtract.\par \vspace{0.5em}\item When a student multiplied the number \(66\) by the repeating decimal, 

\begin{equation*}
\underline{1}.\underline{a} \ \underline{b} \ \underline{a} \ \underline{b}\ldots=\underline{1}.\overline{\underline{a} \ \underline{b}},
\end{equation*}
 
where \(a\) and \(b\) are digits, he did not notice the notation and just multiplied \(66\) times \(\underline{1}.\underline{a} \ \underline{b}.\) Later he found that his answer is \(0.5\) less than the correct answer. What is the \(2\)-digit number \(\underline{a} \ \underline{b}?\)

\(\textbf{(A) }15 \qquad \textbf{(B) }30 \qquad \textbf{(C) }45 \qquad \textbf{(D) }60 \qquad \textbf{(E) }75\)\par \vspace{0.5em}\item A deck of cards has only red cards and black cards. The probability of a randomly chosen card being red is \(\frac13\). When \(4\) black cards are added to the deck, the probability of choosing red becomes \(\frac14\). How many cards were in the deck originally?

\(\textbf{(A) }6 \qquad \textbf{(B) }9 \qquad \textbf{(C) }12 \qquad \textbf{(D) }15 \qquad \textbf{(E) }18\)\par \vspace{0.5em}\item What is the least possible value of \((xy-1)^2+(x+y)^2\) for all real numbers \(x\) and \(y?\)

\(\textbf{(A) }0 \qquad \textbf{(B) }\frac14 \qquad \textbf{(C) }\frac12 \qquad \textbf{(D) }1 \qquad \textbf{(E) }2\)\par \vspace{0.5em}\item A sequence of numbers is defined by \(D_0=0,D_1=0,D_2=1\) and \(D_n=D_{n-1}+D_{n-3}\) for \(n\ge 3\). What are the parities (evenness or oddness) of the triple of numbers \((D_{2021},D_{2022},D_{2023})\), where \(E\) denotes even and \(O\) denotes odd?

\(\textbf{(A) }(O,E,O) \qquad \textbf{(B) }(E,E,O) \qquad \textbf{(C) }(E,O,E) \qquad \textbf{(D) }(O,O,E) \qquad \textbf{(E) }(O,O,O)\)\par \vspace{0.5em}\item Which of the following is equivalent to
\begin{equation*}
(2+3)(2^2+3^2)(2^4+3^4)(2^8+3^8)(2^{16}+3^{16})(2^{32}+3^{32})(2^{64}+3^{64})?
\end{equation*}

\(\textbf{(A) }3^{127}+2^{127} \qquad \textbf{(B) }3^{127}+2^{127}+2\cdot 3^{63}+3\cdot 2^{63} \qquad \textbf{(C) }3^{128}-2^{128} \qquad \textbf{(D) }3^{128}+2^{128} \qquad \textbf{(E) }5^{127}\)\par \vspace{0.5em}\item Two right circular cones with vertices facing down as shown in the figure below contain the same amount of liquid. The radii of the tops of the liquid surfaces are \(3\) cm and \(6\) cm. Into each cone is dropped a spherical marble of radius \(1\) cm, which sinks to the bottom and is completely submerged without spilling any liquid. What is the ratio of the rise of the liquid level in the narrow cone to the rise of the liquid level in the wide cone?


\begin{center}
\begin{asy}
import olympiad;
import cse5;
size(350);
defaultpen(linewidth(0.8));
real h1 = 10, r = 3.1, s=0.75;
pair P = (r,h1), Q = (-r,h1), Pp = s * P, Qp = s * Q;
path e = ellipse((0,h1),r,0.9), ep = ellipse((0,h1*s),r*s,0.9);
draw(ellipse(origin,r*(s-0.1),0.8));
fill(ep,gray(0.8));
fill(origin--Pp--Qp--cycle,gray(0.8));
draw((-r,h1)--(0,0)--(r,h1)^^e);
draw(subpath(ep,0,reltime(ep,0.5)),linetype("4 4"));
draw(subpath(ep,reltime(ep,0.5),reltime(ep,1)));
draw(Qp--(0,Qp.y),Arrows(size=8));
draw(origin--(0,12),linetype("4 4"));
draw(origin--(r*(s-0.1),0));
label("$3$",(-0.9,h1*s),N,fontsize(10));

real h2 = 7.5, r = 6, s=0.6, d = 14;
pair P = (d+r-0.05,h2-0.15), Q = (d-r+0.05,h2-0.15), Pp = s * P + (1-s)*(d,0), Qp = s * Q + (1-s)*(d,0);
path e = ellipse((d,h2),r,1), ep = ellipse((d,h2*s+0.09),r*s,1);
draw(ellipse((d,0),r*(s-0.1),0.8));
fill(ep,gray(0.8));
fill((d,0)--Pp--Qp--cycle,gray(0.8));
draw(P--(d,0)--Q^^e);
draw(subpath(ep,0,reltime(ep,0.5)),linetype("4 4"));
draw(subpath(ep,reltime(ep,0.5),reltime(ep,1)));
draw(Qp--(d,Qp.y),Arrows(size=8));
draw((d,0)--(d,10),linetype("4 4"));
draw((d,0)--(d+r*(s-0.1),0));
label("$6$",(d-r/4,h2*s-0.06),N,fontsize(10));
\end{asy}
\end{center}


\(\textbf{(A) }1:1 \qquad \textbf{(B) }47:43 \qquad \textbf{(C) }2:1 \qquad \textbf{(D) }40:13 \qquad \textbf{(E) }4:1\)\par \vspace{0.5em}\item A laser is placed at the point \((3,5)\). The laser beam travels in a straight line. Larry wants the beam to hit and bounce off the \(y\)-axis, then hit and bounce off the \(x\)-axis, then hit the point \((7,5)\). What is the total distance the beam will travel along this path?

\(\textbf{(A) }2\sqrt{10} \qquad \textbf{(B) }5\sqrt2 \qquad \textbf{(C) }10\sqrt2 \qquad \textbf{(D) }15\sqrt2 \qquad \textbf{(E) }10\sqrt5\)\par \vspace{0.5em}\item All the roots of the polynomial \(z^6-10z^5+Az^4+Bz^3+Cz^2+Dz+16\) are positive integers, possibly repeated. What is the value of \(B\)?

\(\textbf{(A) }{-}88 \qquad \textbf{(B) }{-}80 \qquad \textbf{(C) }{-}64 \qquad \textbf{(D) }{-}41\qquad \textbf{(E) }{-}40\)\par \vspace{0.5em}\item Of the following complex numbers \(z\), which one has the property that \(z^5\) has the greatest real part?

\(\textbf{(A) }{-}2 \qquad \textbf{(B) }{-}\sqrt3+i \qquad \textbf{(C) }{-}\sqrt2+\sqrt2 i \qquad \textbf{(D) }{-}1+\sqrt3 i\qquad \textbf{(E) }2i\)\par \vspace{0.5em}\item What is the value of
\begin{equation*}
\left(\sum_{k=1}^{20} \log_{5^k} 3^{k^2}\right)\cdot\left(\sum_{k=1}^{100} \log_{9^k} 25^k\right)?
\end{equation*}

\(\textbf{(A) }21 \qquad \textbf{(B) }100\log_5 3 \qquad \textbf{(C) }200\log_3 5 \qquad \textbf{(D) }2{,}200\qquad \textbf{(E) }21{,}000\)\par \vspace{0.5em}\item A choir director must select a group of singers from among his \(6\) tenors and \(8\) basses. The only
requirements are that the difference between the numbers of tenors and basses must be a multiple
of \(4\), and the group must have at least one singer. Let \(N\) be the number of different groups that could be
selected. What is the remainder when \(N\) is divided by \(100\)?

\(\textbf{(A) } 47\qquad\textbf{(B) } 48\qquad\textbf{(C) } 83\qquad\textbf{(D) } 95\qquad\textbf{(E) } 96\qquad\)\par \vspace{0.5em}\item In the following list of numbers, the integer \(n\) appears \(n\) times in the list for \(1\le n \le 200\).
\begin{equation*}
1,2,2,3,3,3,4,4,4,\ldots,200,200,\ldots,200
\end{equation*}
What is the median of the numbers in this list?

\(\textbf{(A) }100.5 \qquad \textbf{(B) }134 \qquad \textbf{(C) }142 \qquad \textbf{(D) }150.5\qquad \textbf{(E) }167\)\par \vspace{0.5em}\item Trapezoid \(ABCD\) has \(\overline{AB}\parallel\overline{CD},BC=CD=43\), and \(\overline{AD}\perp\overline{BD}\). Let \(O\) be the intersection of the diagonals \(\overline{AC}\) and \(\overline{BD}\), and let \(P\) be the midpoint of \(\overline{BD}\). Given that \(OP=11\), the length of \(AD\) can be written in the form \(m\sqrt{n}\), where \(m\) and \(n\) are positive integers and \(n\) is not divisible by the square of any prime. What is \(m+n\)?

\(\textbf{(A) }65 \qquad \textbf{(B) }132 \qquad \textbf{(C) }157 \qquad \textbf{(D) }194\qquad \textbf{(E) }215\)\par \vspace{0.5em}\item Let \(f\) be a function defined on the set of positive rational numbers with the property that \(f(a\cdot b)=f(a)+f(b)\) for all positive rational numbers \(a\) and \(b\). Suppose that \(f\) also has the property that \(f(p)=p\) for every prime number \(p\). For which of the following numbers \(x\) is \(f(x)<0\)?

\(\textbf{(A) }\frac{17}{32} \qquad \textbf{(B) }\frac{11}{16} \qquad \textbf{(C) }\frac79 \qquad \textbf{(D) }\frac76\qquad \textbf{(E) }\frac{25}{11}\)\par \vspace{0.5em}\item How many solutions does the equation \(\sin \left( \frac{\pi}2 \cos x\right)=\cos \left( \frac{\pi}2 \sin x\right)\) have in the closed interval \([0,\pi]\)?

\(\textbf{(A) }0 \qquad \textbf{(B) }1 \qquad \textbf{(C) }2 \qquad \textbf{(D) }3\qquad \textbf{(E) }4\)\par \vspace{0.5em}\item Suppose that on a parabola with vertex \(V\) and a focus \(F\) there exists a point \(A\) such that \(AF=20\) and \(AV=21\). What is the sum of all possible values of the length \(FV?\)

\(\textbf{(A) }13 \qquad \textbf{(B) }\frac{40}3 \qquad \textbf{(C) }\frac{41}3 \qquad \textbf{(D) }14\qquad \textbf{(E) }\frac{43}3\)\par \vspace{0.5em}\item The five solutions to the equation
\begin{equation*}
(z-1)(z^2+2z+4)(z^2+4z+6)=0
\end{equation*}
 may be written in the form \(x_k+y_ki\) for \(1\le k\le 5,\) where \(x_k\) and \(y_k\) are real. Let \(\mathcal E\) be the unique ellipse that passes through the points \((x_1,y_1),(x_2,y_2),(x_3,y_3),(x_4,y_4),\) and \((x_5,y_5)\). The eccentricity of \(\mathcal E\) can be written in the form \(\sqrt{\frac mn}\), where \(m\) and \(n\) are relatively prime positive integers. What is \(m+n\)? (Recall that the eccentricity of an ellipse \(\mathcal E\) is the ratio \(\frac ca\), where \(2a\) is the length of the major axis of \(\mathcal E\) and \(2c\) is the is the distance between its two foci.)

\(\textbf{(A) }7 \qquad \textbf{(B) }9 \qquad \textbf{(C) }11 \qquad \textbf{(D) }13\qquad \textbf{(E) }15\)\par \vspace{0.5em}\item Suppose that the roots of the polynomial \(P(x)=x^3+ax^2+bx+c\) are \(\cos \frac{2\pi}7,\cos \frac{4\pi}7,\) and \(\cos \frac{6\pi}7\), where angles are in radians. What is \(abc\)?

\(\textbf{(A) }{-}\frac{3}{49} \qquad \textbf{(B) }{-}\frac{1}{28} \qquad \textbf{(C) }\frac{\sqrt[3]7}{64} \qquad \textbf{(D) }\frac{1}{32}\qquad \textbf{(E) }\frac{1}{28}\)\par \vspace{0.5em}\item Frieda the frog begins a sequence of hops on a \(3\times3\) grid of squares, moving one square on each hop and choosing at random the direction of each hop up, down, left, or right. She does not hop diagonally. When the direction of a hop would take Frieda off the grid, she "wraps around" and jumps to the opposite edge. For example if Frieda begins in the center square and makes two hops "up", the first hop would place her in the top row middle square, and the second hop would cause Frieda to jump to the opposite edge, landing in the bottom row middle square. Suppose Frieda starts from the center square, makes at most four hops at random, and stops hopping if she lands on a corner square. What is the probability that she reaches a corner square on one of the four hops?

\(\textbf{(A) }\frac{9}{16} \qquad \textbf{(B) }\frac{5}{8} \qquad \textbf{(C) }\frac34 \qquad \textbf{(D) }\frac{25}{32}\qquad \textbf{(E) }\frac{13}{16}\)\par \vspace{0.5em}\item Semicircle \(\Gamma\) has diameter \(\overline{AB}\) of length \(14\). Circle \(\Omega\) lies tangent to \(\overline{AB}\) at a point \(P\) and intersects \(\Gamma\) at points \(Q\) and \(R\). If \(QR=3\sqrt3\) and \(\angle QPR=60^\circ\), then the area of \(\triangle PQR\) equals \(\tfrac{a\sqrt{b}}{c}\), where \(a\) and \(c\) are relatively prime positive integers, and \(b\) is a positive integer not divisible by the square of any prime. What is \(a+b+c\)?

\(\textbf{(A) }110 \qquad \textbf{(B) }114 \qquad \textbf{(C) }118 \qquad \textbf{(D) }122\qquad \textbf{(E) }126\)\par \vspace{0.5em}\item Let \(d(n)\) denote the number of positive integers that divide \(n\), including \(1\) and \(n\). For example, \(d(1)=1,d(2)=2,\) and \(d(12)=6\). (This function is known as the divisor function.) Let
\begin{equation*}
f(n)=\frac{d(n)}{\sqrt [3]n}.
\end{equation*}
There is a unique positive integer \(N\) such that \(f(N)>f(n)\) for all positive integers \(n\ne N\). What is the sum of the digits of \(N?\)

\(\textbf{(A) }5 \qquad \textbf{(B) }6 \qquad \textbf{(C) }7 \qquad \textbf{(D) }8\qquad \textbf{(E) }9\)\par \vspace{0.5em}\end{enumerate}
\end{document}
