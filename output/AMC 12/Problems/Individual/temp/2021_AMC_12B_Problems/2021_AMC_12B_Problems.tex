
\documentclass{article}
\usepackage{amsmath, amssymb}
\usepackage{geometry}
\geometry{a4paper, margin=0.75in}
\usepackage{enumitem}
\usepackage[hypertexnames=true, linktoc=all]{hyperref}
\usepackage{fancyhdr}
\usepackage{tikz}
\usepackage{graphicx}
\usepackage{asymptote}
\usepackage{arcs}
\usepackage{xwatermark}
\begin{asydef}
  // Global Asymptote settings
  settings.outformat = "pdf";
  settings.render = 0;
  settings.prc = false;
  import olympiad;
  import cse5;
  size(8cm);
\end{asydef}
\pagestyle{fancy}
\fancyhead[L]{\textbf{AMC 12 Problems}}
\fancyhead[R]{\textbf{2021}}
\fancyfoot[C]{\thepage}
\renewcommand{\headrulewidth}{0.4pt}
\renewcommand{\footrulewidth}{0.4pt}

\title{AMC 12 Problems \\ 2021}
\date{}
\begin{document}\maketitle\thispagestyle{fancy}\newpage\section*{2021 AMC 12B}
\begin{enumerate}[label=\arabic*., itemsep=0.5em]
\item How many integer values of \(x\) satisfy \(|x|<3\pi?\)

\(\textbf{(A) }9 \qquad \textbf{(B) }10 \qquad \textbf{(C) }18 \qquad \textbf{(D) }19 \qquad \textbf{(E) }20\)\par \vspace{0.5em}\item At a math contest, \(57\) students are wearing blue shirts, and another \(75\) students are wearing yellow shirts. The \(132\) students are assigned into \(66\) pairs. In exactly \(23\) of these pairs, both students are wearing blue shirts. In how many pairs are both students wearing yellow shirts?

\(\textbf{(A) }23 \qquad \textbf{(B) }32 \qquad \textbf{(C) }37 \qquad \textbf{(D) }41 \qquad \textbf{(E) }64\)\par \vspace{0.5em}\item Suppose
\begin{equation*}
2+\frac{1}{1+\frac{1}{2+\frac{2}{3+x}}}=\frac{144}{53}.
\end{equation*}
What is the value of \(x?\)

\(\textbf{(A) }\frac34 \qquad \textbf{(B) }\frac78 \qquad \textbf{(C) }\frac{14}{15} \qquad \textbf{(D) }\frac{37}{38} \qquad \textbf{(E) }\frac{52}{53}\)\par \vspace{0.5em}\item Ms. Blackwell gives an exam to two classes. The mean of the scores of the students in the morning class is \(84\), and the afternoon class's mean score is \(70\). The ratio of the number of students in the morning class to the number of students in the afternoon class is \(\frac34\). What is the mean of the score of all the students?

\(\textbf{(A) }74 \qquad \textbf{(B) }75 \qquad \textbf{(C) }76 \qquad \textbf{(D) }77 \qquad \textbf{(E) }78\)\par \vspace{0.5em}\item The point \(P(a,b)\) in the \(xy\)-plane is first rotated counterclockwise by \(90^\circ\) around the point \((1,5)\) and then reflected about the line \(y=-x\). The image of \(P\) after these two transformations is at \((-6,3)\). What is \(b-a?\)

\(\textbf{(A) }1 \qquad \textbf{(B) }3 \qquad \textbf{(C) }5 \qquad \textbf{(D) }7 \qquad \textbf{(E) }9\)\par \vspace{0.5em}\item An inverted cone with base radius \(12 \text{cm}\) and height \(18\text{cm}\) is full of water. The water is poured into a tall cylinder whose horizontal base has a radius of \(24\text{cm}\). What is the height in centimeters of the water in the cylinder?

\(\textbf{(A) }1.5 \qquad \textbf{(B) }3 \qquad \textbf{(C) }4 \qquad \textbf{(D) }4.5 \qquad \textbf{(E) }6\)\par \vspace{0.5em}\item Let \(N=34\cdot34\cdot63\cdot270.\) What is the ratio of the sum of the odd divisors of \(N\) to the sum of the even divisors of \(N?\)

\(\textbf{(A) }1:16 \qquad \textbf{(B) }1:15 \qquad \textbf{(C) }1:14 \qquad \textbf{(D) }1:8 \qquad \textbf{(E) }1:3\)\par \vspace{0.5em}\item Three equally spaced parallel lines intersect a circle, creating three chords of lengths \(38,38,\) and \(34\). What is the distance between two adjacent parallel lines?

\(\textbf{(A) }5\frac12 \qquad \textbf{(B) }6 \qquad \textbf{(C) }6\frac12 \qquad \textbf{(D) }7 \qquad \textbf{(E) }7\frac12\)\par \vspace{0.5em}\item What is the value of
\begin{equation*}
\frac{\log_2 80}{\log_{40}2}-\frac{\log_2 160}{\log_{20}2}?
\end{equation*}

\(\textbf{(A) }0 \qquad \textbf{(B) }1 \qquad \textbf{(C) }\frac54 \qquad \textbf{(D) }2 \qquad \textbf{(E) }\log_2 5\)\par \vspace{0.5em}\item Two distinct numbers are selected from the set \(\{1,2,3,4,\dots,36,37\}\) so that the sum of the remaining \(35\) numbers is the product of these two numbers. What is the difference of these two numbers?

\(\textbf{(A) }5 \qquad \textbf{(B) }7 \qquad \textbf{(C) }8\qquad \textbf{(D) }9 \qquad \textbf{(E) }10\)\par \vspace{0.5em}\item Triangle \(ABC\) has \(AB=13,BC=14\) and \(AC=15\). Let \(P\) be the point on \(\overline{AC}\) such that \(PC=10\). There are exactly two points \(D\) and \(E\) on line \(BP\) such that quadrilaterals \(ABCD\) and \(ABCE\) are trapezoids. What is the distance \(DE?\)

\(\textbf{(A) }\frac{42}5 \qquad \textbf{(B) }6\sqrt2 \qquad \textbf{(C) }\frac{84}5\qquad \textbf{(D) }12\sqrt2 \qquad \textbf{(E) }18\)\par \vspace{0.5em}\item Suppose that \(S\) is a finite set of positive integers. If the greatest integer in \(S\) is removed from \(S\), then the average value (arithmetic mean) of the integers remaining is \(32\). If the least integer in \(S\) is also removed, then the average value of the integers remaining is \(35\). If the greatest integer is then returned to the set, the average value of the integers rises to \(40\). The greatest integer in the original set \(S\) is \(72\) greater than the least integer in \(S\). What is the average value of all the integers in the set \(S\)?

\(\textbf{(A) }36.2 \qquad \textbf{(B) }36.4 \qquad \textbf{(C) }36.6\qquad \textbf{(D) }36.8 \qquad \textbf{(E) }37\)\par \vspace{0.5em}\item How many values of \(\theta\) in the interval \(0<\theta\le 2\pi\) satisfy
\begin{equation*}
1-3\sin\theta+5\cos3\theta = 0?
\end{equation*}

\(\textbf{(A) }2 \qquad \textbf{(B) }4 \qquad \textbf{(C) }5\qquad \textbf{(D) }6 \qquad \textbf{(E) }8\)\par \vspace{0.5em}\item Let \(ABCD\) be a rectangle and let \(\overline{DM}\) be a segment perpendicular to the plane of \(ABCD\). Suppose that \(\overline{DM}\) has integer length, and the lengths of \(\overline{MA},\overline{MC},\) and \(\overline{MB}\) are consecutive odd positive integers (in this order). What is the volume of pyramid \(MABCD?\)

\(\textbf{(A) }24\sqrt5 \qquad \textbf{(B) }60 \qquad \textbf{(C) }28\sqrt5\qquad \textbf{(D) }66 \qquad \textbf{(E) }8\sqrt{70}\)\par \vspace{0.5em}\item The figure is constructed from \(11\) line segments, each of which has length \(2\). The area of pentagon \(ABCDE\) can be written as \(\sqrt{m} + \sqrt{n}\), where \(m\) and \(n\) are positive integers. What is \(m + n ?\)

\begin{center}
\begin{asy}
import olympiad;
import cse5;
/* Made by samrocksnature */ pair A=(-2.4638,4.10658); pair B=(-4,2.6567453480756127); pair C=(-3.47132,0.6335248637894945); pair D=(-1.464483379039766,0.6335248637894945); pair E=(-0.956630463955801,2.6567453480756127); pair F=(-2,2); pair G=(-3,2); draw(A--B--C--D--E--A); draw(A--F--A--G); draw(B--F--C); draw(E--G--D); label("A",A,N); label("B",B,W); label("C",C,W); label("D",D,dir(0)); label("E",E,dir(0)); dot(A^^B^^C^^D^^E^^F^^G);
\end{asy}
\end{center}

\(\textbf{(A)} ~20 \qquad\textbf{(B)} ~21 \qquad\textbf{(C)} ~22 \qquad\textbf{(D)} ~23 \qquad\textbf{(E)} ~24\)\par \vspace{0.5em}\item Let \(g(x)\) be a polynomial with leading coefficient \(1,\) whose three roots are the reciprocals of the three roots of \(f(x)=x^3+ax^2+bx+c,\) where \(1<a<b<c.\) What is \(g(1)\) in terms of \(a,b,\) and \(c?\)

\(\textbf{(A) }\frac{1+a+b+c}c \qquad \textbf{(B) }1+a+b+c \qquad \textbf{(C) }\frac{1+a+b+c}{c^2}\qquad \textbf{(D) }\frac{a+b+c}{c^2} \qquad \textbf{(E) }\frac{1+a+b+c}{a+b+c}\)\par \vspace{0.5em}\item Let \(ABCD\) be an isosceles trapezoid having parallel bases \(\overline{AB}\) and \(\overline{CD}\) with \(AB>CD.\) Line segments from a point inside \(ABCD\) to the vertices divide the trapezoid into four triangles whose areas are \(2, 3, 4,\) and \(5\) starting with the triangle with base \(\overline{CD}\) and moving clockwise as shown in the diagram below. What is the ratio \(\frac{AB}{CD}?\)

\begin{center}
\begin{asy}
import olympiad;
import cse5;
unitsize(100);
pair A=(-1, 0), B=(1, 0), C=(0.3, 0.9), D=(-0.3, 0.9), P=(0.2, 0.5), E=(0.1, 0.75), F=(0.4, 0.5), G=(0.15, 0.2), H=(-0.3, 0.5); 
draw(A--B--C--D--cycle, black); 
draw(A--P, black);
draw(B--P, black);
draw(C--P, black);
draw(D--P, black);
label("$A$",A,(-1,0));
label("$B$",B,(1,0));
label("$C$",C,(1,-0));
label("$D$",D,(-1,0));
label("$2$",E,(0,0));
label("$3$",F,(0,0));
label("$4$",G,(0,0));
label("$5$",H,(0,0));
dot(A^^B^^C^^D^^P);
\end{asy}
\end{center}

\(\textbf{(A)}\: 3\qquad\textbf{(B)}\: 2+\sqrt{2}\qquad\textbf{(C)}\: 1+\sqrt{6}\qquad\textbf{(D)}\: 2\sqrt{3}\qquad\textbf{(E)}\: 3\sqrt{2}\)\par \vspace{0.5em}\item Let \(z\) be a complex number satisfying \(12|z|^2=2|z+2|^2+|z^2+1|^2+31.\) What is the value of \(z+\frac 6z?\)

\(\textbf{(A) }-2 \qquad \textbf{(B) }-1 \qquad \textbf{(C) }\frac12\qquad \textbf{(D) }1 \qquad \textbf{(E) }4\)\par \vspace{0.5em}\item Two fair dice, each with at least \(6\) faces are rolled. On each face of each die is printed a distinct integer from \(1\) to the number of faces on that die, inclusive. The probability of rolling a sum of \(7\) is \(\frac34\) of the probability of rolling a sum of \(10,\) and the probability of rolling a sum of \(12\) is \(\frac{1}{12}\). What is the least possible number of faces on the two dice combined?

\(\textbf{(A) }16 \qquad \textbf{(B) }17 \qquad \textbf{(C) }18\qquad \textbf{(D) }19 \qquad \textbf{(E) }20\)\par \vspace{0.5em}\item Let \(Q(z)\) and \(R(z)\) be the unique polynomials such that
\begin{equation*}
z^{2021}+1=(z^2+z+1)Q(z)+R(z)
\end{equation*}
and the degree of \(R\) is less than \(2.\) What is \(R(z)?\)

\(\textbf{(A) }{-}z \qquad \textbf{(B) }{-}1 \qquad \textbf{(C) }2021\qquad \textbf{(D) }z+1 \qquad \textbf{(E) }2z+1\)\par \vspace{0.5em}\item Let \(S\) be the sum of all positive real numbers \(x\) for which
\begin{equation*}
x^{2^{\sqrt2}}=\sqrt2^{2^x}.
\end{equation*}
Which of the following statements is true?

\(\textbf{(A) }S<\sqrt2 \qquad \textbf{(B) }S=\sqrt2 \qquad \textbf{(C) }\sqrt2<S<2\qquad \textbf{(D) }2\le S<6 \qquad \textbf{(E) }S\ge 6\)\par \vspace{0.5em}\item Arjun and Beth play a game in which they take turns removing one brick or two adjacent bricks from one "wall" among a set of several walls of bricks, with gaps possibly creating new walls. The walls are one brick tall. For example, a set of walls of sizes \(4\) and \(2\) can be changed into any of the following by one move: \((3,2),(2,1,2),(4),(4,1),(2,2),\) or \((1,1,2).\)


\begin{center}
\begin{asy}
import olympiad;
import cse5;
unitsize(4mm); real[] boxes = {0,1,2,3,5,6,13,14,15,17,18,21,22,24,26,27,30,31,32,33}; for(real i:boxes){ 	draw(box((i,0),(i+1,3))); } draw((8,1.5)--(12,1.5),Arrow()); defaultpen(fontsize(20pt)); label(",",(20,0)); label(",",(29,0)); label(",...",(35.5,0));
\end{asy}
\end{center}


Arjun plays first, and the player who removes the last brick wins. For which starting configuration is there a strategy that guarantees a win for Beth?

\(\textbf{(A) }(6,1,1) \qquad \textbf{(B) }(6,2,1) \qquad \textbf{(C) }(6,2,2)\qquad \textbf{(D) }(6,3,1) \qquad \textbf{(E) }(6,3,2)\)\par \vspace{0.5em}\item Three balls are randomly and independently tossed into bins numbered with the positive integers so that for each ball, the probability that it is tossed into bin \(i\) is \(2^{-i}\) for \(i=1,2,3,....\) More than one ball is allowed in each bin. The probability that the balls end up evenly spaced in distinct bins is \(\frac pq,\) where \(p\) and \(q\) are relatively prime positive integers. (For example, the balls are evenly spaced if they are tossed into bins \(3,17,\) and \(10.\)) What is \(p+q?\)

\(\textbf{(A) }55 \qquad \textbf{(B) }56 \qquad \textbf{(C) }57\qquad \textbf{(D) }58 \qquad \textbf{(E) }59\)\par \vspace{0.5em}\item Let \(ABCD\) be a parallelogram with area \(15\). Points \(P\) and \(Q\) are the projections of \(A\) and \(C,\) respectively, onto the line \(BD;\) and points \(R\) and \(S\) are the projections of \(B\) and \(D,\) respectively, onto the line \(AC.\) See the figure, which also shows the relative locations of these points.


\begin{center}
\begin{asy}
import olympiad;
import cse5;
size(350);
defaultpen(linewidth(0.8)+fontsize(11));
real theta = aTan(1.25/2);
pair A = 2.5*dir(180+theta), B = (3.35,0), C = -A, D = -B, P = foot(A,B,D), Q = -P, R = foot(B,A,C), S = -R;
draw(A--B--C--D--A^^B--D^^R--S^^rightanglemark(A,P,D,6)^^rightanglemark(C,Q,D,6));
draw(B--R^^C--Q^^A--P^^D--S,linetype("4 4"));
dot("$A$",A,dir(270));
dot("$B$",B,E);
dot("$C$",C,N);
dot("$D$",D,W);
dot("$P$",P,SE);
dot("$Q$",Q,NE);
dot("$R$",R,N);
dot("$S$",S,dir(270));
\end{asy}
\end{center}


Suppose \(PQ=6\) and \(RS=8,\) and let \(d\) denote the length of \(\overline{BD},\) the longer diagonal of \(ABCD.\) Then \(d^2\) can be written in the form \(m+n\sqrt p,\) where \(m,n,\) and \(p\) are positive integers and \(p\) is not divisible by the square of any prime. What is \(m+n+p?\)

\(\textbf{(A) }81 \qquad \textbf{(B) }89 \qquad \textbf{(C) }97\qquad \textbf{(D) }105 \qquad \textbf{(E) }113\)\par \vspace{0.5em}\item Let \(S\) be the set of lattice points in the coordinate plane, both of whose coordinates are integers between \(1\) and \(30,\) inclusive. Exactly \(300\) points in \(S\) lie on or below a line with equation \(y=mx.\) The possible values of \(m\) lie in an interval of length \(\frac ab,\) where \(a\) and \(b\) are relatively prime positive integers. What is \(a+b?\)

\(\textbf{(A) }31 \qquad \textbf{(B) }47 \qquad \textbf{(C) }62\qquad \textbf{(D) }72 \qquad \textbf{(E) }85\)\par \vspace{0.5em}
\end{enumerate}

\end{document}
