
\documentclass{article}
\usepackage{amsmath, amssymb}
\usepackage{geometry}
\geometry{a4paper, margin=0.75in}
\usepackage{enumitem}
\usepackage[hypertexnames=true, linktoc=all]{hyperref}
\usepackage{fancyhdr}
\usepackage{tikz}
\usepackage{graphicx}
\usepackage{asymptote}
\usepackage{arcs}
\usepackage{xwatermark}
\begin{asydef}
  // Global Asymptote settings
  settings.outformat = "pdf";
  settings.render = 0;
  settings.prc = false;
  import olympiad;
  import cse5;
  size(8cm);
\end{asydef}
\pagestyle{fancy}
\fancyhead[L]{\textbf{AMC 12 Problems}}
\fancyhead[R]{\textbf{2018}}
\fancyfoot[C]{\thepage}
\renewcommand{\headrulewidth}{0.4pt}
\renewcommand{\footrulewidth}{0.4pt}

\title{AMC 12 Problems \\ 2018}
\date{}
\begin{document}\maketitle\thispagestyle{fancy}\newpage\section*{2018 AMC 12A}
\begin{enumerate}[label=\arabic*., itemsep=0.5em]
\item A large urn contains \(100\) balls, of which \(36 \%\) are red and the rest are blue. How many of the blue balls must be removed so that the percentage of red balls in the urn will be \(72 \%\)? (No red balls are to be removed.)

\( \textbf{(A)}\ 28 \qquad\textbf{(B)}\  32 \qquad\textbf{(C)}\  36 \qquad\textbf{(D)}\ 
 50 \qquad\textbf{(E)}\ 64 \)\par \vspace{0.5em}\item While exploring a cave, Carl comes across a collection of \(5\)-pound rocks worth \(\$14\) each, \(4\)-pound rocks worth \(\$11\) each, and \(1\)-pound rocks worth \(\$2\) each. There are at least \(20\) of each size. He can carry at most \(18\) pounds. What is the maximum value, in dollars, of the rocks he can carry out of the cave?

\(\textbf{(A) } 48 \qquad \textbf{(B) } 49 \qquad \textbf{(C) } 50 \qquad \textbf{(D) } 51 \qquad \textbf{(E) } 52 \)\par \vspace{0.5em}\item How many ways can a student schedule \(3\) mathematics courses -- algebra, geometry, and number theory -- in a \(6\)-period day if no two mathematics courses can be taken in consecutive periods? (What courses the student takes during the other \(3\) periods is of no concern here.)

\(\textbf{(A) }3\qquad\textbf{(B) }6\qquad\textbf{(C) }12\qquad\textbf{(D) }18\qquad\textbf{(E) }24\)\par \vspace{0.5em}\item Alice, Bob, and Charlie were on a hike and were wondering how far away the nearest town was. When Alice said, "We are at least \(6\) miles away," Bob replied, "We are at most \(5\) miles away." Charlie then remarked, "Actually the nearest town is at most \(4\) miles away." It turned out that none of the three statements were true. Let \(d\) be the distance in miles to the nearest town. Which of the following intervals is the set of all possible values of \(d\)?

\(\textbf{(A) }   (0,4)   \qquad        \textbf{(B) }   (4,5)   \qquad    \textbf{(C) }   (4,6)   \qquad   \textbf{(D) }  (5,6)  \qquad  \textbf{(E) }   (5,\infty) \)\par \vspace{0.5em}\item What is the sum of all possible values of \(k\) for which the polynomials \(x^2 - 3x + 2\) and \(x^2 - 5x + k\) have a root in common?

\(\textbf{(A) }3 \qquad\textbf{(B) }4 \qquad\textbf{(C) }5 \qquad\textbf{(D) }6 \qquad\textbf{(E) }10 \qquad\)\par \vspace{0.5em}\item For positive integers \(m\) and \(n\) such that \(m+10<n+1\), both the mean and the median of the set \(\{m, m+4, m+10, n+1, n+2, 2n\}\) are equal to \(n\). What is \(m+n\)?

\(\textbf{(A) }20\qquad\textbf{(B) }21\qquad\textbf{(C) }22\qquad\textbf{(D) }23\qquad\textbf{(E) }24\)\par \vspace{0.5em}\item For how many (not necessarily positive) integer values of \(n\) is the value of \(4000\cdot \left(\tfrac{2}{5}\right)^n\) an integer?

\(
\textbf{(A) }3 \qquad
\textbf{(B) }4 \qquad
\textbf{(C) }6 \qquad
\textbf{(D) }8 \qquad
\textbf{(E) }9 \qquad
\)\par \vspace{0.5em}\item All of the triangles in the diagram below are similar to isosceles triangle \(ABC\), in which \(AB=AC\). Each of the \(7\) smallest triangles has area \(1,\) and \(\triangle ABC\) has area \(40\). What is the area of trapezoid \(DBCE\)?


\begin{center}
\begin{asy}
import olympiad;
import cse5;
unitsize(5);
dot((0,0));
dot((60,0));
dot((50,10));
dot((10,10));
dot((30,30));
draw((0,0)--(60,0)--(50,10)--(30,30)--(10,10)--(0,0));
draw((10,10)--(50,10));
label("$B$",(0,0),SW);
label("$C$",(60,0),SE);
label("$E$",(50,10),E);
label("$D$",(10,10),W);
label("$A$",(30,30),N);
draw((10,10)--(15,15)--(20,10)--(25,15)--(30,10)--(35,15)--(40,10)--(45,15)--(50,10));
draw((15,15)--(45,15));
\end{asy}
\end{center}


\(\textbf{(A) }   16   \qquad        \textbf{(B) }   18   \qquad    \textbf{(C) }   20   \qquad   \textbf{(D) }  22 \qquad  \textbf{(E) }   24 \)\par \vspace{0.5em}\item Which of the following describes the largest subset of values of \(y\) within the closed interval \([0,\pi]\) for which

\begin{equation*}
\sin(x+y)\leq \sin(x)+\sin(y)
\end{equation*}
for every \(x\) between \(0\) and \(\pi\), inclusive?

\(\textbf{(A) } y=0 \qquad \textbf{(B) } 0\leq y\leq \frac{\pi}{4} \qquad \textbf{(C) } 0\leq y\leq \frac{\pi}{2} \qquad \textbf{(D) } 0\leq y\leq \frac{3\pi}{4} \qquad \textbf{(E) } 0\leq y\leq \pi \)\par \vspace{0.5em}\item How many ordered pairs of real numbers \((x,y)\) satisfy the following system of equations?

\begin{align*}
x+3y&=3 \\
\big||x|-|y|\big|&=1
\end{align*}

\(\textbf{(A) } 1 \qquad 
\textbf{(B) } 2 \qquad 
\textbf{(C) } 3 \qquad 
\textbf{(D) } 4 \qquad 
\textbf{(E) } 8 \)\par \vspace{0.5em}\item A paper triangle with sides of lengths \(3,4,\) and \(5\) inches, as shown, is folded so that point \(A\) falls on point \(B\). What is the length in inches of the crease?

\begin{center}
\begin{asy}
import olympiad;
import cse5;
draw((0,0)--(4,0)--(4,3)--(0,0));
label("$A$", (0,0), SW);
label("$B$", (4,3), NE);
label("$C$", (4,0), SE);
label("$4$", (2,0), S);
label("$3$", (4,1.5), E);
label("$5$", (2,1.5), NW);
fill(origin--(0,0)--(4,3)--(4,0)--cycle, gray);
\end{asy}
\end{center}

\(\textbf{(A) }   1+\frac12 \sqrt2   \qquad        \textbf{(B) }   \sqrt3   \qquad    \textbf{(C) }   \frac74   \qquad   \textbf{(D) }  \frac{15}{8} \qquad  \textbf{(E) }   2 \)\par \vspace{0.5em}\item Let \(S\) be a set of \(6\) integers taken from \(\{1,2,\dots,12\}\) with the property that if \(a\) and \(b\) are elements of \(S\) with \(a<b\), then \(b\) is not a multiple of \(a\). What is the least possible value of an element in \(S\)?

\(\textbf{(A)}\ 2\qquad\textbf{(B)}\ 3\qquad\textbf{(C)}\ 4\qquad\textbf{(D)}\ 5\qquad\textbf{(E)}\ 7\)\par \vspace{0.5em}\item How many nonnegative integers can be written in the form 
\begin{equation*}
a_7\cdot3^7+a_6\cdot3^6+a_5\cdot3^5+a_4\cdot3^4+a_3\cdot3^3+a_2\cdot3^2+a_1\cdot3^1+a_0\cdot3^0,
\end{equation*}

where \(a_i\in \{-1,0,1\}\) for \(0\le i \le 7\)?

\(\textbf{(A) } 512 \qquad 
\textbf{(B) } 729 \qquad 
\textbf{(C) } 1094 \qquad 
\textbf{(D) } 3281 \qquad 
\textbf{(E) } 59,048 \)\par \vspace{0.5em}\item The solution to the equation \(\log_{3x} 4 = \log_{2x} 8\), where \(x\) is a positive real number other than \(\frac{1}{3}\) or \(\frac{1}{2}\), can be written as \(\frac {p}{q}\) where \(p\) and \(q\) are relatively prime positive integers. What is \(p + q\)?

\(\textbf{(A) } 5   \qquad    
\textbf{(B) } 13   \qquad    
\textbf{(C) } 17   \qquad   
\textbf{(D) } 31 \qquad  
\textbf{(E) } 35 \)\par \vspace{0.5em}\item A scanning code consists of a \(7 \times 7\) grid of squares, with some of its squares colored black and the rest colored white. There must be at least one square of each color in this grid of \(49\) squares. A scanning code is called \(\textit{symmetric}\) if its look does not change when the entire square is rotated by a multiple of \(90 ^{\circ}\) counterclockwise around its center, nor when it is reflected across a line joining opposite corners or a line joining midpoints of opposite sides. What is the total number of possible symmetric scanning codes?

\(\textbf{(A)} \text{ 510} \qquad \textbf{(B)} \text{ 1022} \qquad \textbf{(C)} \text{ 8190} \qquad \textbf{(D)} \text{ 8192} \qquad \textbf{(E)} \text{ 65,534}\)\par \vspace{0.5em}\item Which of the following describes the set of values of \(a\) for which the curves \(x^2+y^2=a^2\) and \(y=x^2-a\) in the real \(xy\)-plane intersect at exactly \(3\) points?

\(
\textbf{(A) }a=\frac14 \qquad
\textbf{(B) }\frac14 < a < \frac12 \qquad
\textbf{(C) }a>\frac14 \qquad
\textbf{(D) }a=\frac12 \qquad
\textbf{(E) }a>\frac12 \qquad
\)\par \vspace{0.5em}\item Farmer Pythagoras has a field in the shape of a right triangle. The right triangle's legs have lengths \(3\) and \(4\) units. In the corner where those sides meet at a right angle, he leaves a small unplanted square \(S\) so that from the air it looks like the right angle symbol. The rest of the field is planted. The shortest distance from \(S\) to the hypotenuse is \(2\) units. What fraction of the field is planted?


\begin{center}
\begin{asy}
import olympiad;
import cse5;
/* Edited by MRENTHUSIASM */
size(160);
pair A, B, C, D, F;
A = origin;
B = (4,0);
C = (0,3);
D = (2/7,2/7);
F = foot(D,B,C);
fill(A--(2/7,0)--D--(0,2/7)--cycle, lightgray);
draw(A--B--C--cycle);
draw((2/7,0)--D--(0,2/7));
label("$4$", midpoint(A--B), N);
label("$3$", midpoint(A--C), E);
label("$2$", midpoint(D--F), SE);
label("$S$", midpoint(A--D));
draw(D--F, dashed);
\end{asy}
\end{center}


\(\textbf{(A) }   \frac{25}{27}   \qquad        \textbf{(B) }   \frac{26}{27}   \qquad    \textbf{(C) }   \frac{73}{75}   \qquad   \textbf{(D) } \frac{145}{147} \qquad  \textbf{(E) }   \frac{74}{75} \)\par \vspace{0.5em}\item Triangle \(ABC\) with \(AB=50\) and \(AC=10\) has area \(120\). Let \(D\) be the midpoint of \(\overline{AB}\), and let \(E\) be the midpoint of \(\overline{AC}\). The angle bisector of \(\angle BAC\) intersects \(\overline{DE}\) and \(\overline{BC}\) at \(F\) and \(G\), respectively. What is the area of quadrilateral \(FDBG\)?

\(
\textbf{(A) }60 \qquad
\textbf{(B) }65 \qquad
\textbf{(C) }70 \qquad
\textbf{(D) }75 \qquad
\textbf{(E) }80 \qquad
\)\par \vspace{0.5em}\item Let \(A\) be the set of positive integers that have no prime factors other than \(2\), \(3\), or \(5\). The infinite sum 
\begin{equation*}
\frac{1}{1} + \frac{1}{2} + \frac{1}{3} + \frac{1}{4} + \frac{1}{5} + \frac{1}{6} + \frac{1}{8} + \frac{1}{9} + \frac{1}{10} + \frac{1}{12} + \frac{1}{15} + \frac{1}{16} + \frac{1}{18} + \frac{1}{20} + \cdots
\end{equation*}
of the reciprocals of the elements of \(A\) can be expressed as \(\frac{m}{n}\), where \(m\) and \(n\) are relatively prime positive integers. What is \(m+n\)?

\(\textbf{(A) } 16 \qquad \textbf{(B) } 17 \qquad \textbf{(C) } 19 \qquad \textbf{(D) } 23 \qquad \textbf{(E) } 36\)\par \vspace{0.5em}\item Triangle \(ABC\) is an isosceles right triangle with \(AB=AC=3\). Let \(M\) be the midpoint of hypotenuse \(\overline{BC}\). Points \(I\) and \(E\) lie on sides \(\overline{AC}\) and \(\overline{AB}\), respectively, so that \(AI>AE\) and \(AIME\) is a cyclic quadrilateral. Given that triangle \(EMI\) has area \(2\), the length \(CI\) can be written as \(\frac{a-\sqrt{b}}{c}\), where \(a\), \(b\), and \(c\) are positive integers and \(b\) is not divisible by the square of any prime. What is the value of \(a+b+c\)?

\(
\textbf{(A) }9 \qquad
\textbf{(B) }10 \qquad
\textbf{(C) }11 \qquad
\textbf{(D) }12 \qquad
\textbf{(E) }13 \qquad
\)\par \vspace{0.5em}\item Which of the following polynomials has the greatest real root?

\(\textbf{(A) }   x^{19}+2018x^{11}+1   \qquad        \textbf{(B) }   x^{17}+2018x^{11}+1   \qquad    \textbf{(C) }   x^{19}+2018x^{13}+1   \qquad   \textbf{(D) }  x^{17}+2018x^{13}+1 \qquad  \textbf{(E) }   2019x+2018 \)\par \vspace{0.5em}\item The solutions to the equations \(z^2=4+4\sqrt{15}i\) and \(z^2=2+2\sqrt 3i,\) where \(i=\sqrt{-1},\) form the vertices of a parallelogram in the complex plane. The area of this parallelogram can be written in the form \(p\sqrt q-r\sqrt s,\) where \(p,\) \(q,\) \(r,\) and \(s\) are positive integers and neither \(q\) nor \(s\) is divisible by the square of any prime number. What is \(p+q+r+s?\)

\(\textbf{(A) } 20 \qquad 
\textbf{(B) } 21 \qquad 
\textbf{(C) } 22 \qquad 
\textbf{(D) } 23 \qquad 
\textbf{(E) } 24 \)\par \vspace{0.5em}\item In \(\triangle PAT,\) \(\angle P=36^{\circ},\) \(\angle A=56^{\circ},\) and \(PA=10.\) Points \(U\) and \(G\) lie on sides \(\overline{TP}\) and \(\overline{TA},\) respectively, so that \(PU=AG=1.\) Let \(M\) and \(N\) be the midpoints of segments \(\overline{PA}\) and \(\overline{UG},\) respectively. What is the degree measure of the acute angle formed by lines \(MN\) and \(PA?\)

\(\textbf{(A) } 76 \qquad 
\textbf{(B) } 77 \qquad 
\textbf{(C) } 78 \qquad 
\textbf{(D) } 79 \qquad 
\textbf{(E) } 80 \)\par \vspace{0.5em}\item Alice, Bob, and Carol play a game in which each of them chooses a real number between \(0\) and \(1.\) The winner of the game is the one whose number is between the numbers chosen by the other two players. Alice announces that she will choose her number uniformly at random from all the numbers between \(0\) and \(1,\) and Bob announces that he will choose his number uniformly at random from all the numbers between \(\tfrac{1}{2}\) and \(\tfrac{2}{3}.\) Armed with this information, what number should Carol choose to maximize her chance of winning?

\(
\textbf{(A) }\frac{1}{2}\qquad
\textbf{(B) }\frac{13}{24} \qquad
\textbf{(C) }\frac{7}{12} \qquad
\textbf{(D) }\frac{5}{8} \qquad
\textbf{(E) }\frac{2}{3}\qquad
\)\par \vspace{0.5em}\item For a positive integer \(n\) and nonzero digits \(a\), \(b\), and \(c\), let \(A_n\) be the \(n\)-digit integer each of whose digits is equal to \(a\); let \(B_n\) be the \(n\)-digit integer each of whose digits is equal to \(b\), and let \(C_n\) be the \(2n\)-digit (not \(n\)-digit) integer each of whose digits is equal to \(c\). What is the greatest possible value of \(a + b + c\) for which there are at least two values of \(n\) such that \(C_n - B_n = A_n^2\)?

\(\textbf{(A) } 12 \qquad \textbf{(B) } 14 \qquad \textbf{(C) } 16 \qquad \textbf{(D) } 18 \qquad \textbf{(E) } 20\)\par \vspace{0.5em}
\end{enumerate}

\end{document}
