
\documentclass{article}
\usepackage{amsmath, amssymb}
\usepackage{geometry}
\geometry{a4paper, margin=0.75in}
\usepackage{enumitem}
\usepackage[hypertexnames=true, linktoc=all]{hyperref}
\usepackage{fancyhdr}
\usepackage{tikz}
\usepackage{graphicx}
\usepackage{asymptote}
\usepackage{arcs}
\usepackage{xwatermark}
\begin{asydef}
  // Global Asymptote settings
  settings.outformat = "pdf";
  settings.render = 0;
  settings.prc = false;
  import olympiad;
  import cse5;
  size(8cm);
\end{asydef}
\pagestyle{fancy}
\fancyhead[L]{\textbf{AMC 12 Problems}}
\fancyhead[R]{\textbf{2023-2024}}
\fancyfoot[C]{\thepage}
\renewcommand{\headrulewidth}{0.4pt}
\renewcommand{\footrulewidth}{0.4pt}

\title{AMC 12 Problems \\ 2023-2024}
\date{}
\begin{document}\maketitle\thispagestyle{fancy}\tableofcontents\newpage\section*{2023 AMC 12A Problems}
\addcontentsline{toc}{section}{2023 AMC 12A Problems}
\begin{enumerate}[label=\arabic*., itemsep=0.5em]\item Cities \(A\) and \(B\) are \(45\) miles apart. Alicia lives in \(A\) and Beth lives in \(B\). Alicia bikes towards \(B\) at 18 miles per hour. Leaving at the same time, Beth bikes toward \(A\) at 12 miles per hour. How many miles from City \(A\) will they be when they meet?

\(\textbf{(A) }20\qquad\textbf{(B) }24\qquad\textbf{(C) }25\qquad\textbf{(D) }26\qquad\textbf{(E) }27\)\par \vspace{0.5em}\item The weight of \(\frac{1}{3}\) of a large pizza together with \(3 \frac{1}{2}\) cups of orange slices is the same weight of \(\frac{3}{4}\) of a large pizza together with \(\frac{1}{2}\) cups of orange slices. A cup of orange slices weigh \(\frac{1}{4}\) of a pound. What is the weight, in pounds, of a large pizza?

\(\textbf{(A) }1\frac{4}{5}\qquad\textbf{(B) }2\qquad\textbf{(C) }2\frac{2}{5}\qquad\textbf{(D) }3\qquad\textbf{(E) }3\frac{3}{5}\)\par \vspace{0.5em}\item How many positive perfect squares less than \(2023\) are divisible by \(5\)?

\(\textbf{(A) }8\qquad\textbf{(B) }9\qquad\textbf{(C) }10\qquad\textbf{(D) }11\qquad\textbf{(E) }12\)\par \vspace{0.5em}\item How many digits are in the base-ten representation of \(8^5 \cdot 5^{10} \cdot 15^5\)?

\(\textbf{(A)}~14\qquad\textbf{(B)}~15\qquad\textbf{(C)}~16\qquad\textbf{(D)}~17\qquad\textbf{(E)}~18\qquad\)\par \vspace{0.5em}\item Janet rolls a standard \(6\)-sided die \(4\) times and keeps a running total of the numbers she rolls. What is the probability that at some point, her running total will equal \(3?\)

\(\textbf{(A) }\frac{2}{9}\qquad\textbf{(B) }\frac{49}{216}\qquad\textbf{(C) }\frac{25}{108}\qquad\textbf{(D) }\frac{17}{72}\qquad\textbf{(E) }\frac{13}{54}\)\par \vspace{0.5em}\item Points \(A\) and \(B\) lie on the graph of \(y=\log_{2}x\). The midpoint of \(\overline{AB}\) is \((6, 2)\). What is the positive difference between the \(x\)-coordinates of \(A\) and \(B\)?

\(\textbf{(A)}~2\sqrt{11}\qquad\textbf{(B)}~4\sqrt{3}\qquad\textbf{(C)}~8\qquad\textbf{(D)}~4\sqrt{5}\qquad\textbf{(E)}~9\)\par \vspace{0.5em}\item A digital display shows the current date as an \(8\)-digit integer consisting of a \(4\)-digit year, followed by a \(2\)-digit month, followed by a \(2\)-digit date within the month. For example, Arbor Day this year is displayed as \(20230428\). For how many dates in \(2023\) will each digit appear an even number of times in the 8-digital display for that date?

\(\textbf{(A)}~5\qquad\textbf{(B)}~6\qquad\textbf{(C)}~7\qquad\textbf{(D)}~8\qquad\textbf{(E)}~9\)\par \vspace{0.5em}\item Maureen is keeping track of the mean of her quiz scores this semester. If Maureen scores an \(11\) on the next quiz, her mean will increase by \(1\). If she scores an \(11\) on each of the next three quizzes, her mean will increase by \(2\). What is the mean of her quiz scores currently?

\(\textbf{(A) }4\qquad\textbf{(B) }5\qquad\textbf{(C) }6\qquad\textbf{(D) }7\qquad\textbf{(E) }8\)\par \vspace{0.5em}\item A square of area \(2\) is inscribed in a square of area \(3\), creating four congruent triangles, as shown below. What is the ratio of the shorter leg to the longer leg in the shaded right triangle?

\begin{center}
\begin{asy}
import olympiad;
import cse5;
size(200);
defaultpen(linewidth(0.6pt)+fontsize(10pt));
real y = sqrt(3);
pair A,B,C,D,E,F,G,H;
A = (0,0);
B = (0,y);
C = (y,y);
D = (y,0);
E = ((y + 1)/2,y);
F = (y, (y - 1)/2);
G = ((y - 1)/2, 0);
H = (0,(y + 1)/2);
fill(H--B--E--cycle, gray);
draw(A--B--C--D--cycle);
draw(E--F--G--H--cycle);
\end{asy}
\end{center}


\(\textbf{(A) }\frac15\qquad\textbf{(B) }\frac14\qquad\textbf{(C) }2-\sqrt3\qquad\textbf{(D) }\sqrt3-\sqrt2\qquad\textbf{(E) }\sqrt2-1\)\par \vspace{0.5em}\item Positive real numbers \(x\) and \(y\) satisfy \(y^3 = x^2\) and \((y-x)^2 = 4y^2\). What is \(x+y\)?

\(\textbf{(A)}\ 12 \qquad \textbf{(B)}\ 18 \qquad \textbf{(C)}\ 24 \qquad \textbf{(D)}\ 36 \qquad \textbf{(E)}\ 42\)\par \vspace{0.5em}\item What is the degree measure of the acute angle formed by lines with slopes \(2\) and \(\tfrac{1}{3}\)?

\(\textbf{(A)}~30\qquad\textbf{(B)}~37.5\qquad\textbf{(C)}~45\qquad\textbf{(D)}~52.5\qquad\textbf{(E)}~60\)\par \vspace{0.5em}\item What is the value of

\begin{equation*}
2^3 - 1^3 + 4^3 - 3^3 + 6^3 - 5^3 + \dots + 18^3 - 17^3?
\end{equation*}


\(\textbf{(A) } 2023 \qquad\textbf{(B) } 2679 \qquad\textbf{(C) } 2941 \qquad\textbf{(D) } 3159 \qquad\textbf{(E) } 3235\)\par \vspace{0.5em}\item In a table tennis tournament every participant played every other participant exactly once. Although there were twice as many right-handed players as left-handed players, the number of games won by left-handed players was \(40\%\) more than the number of games won by right-handed players. (There were no ties and no ambidextrous players.) What is the total number of games played?

\(\textbf{(A) }15\qquad\textbf{(B) }36\qquad\textbf{(C) }45\qquad\textbf{(D) }48\qquad\textbf{(E) }66\)\par \vspace{0.5em}\item How many complex numbers satisfy the equation \(z^{5}=\overline{z}\), where \(\overline{z}\) is the conjugate of the complex number \(z\)?

\(\textbf{(A)}~2\qquad\textbf{(B)}~3\qquad\textbf{(C)}~5\qquad\textbf{(D)}~6\qquad\textbf{(E)}~7\)\par \vspace{0.5em}\item Usain is walking for exercise by zigzagging across a \(100\)-meter by \(30\)-meter rectangular field, beginning at point \(A\) and ending on the segment \(\overline{BC}\). He wants to increase the distance walked by zigzagging as shown in the figure below \((APQRS)\). What angle \(\theta\)\(\angle PAB=\angle QPC=\angle RQB=\cdots\) will produce in a length that is \(120\) meters? (This figure is not drawn to scale. Do not assume that the zigzag path has exactly four segments as shown; there could be more or fewer.)


\begin{center}
\begin{asy}
import olympiad;
import cse5;
import olympiad;
draw((-50,15)--(50,15));
draw((50,15)--(50,-15));
draw((50,-15)--(-50,-15));
draw((-50,-15)--(-50,15));
draw((-50,-15)--(-22.5,15));
draw((-22.5,15)--(5,-15));
draw((5,-15)--(32.5,15));
draw((32.5,15)--(50,-4.090909090909));
label("$\theta$", (-41.5,-10.5));
label("$\theta$", (-13,10.5));
label("$\theta$", (15.5,-10.5));
label("$\theta$", (43,10.5));
dot((-50,15));
dot((-50,-15));
dot((50,15));
dot((50,-15));
dot((50,-4.09090909090909));
label("$D$",(-58,15));
label("$A$",(-58,-15));
label("$C$",(58,15));
label("$B$",(58,-15));
label("$S$",(58,-4.0909090909));
dot((-22.5,15));
dot((5,-15));
dot((32.5,15));
label("$P$",(-22.5,23));
label("$Q$",(5,-23));
label("$R$",(32.5,23));
\end{asy}
\end{center}


\(\textbf{(A)}~\arccos\frac{5}{6}\qquad\textbf{(B)}~\arccos\frac{4}{5}\qquad\textbf{(C)}~\arccos\frac{3}{10}\qquad\textbf{(D)}~\arcsin\frac{4}{5}\qquad\textbf{(E)}~\arcsin\frac{5}{6}\)\par \vspace{0.5em}\item Consider the set of complex numbers \(z\) satisfying \(|1+z+z^{2}|=4\). The maximum value of the imaginary part of \(z\) can be written in the form \(\tfrac{\sqrt{m}}{n}\), where \(m\) and \(n\) are relatively prime positive integers. What is \(m+n\)?

\(\textbf{(A)}~20\qquad\textbf{(B)}~21\qquad\textbf{(C)}~22\qquad\textbf{(D)}~23\qquad\textbf{(E)}~24\)\par \vspace{0.5em}\item Flora the frog starts at \(0\) on the number line and makes a sequence of jumps to the right. In any one jump, independent of previous jumps, Flora leaps a positive integer distance \(m\) with probability \(\frac{1}{2^m}\). What is the probability that Flora will eventually land at \(10\)?

\(\textbf{(A) } \frac{5}{512} \qquad \textbf{(B) } \frac{45}{1024} \qquad \textbf{(C) } \frac{127}{1024} \qquad \textbf{(D) } \frac{511}{1024} \qquad \textbf{(E) } \frac{1}{2}\)\par \vspace{0.5em}\item Circle \(C_1\) and \(C_2\) each have radius \(1\), and the distance between their centers is \(\frac{1}{2}\). Circle \(C_3\) is the largest circle internally tangent to both \(C_1\) and \(C_2\). Circle \(C_4\) is internally tangent to both \(C_1\) and \(C_2\) and externally tangent to \(C_3\). What is the radius of \(C_4\)?


\begin{center}
\begin{asy}
import olympiad;
import cse5;
import olympiad; 
size(10cm); 
draw(circle((0,0),0.75)); 
draw(circle((-0.25,0),1)); 
draw(circle((0.25,0),1)); 
draw(circle((0,6/7),3/28)); 
pair A = (0,0), B = (-0.25,0), C = (0.25,0), D = (0,6/7), E = (-0.95710678118, 0.70710678118), F = (0.95710678118, -0.70710678118);
dot(B^^C); 
draw(B--E, dashed);
draw(C--F, dashed);
draw(B--C); 
label("$C_4$", D); 
label("$C_1$", (-1.375, 0)); 
label("$C_2$", (1.375,0));
label("$\frac{1}{2}$", (0, -.125));
label("$C_3$", (-0.4, -0.4));
label("$1$", (-.85, 0.70));
label("$1$", (.85, -.7));
import olympiad; 
markscalefactor=0.005;
\end{asy}
\end{center}


\(\textbf{(A) } \frac{1}{14} \qquad \textbf{(B) } \frac{1}{12} \qquad \textbf{(C) } \frac{1}{10} \qquad \textbf{(D) } \frac{3}{28} \qquad \textbf{(E) } \frac{1}{9}\)\par \vspace{0.5em}\item What is the product of all the solutions to the equation 
\begin{equation*}
\log_{7x}2023 \cdot \log_{289x} 2023 = \log_{2023x} 2023?
\end{equation*}


\(\textbf{(A) }(\log_{2023}7 \cdot \log_{2023}289)^2 \qquad\textbf{(B) }\log_{2023}7 \cdot \log_{2023}289\qquad\textbf{(C) } 1
\\
\\
\textbf{(D) }\log_{7}2023 \cdot \log_{289}2023\qquad\textbf{(E) }(\log_{7}2023 \cdot \log_{289}2023)^2\)\par \vspace{0.5em}\item Rows 1, 2, 3, 4, and 5 of a triangular array of integers are shown below:


\begin{center}
\begin{asy}
import olympiad;
import cse5;
size(4.5cm);
label("$1$", (0,0));
label("$1$", (-0.5,-2/3));
label("$1$", (0.5,-2/3));
label("$1$", (-1,-4/3));
label("$3$", (0,-4/3));
label("$1$", (1,-4/3));
label("$1$", (-1.5,-2));
label("$5$", (-0.5,-2));
label("$5$", (0.5,-2));
label("$1$", (1.5,-2));
label("$1$", (-2,-8/3));
label("$7$", (-1,-8/3));
label("$11$", (0,-8/3));
label("$7$", (1,-8/3));
label("$1$", (2,-8/3));
\end{asy}
\end{center}


Each row after the first row is formed by placing a 1 at each end of the row, and each interior entry is 1 greater than the sum of the two numbers diagonally above it in the previous row. What is the units digit of the sum of the 2023 numbers in the 2023rd row?

\(\textbf{(A) }1\qquad\textbf{(B) }3\qquad\textbf{(C) }5\qquad\textbf{(D) }7\qquad\textbf{(E) }9\)\par \vspace{0.5em}\item If \(A\) and \(B\) are vertices of a polyhedron, define the distance \(d(A, B)\) to be the minimum number of edges of the polyhedron one must traverse in order to connect \(A\) and \(B\). For example, if \(\overline{AB}\) is an edge of the polyhedron, then \(d(A, B) = 1\), but if \(\overline{AC}\) and \(\overline{CB}\) are edges and \(\overline{AB}\) is not an edge, then \(d(A, B) = 2\). Let \(Q\), \(R\), and \(S\) be randomly chosen distinct vertices of a regular icosahedron (regular polyhedron made up of 20 equilateral triangles). What is the probability that \(d(Q, R) > d(R, S)\)?

\(\textbf{(A)}~\frac{7}{22}\qquad\textbf{(B)}~\frac13\qquad\textbf{(C)}~\frac38\qquad\textbf{(D)}~\frac5{12}\qquad\textbf{(E)}~\frac12\)\par \vspace{0.5em}\item Let \(f\) be the unique function defined on the positive integers such that
\begin{equation*}
\sum_{d\mid n}d\cdot f\left(\frac{n}{d}\right)=1
\end{equation*}
for all positive integers \(n\), where the sum is taken over all positive divisors of \(n\). What is \(f(2023)\)?

\(\textbf{(A)}~-1536\qquad\textbf{(B)}~96\qquad\textbf{(C)}~108\qquad\textbf{(D)}~116\qquad\textbf{(E)}~144\)\par \vspace{0.5em}\item How many ordered pairs of positive real numbers \((a,b)\) satisfy the equation

\begin{equation*}
(1+2a)(2+2b)(2a+b) = 32ab?
\end{equation*}


\(\textbf{(A) }0\qquad\textbf{(B) }1\qquad\textbf{(C) }2\qquad\textbf{(D) }3\qquad\textbf{(E) }\text{an infinite number}\)\par \vspace{0.5em}\item Let \(K\) be the number of sequences \(A_1\), \(A_2\), \(\dots\), \(A_n\) such that \(n\) is a positive integer less than or equal to \(10\), each \(A_i\) is a subset of \(\{1, 2, 3, \dots, 10\}\), and \(A_{i-1}\) is a subset of \(A_i\) for each \(i\) between \(2\) and \(n\), inclusive. For example, \(\{\}\), \(\{5, 7\}\), \(\{2, 5, 7\}\), \(\{2, 5, 7\}\), \(\{2, 5, 6, 7, 9\}\) is one such sequence, with \(n = 5\).What is the remainder when \(K\) is divided by \(10\)?

\(\textbf{(A) } 1 \qquad \textbf{(B) } 3 \qquad \textbf{(C) } 5 \qquad \textbf{(D) } 7 \qquad \textbf{(E) } 9\)\par \vspace{0.5em}\item There is a unique sequence of integers \(a_1, a_2, \cdots a_{2023}\) such that

\begin{equation*}
\tan2023x = \frac{a_1 \tan x + a_3 \tan^3 x + a_5 \tan^5 x + \cdots + a_{2023} \tan^{2023} x}{1 + a_2 \tan^2 x + a_4 \tan^4 x \cdots + a_{2022} \tan^{2022} x}
\end{equation*}
whenever \(\tan 2023x\) is defined. What is \(a_{2023}?\)

\(\textbf{(A) } -2023 \qquad\textbf{(B) } -2022 \qquad\textbf{(C) } -1 \qquad\textbf{(D) } 1 \qquad\textbf{(E) } 2023\)\par \vspace{0.5em}\end{enumerate}\newpage\section*{2023 AMC 12B Problems}
\addcontentsline{toc}{section}{2023 AMC 12B Problems}
\begin{enumerate}[label=\arabic*., itemsep=0.5em]\item Mrs. Jones is pouring orange juice into four identical glasses for her four sons. She fills the first three glasses completely but runs out of juice when the fourth glass is only \(\frac{1}{3}\) full. What fraction of a glass must Mrs. Jones pour from each of the first three glasses into the fourth glass so that all four glasses will have the same amount of juice?

\(\textbf{(A) }\frac{1}{12}\qquad\textbf{(B) }\frac{1}{4}\qquad\textbf{(C) }\frac{1}{6}\qquad\textbf{(D) }\frac{1}{8}\qquad\textbf{(E) }\frac{2}{9}\)\par \vspace{0.5em}\item Carlos went to a sports store to buy running shoes. Running shoes were on sale, with prices reduced by \(20\%\)on every pair of shoes. Carlos also knew that he had to pay a \(7.5\%\) sales tax on the discounted price. He had \(43\) dollars. What is the original (before discount) price of the most expensive shoes he could afford to buy?

\(\textbf{(A) }46\qquad\textbf{(B) }50\qquad\textbf{(C) }48\qquad\textbf{(D) }47\qquad\textbf{(E) }49\)\par \vspace{0.5em}\item A \(3-4-5\) right triangle is inscribed in circle \(A\), and a \(5-12-13\) right triangle is inscribed in circle \(B\). What is the ratio of the area of circle \(A\) to the area of circle \(B\)?

\(\textbf{(A)}~\frac{9}{25}\qquad\textbf{(B)}~\frac{1}{9}\qquad\textbf{(C)}~\frac{1}{5}\qquad\textbf{(D)}~\frac{25}{169}\qquad\textbf{(E)}~\frac{4}{25}\)\par \vspace{0.5em}\item Jackson's paintbrush makes a narrow strip with a width of \(6.5\) millimeters. Jackson has enough paint to make a strip \(25\) meters long. How many square centimeters of paper could Jackson cover with paint?

\(\textbf{(A) }162,500\qquad\textbf{(B) }162.5\qquad\textbf{(C) }1,625\qquad\textbf{(D) }1,625,000\qquad\textbf{(E) }16,250\)\par \vspace{0.5em}\item You are playing a game. A \(2 \times 1\) rectangle covers two adjacent squares (oriented either horizontally or vertically) of a \(3 \times 3\) grid of squares, but you are not told which two squares are covered. Your goal is to find at least one square that is covered by the rectangle. A "turn" consists of you guessing a square, after which you are told whether that square is covered by the hidden rectangle. What is the minimum number of turns you need to ensure that at least one of your guessed squares is covered by the rectangle?

\(\textbf{(A)}~3\qquad\textbf{(B)}~5\qquad\textbf{(C)}~4\qquad\textbf{(D)}~8\qquad\textbf{(E)}~6\)\par \vspace{0.5em}\item When the roots of the polynomial


\begin{equation*}
P(x)  = (x-1)^1 (x-2)^2 (x-3)^3 \cdot \cdot \cdot (x-10)^{10}
\end{equation*}


are removed from the number line, what remains is the union of 11 disjoint open intervals. On how many of these intervals is \(P(x)\) positive?

\(\textbf{(A)}~3\qquad\textbf{(B)}~7\qquad\textbf{(C)}~6\qquad\textbf{(D)}~4\qquad\textbf{(E)}~5\)\par \vspace{0.5em}\item For how many integers \(n\) does the expression
\begin{equation*}
\sqrt{\frac{\log (n^2) - (\log n)^2}{\log n - 3}}
\end{equation*}
represent a real number, where log denotes the base \(10\) logarithm?

\(\textbf{(A) }900 \qquad \textbf{(B) }3\qquad \textbf{(C) }902 \qquad \textbf{(D) } 2  \qquad \textbf{(E) }901\)\par \vspace{0.5em}\item How many nonempty subsets \(B\) of \(\{0, 1, 2, 3, \dots, 12\}\) have the property that the number of elements in \(B\) is equal to the least element of \(B\)? For example, \(B = \{4, 6, 8, 11\}\) satisfies the condition.

\(\textbf{(A)}\ 256 \qquad\textbf{(B)}\ 136 \qquad\textbf{(C)}\ 108 \qquad\textbf{(D)}\ 144 \qquad\textbf{(E)}\ 156\)\par \vspace{0.5em}\item What is the area of the region in the coordinate plane defined by

\(\left||x|-1\right|+\left||y|-1\right|\leq 1?\)

\(\textbf{(A)}~2\qquad\textbf{(B)}~8\qquad\textbf{(C)}~4\qquad\textbf{(D)}~15\qquad\textbf{(E)}~12\)\par \vspace{0.5em}\item In the \(xy\)-plane, a circle of radius \(4\) with center on the positive \(x\)-axis is tangent to the \(y\)-axis at the origin, and a circle with radius \(10\) with center on the positive \(y\)-axis is tangent to the \(x\)-axis at the origin. What is the slope of the line passing through the two points at which these circles intersect?

\(\textbf{(A)}\ \dfrac{2}{7} \qquad\textbf{(B)}\ \dfrac{3}{7}  \qquad\textbf{(C)}\ \dfrac{2}{\sqrt{29}}  \qquad\textbf{(D)}\ \dfrac{1}{\sqrt{29}}  \qquad\textbf{(E)}\ \dfrac{2}{5}\)\par \vspace{0.5em}\item What is the maximum area of an isosceles trapezoid that has legs of length \(1\) and one base twice as long as the other?

\(\textbf{(A) }\frac 54 \qquad \textbf{(B) } \frac 87 \qquad \textbf{(C)} \frac{5\sqrt2}4 \qquad \textbf{(D) } \frac 32  \qquad \textbf{(E) } \frac{3\sqrt3}4\)\par \vspace{0.5em}\item For complex numbers \(u=a+bi\) and \(v=c+di\), define the binary operation \(\otimes\) by
\begin{equation*}
u\otimes v=ac+bdi.
\end{equation*}
Suppose \(z\) is a complex number such that \(z\otimes z=z^{2}+40\). What is \(|z|\)?

\(\textbf{(A) }2\qquad\textbf{(B) }5\qquad\textbf{(C) }\sqrt{5}\qquad\textbf{(D) }\sqrt{10}\qquad\textbf{(E) }5\sqrt{2}\)\par \vspace{0.5em}\item A rectangular box \(P\) has distinct edge lengths \(a\), \(b\), and \(c\). The sum of the lengths of all \(12\) edges of \(P\) is \(13\), the sum of the areas of all \(6\) faces of \(P\) is \(\frac{11}{2}\), and the volume of \(P\) is \(\frac{1}{2}\). What is the length of the longest interior diagonal connecting two vertices of \(P\)?

\(\textbf{(A)}~2\qquad\textbf{(B)}~\frac{3}{8}\qquad\textbf{(C)}~\frac{9}{8}\qquad\textbf{(D)}~\frac{9}{4}\qquad\textbf{(E)}~\frac{3}{2}\)\par \vspace{0.5em}\item For how many ordered pairs \((a,b)\) of integers does the polynomial \(x^3+ax^2+bx+6\) have \(3\) distinct integer roots?

\(\textbf{(A)}\ 5 \qquad\textbf{(B)}\ 6 \qquad\textbf{(C)}\ 8 \qquad\textbf{(D)}\ 7 \qquad\textbf{(E)}\ 4\)\par \vspace{0.5em}\item Suppose \(a\), \(b\), and \(c\) are positive integers such that
\begin{equation*}
\frac{a}{14}+\frac{b}{15}=\frac{c}{210}.
\end{equation*}
Which of the following statements are necessarily true?

I. If \(\gcd(a,14)=1\) or \(\gcd(b,15)=1\) or both, then \(\gcd(c,210)=1\).

II. If \(\gcd(c,210)=1\), then \(\gcd(a,14)=1\) or \(\gcd(b,15)=1\) or both.

III. \(\gcd(c,210)=1\) if and only if \(\gcd(a,14)=\gcd(b,15)=1\).

\(\textbf{(A)}~\text{I, II, and III}\qquad\textbf{(B)}~\text{I only}\qquad\textbf{(C)}~\text{I and II only}\qquad\textbf{(D)}~\text{III only}\qquad\textbf{(E)}~\text{II and III only}\)\par \vspace{0.5em}\item In Coinland, there are three types of coins, each worth \(6, 10,\) and \(15.\) What is the sum of the digits of the maximum amount of money that is impossible to have?

\(\textbf{(A) }8\qquad\textbf{(B) }10\qquad\textbf{(C) }7\qquad\textbf{(D) }11\qquad\textbf{(E) }9\)\par \vspace{0.5em}\item Triangle \(ABC\) has side lengths in arithmetic progression, and the smallest side has length \(6\). If the triangle has an angle of \(120^\circ,\) what is the area of \(ABC\)?

\(\textbf{(A) }12\sqrt{3}\qquad\textbf{(B) }8\sqrt{6}\qquad\textbf{(C) }14\sqrt{2}\qquad\textbf{(D) }20\sqrt{2}\qquad\textbf{(E) }15\sqrt{3}\)\par \vspace{0.5em}\item Last academic year Yolanda and Zelda took different courses that did not necessarily administer the same number of quizzes during each of the two semesters. Yolanda's average on all the quizzes she took during the first semester was \(3\) points higher than Zelda's average on all the quizzes she took during the first semester. Yolanda's average on all the quizzes she took during the second semester was \(18\) points higher than her average for the first semester and was again \(3\) points higher than Zelda's average on all the quizzes Zelda took during her second semester. Which one of the following statements cannot possibly be true?

\(\textbf{(A)}\) Yolanda's quiz average for the academic year was \(22\) points higher than Zelda's.

\(\textbf{(B)}\) Zelda's quiz average for the academic year was higher than Yolanda's.

\(\textbf{(C)}\) Yolanda's quiz average for the academic year was \(3\) points higher than Zelda's.

\(\textbf{(D)}\) Zelda's quiz average for the academic year equaled Yolanda's.

\(\textbf{(E)}\) If Zelda had scored \(3\) points higher on each quiz she took, then she would have had the same average for the academic year as Yolanda.\par \vspace{0.5em}\item Each of \(2023\) balls is placed in one of \(3\) bins. Which of the following is closest to the probability that each of the bins will contain an odd number of balls?

\(\textbf{(A) } \frac{2}{3} \qquad \textbf{(B) } \frac{3}{10} \qquad \textbf{(C) } \frac{1}{2} \qquad \textbf{(D) } \frac{1}{3} \qquad \textbf{(E) } \frac{1}{4}\)\par \vspace{0.5em}\item Cyrus the frog jumps \(2\) units in a direction, then \(2\) more in another direction. What is the probability that he lands less than \(1\) unit away from his starting position?

\(\textbf{(A)}~\frac{1}{6}\qquad\textbf{(B)}~\frac{1}{5}\qquad\textbf{(C)}~\frac{\sqrt{3}}{8}\qquad\textbf{(D)}~\frac{\arctan \frac{1}{2}}{\pi}\qquad\textbf{(E)}~\frac{2\arcsin \frac{1}{4}}{\pi}\)\par \vspace{0.5em}\item A lampshade is made in the form of the lateral surface of the frustum of a right circular cone. The height of the frustum is \(3\sqrt3\) inches, its top diameter is \(6\) inches, and its bottom diameter is \(12\) inches. A bug is at the bottom of the lampshade and there is a glob of honey on the top edge of the lampshade at the spot farthest from the bug. The bug wants to crawl to the honey, but it must stay on the surface of the lampshade. What is the length in inches of its shortest path to the honey?

\(\textbf{(A) } 6 + 3\pi\qquad \textbf{(B) }6 + 6\pi\qquad \textbf{(C) } 6\sqrt3 \qquad \textbf{(D) } 6\sqrt5 \qquad \textbf{(E) } 6\sqrt3 + \pi\)\par \vspace{0.5em}\item A real-valued function \(f\) has the property that for all real numbers \(a\) and \(b,\)
\begin{equation*}
f(a + b)  + f(a - b) = 2f(a) f(b).
\end{equation*}
Which one of the following cannot be the value of \(f(1)?\)

\(\textbf{(A) } 0 \qquad \textbf{(B) } 1 \qquad \textbf{(C) } -1 \qquad \textbf{(D) } 2 \qquad \textbf{(E) } -2\)\par \vspace{0.5em}\item When \(n\) standard six-sided dice are rolled, the product of the numbers rolled can be any of \(936\) possible values. What is \(n\)?

\(\textbf{(A)}~11\qquad\textbf{(B)}~6\qquad\textbf{(C)}~8\qquad\textbf{(D)}~10\qquad\textbf{(E)}~9\)\par \vspace{0.5em}\item Suppose that \(a\), \(b\), \(c\) and \(d\) are positive integers satisfying all of the following relations.


\begin{equation*}
abcd=2^6\cdot 3^9\cdot 5^7
\end{equation*}


\begin{equation*}
\text{lcm}(a,b)=2^3\cdot 3^2\cdot 5^3
\end{equation*}


\begin{equation*}
\text{lcm}(a,c)=2^3\cdot 3^3\cdot 5^3
\end{equation*}


\begin{equation*}
\text{lcm}(a,d)=2^3\cdot 3^3\cdot 5^3
\end{equation*}


\begin{equation*}
\text{lcm}(b,c)=2^1\cdot 3^3\cdot 5^2
\end{equation*}


\begin{equation*}
\text{lcm}(b,d)=2^2\cdot 3^3\cdot 5^2
\end{equation*}


\begin{equation*}
\text{lcm}(c,d)=2^2\cdot 3^3\cdot 5^2
\end{equation*}


What is \(\text{gcd}(a,b,c,d)\)?

\(\textbf{(A)}~30\qquad\textbf{(B)}~45\qquad\textbf{(C)}~3\qquad\textbf{(D)}~15\qquad\textbf{(E)}~6\)\par \vspace{0.5em}\item A regular pentagon with area \(\sqrt{5}+1\) is printed on paper and cut out. The five vertices of the pentagon are folded into the center of the pentagon, creating a smaller pentagon. What is the area of the new pentagon?

\(\textbf{(A)}~4-\sqrt{5}\qquad\textbf{(B)}~\sqrt{5}-1\qquad\textbf{(C)}~8-3\sqrt{5}\qquad\textbf{(D)}~\frac{\sqrt{5}+1}{2}\qquad\textbf{(E)}~\frac{2+\sqrt{5}}{3}\)\par \vspace{0.5em}\end{enumerate}\newpage\section*{2024 AMC 12A Problems}
\addcontentsline{toc}{section}{2024 AMC 12A Problems}
\begin{enumerate}[label=\arabic*., itemsep=0.5em]\item What is the value of \(9901\cdot101-99\cdot10101?\)

\(\textbf{(A)}~2\qquad\textbf{(B)}~20\qquad\textbf{(C)}~200\qquad\textbf{(D)}~202\qquad\textbf{(E)}~2020\)\par \vspace{0.5em}\item A model used to estimate the time it will take to hike to the top of the mountain on a trail is of the form \(T=aL+bG,\) where \(a\) and \(b\) are constants, \(T\) is the time in minutes, \(L\) is the length of the trail in miles, and \(G\) is the altitude gain in feet. The model estimates that it will take \(69\) minutes to hike to the top if a trail is \(1.5\) miles long and ascends \(800\) feet, as well as if a trail is \(1.2\) miles long and ascends \(1100\) feet. How many minutes does the model estimates it will take to hike to the top if the trail is \(4.2\) miles long and ascends \(4000\) feet?

\(\textbf{(A) }240\qquad\textbf{(B) }246\qquad\textbf{(C) }252\qquad\textbf{(D) }258\qquad\textbf{(E) }264\)\par \vspace{0.5em}\item The number \(2024\) is written as the sum of not necessarily distinct two-digit numbers. What is the least number of two-digit numbers needed to write this sum?

\(\textbf{(A) }20\qquad\textbf{(B) }21\qquad\textbf{(C) }22\qquad\textbf{(D) }23\qquad\textbf{(E) }24\)\par \vspace{0.5em}\item What is the least value of \(n\) such that \(n!\) is a multiple of \(2024\)?

\(
\textbf{(A) }11 \qquad
\textbf{(B) }21 \qquad
\textbf{(C) }22 \qquad
\textbf{(D) }23 \qquad
\textbf{(E) }253 \qquad
\)\par \vspace{0.5em}\item A data set containing \(20\) numbers, some of which are \(6\), has mean \(45\). When all the 6s are removed, the data set has mean \(66\). How many 6s were in the original data set?

\(\textbf{(A) }4\qquad\textbf{(B) }5\qquad\textbf{(C) }6\qquad\textbf{(D) }7\qquad\textbf{(E) }8\)\par \vspace{0.5em}\item The product of three integers is \(60\). What is the least possible positive sum of the three integers?

\(\textbf{(A) } 2 \qquad \textbf{(B) } 3 \qquad \textbf{(C) } 5 \qquad \textbf{(D) } 6 \qquad \textbf{(E) } 13\)\par \vspace{0.5em}\item In \(\Delta ABC\), \(\angle ABC = 90^\circ\) and \(BA = BC = \sqrt{2}\). Points \(P_1, P_2, \dots, P_{2024}\) lie on hypotenuse \(\overline{AC}\) so that \(AP_1= P_1P_2 = P_2P_3 = \dots = P_{2023}P_{2024} = P_{2024}C\). What is the length of the vector sum

\begin{equation*}
\overrightarrow{BP_1} + \overrightarrow{BP_2} + \overrightarrow{BP_3} + \dots + \overrightarrow{BP_{2024}}?
\end{equation*}

\(
\textbf{(A) }1011 \qquad
\textbf{(B) }1012 \qquad
\textbf{(C) }2023 \qquad
\textbf{(D) }2024 \qquad
\textbf{(E) }2025 \qquad
\)\par \vspace{0.5em}\item How many angles \(\theta\) with \(0\le\theta\le2\pi\) satisfy \(\log(\sin(3\theta))+\log(\cos(2\theta))=0\)?  

\( \textbf{(A) }0 \qquad \textbf{(B) }1 \qquad \textbf{(C) }2 \qquad \textbf{(D) }3 \qquad \textbf{(E) }4 \qquad \)\par \vspace{0.5em}\item Let \(M\) be the greatest integer such that both \(M + 1213\) and \(M + 3773\) are perfect squares. What is the units digit of \(M\)?

\(
\textbf{(A) }1 \qquad
\textbf{(B) }2 \qquad
\textbf{(C) }3 \qquad
\textbf{(D) }6 \qquad
\textbf{(E) }8 \qquad
\)\par \vspace{0.5em}\item Let \(\alpha\) be the radian measure of the smallest angle in a \(3{-}4{-}5\) right triangle. Let \(\beta\) be the radian measure of the smallest angle in a \(7{-}24{-}25\) right triangle. In terms of \(\alpha\), what is \(\beta\)?

\(
\textbf{(A) }\frac{\alpha}{3}\qquad
\textbf{(B) }\alpha - \frac{\pi}{8}\qquad
\textbf{(C) }\frac{\pi}{2} - 2\alpha \qquad
\textbf{(D) }\frac{\alpha}{2}\qquad
\textbf{(E) }\pi - 4\alpha\qquad
\)\par \vspace{0.5em}\item There are exactly \(K\) positive integers \(b\) with \(5 \leq b \leq 2024\) such that the base-\(b\) integer \(2024_b\) is divisible by \(16\) (where \(16\) is in base ten). What is the sum of the digits of \(K\)?

\(\textbf{(A) }16\qquad\textbf{(B) }17\qquad\textbf{(C) }18\qquad\textbf{(D) }20\qquad\textbf{(E) }21\)\par \vspace{0.5em}\item The first three terms of a geometric sequence are the integers \(a,\,720,\) and \(b,\) where \(a<720<b.\) What is the sum of the digits of the least possible value of \(b?\)

\(\textbf{(A) } 9 \qquad \textbf{(B) } 12 \qquad \textbf{(C) } 16 \qquad \textbf{(D) } 18 \qquad \textbf{(E) } 21\)\par \vspace{0.5em}\item The graph of \(y=e^{x+1}+e^{-x}-2\) has an axis of symmetry. What is the reflection of the point \((-1,\tfrac{1}{2})\) over this axis?

\(\textbf{(A) }\left(-1,-\frac{3}{2}\right)\qquad\textbf{(B) }(-1,0)\qquad\textbf{(C) }\left(-1,\tfrac{1}{2}\right)\qquad\textbf{(D) }\left(0,\frac{1}{2}\right)\qquad\textbf{(E) }\left(3,\frac{1}{2}\right)\)\par \vspace{0.5em}\item The numbers, in order, of each row and the numbers, in order, of each column of a \(5 \times 5\) array of integers form an arithmetic progression of length \(5{.}\) The numbers in positions \((5, 5), \,(2,4),\,(4,3),\) and \((3, 1)\) are \(0, 48, 16,\) and \(12{,}\) respectively. What number is in position \((1, 2)?\)

\begin{equation*}
\begin{bmatrix} . & ? &.&.&. \\ .&.&.&48&.\\ 12&.&.&.&.\\ .&.&16&.&.\\ .&.&.&.&0\end{bmatrix}
\end{equation*}

\(\textbf{(A) } 19 \qquad \textbf{(B) } 24 \qquad \textbf{(C) } 29 \qquad \textbf{(D) } 34 \qquad \textbf{(E) } 39\)\par \vspace{0.5em}\item The roots of \(x^3 + 2x^2 - x + 3\) are \(p, q,\) and \(r.\) What is the value of 
\begin{equation*}
(p^2 + 4)(q^2 + 4)(r^2 + 4)?
\end{equation*}

\(\textbf{(A) } 64 \qquad \textbf{(B) } 75 \qquad \textbf{(C) } 100 \qquad \textbf{(D) } 125 \qquad \textbf{(E) } 144\)\par \vspace{0.5em}\item A set of \(12\) tokens ---- \(3\) red, \(2\) white, \(1\) blue, and \(6\) black ---- is to be distributed at random to \(3\) game players, \(4\) tokens per player. The probability that some player gets all the red tokens, another gets all the white tokens, and the remaining player gets the blue token can be written as \(\frac{m}{n}\), where \(m\) and \(n\) are relatively prime positive integers. What is \(m+n\)?

\(
\textbf{(A) }387 \qquad
\textbf{(B) }388 \qquad
\textbf{(C) }389 \qquad
\textbf{(D) }390 \qquad
\textbf{(E) }391 \qquad
\)\par \vspace{0.5em}\item Integers \(a\), \(b\), and \(c\) satisfy \(ab + c = 100\), \(bc + a = 87\), and \(ca + b = 60\). What is \(ab + bc + ca\)?

\(
\textbf{(A) }212 \qquad
\textbf{(B) }247 \qquad
\textbf{(C) }258 \qquad
\textbf{(D) }276 \qquad
\textbf{(E) }284 \qquad
\)\par \vspace{0.5em}\item On top of a rectangular card with sides of length \(1\) and \(2+\sqrt{3}\), an identical card is placed so that two of their diagonals line up, as shown (\(\overline{AC}\), in this case).


\begin{center}
\begin{asy}
import olympiad;
import cse5;
defaultpen(fontsize(12)+0.85); size(150);
real h=2.25;
pair C=origin,B=(0,h),A=(1,h),D=(1,0),Dp=reflect(A,C)*D,Bp=reflect(A,C)*B;
pair L=extension(A,Dp,B,C),R=extension(Bp,C,A,D);
draw(L--B--A--Dp--C--Bp--A);
draw(C--D--R);
draw(L--C^^R--A,dashed+0.6);
draw(A--C,black+0.6);
dot("$C$",C,2*dir(C-R)); dot("$A$",A,1.5*dir(A-L)); dot("$B$",B,dir(B-R));
\end{asy}
\end{center}


Continue the process, adding a third card to the second, and so on, lining up successive diagonals after rotating clockwise. In total, how many cards must be used until a vertex of a new card lands exactly on the vertex labeled \(B\) in the figure?

\(\textbf{(A) }6\qquad\textbf{(B) }8\qquad\textbf{(C) }10\qquad\textbf{(D) }12\qquad\textbf{(E) }\text{No new vertex will land on }B.\)\par \vspace{0.5em}\item Cyclic quadrilateral \(ABCD\) has lengths \(BC=CD=3\) and \(DA=5\) with \(\angle CDA=120^\circ\). What is the length of the shorter diagonal of \(ABCD\)?

\(
\textbf{(A) }\frac{31}7 \qquad
\textbf{(B) }\frac{33}7 \qquad
\textbf{(C) }5 \qquad
\textbf{(D) }\frac{39}7 \qquad
\textbf{(E) }\frac{41}7 \qquad
\)\par \vspace{0.5em}\item Points \(P\) and \(Q\) are chosen uniformly and independently at random on sides \(\overline {AB}\) and \(\overline{AC},\) respectively, of equilateral triangle \(\Delta ABC.\) Which of the following intervals contains the probability that the area of \(\triangle APQ\) is less than half the area of \(\triangle ABC?\)

\(\textbf{(A) } \left[\frac 38, \frac 12\right] \qquad \textbf{(B) } \left(\frac 12, \frac 23\right] \qquad \textbf{(C) } \left(\frac 23, \frac 34\right] \qquad \textbf{(D) } \left(\frac 34, \frac 78\right] \qquad \textbf{(E) } \left(\frac 78, 1\right]\)\par \vspace{0.5em}\item Suppose that \(a_1 = 2\) and the sequence \((a_n)\) satisfies the recurrence relation 
\begin{equation*}
\frac{a_n -1}{n-1}=\frac{a_{n-1}+1}{(n-1)+1}
\end{equation*}
for all \(n \ge 2.\) What is the greatest integer less than or equal to 
\begin{equation*}
\sum^{100}_{n=1} a_n^2?
\end{equation*}

\(\textbf{(A) } 338{,}550 \qquad \textbf{(B) } 338{,}551 \qquad \textbf{(C) } 338{,}552 \qquad \textbf{(D) } 338{,}553 \qquad \textbf{(E) } 338{,}554\)\par \vspace{0.5em}\item The figure below shows a dotted grid \(8\) cells wide and \(3\) cells tall consisting of \(1''\times1''\) squares. Carl places \(1\)-inch toothpicks along some of the sides of the squares to create a closed loop that does not intersect itself. The numbers in the cells indicate the number of sides of that square that are to be covered by toothpicks, and any number of toothpicks are allowed if no number is written. In how many ways can Carl place the toothpicks?


\begin{center}
\begin{asy}
import olympiad;
import cse5;
size(6cm);
for (int i=0; i<9; ++i) {
  draw((i,0)--(i,3),dotted);
}
for (int i=0; i<4; ++i){
  draw((0,i)--(8,i),dotted);
}
for (int i=0; i<8; ++i) {
  for (int j=0; j<3; ++j) {
    if (j==1) {
      label("1",(i+0.5,1.5));
}}}
\end{asy}
\end{center}


\(\textbf{(A) }130\qquad\textbf{(B) }144\qquad\textbf{(C) }146\qquad\textbf{(D) }162\qquad\textbf{(E) }196\)\par \vspace{0.5em}\item What is the value of 


\begin{equation*}
\tan^2 \frac {\pi}{16} \cdot \tan^2 \frac {3\pi}{16}~ + ~ \tan^2 \frac {\pi}{16} \cdot \tan^2 \frac {5\pi}{16} ~+~\tan^2 \frac {3\pi}{16} \cdot \tan^2 \frac {7\pi}{16} ~+~ \tan^2 \frac {5\pi}{16} \cdot \tan^2 \frac {7\pi}{16}?
\end{equation*}


\(\textbf{(A) } 28 \qquad \textbf{(B) } 68 \qquad \textbf{(C) } 70 \qquad \textbf{(D) } 72 \qquad \textbf{(E) } 84\)\par \vspace{0.5em}\item A \(\textit{disphenoid}\) is a tetrahedron whose triangular faces are congruent to one another. What is the least total surface area of a disphenoid whose faces are scalene triangles with integer side lengths?

\(\textbf{(A) }\sqrt{3}\qquad\textbf{(B) }3\sqrt{15}\qquad\textbf{(C) }15\qquad\textbf{(D) }15\sqrt{7}\qquad\textbf{(E) }24\sqrt{6}\)\par \vspace{0.5em}\item A graph is \(\textit{symmetric}\) about a line if the graph remains unchanged after reflection in that line. For how many quadruples of integers \((a,b,c,d)\), where \(|a|,|b|,|c|,|d|\le5\) and \(c\) and \(d\) are not both \(0\), is the graph of 
\begin{equation*}
y=\frac{ax+b}{cx+d}
\end{equation*}
symmetric about the line \(y=x\)?

\(\textbf{(A) }1282\qquad\textbf{(B) }1292\qquad\textbf{(C) }1310\qquad\textbf{(D) }1320\qquad\textbf{(E) }1330\)\par \vspace{0.5em}\end{enumerate}\newpage\section*{2024 AMC 12B Problems}
\addcontentsline{toc}{section}{2024 AMC 12B Problems}
\begin{enumerate}[label=\arabic*., itemsep=0.5em]\item In a long line of people arranged left to right, the 1013th person from the left is also the 1010th person from the right. How many people are in the line?

\(\textbf{(A) } 2021 \qquad\textbf{(B) } 2022 \qquad\textbf{(C) } 2023 \qquad\textbf{(D) } 2024 \qquad\textbf{(E) } 2025\)\par \vspace{0.5em}\item What is \(10! - 7! \cdot 6!\)?

\(\textbf{(A) }-120 \qquad\textbf{(B) }0 \qquad\textbf{(C) }120 \qquad\textbf{(D) }600 \qquad\textbf{(E) }720 \qquad\)\par \vspace{0.5em}\item For how many integer values of \(x\) is \(|2x|\leq 7\pi?\)

\(\textbf{(A) }16 \qquad\textbf{(B) }17\qquad\textbf{(C) }19\qquad\textbf{(D) }20\qquad\textbf{(E) }21\)\par \vspace{0.5em}\item Balls numbered \(1,2,3,\ldots\) are deposited in \(5\) bins, labeled \(A,B,C,D,\) and \(E\), using the following procedure. Ball \(1\) is deposited in bin \(A\), and balls \(2\) and \(3\) are deposited in \(B\). The next three balls are deposited in bin \(C\), the next \(4\) in bin \(D\), and so on, cycling back to bin \(A\) after balls are deposited in bin \(E\). (For example, \(22,23,\ldots,28\) are deposited in bin \(B\) at step 7 of this process.) In which bin is ball \(2024\) deposited?

\(\textbf{(A) }A\qquad\textbf{(B) }B\qquad\textbf{(C) }C\qquad\textbf{(D) }D\qquad\textbf{(E) }E\)\par \vspace{0.5em}\item In the following expression, Melanie changed some of the plus signs to minus signs:
\begin{equation*}
1 + 3+5+7+\cdots+97+99
\end{equation*}
When the new expression was evaluated, it was negative. What is the least number of plus signs that Melanie could have changed to minus signs?

\(
\textbf{(A) }14 \qquad
\textbf{(B) }15 \qquad
\textbf{(C) }16 \qquad
\textbf{(D) }17 \qquad
\textbf{(E) }18 \qquad
\)\par \vspace{0.5em}\item The national debt of the United States is on track to reach \(5 \cdot 10^{13}\) dollars by \(2033\). How many digits does this number of dollars have when written as a numeral in base \(5\)? (The approximation of \(\log_{10} 5\) as \(0.7\) is sufficient for this problem.)

\(
\textbf{(A) }18 \qquad
\textbf{(B) }20 \qquad
\textbf{(C) }22 \qquad
\textbf{(D) }24 \qquad
\textbf{(E) }26 \qquad
\)\par \vspace{0.5em}\item In the figure below \(WXYZ\) is a rectangle with \(WX=4\) and \(WZ=8\). Point \(M\) lies \(\overline{XY}\), point \(A\) lies on \(\overline{YZ}\), and \(\angle WMA\) is a right angle. The areas of \(\triangle WXM\) and \(\triangle WAZ\) are equal. What is the area of \(\triangle WMA\)?


\begin{center}
\begin{asy}
import olympiad;
import cse5;
pair X = (0, 0);
pair W = (0, 4);
pair Y = (8, 0);
pair Z = (8, 4);
label("$X$", X, dir(180));
label("$W$", W, dir(180));
label("$Y$", Y, dir(0));
label("$Z$", Z, dir(0));

draw(W--X--Y--Z--cycle);
dot(X);
dot(Y);
dot(W);
dot(Z);
pair M = (2, 0);
pair A = (8, 3);
label("$A$", A, dir(0));
dot(M);
dot(A);
draw(W--M--A--cycle);
markscalefactor = 0.05;
draw(rightanglemark(W, M, A));
label("$M$", M, dir(-90));
\end{asy}
\end{center}


\(
\textbf{(A) }13 \qquad
\textbf{(B) }14 \qquad
\textbf{(C) }15 \qquad
\textbf{(D) }16 \qquad
\textbf{(E) }17 \qquad
\)\par \vspace{0.5em}\item What value of \(x\) satisfies
\begin{equation*}
\frac{\log_2x\cdot\log_3x}{\log_2x+\log_3x}=2?
\end{equation*}

\(
\textbf{(A) }25\qquad
\textbf{(B) }32\qquad
\textbf{(C) }36\qquad
\textbf{(D) }42\qquad
\textbf{(E) }48\qquad
\)\par \vspace{0.5em}\item A dartboard is the region \(B\) in the coordinate plane consisting of points \((x,y)\) such that \(|x| + |y| \le 8\) . A target \(T\) is the region where \((x^2 + y^2 - 25)^2 \le 49.\) A dart is thrown and lands at a random point in \(B\). The probability that the dart lands in \(T\) can be expressed as \(\frac{m}{n} \cdot \pi,\) where \(m\) and \(n\) are relatively prime positive integers. What is \(m + n?\)

\(
\textbf{(A) }39 \qquad
\textbf{(B) }71 \qquad
\textbf{(C) }73 \qquad
\textbf{(D) }75 \qquad
\textbf{(E) }135 \qquad
\)\par \vspace{0.5em}\item A list of 9 real numbers consists of \(1\), \(2.2 \), \(3.2 \), \(5.2 \), \(6.2 \), and \(7\), as well as \(x, y,z\) with \(x\leq y\leq z\). The range of the list is \(7\), and the mean and median are both positive integers. How many ordered triples \((x,y,z)\) are possible?

\(\textbf{(A) }1 \qquad\textbf{(B) }2 \qquad\textbf{(C) }3 \qquad\textbf{(D) }4 \qquad\textbf{(E) }\text{infinitely many}\qquad\)\par \vspace{0.5em}\item Let \(x_{n} = \sin^2(n^\circ)\). What is the mean of \(x_{1}, x_{2}, x_{3}, \cdots, x_{90}\)?

\(
\textbf{(A) }\frac{11}{45} \qquad
\textbf{(B) }\frac{22}{45} \qquad
\textbf{(C) }\frac{89}{180} \qquad
\textbf{(D) }\frac{1}{2} \qquad
\textbf{(E) }\frac{91}{180} \qquad
\)\par \vspace{0.5em}\item Suppose \(z\) is a complex number with positive imaginary part, with real part greater than \(1\), and with \(|z| = 2\). In the complex plane, the four points \(0\), \(z\), \(z^{2}\), and \(z^{3}\) are the vertices of a quadrilateral with area \(15\). What is the imaginary part of \(z\)?

\(\textbf{(A)}~\frac{3}{4}\qquad\textbf{(B)}~1\qquad\textbf{(C)}~\frac{4}{3}\qquad\textbf{(D)}~\frac{3}{2}\qquad\textbf{(E)}~\frac{5}{3}\)\par \vspace{0.5em}\item There are real numbers \(x,y,h\) and \(k\) that satisfy the system of equations
\begin{equation*}
x^2 + y^2 - 6x - 8y = h
\end{equation*}

\begin{equation*}
x^2 + y^2 - 10x + 4y = k
\end{equation*}
What is the minimum possible value of \(h+k\)?

\(
\textbf{(A) }-54 \qquad
\textbf{(B) }-46 \qquad
\textbf{(C) }-34 \qquad
\textbf{(D) }-16 \qquad
\textbf{(E) }16 \qquad
\)\par \vspace{0.5em}\item How many different remainders can result when the \(100\)th power of an integer is divided by \(125\)?

\(\textbf{(A) }1 \qquad\textbf{(B) }2 \qquad\textbf{(C) }5 \qquad\textbf{(D) }25 \qquad\textbf{(E) }125 \qquad\)\par \vspace{0.5em}\item A triangle in the coordinate plane has vertices \(A(\log_21,\log_22)\), \(B(\log_23,\log_24)\), and \(C(\log_27,\log_28)\). What is the area of \(\triangle ABC\)?

\(
\textbf{(A) }\log_2\frac{\sqrt3}7\qquad
\textbf{(B) }\log_2\frac3{\sqrt7}\qquad
\textbf{(C) }\log_2\frac7{\sqrt3}\qquad
\textbf{(D) }\log_2\frac{11}{\sqrt7}\qquad
\textbf{(E) }\log_2\frac{11}{\sqrt3}\qquad
\)\par \vspace{0.5em}\item A group of \(16\) people will be partitioned into \(4\) indistinguishable \(4\)-person committees. Each committee will have one chairperson and one secretary. The number of different ways to make these assignments can be written as \(3^{r}M\), where \(r\) and \(M\) are positive integers and \(M\) is not divisible by \(3\). What is \(r\)?

\(
\textbf{(A) }5 \qquad
\textbf{(B) }6 \qquad
\textbf{(C) }7 \qquad
\textbf{(D) }8 \qquad
\textbf{(E) }9 \qquad\)\par \vspace{0.5em}\item Integers \(a\) and \(b\) are randomly chosen without replacement from the set of integers with absolute value not exceeding \(10\). What is the probability that the polynomial \(x^3 + ax^2 + bx + 6\) has \(3\) distinct integer roots?

\(\textbf{(A) }\frac{1}{240} \qquad \textbf{(B) }\frac{1}{221} \qquad \textbf{(C) }\frac{1}{105} \qquad \textbf{(D) }\frac{1}{84} \qquad \textbf{(E) }\frac{1}{63}\)\par \vspace{0.5em}\item The Fibonacci numbers are defined by \(F_1=1,\) \(F_2=1,\) and \(F_n=F_{n-1}+F_{n-2}\) for \(n\geq 3.\) What is
\begin{equation*}
\dfrac{F_2}{F_1}+\dfrac{F_4}{F_2}+\dfrac{F_6}{F_3}+\cdots+\dfrac{F_{20}}{F_{10}}?
\end{equation*}

\(\textbf{(A) }318 \qquad\textbf{(B) }319\qquad\textbf{(C) }320\qquad\textbf{(D) }321\qquad\textbf{(E) }322\)\par \vspace{0.5em}\item Equilateral \(\triangle ABC\) with side length \(14\) is rotated about its center by angle \(\theta\), where \(0 < \theta < 60^{\circ}\), to form \(\triangle DEF\). See the figure. The area of hexagon \(ADBECF\) is \(91\sqrt{3}\). What is \(\tan\theta\)?

\begin{center}
\begin{asy}
import olympiad;
import cse5;
// Credit to shihan for this diagram.

defaultpen(fontsize(13)); size(200);
pair O=(0,0),A=dir(225),B=dir(-15),C=dir(105),D=rotate(38.21,O)*A,E=rotate(38.21,O)*B,F=rotate(38.21,O)*C;
draw(A--B--C--A,gray+0.4);draw(D--E--F--D,gray+0.4); draw(A--D--B--E--C--F--A,black+0.9); dot(O); dot("$A$",A,dir(A)); dot("$B$",B,dir(B)); dot("$C$",C,dir(C)); dot("$D$",D,dir(D)); dot("$E$",E,dir(E)); dot("$F$",F,dir(F));
\end{asy}
\end{center}


\(\textbf{(A)}~\frac{3}{4}\qquad\textbf{(B)}~\frac{5\sqrt{3}}{11}\qquad\textbf{(C)}~\frac{4}{5}\qquad\textbf{(D)}~\frac{11}{13}\qquad\textbf{(E)}~\frac{7\sqrt{3}}{13}\)\par \vspace{0.5em}\item Suppose \(A\), \(B\), and \(C\) are points in the plane with \(AB=40\) and \(AC=42\), and let \(x\) be the length of the line segment from \(A\) to the midpoint of \(\overline{BC}\). Define a function \(f\) by letting \(f(x)\) be the area of \(\triangle ABC\). Then the domain of \(f\) is an open interval \((p,q)\), and the maximum value \(r\) of \(f(x)\) occurs at \(x=s\). What is \(p+q+r+s\)?

\(
\textbf{(A) }909\qquad
\textbf{(B) }910\qquad
\textbf{(C) }911\qquad
\textbf{(D) }912\qquad
\textbf{(E) }913\qquad
\)\par \vspace{0.5em}\item The measures of the smallest angles of three different right triangles sum to \(90^\circ\). All three triangles have side lengths that are primitive Pythagorean triples. Two of them are \(3-4-5\) and \(5-12-13\). What is the perimeter of the third triangle?

\(
\textbf{(A) }40 \qquad
\textbf{(B) }126 \qquad
\textbf{(C) }154 \qquad
\textbf{(D) }176 \qquad
\textbf{(E) }208 \qquad
\)\par \vspace{0.5em}\item Let \(\triangle{ABC}\) be a triangle with integer side lengths and the property that \(\angle{B} = 2\angle{A}\). What is the least possible perimeter of such a triangle?

\(
\textbf{(A) }13 \qquad
\textbf{(B) }14 \qquad
\textbf{(C) }15 \qquad
\textbf{(D) }16 \qquad
\textbf{(E) }17 \qquad
\)\par \vspace{0.5em}\item A right pyramid has regular octagon \(ABCDEFGH\) with side length \(1\) as its base and apex \(V.\) Segments \(\overline{AV}\) and \(\overline{DV}\) are perpendicular. What is the square of the height of the pyramid?

\(
\textbf{(A) }1 \qquad
\textbf{(B) }\frac{1+\sqrt2}{2} \qquad
\textbf{(C) }\sqrt2 \qquad
\textbf{(D) }\frac32 \qquad
\textbf{(E) }\frac{2+\sqrt2}{3} \qquad
\)\par \vspace{0.5em}\item What is the number of ordered triples \((a,b,c)\) of positive integers, with \(a\le b\le c\le 9\), such that there exists a (non-degenerate) triangle \(\triangle ABC\) with an integer inradius for which \(a\), \(b\), and \(c\) are the lengths of the altitudes from \(A\) to \(\overline{BC}\), \(B\) to \(\overline{AC}\), and \(C\) to \(\overline{AB}\), respectively? (Recall that the inradius of a triangle is the radius of the largest possible circle that can be inscribed in the triangle.)

\(
\textbf{(A) }2\qquad
\textbf{(B) }3\qquad
\textbf{(C) }4\qquad
\textbf{(D) }5\qquad
\textbf{(E) }6\qquad
\)\par \vspace{0.5em}\item Pablo will decorate each of \(6\) identical white balls with either a striped or a dotted pattern, using either red or blue paint. He will decide on the color and pattern for each ball by flipping a fair coin for each of the \(12\) decisions he must make. After the paint dries, he will place the \(6\) balls in an urn. Frida will randomly select one ball from the urn and note its color and pattern. The events "the ball Frida selects is red" and "the ball Frida selects is striped" may or may not be independent, depending on the outcome of Pablo's coin flips. The probability that these two events are independent can be written as \(\frac mn,\) where \(m\) and \(n\) are relatively prime positive integers. What is \(m?\) (Recall that two events \(A\) and \(B\) are independent if \(P(A \text{ and }B) = P(A) \cdot P(B).\))

\(\textbf{(A) } 243 \qquad \textbf{(B) } 245 \qquad \textbf{(C) } 247 \qquad \textbf{(D) } 249\qquad \textbf{(E) } 251\)\par \vspace{0.5em}\end{enumerate}\newpage
\end{document}
