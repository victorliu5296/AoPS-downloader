
\documentclass{article}
\usepackage{amsmath, amssymb}
\usepackage{geometry}
\geometry{a4paper, margin=0.75in}
\usepackage{enumitem}
\usepackage[hypertexnames=true, linktoc=all]{hyperref}
\usepackage{fancyhdr}
\usepackage{tikz}
\usepackage{graphicx}
\usepackage{asymptote}
\usepackage{arcs}
\usepackage{xwatermark}
\begin{asydef}
  // Global Asymptote settings
  settings.outformat = "pdf";
  settings.render = 0;
  settings.prc = false;
  import olympiad;
  import cse5;
  size(8cm);
\end{asydef}
\pagestyle{fancy}
\fancyhead[L]{\textbf{AMC 8 Problems}}
\fancyhead[R]{\textbf{2018}}
\fancyfoot[C]{\thepage}
\renewcommand{\headrulewidth}{0.4pt}
\renewcommand{\footrulewidth}{0.4pt}

\title{AMC 8 Problems \\ 2018}
\date{}
\begin{document}\maketitle\thispagestyle{fancy}\newpage\section*{2018 AMC 8}
\begin{enumerate}[label=\arabic*., itemsep=0.5em]
\item An amusement park has a collection of scale models, with ratio \(1 : 20\), of buildings and other sights from around the country. The height of the United States Capitol is 289 feet. What is the height in feet of its replica to the nearest whole number?

\(\textbf{(A) }14\qquad\textbf{(B) }15\qquad\textbf{(C) }16\qquad\textbf{(D) }18\qquad\textbf{(E) }20\)\par \vspace{0.5em}\item What is the value of the product
\begin{equation*}
\left(1+\frac{1}{1}\right)\cdot\left(1+\frac{1}{2}\right)\cdot\left(1+\frac{1}{3}\right)\cdot\left(1+\frac{1}{4}\right)\cdot\left(1+\frac{1}{5}\right)\cdot\left(1+\frac{1}{6}\right)?
\end{equation*}


\(\textbf{(A) }\frac{7}{6}\qquad\textbf{(B) }\frac{4}{3}\qquad\textbf{(C) }\frac{7}{2}\qquad\textbf{(D) }7\qquad\textbf{(E) }8\)\par \vspace{0.5em}\item Students Arn, Bob, Cyd, Dan, Eve, and Fon are arranged in that order in a circle. They start counting: Arn first, then Bob, and so forth. When the number contains a 7 as a digit (such as 47) or is a multiple of 7 that person leaves the circle and the counting continues. Who is the last one present in the circle?

\(\textbf{(A) } \text{Arn}\qquad\textbf{(B) }\text{Bob}\qquad\textbf{(C) }\text{Cyd}\qquad\textbf{(D) }\text{Dan}\qquad \textbf{(E) }\text{Eve}\)\par \vspace{0.5em}\item The twelve-sided figure shown has been drawn on \(1 \text{ cm}\times 1 \text{ cm}\) graph paper. What is the area of the figure in \(\text{cm}^2\)?


\begin{center}
\begin{asy}
import olympiad;
import cse5;
unitsize(8mm);
for (int i=0; i<7; ++i) {
  draw((i,0)--(i,7),gray);
  draw((0,i+1)--(7,i+1),gray);
}
draw((1,3)--(2,4)--(2,5)--(3,6)--(4,5)--(5,5)--(6,4)--(5,3)--(5,2)--(4,1)--(3,2)--(2,2)--cycle,black+2bp);
\end{asy}
\end{center}


\(\textbf{(A) } 12 \qquad \textbf{(B) } 12.5 \qquad \textbf{(C) } 13 \qquad \textbf{(D) } 13.5 \qquad \textbf{(E) } 14\)\par \vspace{0.5em}\item What is the value of \(1+3+5+\cdots+2017+2019-2-4-6-\cdots-2016-2018\)?

\(\textbf{(A) }-1010\qquad\textbf{(B) }-1009\qquad\textbf{(C) }1008\qquad\textbf{(D) }1009\qquad \textbf{(E) }1010\)\par \vspace{0.5em}\item On a trip to the beach, Anh traveled 50 miles on the highway and 10 miles on a coastal access road. He drove three times as fast on the highway as on the coastal road. If Anh spent 30 minutes driving on the coastal road, how many minutes did his entire trip take?

\(\textbf{(A) }50\qquad\textbf{(B) }70\qquad\textbf{(C) }80\qquad\textbf{(D) }90\qquad \textbf{(E) }100\)\par \vspace{0.5em}\item The \(5\)-digit number \(\underline{2}\) \(\underline{0}\) \(\underline{1}\) \(\underline{8}\) \(\underline{U}\) is divisible by \(9\). What is the remainder when this number is divided by \(8\)?

\(\textbf{(A) }1\qquad\textbf{(B) }3\qquad\textbf{(C) }5\qquad\textbf{(D) }6\qquad\textbf{(E) }7\)\par \vspace{0.5em}\item Mr. Garcia asked the members of his health class how many days last week they exercised for at least 30 minutes. The results are summarized in the following bar graph, where the heights of the bars represent the number of students.


\begin{center}
\begin{asy}
import olympiad;
import cse5;
size(8cm);
void drawbar(real x, real h) {
  fill((x-0.15,0.5)--(x+0.15,0.5)--(x+0.15,h)--(x-0.15,h)--cycle,gray);
}
draw((0.5,0.5)--(7.5,0.5)--(7.5,5)--(0.5,5)--cycle);
for (real i=1; i<5; i=i+0.5) {
  draw((0.5,i)--(7.5,i),gray);
}
drawbar(1.0,1.0);
drawbar(2.0,2.0);
drawbar(3.0,1.5);
drawbar(4.0,3.5);
drawbar(5.0,4.5);
drawbar(6.0,2.0);
drawbar(7.0,1.5);
for (int i=1; i<8; ++i) {
  label("$"+string(i)+"$",(i,0.25));
}
for (int i=1; i<9; ++i) {
  label("$"+string(i)+"$",(0.5,0.5*(i+1)),W);
}
label("Number of Days of Exercise",(4,-0.1));
label(rotate(90)*"Number of Students",(-0.1,2.75));
\end{asy}
\end{center}

What was the mean number of days of exercise last week, rounded to the nearest hundredth, reported by the students in Mr. Garcia's class?

\(\textbf{(A) } 3.50 \qquad \textbf{(B) } 3.57 \qquad \textbf{(C) } 4.36 \qquad \textbf{(D) } 4.50 \qquad \textbf{(E) } 5.00\)\par \vspace{0.5em}\item Jenica is tiling the floor of her 12 foot by 16 foot living room. She plans to place one-foot by one-foot square tiles to form a border along the edges of the room and to fill in the rest of the floor with two-foot by two-foot square tiles. How many tiles will she use?

\(\textbf{(A) }48\qquad\textbf{(B) }87\qquad\textbf{(C) }91\qquad\textbf{(D) }96\qquad \textbf{(E) }120\)\par \vspace{0.5em}\item The \(\emph{harmonic mean}\) of a set of non-zero numbers is the reciprocal of the average of the reciprocals of the numbers. What is the harmonic mean of 1, 2, and 4?

\(\textbf{(A) }\frac{3}{7}\qquad\textbf{(B) }\frac{7}{12}\qquad\textbf{(C) }\frac{12}{7}\qquad\textbf{(D) }\frac{7}{4}\qquad \textbf{(E) }\frac{7}{3}\)\par \vspace{0.5em}\item Abby, Bridget, and four of their classmates will be seated in two rows of three for a group picture, as shown.


\begin{eqnarray*}
\text{X}&\quad\text{X}\quad&\text{X} \\
\text{X}&\quad\text{X}\quad&\text{X} 
\end{eqnarray*}


If the seating positions are assigned randomly, what is the probability that Abby and Bridget are adjacent to each other in the same row or the same column?

\(\textbf{(A) } \frac{1}{3} \qquad \textbf{(B) } \frac{2}{5} \qquad \textbf{(C) } \frac{7}{15} \qquad \textbf{(D) } \frac{1}{2} \qquad \textbf{(E) } \frac{2}{3}\)\par \vspace{0.5em}\item The clock in Sri's car, which is not accurate, gains time at a constant rate. One day as he begins shopping, he notes that his car clock and his watch (which is accurate) both say 12:00 noon. When he is done shopping, his watch says 12:30 and his car clock says 12:35. Later that day, Sri loses his watch. He looks at his car clock and it says 7:00. What is the actual time?
\(\textbf{(A) }5:50\qquad\textbf{(B) }6:00\qquad\textbf{(C) }6:30\qquad\textbf{(D) }6:55\qquad \textbf{(E) }8:10\)\par \vspace{0.5em}\item Laila took five math tests, each worth a maximum of 100 points. Laila's score on each test was an integer between 0 and 100, inclusive. Laila received the same score on the first four tests, and she received a higher score on the last test. Her average score on the five tests was 82. How many values are possible for Laila's score on the last test?

\(\textbf{(A) }4\qquad\textbf{(B) }5\qquad\textbf{(C) }9\qquad\textbf{(D) }10\qquad \textbf{(E) }18\)\par \vspace{0.5em}\item Let \(N\) be the greatest five-digit number whose digits have a product of \(120\). What is the sum of the digits of \(N\)?

\(\textbf{(A) }15\qquad\textbf{(B) }16\qquad\textbf{(C) }17
\qquad\textbf{(D) }18\qquad\textbf{(E) }20\)\par \vspace{0.5em}\item In the diagram below, a diameter of each of the two smaller circles is a radius of the larger circle. If the two smaller circles have a combined area of \(1\) square unit, then what is the area of the shaded region, in square units?


\begin{center}
\begin{asy}
import olympiad;
import cse5;
size(4cm);
filldraw(scale(2)*unitcircle,gray,black);
filldraw(shift(-1,0)*unitcircle,white,black);
filldraw(shift(1,0)*unitcircle,white,black);
\end{asy}
\end{center}


\(\textbf{(A) } \frac{1}{4} \qquad \textbf{(B) } \frac{1}{3} \qquad \textbf{(C) } \frac{1}{2} \qquad \textbf{(D) } 1 \qquad \textbf{(E) } \frac{\pi}{2}\)\par \vspace{0.5em}\item Professor Chang has nine different language books lined up on a bookshelf: two Arabic, three German, and four Spanish. How many ways are there to arrange the nine books on the shelf keeping the Arabic books together and keeping the Spanish books together?

\(\textbf{(A) }1440\qquad\textbf{(B) }2880\qquad\textbf{(C) }5760\qquad\textbf{(D) }182,440\qquad \textbf{(E) }362,880\)\par \vspace{0.5em}\item Bella begins to walk from her house toward her friend Ella's house. At the same time, Ella begins to ride her bicycle toward Bella's house. They each maintain a constant speed, and Ella rides 5 times as fast as Bella walks. The distance between their houses is \(2\) miles, which is \(10,560\) feet, and Bella covers \(2 \tfrac{1}{2}\) feet with each step. How many steps will Bella take by the time she meets Ella?

\(\textbf{(A) }704\qquad\textbf{(B) }845\qquad\textbf{(C) }1056\qquad\textbf{(D) }1760\qquad \textbf{(E) }3520\)\par \vspace{0.5em}\item How many positive factors does \(23,232\) have?

\(\textbf{(A) }9\qquad\textbf{(B) }12\qquad\textbf{(C) }28\qquad\textbf{(D) }36\qquad\textbf{(E) }42\)\par \vspace{0.5em}\item In a sign pyramid a cell gets a "+" if the two cells below it have the same sign, and it gets a "-" if the two cells below it have different signs. The diagram below illustrates a sign pyramid with four levels. How many possible ways are there to fill the four cells in the bottom row to produce a "+" at the top of the pyramid?


\begin{center}
\begin{asy}
import olympiad;
import cse5;
unitsize(2cm);
path box = (-0.5,-0.2)--(-0.5,0.2)--(0.5,0.2)--(0.5,-0.2)--cycle;
draw(box); label("$+$",(0,0));
draw(shift(1,0)*box); label("$-$",(1,0));
draw(shift(2,0)*box); label("$+$",(2,0));
draw(shift(3,0)*box); label("$-$",(3,0));
draw(shift(0.5,0.4)*box); label("$-$",(0.5,0.4));
draw(shift(1.5,0.4)*box); label("$-$",(1.5,0.4));
draw(shift(2.5,0.4)*box); label("$-$",(2.5,0.4));
draw(shift(1,0.8)*box); label("$+$",(1,0.8));
draw(shift(2,0.8)*box); label("$+$",(2,0.8));
draw(shift(1.5,1.2)*box); label("$+$",(1.5,1.2));
\end{asy}
\end{center}


\(\textbf{(A) } 2 \qquad \textbf{(B) } 4 \qquad \textbf{(C) } 8 \qquad \textbf{(D) } 12 \qquad \textbf{(E) } 16\)\par \vspace{0.5em}\item In \(\triangle ABC,\) a point \(E\) is on \(\overline{AB}\) with \(AE=1\) and \(EB=2.\) Point \(D\) is on \(\overline{AC}\) so that \(\overline{DE} \parallel \overline{BC}\) and point \(F\) is on \(\overline{BC}\) so that \(\overline{EF} \parallel \overline{AC}.\) What is the ratio of the area of \(CDEF\) to the area of \(\triangle ABC?\)


\begin{center}
\begin{asy}
import olympiad;
import cse5;
size(7cm);
pair A,B,C,DD,EE,FF;
A = (0,0); B = (3,0); C = (0.5,2.5);
EE = (1,0);
DD = intersectionpoint(A--C,EE--EE+(C-B));
FF = intersectionpoint(B--C,EE--EE+(C-A));
draw(A--B--C--A--DD--EE--FF,black+1bp);
label("$A$",A,S); label("$B$",B,S); label("$C$",C,N);
label("$D$",DD,W); label("$E$",EE,S); label("$F$",FF,NE);
label("$1$",(A+EE)/2,S); label("$2$",(EE+B)/2,S);
\end{asy}
\end{center}


\(\textbf{(A) } \frac{4}{9} \qquad \textbf{(B) } \frac{1}{2} \qquad \textbf{(C) } \frac{5}{9} \qquad \textbf{(D) } \frac{3}{5} \qquad \textbf{(E) } \frac{2}{3}\)\par \vspace{0.5em}\item How many positive three-digit integers have a remainder of 2 when divided by 6, a remainder of 5 when divided by 9, and a remainder of 7 when divided by 11?

\(\textbf{(A) }1\qquad\textbf{(B) }2\qquad\textbf{(C) }3\qquad\textbf{(D) }4\qquad \textbf{(E) }5\)\par \vspace{0.5em}\item Point \(E\) is the midpoint of side \(\overline{CD}\) in square \(ABCD,\) and \(\overline{BE}\) meets diagonal \(\overline{AC}\) at \(F.\) The area of quadrilateral \(AFED\) is \(45.\) What is the area of \(ABCD?\)


\begin{center}
\begin{asy}
import olympiad;
import cse5;
size(5cm);
draw((0,0)--(6,0)--(6,6)--(0,6)--cycle);
draw((0,6)--(6,0)); draw((3,0)--(6,6));
label("$A$",(0,6),NW);
label("$B$",(6,6),NE);
label("$C$",(6,0),SE);
label("$D$",(0,0),SW);
label("$E$",(3,0),S);
label("$F$",(4,2),E);
\end{asy}
\end{center}


\(\textbf{(A) } 100 \qquad \textbf{(B) } 108 \qquad \textbf{(C) } 120 \qquad \textbf{(D) } 135 \qquad \textbf{(E) } 144\)\par \vspace{0.5em}\item From a regular octagon, a triangle is formed by connecting three randomly chosen vertices of the octagon. What is the probability that at least one of the sides of the triangle is also a side of the octagon?


\begin{center}
\begin{asy}
import olympiad;
import cse5;
size(3cm);
pair A[];
for (int i=0; i<9; ++i) {
A[i] = rotate(22.5+45*i)*(1,0);
}
filldraw(A[0]--A[1]--A[2]--A[3]--A[4]--A[5]--A[6]--A[7]--cycle,gray,black);
for (int i=0; i<8; ++i) { dot(A[i]); }
\end{asy}
\end{center}


\(\textbf{(A) } \frac{2}{7} \qquad \textbf{(B) } \frac{5}{42} \qquad \textbf{(C) } \frac{11}{14} \qquad \textbf{(D) } \frac{5}{7} \qquad \textbf{(E) } \frac{6}{7}\)\par \vspace{0.5em}\item In the cube \(ABCDEFGH\) with opposite vertices \(C\) and \(E,\) \(J\) and \(I\) are the midpoints of edges \(\overline{FB}\) and \(\overline{HD},\) respectively. Let \(R\) be the ratio of the area of the cross-section \(EJCI\) to the area of one of the faces of the cube. What is \(R^2?\)


\begin{center}
\begin{asy}
import olympiad;
import cse5;
size(6cm);
pair A,B,C,D,EE,F,G,H,I,J;
C = (0,0);
B = (-1,1);
D = (2,0.5);
A = B+D;
G = (0,2);
F = B+G;
H = G+D;
EE = G+B+D;
I = (D+H)/2; J = (B+F)/2;
filldraw(C--I--EE--J--cycle,lightgray,black);
draw(C--D--H--EE--F--B--cycle); 
draw(G--F--G--C--G--H);
draw(A--B,dashed); draw(A--EE,dashed); draw(A--D,dashed);
dot(A); dot(B); dot(C); dot(D); dot(EE); dot(F); dot(G); dot(H); dot(I); dot(J);
label("$A$",A,E);
label("$B$",B,W);
label("$C$",C,S);
label("$D$",D,E);
label("$E$",EE,N);
label("$F$",F,W);
label("$G$",G,N);
label("$H$",H,E);
label("$I$",I,E);
label("$J$",J,W);
\end{asy}
\end{center}


\(\textbf{(A) } \frac{5}{4} \qquad \textbf{(B) } \frac{4}{3} \qquad \textbf{(C) } \frac{3}{2} \qquad \textbf{(D) } \frac{25}{16} \qquad \textbf{(E) } \frac{9}{4}\)\par \vspace{0.5em}\item How many perfect cubes lie between \(2^8+1\) and \(2^{18}+1\), inclusive?

\(\textbf{(A) }4\qquad\textbf{(B) }9\qquad\textbf{(C) }10\qquad\textbf{(D) }57\qquad \textbf{(E) }58\)\par \vspace{0.5em}
\end{enumerate}

\end{document}
