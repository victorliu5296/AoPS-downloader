
\documentclass{article}
\usepackage{amsmath, amssymb}
\usepackage{geometry}
\geometry{a4paper, margin=0.75in}
\usepackage{enumitem}
\usepackage[hypertexnames=true, linktoc=all]{hyperref}
\usepackage{fancyhdr}
\usepackage{tikz}
\usepackage{graphicx}
\usepackage{asymptote}
\usepackage{arcs}
\usepackage{xwatermark}
\begin{asydef}
  // Global Asymptote settings
  settings.outformat = "pdf";
  settings.render = 0;
  settings.prc = false;
  import olympiad;
  import cse5;
  size(8cm);
\end{asydef}
\pagestyle{fancy}
\fancyhead[L]{\textbf{AMC 8 Problems}}
\fancyhead[R]{\textbf{2024}}
\fancyfoot[C]{\thepage}
\renewcommand{\headrulewidth}{0.4pt}
\renewcommand{\footrulewidth}{0.4pt}

\title{AMC 8 Problems \\ 2024}
\date{}
\begin{document}\maketitle\thispagestyle{fancy}\newpage\section*{2024 AMC 8}
\begin{enumerate}[label=\arabic*., itemsep=0.5em]
\item What is the unit's digit of: 
\begin{equation*}
222{,}222-22{,}222-2{,}222-222-22-2?
\end{equation*}

\(\textbf{(A) } 0\qquad\textbf{(B) } 2\qquad\textbf{(C) } 4\qquad\textbf{(D) } 8\qquad\textbf{(E) } 10\)\par \vspace{0.5em}\item What is the value of this expression in decimal form?

\begin{equation*}
\frac{44}{11} + \frac{110}{44} + \frac{44}{1100}
\end{equation*}

\(\textbf{(A) } 6.4\qquad\textbf{(B) } 6.504\qquad\textbf{(C) } 6.54\qquad\textbf{(D) } 6.9\qquad\textbf{(E) } 6.94\)\par \vspace{0.5em}\item Four squares of side length \(4, 7, 9,\) and \(10\) are arranged in increasing size order so that their left edges and bottom edges align. The squares alternate in color white-gray-white-gray, respectively, as shown in the figure. What is the area of the visible gray region in square units?

\begin{center}
\begin{asy}
import olympiad;
import cse5;
size(150);
filldraw((0,0)--(10,0)--(10,10)--(0,10)--cycle,gray(0.7),linewidth(1));
filldraw((0,0)--(9,0)--(9,9)--(0,9)--cycle,white,linewidth(1));
filldraw((0,0)--(7,0)--(7,7)--(0,7)--cycle,gray(0.7),linewidth(1));
filldraw((0,0)--(4,0)--(4,4)--(0,4)--cycle,white,linewidth(1));
draw((11,0)--(11,4),linewidth(1));
draw((11,6)--(11,10),linewidth(1));
label("$10$",(11,5),fontsize(14pt));
draw((10.75,0)--(11.25,0),linewidth(1));
draw((10.75,10)--(11.25,10),linewidth(1));
draw((0,11)--(3,11),linewidth(1));
draw((5,11)--(9,11),linewidth(1));
draw((0,11.25)--(0,10.75),linewidth(1));
draw((9,11.25)--(9,10.75),linewidth(1));
label("$9$",(4,11),fontsize(14pt));
draw((-1,0)--(-1,1),linewidth(1));
draw((-1,3)--(-1,7),linewidth(1));
draw((-1.25,0)--(-0.75,0),linewidth(1));
draw((-1.25,7)--(-0.75,7),linewidth(1));
label("$7$",(-1,2),fontsize(14pt));
draw((0,-1)--(1,-1),linewidth(1));
draw((3,-1)--(4,-1),linewidth(1));
draw((0,-1.25)--(0,-.75),linewidth(1));
draw((4,-1.25)--(4,-.75),linewidth(1));
label("$4$",(2,-1),fontsize(14pt));
\end{asy}
\end{center}

\(\textbf{(A)}\ 42 \qquad \textbf{(B)}\ 45\qquad \textbf{(C)}\ 49\qquad \textbf{(D)}\ 50\qquad \textbf{(E)}\ 52\)\par \vspace{0.5em}\item When Yunji added all the integers from \(1\) to \(9\), she mistakenly left out a number. Her sum turned out to be a square number. What number did Yunji leave out?

\(\textbf{(A) } 5\qquad\textbf{(B) } 6\qquad\textbf{(C) } 7\qquad\textbf{(D) } 8\qquad\textbf{(E) } 9\)\par \vspace{0.5em}\item Aaliyah rolls two standard 6-sided dice. She notices that the product of the two numbers rolled is a multiple of \(6\). Which of the following integers cannot be the sum of the two numbers?

\(\textbf{(A) } 5\qquad\textbf{(B) } 6\qquad\textbf{(C) } 7\qquad\textbf{(D) } 8\qquad\textbf{(E) } 9\)\par \vspace{0.5em}\item Sergei skated around an ice rink, gliding along different paths. The gray lines in the figures below show four of the paths labeled \(P\), \(Q\), \(R\), and \(S\). What is the sorted order of the four paths from shortest to longest?



\(\textbf{(A)}\ P,Q,R,S \qquad \textbf{(B)}\ P,R,S,Q \qquad \textbf{(C)}\ Q,S,P,R \qquad \textbf{(D)}\ R,P,S,Q \qquad \textbf{(E)}\ R,S,P,Q\)\par \vspace{0.5em}\item A \(3\times 7\) rectangle is covered without overlap by 3 shapes of tiles: \(2\times 2\), \(1\times 4\), and \(1\times 1\), shown below. What is the minimum possible number of \(1\times 1\) tiles used?



\(\textbf{(A) } 1\qquad\textbf{(B) } 2\qquad\textbf{(C) } 3\qquad\textbf{(D) } 4\qquad\textbf{(E) } 5\)\par \vspace{0.5em}\item On Monday Taye has \(\$2\). Every day, he either gains \(\$3\) or doubles the amount of money he had on the previous day. How many different dollar amounts could Taye have on Thursday, \(3\) days later?

\(\textbf{(A) } 3\qquad\textbf{(B) } 4\qquad\textbf{(C) } 5\qquad\textbf{(D) } 6\qquad\textbf{(E) } 7\)\par \vspace{0.5em}\item All the marbles in Maria's collection are red, green, or blue. Maria has half as many red marbles as green marbles and twice as many blue marbles as green marbles. Which of the following could be the total number of marbles in Maria's collection?

\(\textbf{(A) } 24\qquad\textbf{(B) } 25\qquad\textbf{(C) } 26\qquad\textbf{(D) } 27\qquad\textbf{(E) } 28\)\par \vspace{0.5em}\item In January 1980 the Moana Loa Observation recorded carbon dioxide \((CO_2)\) levels of 338 ppm (parts per million). Over the years the average \(CO_2\) reading has increased by about 1.515 ppm each year. What is the expected \(CO_2\) level in ppm in January 2030? Round your answer to the nearest integer.

\(\textbf{(A)}\ 399 \qquad \textbf{(B)}\ 414 \qquad \textbf{(C)}\ 420 \qquad \textbf{(D)}\ 444 \qquad \textbf{(E)}\ 459\)\par \vspace{0.5em}\item The coordinates of \(\triangle ABC\) are \(A(5,7)\), \(B(11,7)\), and \(C(3,y)\), with \(y>7\). The area of \(\triangle ABC\) is 12. What is the value of \(y\)?


\begin{center}
\begin{asy}
import olympiad;
import cse5;
// Diagram inaccurate to prevent measuring with ruler.
size(10cm);
draw((3,10)--(11,7)--(5,7)--(3,10));

dot((5,7));
label("$A(5,7)$",(5,7),S);
dot((11,7));
label("$B(11,7)$",(11,7),S);

dot((3,10));
label("$C(3,y)$",(3,10),NW);

// Problem 11: put on here by Andrei.martynau
\end{asy}
\end{center}


\(\textbf{(A) }8\qquad\textbf{(B) }9\qquad\textbf{(C) }10\qquad\textbf{(D) }11\qquad \textbf{(E) }12\)\par \vspace{0.5em}\item Rohan keeps a total of \(90\) guppies in \(4\) fish tanks.

*There is \(1\) more guppy in the \(2\)nd tank than the \(1\)st tank.

*There are \(2\) more guppies in the 3rd tank than the \(2\)nd tank.

*There are \(3\) more guppies in the 4th tank than the \(3\)rd tank. 

How many guppies are in the \(4\)th tank?

\(\textbf{(A)}\ 20 \qquad \textbf{(B)}\ 21 \qquad \textbf{(C)}\ 23 \qquad \textbf{(D)}\ 24 \qquad \textbf{(E)}\ 26\)\par \vspace{0.5em}\item Buzz Bunny is hopping up and down a set of stairs, one step at a time. In how many ways can Buzz Bunny start on the ground, make a sequence of \(6\) hops, and end up back on the ground?
(For example, one sequence of hops is up-up-down-down-up-down.)


\begin{center}
\begin{asy}
import olympiad;
import cse5;
/* AMC8 P13 2024, revised by Teacher David */
/**
 * This Geometry Artwork/Graph is designed using GeoSketch v1.0, 
 * a free software tool created by Tina Yan, William Zhong, and 
 * Teacher David.
 *
 * For more information, please refer to
 *   https://geosketch.org (under construction)
 */
defaultpen(linewidth(1pt));
unitsize(0.3pt);
import graph;
/**
 * Define a quadratic bezier curve function.
 */
typedef pair quad_bezier(real t);
quad_bezier fungen (pair a, pair b, pair c) {
  return new pair (real t) {
    real x = (1-t)*(1-t)*a.x + 2*(1-t)*t*b.x + t*t*c.x;
    real y = (1-t)*(1-t)*a.y + 2*(1-t)*t*b.y + t*t*c.y;
    return (x,y);
  };
}

quad_bezier qb0 = fungen((293,243),(237,276),(239,310));
draw(graph(qb0, 0,1));
quad_bezier qb1 = fungen((239,310),(274,301),(295,254));
draw(graph(qb1, 0,1));
quad_bezier qb2 = fungen((266,294),(260,309),(266,323));
draw(graph(qb2, 0,1));
quad_bezier qb3 = fungen((266,323),(294,311),(302,257));
draw(graph(qb3, 0,1));
quad_bezier qb4 = fungen((333,258),(341,249),(348,244));
draw(graph(qb4, 0,1));
quad_bezier qb5 = fungen((348,244),(355,241),(351,234));
draw(graph(qb5, 0,1));
quad_bezier qb6 = fungen((351,234),(348,226),(338,226));
draw(graph(qb6, 0,1));
quad_bezier qb7 = fungen((351,234),(350,219),(321,208));
draw(graph(qb7, 0,1));
quad_bezier qb8 = fungen((260,247),(135,293),(137,170));
draw(graph(qb8, 0,1));
quad_bezier qb9 = fungen((122,161),(132,147),(148,144));
draw(graph(qb9, 0,1));
quad_bezier qb10 = fungen((148,144),(176,155),(204,146));
draw(graph(qb10, 0,1));
quad_bezier qb11 = fungen((204,146),(216,141),(235,137));
draw(graph(qb11, 0,1));
quad_bezier qb12 = fungen((228,156),(208,160),(188,161));
draw(graph(qb12, 0,1));
quad_bezier qb13 = fungen((319,214),(313,174),(283,168));
draw(graph(qb13, 0,1));
quad_bezier qb14 = fungen((228,156),(242,158),(247,171));
draw(graph(qb14, 0,1));
quad_bezier qb15 = fungen((245,181),(250,158),(266,143));
draw(graph(qb15, 0,1));
quad_bezier qb16 = fungen((266,143),(287,134),(298,135));
draw(graph(qb16, 0,1));
quad_bezier qb17 = fungen((298,135),(309,143),(300,148));
draw(graph(qb17, 0,1));
quad_bezier qb18 = fungen((300,148),(272,150),(270,175));
draw(graph(qb18, 0,1));
quad_bezier qb19 = fungen((282,177),(274,158),(300,148));
draw(graph(qb19, 0,1));

draw(arc((317.8948497854077,245.25965665236052), 19.760615163024095, 143.54947493250435, 40.14574559948477));
draw(arc((282.65584415584414,295.7857142857143), 53.78971270402217, -78.91253214600312, -114.90992209204622));
draw(arc((127.7,168.5), 9.420191080864546, 9.162347045721713, 232.7651660184253));
draw(arc((229.125,145.625), 10.435815732370902, -55.73889710090544, 96.1886159632416));
draw(arc((186.26470588235293,181.5), 20.573313920580237, -85.1615330431756, 85.1615330431756));
filldraw(ellipse((314,235), 13.0, 10.04987562112089), rgb(254,255,255), black);
filldraw(rotate(14.036243467926468,(315,251))*ellipse((315,235), 9.219544457292887, 8.246211251235321),
	rgb(0,0,0), black);

pair o = (400,190);
real len=80;
real height=56;
for (int i=0; i<4; ++i) {
    pair a = (i*len, i*height);
    path p = a -- a+(len,0) -- a+(len, height);
    draw(shift(o)*p);
}
path p = (0,0)--(0,-height)--(4*len,-height);
draw(shift(o)*p);
\end{asy}
\end{center}



\(\textbf{(A)}\ 4 \qquad \textbf{(B)}\ 5 \qquad \textbf{(C)}\ 6 \qquad \textbf{(D)}\ 8 \qquad \textbf{(E)}\ 12\)\par \vspace{0.5em}\item What is the distance of the shortest route from A to Z?


\begin{center}
\begin{asy}
import olympiad;
import cse5;
/* AMC8 P14 2024, by NUMANA: BUI VAN HIEU */
import graph;
unitsize(2cm);
real r=0.25;
// Define the nodes and their positions
pair[] nodes = { (0,0), (2,0), (1,1), (3,1), (4,0), (6,0) };
string[] labels = { "A", "M", "X", "Y", "C", "Z" };

// Draw the nodes as circles with labels
for(int i = 0; i < nodes.length; ++i) {
    draw(circle(nodes[i], r));
    label("$" + labels[i] + "$", nodes[i]);
}
// Define the edges with their node indices and labels
int[][] edges = { {0, 1}, {0, 2}, {2, 1}, {2, 3}, {1, 3}, {1, 4}, {3, 4}, {4, 5}, {3, 5} };
string[] edgeLabels = { "8", "5", "2", "10", "6", "14", "5", "10", "17" };
pair[] edgeLabelsPos = { S, SE, SW, S, SE, S, SW, S, NE};
// Draw the edges with labels
for (int i = 0; i < edges.length; ++i) {
    pair start = nodes[edges[i][0]];
    pair end = nodes[edges[i][1]];
    draw(start + r*dir(end-start) -- end-r*dir(end-start), Arrow);
    label("$" + edgeLabels[i] + "$", midpoint(start -- end),  edgeLabelsPos[i]);
}
// Draw the curved edge with label
draw(nodes[1]+r * dir(-45)..controls (3, -0.75) and (5, -0.75)..nodes[5]+r * dir(-135), Arrow);
label("$25$", midpoint(nodes[1]..controls (3, -0.75) and (5, -0.75)..nodes[5]), 2S);
\end{asy}
\end{center}


\(\textbf{(A)}\ 28 \qquad \textbf{(B)}\ 29 \qquad \textbf{(C)}\ 30 \qquad \textbf{(D)}\ 31 \qquad \textbf{(E)}\ 32\)\par \vspace{0.5em}\item Let the letters \(F,L,Y,B,U,G\) represent distinct digits. Suppose \(\underline{F}~\underline{L}~\underline{Y}~\underline{F}~\underline{L}~\underline{Y}\) is the GREATEST number that satisfies the equation


\begin{equation*}
8\cdot\underline{F}~\underline{L}~\underline{Y}~\underline{F}~\underline{L}~\underline{Y}=\underline{B}~\underline{U}~\underline{G}~\underline{B}~\underline{U}~\underline{G}.
\end{equation*}


What is the value of \(\underline{F}~\underline{L}~\underline{Y}+\underline{B}~\underline{U}~\underline{G}\)?

\(\textbf{(A)}\ 1089 \qquad \textbf{(B)}\ 1098 \qquad \textbf{(C)}\ 1107 \qquad \textbf{(D)}\ 1116 \qquad \textbf{(E)}\ 1125\)\par \vspace{0.5em}\item Minh enters the numbers \(1\) through \(81\) into the cells of a \(9 \times 9\) grid in some order. She calculates the product of the numbers in each row and column. What is the least number of rows and columns that could have a product divisible by \(3\)?

\(\textbf{(A) } 8\qquad\textbf{(B) } 9\qquad\textbf{(C) } 10\qquad\textbf{(D) } 11\qquad\textbf{(E) } 12\)\par \vspace{0.5em}\item A chess king is said to ''attack'' all squares one step away from it (basically any square right next to it in any direction), horizontally, vertically, or diagonally. For instance, a king on the center square of a 3 x 3 grid attacks all 8 other squares, as shown below. Suppose a white king and a black king are placed on different squares of 3 x 3 grid so that they do not attack each other. In how many ways can this be done?


\begin{center}
\begin{asy}
import olympiad;
import cse5;
/* AMC8 P17 2024, revised by Teacher David */
unitsize(29pt);
import math;
add(grid(3,3));

pair [] a = {(0.5,0.5), (0.5, 1.5), (0.5, 2.5), (1.5, 2.5), (2.5,2.5), (2.5,1.5), (2.5,0.5), (1.5,0.5)};

for (int i=0; i<a.length; ++i) {
    pair x = (1.5,1.5) + 0.4*dir(225-45*i);
    draw(x -- a[i], arrow=EndArrow());
}

label("$K$", (1.5,1.5));
\end{asy}
\end{center}


\(\textbf{(A)}\ 20 \qquad \textbf{(B)}\ 24 \qquad \textbf{(C)}\ 27 \qquad \textbf{(D)}\ 28 \qquad \textbf{(E)}\ 32\)\par \vspace{0.5em}\item Three concentric circles centered at \(O\) have a radius of \(1,2,\) and \(3\). Points \(B\) and \(C\) lie on the largest circle. The region between the two smaller circles is shaded, as is the portion of the region between the two larger circles bounded by central angle \(BOC\), as shown in the figure below. Suppose the shaded and unshaded regions are equal in area. What is the measure of \(\angle{BOC}\) in degrees?


\begin{center}
\begin{asy}
import olympiad;
import cse5;
size(100);
import graph;



draw(circle((0,0),3));
real radius = 3;
real angleStart = -54;  // starting angle of the sector
real angleEnd = 54;  // ending angle of the sector
label("$O$",(0,0),W);
pair O = (0, 0);
filldraw(arc(O, radius, angleStart, angleEnd)--O--cycle, lightgray);
filldraw(circle((0,0),2),lightgray);
filldraw(circle((0,0),1),white);
draw((1.763,2.427)--(0,0)--(1.763,-2.427));
label("$B$",(1.763,2.427),NE);
label("$C$",(1.763,-2.427),SE);
\end{asy}
\end{center}


\(\textbf{(A)}\ 108 \qquad \textbf{(B)}\ 120 \qquad \textbf{(C)}\ 135 \qquad \textbf{(D)}\ 144 \qquad \textbf{(E)}\ 150\)\par \vspace{0.5em}\item Jordan owns 15 pairs of sneakers. Three fifths of the pairs are red and the rest are white. Two thirds of the pairs are high-top and the rest are low-top. The red high-top sneakers make up a fraction of the collection. What is the least possible value of this fraction?




\(\textbf{(A) } 0\qquad\textbf{(B) } \dfrac{1}{5} \qquad\textbf{(C) } \dfrac{4}{15} \qquad\textbf{(D) } \dfrac{1}{3} \qquad\textbf{(E) } \dfrac{2}{5}\)\par \vspace{0.5em}\item Any three vertices of the cube \(PQRSTUVW,\) shown in the figure below, can be connected to form a triangle. \((\)For example, vertices \(P, Q,\) and \(R\) can be connected to form \(\triangle{PQR}.)\) How many of these triangles are equilateral and contain \(P\) as a vertex?


\begin{center}
\begin{asy}
import olympiad;
import cse5;
unitsize(4);
pair P,Q,R,S,T,U,V,W;
P=(0,30); Q=(30,30); R=(40,40); S=(10,40); T=(10,10); U=(40,10); V=(30,0); W=(0,0);
draw(W--V); draw(V--Q); draw(Q--P); draw(P--W); draw(T--U); draw(U--R); draw(R--S); draw(S--T); draw(W--T); draw(P--S); draw(V--U); draw(Q--R);
dot(P);
dot(Q);
dot(R);
dot(S);
dot(T);
dot(U);
dot(V);
dot(W);
label("$P$",P,NW);
label("$Q$",Q,NW);
label("$R$",R,NE);
label("$S$",S,N);
label("$T$",T,NE);
label("$U$",U,NE);
label("$V$",V,SE);
label("$W$",W,SW);
\end{asy}
\end{center}


\(\textbf{(A) }0\qquad\textbf{(B) }1\qquad\textbf{(C) }2\qquad\textbf{(D) }3\qquad\textbf{(E) }6\)\par \vspace{0.5em}\item A group of frogs (called an ''army'') is living in a tree. A frog turns green when in the shade and turns yellow
when in the sun. Initially, the ratio of green to yellow frogs was \(3 : 1\). Then \(3\) green frogs moved to the
sunny side and \(5\) yellow frogs moved to the shady side. Now the ratio is \(4 : 1\). What is the difference
between the number of green frogs and the number of yellow frogs now?

\(\textbf{(A) } 10\qquad\textbf{(B) } 12\qquad\textbf{(C) } 16\qquad\textbf{(D) } 20\qquad\textbf{(E) } 24\)\par \vspace{0.5em}\item A  roll of tape is \(4\) inches in diameter and is wrapped around a ring that is \(2\) inches in diameter. A cross section of the tape is shown in the figure below. The tape is \(0.015\) inches thick. If the tape is completely unrolled, approximately how long would it be? Round your answer to the nearest \(100\) inches.


\begin{center}
\begin{asy}
import olympiad;
import cse5;
/* AMC8 P22 2024, revised by Teacher David */
size(150);

pair o = (0,0);
real r1 = 1;
real r2 = 2;

filldraw(circle(o, r2), mediumgray, linewidth(1pt));
filldraw(circle(o, r1), white, linewidth(1pt));

draw((-2,-2.6)--(-2,-2.4));
draw((2,-2.6)--(2,-2.4));
draw((-2,-2.5)--(2,-2.5), L=Label("4 in."));

draw((-1,0)--(1,0), L=Label("2 in.", align=(0,1)), arrow=Arrows());

draw((2,0)--(2,-1.3), linewidth(1pt));
\end{asy}
\end{center}


\(\textbf{(A) } 300\qquad\textbf{(B) } 600\qquad\textbf{(C) } 1200\qquad\textbf{(D) } 1500\qquad\textbf{(E) } 1800\)\par \vspace{0.5em}\item Rodrigo has a very large sheet of graph paper. First he draws a line segment connecting point \((0,4)\) to point \((2,0)\) and colors the \(4\) cells whose interiors intersect the segment, as shown below. Next Rodrigo draws a line segment connecting point \((2000,3000)\) to point \((5000,8000)\). How many cells will he color this time?


\begin{center}
\begin{asy}
import olympiad;
import cse5;
filldraw((0,4)--(1,4)--(1,3)--(0,3)--cycle, gray(.75), gray(.5)+linewidth(1));
filldraw((0,3)--(1,3)--(1,2)--(0,2)--cycle, gray(.75), gray(.5)+linewidth(1));
filldraw((1,2)--(2,2)--(2,1)--(1,1)--cycle, gray(.75), gray(.5)+linewidth(1));
filldraw((1,1)--(2,1)--(2,0)--(1,0)--cycle, gray(.75), gray(.5)+linewidth(1));

draw((-1,5)--(-1,-1),gray(.9));
draw((0,5)--(0,-1),gray(.9));
draw((1,5)--(1,-1),gray(.9));
draw((2,5)--(2,-1),gray(.9));
draw((3,5)--(3,-1),gray(.9));
draw((4,5)--(4,-1),gray(.9));
draw((5,5)--(5,-1),gray(.9));

draw((-1,5)--(5, 5),gray(.9));
draw((-1,4)--(5,4),gray(.9));
draw((-1,3)--(5,3),gray(.9));
draw((-1,2)--(5,2),gray(.9));
draw((-1,1)--(5,1),gray(.9));
draw((-1,0)--(5,0),gray(.9));
draw((-1,-1)--(5,-1),gray(.9));


dot((0,4));
label("$(0,4)$",(0,4),NW);

dot((2,0));
label("$(2,0)$",(2,0),SE);

draw((0,4)--(2,0));

draw((-1,0) -- (5,0), arrow=Arrow);
draw((0,-1) -- (0,5), arrow=Arrow);
\end{asy}
\end{center}


\(\textbf{(A) }6000\qquad\textbf{(B) }6500\qquad\textbf{(C) }7000\qquad\textbf{(D) }7500\qquad\textbf{(E) }8000\)\par \vspace{0.5em}\item Jean has made a piece of stained glass art in the shape of two mountains, as shown in the figure below. One mountain peak is \(8\) feet high while the other peak is \(12\) feet high. Each peak forms a \(90^\circ\) angle, and the straight sides form a \(45^\circ\) angle with the ground. The artwork has an area of \(183\) square feet. The sides of the mountain meet at an intersection point near the center of the artwork, \(h\) feet above the ground. What is the value of \(h?\)


\begin{center}
\begin{asy}
import olympiad;
import cse5;
unitsize(.3cm);
filldraw((0,0)--(8,8)--(11,5)--(18,12)--(30,0)--cycle,gray(0.7),linewidth(1));
draw((-1,0)--(-1,8),linewidth(.75));
draw((-1.4,0)--(-.6,0),linewidth(.75));
draw((-1.4,8)--(-.6,8),linewidth(.75));
label("$8$",(-1,4),W);
label("$12$",(31,6),E);
draw((-1,8)--(8,8),dashed);
draw((31,0)--(31,12),linewidth(.75));
draw((30.6,0)--(31.4,0),linewidth(.75));
draw((30.6,12)--(31.4,12),linewidth(.75));
draw((31,12)--(18,12),dashed);
label("$45^{\circ}$",(.75,0),NE,fontsize(10pt));
label("$45^{\circ}$",(29.25,0),NW,fontsize(10pt));
draw((8,8)--(7.5,7.5)--(8,7)--(8.5,7.5)--cycle);
draw((18,12)--(17.5,11.5)--(18,11)--(18.5,11.5)--cycle);
draw((11,5)--(11,0),dashed);
label("$h$",(11,2.5),E);
\end{asy}
\end{center}


\(\textbf{(A)}\ 4 \qquad \textbf{(B)}\ 5 \qquad \textbf{(C)}\ 4\sqrt{2} \qquad \textbf{(D)}\ 6 \qquad \textbf{(E)}\ 5\sqrt{2}\)\par \vspace{0.5em}\item A airplane has \(4\) rows of seats with \(3\) seats in each row. Eight passengers have boarded the plane and are distributed randomly among the seats. A married couple is next to board. What is the probability there will be \(2\) adjacent seats in the same row for the couple?



\(\textbf{(A)}\ \dfrac{8}{15} \qquad \textbf{(B)}\ \dfrac{32}{55} \qquad \textbf{(C)}\ \dfrac{20}{33} \qquad \textbf{(D)}\ \dfrac{34}{55} \qquad \textbf{(E)}\ \dfrac{8}{11}\)\par \vspace{0.5em}
\end{enumerate}

\end{document}
