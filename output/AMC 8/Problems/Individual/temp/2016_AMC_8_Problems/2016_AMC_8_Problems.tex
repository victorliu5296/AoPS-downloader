
\documentclass{article}
\usepackage{amsmath, amssymb}
\usepackage{geometry}
\geometry{a4paper, margin=0.75in}
\usepackage{enumitem}
\usepackage[hypertexnames=true, linktoc=all]{hyperref}
\usepackage{fancyhdr}
\usepackage{tikz}
\usepackage{graphicx}
\usepackage{asymptote}
\usepackage{arcs}
\usepackage{xwatermark}
\begin{asydef}
  // Global Asymptote settings
  settings.outformat = "pdf";
  settings.render = 0;
  settings.prc = false;
  import olympiad;
  import cse5;
  size(8cm);
\end{asydef}
\pagestyle{fancy}
\fancyhead[L]{\textbf{AMC 8 Problems}}
\fancyhead[R]{\textbf{2016}}
\fancyfoot[C]{\thepage}
\renewcommand{\headrulewidth}{0.4pt}
\renewcommand{\footrulewidth}{0.4pt}

\title{AMC 8 Problems \\ 2016}
\date{}
\begin{document}\maketitle\thispagestyle{fancy}\newpage\section*{2016 AMC 8}\begin{enumerate}[label=\arabic*., itemsep=0.5em]\item The longest professional tennis match lasted a total of 11 hours and 5 minutes. How many minutes is that?

\(\textbf{(A) } 605 \qquad\textbf{(B) } 655\qquad\textbf{(C) } 665\qquad\textbf{(D) } 1005\qquad \textbf{(E) } 1105\)\par \vspace{0.5em}\item In rectangle \(ABCD\), \(AB=6\) and \(AD=8\).  Point \(M\) is the midpoint of \(\overline{AD}\).  What is the area of \(\triangle AMC\)?


\begin{center}
\begin{asy}
import olympiad;
import cse5;
draw((0,4)--(0,0)--(6,0)--(6,8)--(0,8)--(0,4)--(6,8)--(0,0));
label("$A$", (0,0), SW);
label("$B$", (6, 0), SE);
label("$C$", (6,8), NE);
label("$D$", (0, 8), NW);
label("$M$", (0, 4), W);
label("$4$", (0, 2), W);
label("$6$", (3, 0), S);
\end{asy}
\end{center}


\(\textbf{(A) }12\qquad\textbf{(B) }15\qquad\textbf{(C) }18\qquad\textbf{(D) }20\qquad \textbf{(E) }24\)

[[2016 AMC 8 Problems/Problem 2|Solution
]]\par \vspace{0.5em}\item Four students take an exam. Three of their scores are \(70\), \(80\), and \(90\). If the average of their four scores is \(70\), then what is the remaining score??

\(\textbf{(A) }40\qquad\textbf{(B) }50\qquad\textbf{(C) }55\qquad\textbf{(D) }60\qquad \textbf{(E) }70\)

[[2016 AMC 8 Problems/Problem 3|Solution
]]\par \vspace{0.5em}\item When Cheenu was a boy, he could run \(15\) miles in \(3\) hours and \(30\) minutes. As an old man, he can now walk \(10\) miles in \(4\) hours. How many minutes longer does it take for him to travel a mile now compared to when he was a boy?

\(\textbf{(A) }6\qquad\textbf{(B) }10\qquad\textbf{(C) }15\qquad\textbf{(D) }18\qquad \textbf{(E) }30\)

[[2016 AMC 8 Problems/Problem 4|Solution
]]\par \vspace{0.5em}\item The number \(N\) is a two-digit number.

 When \(N\) is divided by \(9\), the remainder is \(1\).

 When \(N\) is divided by \(10\), the remainder is \(3\).

What is the remainder when \(N\) is divided by \(11\)?


\(\textbf{(A) }0\qquad\textbf{(B) }2\qquad\textbf{(C) }4\qquad\textbf{(D) }5\qquad \textbf{(E) }7\)

[[2016 AMC 8 Problems/Problem 5|Solution
]]\par \vspace{0.5em}\item The following bar graph represents the length (in letters) of the names of 19 people. What is the median length of these names?

\begin{center}
\begin{asy}
import olympiad;
import cse5;
unitsize(0.9cm);
draw((-0.5,0)--(10,0), linewidth(1.5));
draw((-0.5,1)--(10,1));
draw((-0.5,2)--(10,2));
draw((-0.5,3)--(10,3));
draw((-0.5,4)--(10,4));
draw((-0.5,5)--(10,5));
draw((-0.5,6)--(10,6));
draw((-0.5,7)--(10,7));
label("frequency",(-0.5,8));
label("0", (-1, 0));
label("1", (-1, 1));
label("2", (-1, 2));
label("3", (-1, 3));
label("4", (-1, 4));
label("5", (-1, 5));
label("6", (-1, 6));
label("7", (-1, 7));
filldraw((0,0)--(0,7)--(1,7)--(1,0)--cycle, black);
filldraw((2,0)--(2,3)--(3,3)--(3,0)--cycle, black);
filldraw((4,0)--(4,1)--(5,1)--(5,0)--cycle, black);
filldraw((6,0)--(6,4)--(7,4)--(7,0)--cycle, black);
filldraw((8,0)--(8,4)--(9,4)--(9,0)--cycle, black);
label("3", (0.5, -0.5));
label("4", (2.5, -0.5));
label("5", (4.5, -0.5));
label("6", (6.5, -0.5));
label("7", (8.5, -0.5));
label("name length", (4.5, -1));
\end{asy}
\end{center}


\(\textbf{(A) }3\qquad\textbf{(B) }4\qquad\textbf{(C) }5\qquad\textbf{(D) }6\qquad \textbf{(E) }7\)

[[2016 AMC 8 Problems/Problem 6|Solution
]]\par \vspace{0.5em}\item Which of the following numbers is not a perfect square?

\(\textbf{(A) }1^{2016}\qquad\textbf{(B) }2^{2017}\qquad\textbf{(C) }3^{2018}\qquad\textbf{(D) }4^{2019}\qquad \textbf{(E) }5^{2020}\)

[[2016 AMC 8 Problems/Problem 7|Solution
]]\par \vspace{0.5em}\item Find the value of the expression:

\begin{equation*}
100-98+96-94+92-90+\cdots+8-6+4-2.
\end{equation*}
\(\textbf{(A) }20\qquad\textbf{(B) }40\qquad\textbf{(C) }50\qquad\textbf{(D) }80\qquad \textbf{(E) }100\)

[[2016 AMC 8 Problems/Problem 8|Solution
]]\par \vspace{0.5em}\item What is the sum of the distinct prime integer divisors of \(2016\)?

\(\textbf{(A) }9\qquad\textbf{(B) }12\qquad\textbf{(C) }16\qquad\textbf{(D) }49\qquad \textbf{(E) }63\)

[[2016 AMC 8 Problems/Problem 9|Solution
]]\par \vspace{0.5em}\item Suppose that \(a * b\) means \(3a-b.\) What is the value of \(x\) if

\begin{equation*}
2 * (5 * x)=1
\end{equation*}

\(\textbf{(A) }\frac{1}{10} \qquad\textbf{(B) }2\qquad\textbf{(C) }\frac{10}{3} \qquad\textbf{(D) }10\qquad \textbf{(E) }14.\)

[[2016 AMC 8 Problems/Problem 10|Solution
]]\par \vspace{0.5em}\item Determine how many two-digit numbers satisfy the following property: when the number is added to the number obtained by reversing its digits, the sum is \(132.\)

\(\textbf{(A) }5\qquad\textbf{(B) }7\qquad\textbf{(C) }9\qquad\textbf{(D) }11\qquad \textbf{(E) }12\)

[[2016 AMC 8 Problems/Problem 11|Solution
]]\par \vspace{0.5em}\item Jefferson Middle School has the same number of boys and girls. \(\frac{3}{4}\) of the girls and \(\frac{2}{3}\)
of the boys went on a field trip. What fraction of the students on the field trip were girls?

\(\textbf{(A) }\frac{1}{2}\qquad\textbf{(B) }\frac{9}{17}\qquad\textbf{(C) }\frac{7}{13}\qquad\textbf{(D) }\frac{2}{3}\qquad \textbf{(E) }\frac{14}{15}\)

[[2016 AMC 8 Problems/Problem 12|Solution
]]\par \vspace{0.5em}\item Two different numbers are randomly selected from the set \(\{ - 2, -1, 0, 3, 4, 5\}\) and multiplied together. What is the probability that the product is \(0\)?

\(\textbf{(A) }\dfrac{1}{6}\qquad\textbf{(B) }\dfrac{1}{5}\qquad\textbf{(C) }\dfrac{1}{4}\qquad\textbf{(D) }\dfrac{1}{3}\qquad \textbf{(E) }\dfrac{1}{2}\)

[[2016 AMC 8 Problems/Problem 13|Solution
]]\par \vspace{0.5em}\item Karl's car uses a gallon of gas every \(35\) miles, and his gas tank holds \(14\) gallons when it is full. One day, Karl started with a full tank of gas, 
drove \(350\) miles, bought \(8\) gallons of gas, and continued driving to his destination. When he arrived, his gas tank was half full. How many miles did Karl drive that day? 

\(\textbf{(A)}\mbox{ }525\qquad\textbf{(B)}\mbox{ }560\qquad\textbf{(C)}\mbox{ }595\qquad\textbf{(D)}\mbox{ }665\qquad\textbf{(E)}\mbox{ }735\)

[[2016 AMC 8 Problems/Problem 14|Solution
]]\par \vspace{0.5em}\item What is the largest power of \(2\) that is a divisor of \(13^4 - 11^4\)?

\(\textbf{(A)}\mbox{ }8\qquad \textbf{(B)}\mbox{ }16\qquad \textbf{(C)}\mbox{ }32\qquad \textbf{(D)}\mbox{ }64\qquad \textbf{(E)}\mbox{ }128\)

[[2016 AMC 8 Problems/Problem 15|Solution
]]\par \vspace{0.5em}\item Annie and Bonnie are running laps around a \(400\)-meter oval track. They started at the same time, at the same place, but Annie has pulled ahead because she runs \(25\%\) faster than Bonnie. How many laps will Annie have run when she first passes Bonnie?

\(\textbf{(A) }1\dfrac{1}{4}\qquad\textbf{(B) }3\dfrac{1}{3}\qquad\textbf{(C) }4\qquad\textbf{(D) }5\qquad \textbf{(E) }25\)

[[2016 AMC 8 Problems/Problem 16|Solution
]]\par \vspace{0.5em}\item An ATM password at Fred's Bank is composed of four digits from \(0\) to \(9\), with repeated digits allowable. If no password may begin with the sequence \(9,1,1,\) then how many passwords are possible?

\(\textbf{(A)}\mbox{ }30\qquad\textbf{(B)}\mbox{ }7290\qquad\textbf{(C)}\mbox{ }9000\qquad\textbf{(D)}\mbox{ }9990\qquad\textbf{(E)}\mbox{ }9999\)

[[2016 AMC 8 Problems/Problem 17|Solution
]]\par \vspace{0.5em}\item In an All-Area track meet, \(216\) sprinters enter a \(100-\)meter dash competition. The track has \(6\) lanes, so only \(6\) sprinters can compete at a time. At the end of each race, the five non-winners are eliminated, and the winner will compete again in a later race. How many races are needed to determine the champion sprinter?

\(\textbf{(A)}\mbox{ }36\qquad\textbf{(B)}\mbox{ }42\qquad\textbf{(C)}\mbox{ }43\qquad\textbf{(D)}\mbox{ }60\qquad\textbf{(E)}\mbox{ }72\)

[[2016 AMC 8 Problems/Problem 18|Solution
]]\par \vspace{0.5em}\item The sum of \(25\) consecutive even integers is \(10,000\). What is the largest of these \(25\) consecutive integers?

\(\textbf{(A)}\mbox{ }360\qquad\textbf{(B)}\mbox{ }388\qquad\textbf{(C)}\mbox{ }412\qquad\textbf{(D)}\mbox{ }416\qquad\textbf{(E)}\mbox{ }424\)

[[2016 AMC 8 Problems/Problem 19|Solution
]]\par \vspace{0.5em}\item The least common multiple of \(a\) and \(b\) is \(12\), and the least common multiple of \(b\) and \(c\) is \(15\). What is the least possible value of the least common multiple of \(a\) and \(c\)?

\(\textbf{(A) }20\qquad\textbf{(B) }30\qquad\textbf{(C) }60\qquad\textbf{(D) }120\qquad \textbf{(E) }180\)

[[2016 AMC 8 Problems/Problem 20|Solution
]]\par \vspace{0.5em}\item A top hat contains 3 red chips and 2 green chips. Chips are drawn randomly, one at a time without replacement, until all 3 of the reds are drawn or until both green chips are drawn. What is the probability that the 3 reds are drawn?

\(\textbf{(A) }\dfrac{3}{10}\qquad\textbf{(B) }\dfrac{2}{5}\qquad\textbf{(C) }\dfrac{1}{2}\qquad\textbf{(D) }\dfrac{3}{5}\qquad \textbf{(E) }\dfrac{7}{10}\)

[[2016 AMC 8 Problems/Problem 21|Solution
]]\par \vspace{0.5em}\item Rectangle \(DEFA\) below is a \(3 \times 4\) rectangle with \(DC=CB=BA=1\). What is the area of the "bat wings" (shaded region)?
\begin{center}
\begin{asy}
import olympiad;
import cse5;
draw((0,0)--(3,0)--(3,4)--(0,4)--(0,0)--(2,4)--(3,0));
draw((3,0)--(1,4)--(0,0));
fill((0,0)--(1,4)--(1.5,3)--cycle, black);
fill((3,0)--(2,4)--(1.5,3)--cycle, black);
label("$A$",(3.05,4.2));
label("$B$",(2,4.2));
label("$C$",(1,4.2));
label("$D$",(0,4.2));
label("$E$", (0,-0.2));
label("$F$", (3,-0.2));
\end{asy}
\end{center}


\(\textbf{(A) }2\qquad\textbf{(B) }2 \frac{1}{2}\qquad\textbf{(C) }3\qquad\textbf{(D) }3 \frac{1}{2}\qquad \textbf{(E) }4\)

[[2016 AMC 8 Problems/Problem 22|Solution
]]\par \vspace{0.5em}\item Two congruent circles centered at points \(A\) and \(B\) each pass through the other circle's center. The line containing both \(A\) and \(B\) is extended to intersect the circles at points \(C\) and \(D\). The circles intersect at two points, one of which is \(E\). What is the degree measure of \(\angle CED\)?

\(\textbf{(A) }90\qquad\textbf{(B) }105\qquad\textbf{(C) }120\qquad\textbf{(D) }135\qquad \textbf{(E) }150\)

[[2016 AMC 8 Problems/Problem 23|Solution
]]\par \vspace{0.5em}\item The digits \(1\), \(2\), \(3\), \(4\), and \(5\) are each used once to write a five-digit number \(PQRST\). The three-digit number \(PQR\) is divisible by \(4\), the three-digit number \(QRS\) is divisible by \(5\), and the three-digit number \(RST\) is divisible by \(3\). What is \(P\)?

\(\textbf{(A) }1\qquad\textbf{(B) }2\qquad\textbf{(C) }3\qquad\textbf{(D) }4\qquad \textbf{(E) }5\)

[[2016 AMC 8 Problems/Problem 24|Solution
]]

==Problem 25== 

A semicircle is inscribed in an isosceles triangle with base \(16\) and height \(15\) so that the diameter of the semicircle is contained in the base of the triangle as shown. What is the radius of the semicircle?


\begin{center}
\begin{asy}
import olympiad;
import cse5;
draw((0,0)--(8,15)--(16,0)--(0,0));
draw(arc((8,0),7.0588,0,180));
\end{asy}
\end{center}


\(\textbf{(A) }4 \sqrt{3}\qquad\textbf{(B) } \dfrac{120}{17}\qquad\textbf{(C) }10\qquad\textbf{(D) }\dfrac{17\sqrt{2}}{2}\qquad \textbf{(E)} \dfrac{17\sqrt{3}}{2}\)

[[2016 AMC 8 Problems/Problem 25|Solution
]]\par \vspace{0.5em}\end{enumerate}
\end{document}
