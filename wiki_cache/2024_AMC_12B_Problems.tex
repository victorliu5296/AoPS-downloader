{{AMC12 Problems|year=2024|ab=B}}

==Problem 1==
In a long line of people arranged left to right, the 1013th person from the left is also the 1010th person from the right. How many people are in the line?

<math>\textbf{(A) } 2021 \qquad\textbf{(B) } 2022 \qquad\textbf{(C) } 2023 \qquad\textbf{(D) } 2024 \qquad\textbf{(E) } 2025</math>

[[2024 AMC 12B Problems/Problem 1|Solution]]

==Problem 2==
What is <math>10! - 7! \cdot 6!</math>?

<math>\textbf{(A) }-120 \qquad\textbf{(B) }0 \qquad\textbf{(C) }120 \qquad\textbf{(D) }600 \qquad\textbf{(E) }720 \qquad</math>

[[2024 AMC 12B Problems/Problem 2|Solution]]

==Problem 3==
For how many integer values of <math>x</math> is <math>|2x|\leq 7\pi?</math>

<math>\textbf{(A) }16 \qquad\textbf{(B) }17\qquad\textbf{(C) }19\qquad\textbf{(D) }20\qquad\textbf{(E) }21</math>

[[2024 AMC 12B Problems/Problem 3|Solution]]

==Problem 4==

Balls numbered <math>1,2,3,\ldots</math> are deposited in <math>5</math> bins, labeled <math>A,B,C,D,</math> and <math>E</math>, using the following procedure. Ball <math>1</math> is deposited in bin <math>A</math>, and balls <math>2</math> and <math>3</math> are deposited in <math>B</math>. The next three balls are deposited in bin <math>C</math>, the next <math>4</math> in bin <math>D</math>, and so on, cycling back to bin <math>A</math> after balls are deposited in bin <math>E</math>. (For example, <math>22,23,\ldots,28</math> are deposited in bin <math>B</math> at step 7 of this process.) In which bin is ball <math>2024</math> deposited?

<math>\textbf{(A) }A\qquad\textbf{(B) }B\qquad\textbf{(C) }C\qquad\textbf{(D) }D\qquad\textbf{(E) }E</math>

[[2024 AMC 12B Problems/Problem 4|Solution]]

==Problem 5==

In the following expression, Melanie changed some of the plus signs to minus signs:<cmath> 1 + 3+5+7+\cdots+97+99</cmath>When the new expression was evaluated, it was negative. What is the least number of plus signs that Melanie could have changed to minus signs?

<math>
\textbf{(A) }14 \qquad
\textbf{(B) }15 \qquad
\textbf{(C) }16 \qquad
\textbf{(D) }17 \qquad
\textbf{(E) }18 \qquad
</math>

[[2024 AMC 12B Problems/Problem 5|Solution]]

==Problem 6==
The national debt of the United States is on track to reach <math>5 \cdot 10^{13}</math> dollars by <math>2033</math>. How many digits does this number of dollars have when written as a numeral in base <math>5</math>? (The approximation of <math>\log_{10} 5</math> as <math>0.7</math> is sufficient for this problem.)

<math>
\textbf{(A) }18 \qquad
\textbf{(B) }20 \qquad
\textbf{(C) }22 \qquad
\textbf{(D) }24 \qquad
\textbf{(E) }26 \qquad
</math>

[[2024 AMC 12B Problems/Problem 6|Solution]]

==Problem 7==
In the figure below <math>WXYZ</math> is a rectangle with <math>WX=4</math> and <math>WZ=8</math>. Point <math>M</math> lies <math>\overline{XY}</math>, point <math>A</math> lies on <math>\overline{YZ}</math>, and <math>\angle WMA</math> is a right angle. The areas of <math>\triangle WXM</math> and <math>\triangle WAZ</math> are equal. What is the area of <math>\triangle WMA</math>?

<asy>
pair X = (0, 0);
pair W = (0, 4);
pair Y = (8, 0);
pair Z = (8, 4);
label("$X$", X, dir(180));
label("$W$", W, dir(180));
label("$Y$", Y, dir(0));
label("$Z$", Z, dir(0));

draw(W--X--Y--Z--cycle);
dot(X);
dot(Y);
dot(W);
dot(Z);
pair M = (2, 0);
pair A = (8, 3);
label("$A$", A, dir(0));
dot(M);
dot(A);
draw(W--M--A--cycle);
markscalefactor = 0.05;
draw(rightanglemark(W, M, A));
label("$M$", M, dir(-90));
</asy>

<math>
\textbf{(A) }13 \qquad
\textbf{(B) }14 \qquad
\textbf{(C) }15 \qquad
\textbf{(D) }16 \qquad
\textbf{(E) }17 \qquad
</math>

[[2024 AMC 12B Problems/Problem 7|Solution]]

==Problem 8==
What value of <math>x</math> satisfies<cmath>\frac{\log_2x\cdot\log_3x}{\log_2x+\log_3x}=2?</cmath>
<math>
\textbf{(A) }25\qquad
\textbf{(B) }32\qquad
\textbf{(C) }36\qquad
\textbf{(D) }42\qquad
\textbf{(E) }48\qquad
</math>

[[2024 AMC 12B Problems/Problem 8|Solution]]

==Problem 9==
A dartboard is the region <math>B</math> in the coordinate plane consisting of points <math>(x,y)</math> such that <math>|x| + |y| \le 8</math> . A target <math>T</math> is the region where <math>(x^2 + y^2 - 25)^2 \le 49.</math> A dart is thrown and lands at a random point in <math>B</math>. The probability that the dart lands in <math>T</math> can be expressed as <math>\frac{m}{n} \cdot \pi,</math> where <math>m</math> and <math>n</math> are relatively prime positive integers. What is <math>m + n?</math>

<math>
\textbf{(A) }39 \qquad
\textbf{(B) }71 \qquad
\textbf{(C) }73 \qquad
\textbf{(D) }75 \qquad
\textbf{(E) }135 \qquad
</math>

[[2024 AMC 12B Problems/Problem 9|Solution]]

==Problem 10==

A list of 9 real numbers consists of <math>1</math>, <math>2.2 </math>, <math>3.2 </math>, <math>5.2 </math>, <math>6.2 </math>, and <math>7</math>, as well as <math>x, y,z</math> with <math>x\leq y\leq z</math>. The range of the list is <math>7</math>, and the mean and median are both positive integers. How many ordered triples <math>(x,y,z)</math> are possible?

<math>\textbf{(A) }1 \qquad\textbf{(B) }2 \qquad\textbf{(C) }3 \qquad\textbf{(D) }4 \qquad\textbf{(E) }\text{infinitely many}\qquad</math>

[[2024 AMC 12B Problems/Problem 10|Solution]]

==Problem 11==
Let <math>x_{n} = \sin^2(n^\circ)</math>. What is the mean of <math>x_{1}, x_{2}, x_{3}, \cdots, x_{90}</math>?

<math>
\textbf{(A) }\frac{11}{45} \qquad
\textbf{(B) }\frac{22}{45} \qquad
\textbf{(C) }\frac{89}{180} \qquad
\textbf{(D) }\frac{1}{2} \qquad
\textbf{(E) }\frac{91}{180} \qquad
</math>

[[2024 AMC 12B Problems/Problem 11|Solution]]

==Problem 12==
Suppose <math>z</math> is a complex number with positive imaginary part, with real part greater than <math>1</math>, and with <math>|z| = 2</math>. In the complex plane, the four points <math>0</math>, <math>z</math>, <math>z^{2}</math>, and <math>z^{3}</math> are the vertices of a quadrilateral with area <math>15</math>. What is the imaginary part of <math>z</math>?

<math>\textbf{(A)}~\frac{3}{4}\qquad\textbf{(B)}~1\qquad\textbf{(C)}~\frac{4}{3}\qquad\textbf{(D)}~\frac{3}{2}\qquad\textbf{(E)}~\frac{5}{3}</math>

[[2024 AMC 12B Problems/Problem 12|Solution]]

==Problem 13==
There are real numbers <math>x,y,h</math> and <math>k</math> that satisfy the system of equations<cmath>x^2 + y^2 - 6x - 8y = h</cmath><cmath>x^2 + y^2 - 10x + 4y = k</cmath>What is the minimum possible value of <math>h+k</math>?

<math>
\textbf{(A) }-54 \qquad
\textbf{(B) }-46 \qquad
\textbf{(C) }-34 \qquad
\textbf{(D) }-16 \qquad
\textbf{(E) }16 \qquad
</math>

[[2024 AMC 12B Problems/Problem 13|Solution]]

==Problem 14==
How many different remainders can result when the <math>100</math>th power of an integer is divided by <math>125</math>?

<math>\textbf{(A) }1 \qquad\textbf{(B) }2 \qquad\textbf{(C) }5 \qquad\textbf{(D) }25 \qquad\textbf{(E) }125 \qquad</math>

[[2024 AMC 12B Problems/Problem 14|Solution]]

==Problem 15==
A triangle in the coordinate plane has vertices <math>A(\log_21,\log_22)</math>, <math>B(\log_23,\log_24)</math>, and <math>C(\log_27,\log_28)</math>. What is the area of <math>\triangle ABC</math>?

<math>
\textbf{(A) }\log_2\frac{\sqrt3}7\qquad
\textbf{(B) }\log_2\frac3{\sqrt7}\qquad
\textbf{(C) }\log_2\frac7{\sqrt3}\qquad
\textbf{(D) }\log_2\frac{11}{\sqrt7}\qquad
\textbf{(E) }\log_2\frac{11}{\sqrt3}\qquad
</math>

[[2024 AMC 12B Problems/Problem 15|Solution]]

==Problem 16==

A group of <math>16</math> people will be partitioned into <math>4</math> indistinguishable <math>4</math>-person committees. Each committee will have one chairperson and one secretary. The number of different ways to make these assignments can be written as <math>3^{r}M</math>, where <math>r</math> and <math>M</math> are positive integers and <math>M</math> is not divisible by <math>3</math>. What is <math>r</math>?

<math>
\textbf{(A) }5 \qquad
\textbf{(B) }6 \qquad
\textbf{(C) }7 \qquad
\textbf{(D) }8 \qquad
\textbf{(E) }9 \qquad</math>

[[2024 AMC 12B Problems/Problem 16|Solution]]
==Problem 17==
Integers <math>a</math> and <math>b</math> are randomly chosen without replacement from the set of integers with absolute value not exceeding <math>10</math>. What is the probability that the polynomial <math>x^3 + ax^2 + bx + 6</math> has <math>3</math> distinct integer roots?

<math>\textbf{(A) }\frac{1}{240} \qquad \textbf{(B) }\frac{1}{221} \qquad \textbf{(C) }\frac{1}{105} \qquad \textbf{(D) }\frac{1}{84} \qquad \textbf{(E) }\frac{1}{63}</math>

[[2024 AMC 12B Problems/Problem 17|Solution]]

==Problem 18==
The Fibonacci numbers are defined by <math>F_1=1,</math> <math>F_2=1,</math> and <math>F_n=F_{n-1}+F_{n-2}</math> for <math>n\geq 3.</math> What is<cmath>\dfrac{F_2}{F_1}+\dfrac{F_4}{F_2}+\dfrac{F_6}{F_3}+\cdots+\dfrac{F_{20}}{F_{10}}?</cmath>
<math>\textbf{(A) }318 \qquad\textbf{(B) }319\qquad\textbf{(C) }320\qquad\textbf{(D) }321\qquad\textbf{(E) }322</math>

[[2024 AMC 12B Problems/Problem 18|Solution]]

==Problem 19==
Equilateral <math>\triangle ABC</math> with side length <math>14</math> is rotated about its center by angle <math>\theta</math>, where <math>0 < \theta < 60^{\circ}</math>, to form <math>\triangle DEF</math>. See the figure. The area of hexagon <math>ADBECF</math> is <math>91\sqrt{3}</math>. What is <math>\tan\theta</math>?
<asy>
// Credit to shihan for this diagram.

defaultpen(fontsize(13)); size(200);
pair O=(0,0),A=dir(225),B=dir(-15),C=dir(105),D=rotate(38.21,O)*A,E=rotate(38.21,O)*B,F=rotate(38.21,O)*C;
draw(A--B--C--A,gray+0.4);draw(D--E--F--D,gray+0.4); draw(A--D--B--E--C--F--A,black+0.9); dot(O); dot("$A$",A,dir(A)); dot("$B$",B,dir(B)); dot("$C$",C,dir(C)); dot("$D$",D,dir(D)); dot("$E$",E,dir(E)); dot("$F$",F,dir(F));
</asy>

<math>\textbf{(A)}~\frac{3}{4}\qquad\textbf{(B)}~\frac{5\sqrt{3}}{11}\qquad\textbf{(C)}~\frac{4}{5}\qquad\textbf{(D)}~\frac{11}{13}\qquad\textbf{(E)}~\frac{7\sqrt{3}}{13}</math>

[[2024 AMC 12B Problems/Problem 19|Solution]]

==Problem 20==
Suppose <math>A</math>, <math>B</math>, and <math>C</math> are points in the plane with <math>AB=40</math> and <math>AC=42</math>, and let <math>x</math> be the length of the line segment from <math>A</math> to the midpoint of <math>\overline{BC}</math>. Define a function <math>f</math> by letting <math>f(x)</math> be the area of <math>\triangle ABC</math>. Then the domain of <math>f</math> is an open interval <math>(p,q)</math>, and the maximum value <math>r</math> of <math>f(x)</math> occurs at <math>x=s</math>. What is <math>p+q+r+s</math>?

<math>
\textbf{(A) }909\qquad
\textbf{(B) }910\qquad
\textbf{(C) }911\qquad
\textbf{(D) }912\qquad
\textbf{(E) }913\qquad
</math>

[[2024 AMC 12B Problems/Problem 20|Solution]]

==Problem 21==
The measures of the smallest angles of three different right triangles sum to <math>90^\circ</math>. All three triangles have side lengths that are primitive Pythagorean triples. Two of them are <math>3-4-5</math> and <math>5-12-13</math>. What is the perimeter of the third triangle?

<math>
\textbf{(A) }40 \qquad
\textbf{(B) }126 \qquad
\textbf{(C) }154 \qquad
\textbf{(D) }176 \qquad
\textbf{(E) }208 \qquad
</math>

[[2024 AMC 12B Problems/Problem 21|Solution]]

==Problem 22==
Let <math>\triangle{ABC}</math> be a triangle with integer side lengths and the property that <math>\angle{B} = 2\angle{A}</math>. What is the least possible perimeter of such a triangle?

<math>
\textbf{(A) }13 \qquad
\textbf{(B) }14 \qquad
\textbf{(C) }15 \qquad
\textbf{(D) }16 \qquad
\textbf{(E) }17 \qquad
</math>

[[2024 AMC 12B Problems/Problem 22|Solution]]

==Problem 23==
A right pyramid has regular octagon <math>ABCDEFGH</math> with side length <math>1</math> as its base and apex <math>V.</math> Segments <math>\overline{AV}</math> and <math>\overline{DV}</math> are perpendicular. What is the square of the height of the pyramid?

<math>
\textbf{(A) }1 \qquad
\textbf{(B) }\frac{1+\sqrt2}{2} \qquad
\textbf{(C) }\sqrt2 \qquad
\textbf{(D) }\frac32 \qquad
\textbf{(E) }\frac{2+\sqrt2}{3} \qquad
</math>

[[2024 AMC 12B Problems/Problem 23|Solution]]

==Problem 24==
What is the number of ordered triples <math>(a,b,c)</math> of positive integers, with <math>a\le b\le c\le 9</math>, such that there exists a (non-degenerate) triangle <math>\triangle ABC</math> with an integer inradius for which <math>a</math>, <math>b</math>, and <math>c</math> are the lengths of the altitudes from <math>A</math> to <math>\overline{BC}</math>, <math>B</math> to <math>\overline{AC}</math>, and <math>C</math> to <math>\overline{AB}</math>, respectively? (Recall that the inradius of a triangle is the radius of the largest possible circle that can be inscribed in the triangle.)

<math>
\textbf{(A) }2\qquad
\textbf{(B) }3\qquad
\textbf{(C) }4\qquad
\textbf{(D) }5\qquad
\textbf{(E) }6\qquad
</math>

[[2024 AMC 12B Problems/Problem 24|Solution]]

==Problem 25==

Pablo will decorate each of <math>6</math> identical white balls with either a striped or a dotted pattern, using either red or blue paint. He will decide on the color and pattern for each ball by flipping a fair coin for each of the <math>12</math> decisions he must make. After the paint dries, he will place the <math>6</math> balls in an urn. Frida will randomly select one ball from the urn and note its color and pattern. The events "the ball Frida selects is red" and "the ball Frida selects is striped" may or may not be independent, depending on the outcome of Pablo's coin flips. The probability that these two events are independent can be written as <math>\frac mn,</math> where <math>m</math> and <math>n</math> are relatively prime positive integers. What is <math>m?</math> (Recall that two events <math>A</math> and <math>B</math> are independent if <math>P(A \text{ and }B) = P(A) \cdot P(B).</math>)

<math>\textbf{(A) } 243 \qquad \textbf{(B) } 245 \qquad \textbf{(C) } 247 \qquad \textbf{(D) } 249\qquad \textbf{(E) } 251</math>

[[2024 AMC 12B Problems/Problem 25|Solution]]

==See also==
{{AMC12 box|year=2024|ab=B|before=[[2024 AMC 12A Problems]]|after=[[2025 AMC 12A Problems]]}}

[[AMC 12]]

[[AMC 12 Problems and Solutions]]

[[Mathematics competitions]]

[[Mathematics competition resources]]